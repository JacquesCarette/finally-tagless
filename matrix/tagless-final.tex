
\begin{document}

\section{Introduction}

See Section 1 of Walid02. Our subject area is that of a definitional DSL
interpreter.
Define Staging, meta-language and object language. (Walid uses
`subject' language).


\subsection{The Tag Problem}

\comment{see tagless_interp1.ml, module Tagfull for the complete code.
If you change the code in here, please adjust the .ml file
accordingly. Let the paper and the accompanying code be in sync.}


Consider the sample language of abstraction, application and boolean
literals with deBruin indices as variables. This is the language used
in the body of Walid02.

The language can be described by the following grammar (using OCaml
syntax)


\begin{code}
  type var = VZ | VS of var
  type exp = V of var | B of bool | A of exp * exp | L of exp
\end{code}

and a sample term is
\begin{code}
  let test1 = A ((L (V VZ)),(B true))
\end{code}

The first attempt at the interpreter: take the object language term
such as test1 and give us the value
\begin{code}
  let rec eval0 env = function 
  | V VZ -> List.hd env
  | V (VS v) -> eval0 (List.tl env) (V v)
  | B b -> b 
  | L e -> fun x -> eval0 (x::env) e
  | A (e1,e2) -> (eval0 env e1) (eval0 env e2) 
\end{code}

If our meta-language (the language in which we wrote test1
and that interpreter) were untyped, the above code would be acceptable.
The |L e| line exhibits interpretative overhead (see descr in Walid02
paper): |eval0 env' e| will be executed every time the result of
evaluating |L e| is applied. Staging can be used to remove the
interpretative overhead. Cite some paper -- see Walid02, Sec 1.1 and
1.2. They say it well.

If we use OCaml or other typed language as the meta-language, 
the function |eval0| is ill-typed. The line |B b|
tells that the function returns a boolean, whereas the next line says
the result of |eval0| is a function. Clearly, the result of the
function can't be both of these types. 
A related problem is the type of the env: since variables
in the language may be either of boolean or function types, a regular
OCaml list cannot be used for holding the corresponding values. 

The solution is to introduce the Universal type (see Walid02 Section
1.3)

\begin{code}
  type u = UB of bool | UA of (u -> u)
\end{code}

We can then write the interpreter, and it works
\begin{code}
  let rec eval env = function
  | V VZ -> List.hd env
  | V (VS v) -> eval (List.tl env) (V v)
  | B b -> UB b
  | L e -> UA (fun x -> eval (x::env) e)
  | A (e1,e2) -> match eval env e1 with UA f -> f (eval env e2)
\end{code}
whose inferred type is |u list -> exp -> u|,

\begin{code}
  let test1r = eval [] test1
  val test1r : u = UB true 
\end{code}

Unfortunately, the result is tagged: |UB true| rather than just
|true|. We also note that the eval function is partial in two
respects: first there is an inexhaustive pattern-matching in the 
line |A (e1,e2)|.
Second, the occurrences of |List.hd| and |List.tl|.
The failure of the latter corresponds to the evaluation of `open'
terms. One can see that if all our terms are closed, List.hd and
List.tl above will never be applied to the empty list. The partial
pattern match in the |A (e1,e2)| line raises an exception when we try to
evaluate something like |A (B true, B false)|, that is, the first
operand is not a function. 

Suppose our object language were typed, too. Then terms like
|A (B true, B false)| would be ill-typed. Thus, if our evaluator
accepted only typed terms, then the missing alternatives in the |A|
rule would not occur. Alas, what is obvious to us is not obvious to
the meta-language. There is still the pattern match, and we still have
to attach and eliminate the UB and UA tags, even when there is no
longer a need for them. See Walid02, Sec 1.4 for more detail.

Of course there is also the issue how to design a type system to type
terms like test1 (GADT seem necessary). Also, how to design the
typesystem to keep track of open/closed status of terms, so we can
statically assure that only closed terms are passed to eval and that
the Nil checking in List.hd and List.tl can be eliminated as well. The
latter is even more interesting problem requiring even more advanced
type systems (cite env classifiers, Nanevski, Rowan Davies).

The problem is also outlined in Simon Peyton-Jones GADT paper, but
only in the context of a first-order language

In short: (put into really introduction or the conclusions?)
It is not satisfactory
to use the Universal type because tagging hurts performance and
necessitates partial pattern matching in the metalanguage, even for a
well-typed object term.  


\subsection{Other solutions}
The tag problem motivated GADTs (SPJ, Xi) and dependent types
(WalidICFP02, Concoqtion). Or, perhaps just refer to related work?
But we need to say something here, to give the reader the feeling of
how advanced solutions were proposed and how rarely they are implemented.


\subsection{The Final Proposal}
We have realized that, if we change the encoding of the term from data
constructors to functions, then we can encode a tagless interpreter of a
typed object language in a metalanguage with a very simple type system.

\comment{see tagless_interp1.ml, module Tagless for the complete code.
If you change the code in here, please adjust the .ml file
accordingly. Let the paper and the accompanying code be in sync.}

Now, let us define our object language (embedded DSL) in a different
way. Instead of using algebraic datatypes to encode the terms of the
language in the meta-language (reminiscent to the initial approach),
we will be using functions as term constructors (final approach):

\begin{code}
  let b (bv:bool) env = bv
  let varZ (vc,_) = vc
  let varS vp (_,envr) = vp envr
  let app e1 e2 env = (e1 env) (e2 env)
  let lam e env = fun x -> e (x,env)
\end{code}

That is all of the interpreter. The previous term |test1| now reads
\begin{code}
  let testf1 = app (lam varZ) (b true)
\end{code}

To evaluate the term, we merely need to apply that to the empty
environment

\begin{code}
  let testf1r = testf1 ()
  val testf1r : bool = true
\end{code}

Now we see that the result has no tags. The above functions |b|,
|varZ|, |app| constitute the evaluator. It is patent that the
interpreter has no tags and no pattern matching. A term |b true| is
evaluated to a boolean and a term |lam varZ| is evaluated to a
function, with no tags attached. The application rule than applies that
function without any need for pattern-matching. The evaluator has
another feature: evaluation of open terms leads to a type error rather
than a run-time error. Indeed, the following gives a type error

\begin{code}
  let testf1bad = app (lam (varS varZ)) (b true)
  let testf1badr = testf1bad ()
\end{code}
that the initial environment should be a tuple rather than () --
meaning the term requires non-empty initial environment, the term is
open.

Thus all the errors that previously reported at run-time (open terms,
applications of non-functions) are now reported at compile
time. Running of the interpreter assuredly gives no run-time errors
because its code uses no run-time error-raising constructs.



The interpreter uses no Universal type or pattern matching, and it
is obvious to the compiler that non-termination can only result from
interpreting fix.  If the source language is strongly normalizing, then
our interpreter is total.

We should emphasize that not only our code can't have pattern-match
errors, that fact is apparent to the compiler itself. It is one thing
to write a code with pattern-matching and claim (and prove) that
missing cases cannot occur. However true that may be, the fact the
missing cases cannot occur is not evident to the compiler, which still
has to generate code to report missing cases or compile in the error
generation. That's where efficiency is lost, and Walid's papers on tag
elimination have numbers showing that performance is lost indeed. 
In our code, there is simply no pattern-mismatch opportunity: the
proof of no bad match is expressed in the code itself. 


In the above code, the interpreter is wired in, built into the
functions |b|, |app|, etc. We explain how to abstract the interpreter
below, to make it possible to `evaluate' the same term in many different
ways (e.g., interpret term, compile it, compute its size or depth,
etc.)

\subsection{Contributions}
The term `constructor' functions lam, app, etc. look like free
variables in the encoding of an object term.  Defining these functions
differently gives rise to different interpreters.  Given the same term
representation but varying the interpreters, we can
	- evaluate the term to a value in the metalanguage
	- determine the size or depth of a term
	- compute a _CBN_ CPS transformation of a term
	- `compile' the term. This requires staging, as in MetaOCaml.
	- Surprisingly, it is also possible to write a partial
	evaluator! 

We have implemented the above in (Meta)OCaml and Haskell.

What are our contributions:
0) we interpret/compile/pe the typed source language
   [Ken says that Danvy's source language is untyped]
1) a much clearer implementation
2) a nicely comparable Haskell/MetaOCaml version
3) a functorial Symantics that deals with interp/comp/pe uniformly
4) clear and easy extensibility to more features
- only closed terms can be evaluated
Making sure that we interpret terms that are well-typed and closed --
and we can do that in far simpler type system (e.g., Haskell or
OCaml). It was thought that doing either requires quite advanced
extensions (GADT, Nanevski, Concoqtion).


\section{Our base language: Tagless interpreter}

Introduce the language, 
  The language is simply-typed lambda-calculus with fixpoint,
  integers, booleans and comparison.

  Lam hoas_fn | App e e | Fix hoas_fn |
  I Int | B Bool | Add ie1 ie2 | Mul ie1 ie2 | Leq ie1 ie2 |
  IF b e-then e-else


this time with HOAS rather than deBruin indices. The language is close 
to the language of Xi03, but without the polymorphic lift and more 
constants so we can write better examples, enough to express
Fibonacci, factorial, and power.

How to make encoding flexible: abstract the interpreter
Haskell: type classes, OCaml: functors (give the signature of
Symantics without the 'sv part)

We have implemented the above in (Meta)OCaml and Haskell.  In Haskell,
the term `constructor' functions are defined as methods in a type class
Symantics (the name means that the class interface gives the syntax for
the source language and its instances give the semantics).
	class Symantics r where
	  int :: Int -> r Int
	  add :: r Int -> r Int -> r Int
	  lam :: (r a -> r b) -> r (a->b)
	  app :: r (a->b) -> r a -> r b
	  fix :: (r a -> r a) -> r a
	  ...
So, Haskell infers the type "Symantics r => r Int" for the example term
encoding above.


encoding of a few terms.


\section{Tagless compiler (aka, staged interpreter)}
Staged code in OCaml... Don't show the Haskell code (too much)


\section{Tagless partial evaluator}

the sv type problem: the inductive type function 'dv -> 'sv
its GADT solution. 
its type-level function solution: incope1.hs 
  Need to thank Martin Sulzmann

How can we do without GADT
Now show the Symantics signature in full, with the 'sv types

It's relation to PE and CPS. Currently, our CPS is CBN. Need to
investigate CBV.


Surprisingly, it is also possible to write a partial evaluator!  The
partial evaluator does have tags: whether a value is static or dynamic.
The pattern-matching on this _phase_ tag is always exhaustive; we still
have no universal type and no tags for object types.  The partial
evaluator uses the previously mentioned compiler to `interpret' dynamic
terms.  In this sense, the partial evaluator is a modular extension to
the compiler.


\section{Variations and discussion}

state and imperative features

CPS transformation (interpreting/compiling the term to CBN CPS)

\subsection{Translucent types}
The crucial role of the higher-order type parameter r

The type constructor "r" above represents a particular interpreter.  The
meta-type "r tau" hides how the interpreter represents the object type
"tau" yet exposes enough of the type information so we can type-check
the encoding of an object term without knowing what "r" is.  The checked
term is then well-typed in any interpreter.  Each instance of Symantics
instantiates "r" to interpret terms in a particular way.

\section{Self-interpretation}



\section{Related work}
Walid's Jones optimality paper. See his own ICFP02 paper that
described what he achieved then and what he _did not_. He did not
eliminate all the tags.

Walid's ICFP02

Walid's PEPM07 (Concoqtion paper)? At leats at the Concoqtion
presentation at WG2.11, Emir used tagless interpreter as the main and
only example.

Danvy's self-interpreter
In sharp contrast to Danvy's paper, all the evaluators in our case
cannot produce any pattern-match error! Except for the partial
evaluator, there are NO patterns to match. And this is critical for
tag elimination -- pattern matching means there are no tags. And in
PE, the pattern-match deals with phases rather than with types. Also,
in our case the exhaustiveness of the pattern match is apparent to the
compiler -- that's why there are no tags.

Simon Peyton-Jones GADTs (the motivation for GADT is tagless
interpreter, but only for the first-order language)

Xi's POPL03 (He uses a higher-order language, one of the motivations
for his GRDT)

Asai; Simii et al.

"Boxes go Bananas" by Washburn and Weirich (intended or otherwise)


Typing Dep Typing plus two messages on Haskell-Cafe: typechecking 
from untyped to the typed form, ready to be interpreted. Walid's ICFP02
uses dept types


\section{Conclusions}

Our approach showed better the meaning of GADT and when they can be
avoided.

\end{document}
