\documentclass[preprint]{sigplanconf}
\usepackage{amsmath}
\usepackage{natbib1}

\newcommand{\jacques}[1]{{\it [Jacques says: #1]}}
\newcommand{\oleg}[1]{{\it [Oleg says: #1]}}
\newcommand{\ccshan}[1]{{\it [Ken says: #1]}}

\usepackage[compact]{fancyvrb1}
\DefineShortVerb{\|}
\DefineVerbatimEnvironment{code}{Verbatim}{xleftmargin=1em,fontsize=\small}

\begin{document}

\conferenceinfo{ICFP '07}{Freiburg, Germany} 
\copyrightyear{2007} 
\copyrightdata{[to be supplied]} 

\titlebanner{banner above paper title}        % These are ignored unless
\preprintfooter{short description of paper}   % 'preprint' option specified.

\title{Staged typed type-preserving tagless interpreters, \emph{finally}}
\subtitle{Subtitle Text, if any}

\authorinfo{Jacques Carette}
           {McMaster University}
           {carette@mcmaster.ca}
\authorinfo{Oleg Kiselyov}
           {FNMOC}
           {oleg@pobox.com}
\authorinfo{Chung-chieh Shan}
           {Rutgers University}
           {ccshan@cs.rutgers.edu}

\maketitle

\begin{abstract}
This is the text of the abstract.
\end{abstract}

%\category{CR-number}{subcategory}{third-level}
%\terms term1, term2
%\keywords keyword1, keyword2

\section{Introduction}

See Section 1 of Walid02. Our subject area is that of a definitional DSL
interpreter.
Define Staging, meta-language and object language. (Walid uses
`subject' language).

We have programmed out interpreters in OCaml/MetaOCaml (cite
www.metaocaml.org) and Haskell. The complete code is available 
% connot give URL in the ICFP paper due to double-blind reviews
in the supplemental material to the paper. We will use both
(Meta)OCaml and Haskell for our examples throughout the paper.


\subsection{The Tag Problem}

\oleg{see tagless\_interp1.ml, module Tagfull for the complete code.
If you change the code in here, please adjust the .ml file
accordingly. Let the paper and the accompanying code be in sync.}


Consider the sample language of abstraction, application and boolean
literals with deBruin indices as variables. This is the language used
in the body of Walid02.

The language can be described by the following grammar (using OCaml
syntax)


\begin{code}
  type var = VZ | VS of var
  type exp = V of var | B of bool | A of exp * exp | L of exp
\end{code}

and a sample term is
\begin{code}
  let test1 = A ((L (V VZ)),(B true))
\end{code}

The first attempt at the interpreter, the function eval. It takes the
object language term such as test1 and gives us its value.
The first argument to |eval| is the environment, initially empty,
which is the sequence of values associated with free variables in the
interpreted code.
\begin{code}
  let rec eval0 env = function 
  | V VZ -> List.hd env
  | V (VS v) -> eval0 (List.tl env) (V v)
  | B b -> b 
  | L e -> fun x -> eval0 (x::env) e
  | A (e1,e2) -> (eval0 env e1) (eval0 env e2) 
\end{code}

Suppose our our meta-language (the language in which we wrote test1
and that interpreter) were untyped. The above code would be acceptable.
The |L e| line exhibits interpretative overhead (see descr in Walid02
paper): |eval0 env' e| will be executed every time the result of
evaluating |L e| is applied. Staging can be used to remove the
interpretative overhead. Cite some paper -- see Walid02, Sec 1.1 and
1.2. They say it well.

If we use OCaml or other typed language as the meta-language, 
the function |eval0| is ill-typed. The line |B b|
tells that the function returns a boolean, whereas the next line says
the result of |eval0| is a function. Clearly, the result of the
function can't be both of these types. 
A related problem is the type of the env: since variables
in the language may be either of boolean or function types, a regular
OCaml list cannot be used for holding the corresponding values. 

The solution is to introduce the Universal type (see Walid02 Section
1.3)

\begin{code}
  type u = UB of bool | UA of (u -> u)
\end{code}

We can then write the interpreter, which is accepted by the typecheker
\begin{code}
  let rec eval env = function
  | V VZ -> List.hd env
  | V (VS v) -> eval (List.tl env) (V v)
  | B b -> UB b
  | L e -> UA (fun x -> eval (x::env) e)
  | A (e1,e2) -> match eval env e1 with UA f -> f (eval env e2)
\end{code}
and whose inferred type is |u list -> exp -> u|. Now we can evaluate
our sample |test1| term:

\begin{code}
  let test1r = eval [] test1
  val test1r : u = UB true 
\end{code}

Unfortunately, the result is tagged: |UB true| rather than just
|true|. We also note that the eval function is partial in two
respects: first there is an inexhaustive pattern-matching in the 
line |A (e1,e2)|.
Second, the occurrences of |List.hd| and |List.tl|.
The failure of the latter corresponds to the evaluation of `open'
terms. One can see that if all our terms are closed, |List.hd| and
|List.tl| above will never be applied to the empty list. The partial
pattern match in the |A (e1,e2)| line raises an exception when we try to
evaluate something like |A (B true, B false)|, that is, the first
operand is not a function. 

Suppose our object language were typed, too. Then terms like
|A (B true, B false)| would be ill-typed. Thus, if our evaluator
accepted only typed terms, then the missing alternatives in the |A|
line would not occur. Alas, what is obvious to us is not obvious to
the meta-language. There is still the pattern match, and we still have
to attach and eliminate the UB and UA tags, even when there is no
longer a need for them. See Walid02, Sec 1.4 for more detail.

Of course there is also the issue how to design a type system to type
terms like test1 (GADT seem necessary). Also, how to design the
typesystem to keep track of open/closed status of terms, so we can
statically assure that only closed terms are passed to eval and that
the empty list checking in |List.hd| and |List.tl| can be eliminated
as well. The latter is even more interesting problem requiring even
more advanced type systems (cite env classifiers, Nanevski, Rowan
Davies).

The problem is also outlined in Simon Peyton-Jones GADT paper, but
only in the context of a first-order language

In short: (put into really introduction or the conclusions?)
It is not satisfactory
to use the Universal type because tagging hurts performance and
necessitates partial pattern matching in the metalanguage, even for a
well-typed object term.  


\subsection{Other solutions}
The tag problem motivated GADTs (SPJ, Xi) and dependent types
(WalidICFP02, Concoqtion). Or, perhaps just refer to related work?
But we need to say something here, to give the reader the feeling of
how advanced solutions were proposed and how rarely they are implemented.


\subsection{The `Final' Proposal}
We have realized that, if we change the encoding of the term from data
constructors to functions, then we can encode a tagless interpreter of a
typed object language in a metalanguage with a very simple type system.

\oleg{see tagless\_interp1.ml, module Tagless for the complete code.
If you change the code in here, please adjust the .ml file
accordingly. Let the paper and the accompanying code be in sync.}

Now, let us define our object language (embedded DSL) in a different
way. Instead of using algebraic datatypes to encode the terms of the
language in the meta-language (reminiscent to the initial approach),
we will be using functions as term constructors (the final approach):

\begin{code}
  let b (bv:bool) env = bv
  let varZ (vc,_) = vc
  let varS vp (_,envr) = vp envr
  let app e1 e2 env = (e1 env) (e2 env)
  let lam e env = fun x -> e (x,env)
\end{code}

That is all of the interpreter. Our sample term now reads
\begin{code}
  let testf1 = app (lam varZ) (b true)
\end{code}
which is almost the same as before, only written with lower-case
identifiers. To evaluate the term, we merely need to apply 
the term to the empty environment

\begin{code}
  let testf1r = testf1 ()
  val testf1r : bool = true
\end{code}

The result now has no tags. The above functions |b|,
|varZ|, |app| constitute the evaluator. It is patent that the
interpreter has no tags and no pattern matching. The term |b true| is
evaluated to a boolean and a term |lam varZ| is evaluated to a
function, with no tags attached. The application rule applies that
function without any need for pattern-matching. The evaluator has
another feature: evaluation of open terms leads to a type error rather
than a run-time error. Indeed, the following gives a type error

\begin{code}
  let testf1bad = app (lam (varS varZ)) (b true)
  let testf1badr = testf1bad ()
\end{code}
that the initial environment should be a tuple rather than () --
meaning the term requires non-empty initial environment: the term is
open.

Thus all the errors that were previously reported at run-time (open terms,
applications of non-functions) are now reported at compile
time. Running of the interpreter assuredly gives no run-time errors
because its code uses no operations that could possibly raise a
run-time error.



The interpreter uses no Universal type or pattern matching, and it
is obvious to the compiler that non-termination can only result from
interpreting |fix|.  If the source language is strongly normalizing, then
our interpreter is total.

We should emphasize that not only our code can't have pattern-match
errors, that fact is apparent to the compiler itself. It is one thing
to write a code with pattern-matching and claim (and prove) that
missing cases cannot occur. However true that may be, the fact the
missing cases cannot occur is not evident to the compiler, which still
has to generate code to report missing cases or compile in the error
generation. That's where efficiency is lost, and Walid's papers on tag
elimination have numbers showing that performance is lost indeed. 
In our code, there is simply no pattern-mismatch opportunity: the
proof of no bad match is expressed in the code itself. 


In the above code, the interpreter is wired in, built into the
functions |b|, |app|, etc. We explain how to abstract the interpreter
below, to make it possible to `evaluate' the same term in many different
ways (e.g., interpret term, compile it, compute its size or depth,
etc.)

\subsection{Contributions}
The term `constructor' functions lam, app, etc. look like free
variables in the encoding of an object term.  Defining these functions
differently gives rise to different interpreters.  Given the same term
representation but varying the interpreters, we can
	- evaluate the term to a value in the metalanguage
	- determine the size or depth of a term
        - compute a \emph{CBN} CPS transformation of a term
	- `compile' the term. This requires staging, as in MetaOCaml.
	- Surprisingly, it is also possible to write a partial
	evaluator! 

We have implemented the above in (Meta)OCaml and Haskell.

What are our contributions:
0) we interpret/compile/pe the typed source language
   [Ken says that Danvy's source language is untyped]
1) a much clearer implementation
2) a nicely comparable Haskell/MetaOCaml version
3) a functorial Symantics that deals with interp/comp/pe uniformly
4) clear and easy extensibility to more features
- only closed terms can be evaluated
Making sure that we interpret terms that are well-typed and closed --
and we can do that in far simpler type system (e.g., Haskell or
OCaml). It was thought that doing either requires quite advanced
extensions (GADT, Nanevski, Concoqtion).

The structure of our paper is as follows. Section XXX ...
We finally conclude.


\section{Our base language: Tagless interpreter}

Introduce the language, The language is simply-typed lambda-calculus
with fixpoint, integers, booleans and comparison, whose BNF grammar is

\begin{code}
  e ::= Lam hoas_fn | App e e | Fix hoas_fn |
  I int_literal | B bool_literal | Add e1 e2 | Mul e1 e2 | Leq e1 e2 |
  IF b e-then e-else
\end{code}

This time we use the higher-order abstract syntax (cite: but not
Elliott. Cite Dale Miller!) rather than deBruijn indices.  The
language is close to the language of Xi03, without the polymorphic
lift but with more constants so we can write better examples, enough to
express Fibonacci, factorial, and power.

How to make encoding flexible: abstract the interpreter
We have implemented the above in (Meta)OCaml and Haskell.  In Haskell,
the term `constructor' functions are defined as methods in a type class
Symantics (the name means that the class interface gives the syntax for
the source language and its instances give the semantics).



\begin{code}
class Symantics repr where
    int :: Int -> repr Int                -- int literal
    bool :: Bool -> repr Bool             -- bool literal

    lam :: (repr a -> repr b) -> repr (a->b)
    app :: repr (a->b) -> repr a -> repr b
    fix :: (repr a -> repr a) -> repr a

    add :: repr Int -> repr Int -> repr Int
    mul :: repr Int -> repr Int -> repr Int
    if_ :: repr Bool -> repr a -> repr a -> repr a
    leq :: repr Int -> repr Int -> repr Bool
\end{code}

Our sample code test1 now reads |app (lam (\x -> x)) (bool True)|,
whose inferred type is |Symantics repr => repr Bool|.

encoding of a few terms.

\begin{code}
testpowfix () = 
  lam (\x -> fix (\self -> lam (\n ->
        if_ (leq n (int 0)) (int 1)
            (mul x (app self (add n (int (-1))))))))
testpowfix7 () = 
  lam (\x -> app (app (testpowfix ()) x) (int 7))
\end{code}
The dummy unit |()| argument above is to avoid the monomorphism
restriction and so that the type of |testpowfix| and |testpowfix7|
would remain polymorphic in |repr|. Instead of supplying the dummy
argument, we could have given the terms explicit polymorphic
sugnature. We however prefer if the typechecker of the meta-language
(Haskell) inferred the types of the object terms for us. We could also
avoid the |()| argument by switching off the monomophism restriction
with a compiler flag.


OCaml: functors.

\begin{code}
module type Symantics = sig
  type ('c,'dv) repr
  val int  : int  -> ('c,int) repr
  val bool : bool -> ('c,bool) repr
  val add  : ('c,int) repr -> ('c,int) repr -> ('c,int) repr
  ...
  val if_ : ('c,bool) repr ->
             (unit -> ('c,'da) repr) ->
             (unit -> ('c,'da) repr) -> ('c,'da) repr 

  val lam : (('c,'da) repr -> ('c,'db) repr) 
          -> ('c,'da->'db) repr
  val app : ('c,'da->'db) repr
    -> ('c,'da) repr -> ('c,'db) repr
  val fix : (('c,'da->'db) repr -> ('c,'da->'db) repr) 
            -> ('c,'da->'db) repr
end
\end{code}

Some differences are in the additional type parameter |c| (the
environment classifier, Walid03) to be used later when we get to code
generation (required by MetaOCaml, part of the type of the
code). MetaOCaml is a call-by-value language, and thus we have to eta-expand
the fix-point combinator and encapsulate the branches of the |if_|
into thunks to prevent their evaluation. 

The power example and our sample term |test1| are encoded as
\begin{code}
module EX(S: Symantics) = struct
 open S
 let test1 () = app (lam (fun x -> x)) (bool true)
 let testpowfix () = 
   lam (fun x ->fix (fun self -> lam (fun n ->
         if_ (leq n (int 0)) (fun () -> int 1)
             (fun () -> mul x (app self (add n (int (-1))))))))

 let testpowfix7 = 
    lam (fun x -> app (app (testpowfix ()) x) (int 7))
end
\end{code}
The dummy unit argument in |testpowfix| is the artefact of MetaOCaml
(which we will be using to compile our tests in Sections below). To be
more precise, the environment qualifier (which appears as the |'c|
type parameter in the signature of Symantic functions) must remain
uninstantiated in order for us to `run' the code. That means, the type
must be polymorhic. The value restriction then dictates 
the definitions of our object terms must look syntactically like
values. Alternatively to giving the |()| argument, we could have used
rank-2 record types of OCaml to maintain the necessary polymorphism.

Thus the object expressions (expressions of the embedded DSL) have in
OCaml the type of the functor, from |Symantics| to the value. This
essentially the same as |Symantics repr => | constraint in the case of
the embedding of the DSL in Haskell.



\section{Tagless compiler (aka, staged interpreter)}
Staged code in OCaml... Don't show the Haskell code (too much)


\section{Tagless partial evaluator}

the sv type problem: the inductive type function 'dv -> 'sv
its GADT solution. 
its type-level function solution: incope1.hs 

How can we do without GADT
Now show the Symantics signature in full, with the 'sv types

It's relation to PE and CPS. Currently, our CPS is CBN. Need to
investigate CBV.


Surprisingly, it is also possible to write a partial evaluator!  The
partial evaluator does have tags: whether a value is static or dynamic.
The pattern-matching on this \emph{phase} tag is always exhaustive; we still
have no universal type and no tags for object types.  The partial
evaluator uses the previously mentioned compiler to `interpret' dynamic
terms.  In this sense, the partial evaluator is a modular extension to
the compiler.


\section{Variations and discussion}

state and imperative features

CPS transformation (interpreting/compiling the term to CBN CPS)

We must formulate some propositions: a typed term makes progress in
any interpreter. The reason it is typed in any interpreter:
|forall r. Symantics r => r tau| directly says that in every model the
term is typed. 

\subsection{Translucent types}
The crucial role of the higher-order type parameter r

The type constructor "r" above represents a particular interpreter.  The
meta-type "r tau" hides how the interpreter represents the object type
"tau" yet exposes enough of the type information so we can type-check
the encoding of an object term without knowing what "r" is.  The checked
term is then well-typed in any interpreter.  Each instance of Symantics
instantiates "r" to interpret terms in a particular way.

\section{Self-interpretation}


Self-interpretation in our framework is
a bit tricky due to a need for polymorphism.  An encoded term should
be polymorphic in the interpreter type-constructor "r".  Further, the
functions "lam", "app", etc. above should be polymorphic in the object
types "a" and "b".  Thus the object language seems to require rank-2
polymorphism; and so type annotations will be inevitable.

To neatly bypass the need for rank-2 polymorphism, we use the notion
of let-polymorphism and a `hole', as follows.  A partial evaluator (or
a compiler or interpreter) is an evaluation context that (let-)binds
variables named "app", "lam", etc.  For the object language to
take advantage of let-bound polymorphism in the metalanguage, it
is crucial that we encode object "let" to meta "let".  We achieve
self-interpretation in that the following evaluation context in the
object language defines an interpreter for encoded object terms.
\begin{code}
    let lam = \f -> f in
    let app = \f x -> f x in
    let add = \m n -> m + n in
    let int = \i -> i in
    [hole]
\end{code}
The obvious notion of optimality in this setup is easy to achieve.  For
example, given the context above, our partial evaluator turns
\begin{code}
    app (lam (\x -> add x (int 1))) (int 3)
\end{code}
into the number 4, as desired.  We can also encode the self-interpreter
above as the following evaluation context in the object language.
\begin{code}
    let lam = lam (\f -> f) in
    let app = lam (\f -> lam (\x -> app f x)) in
    let add = lam (\m -> lam (\n -> add m n)) in
    let int = lam (\i -> i) in
    [hole]
\end{code}
The "let" and the hole "[hole]" are meta-constructions, which must map to
themselves in any interpretation.  


\section{Related work}
Walid's Jones optimality paper. See his own ICFP02 paper that
described what he achieved then and what he \emph{did not}. He did not
eliminate all the tags.

Walid's ICFP02

Walid's PEPM07 (Concoqtion paper)? At leats at the Concoqtion
presentation at WG2.11, Emir used tagless interpreter as the main and
only example.

Danvy's self-interpreter
In sharp contrast to Danvy's paper, all the evaluators in our case
cannot produce any pattern-match error! Except for the partial
evaluator, there are NO patterns to match. And this is critical for
tag elimination -- pattern matching means there are no tags. And in
PE, the pattern-match deals with phases rather than with types. Also,
in our case the exhaustiveness of the pattern match is apparent to the
compiler -- that's why there are no tags.

Simon Peyton-Jones GADTs (the motivation for GADT is tagless
interpreter, but only for the first-order language)

Xi's POPL03 (He uses a higher-order language, one of the motivations
for his GRDT)

Asai; Simii et al.

"Boxes go Bananas" by Washburn and Weirich (intended or otherwise)


Typing Dep Typing plus two messages on Haskell-Cafe: typechecking 
from untyped to the typed form, ready to be interpreted. Walid's ICFP02
uses dept types


\section{Conclusions}

Our approach showed better the meaning of GADT and when they can be
avoided.

\subsection*{Acknowledgments}
We woul like to tank Martin Sulzmann for helpful discussions.

\end{document}
