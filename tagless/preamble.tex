\usepackage{amsmath}
\usepackage{amssymb}
\usepackage{comment}
\usepackage{hyphenat}
\usepackage{url}
\usepackage{prooftree1} \proofrulebaseline=1.5ex \proofdotseparation=.75ex
\usepackage{mdwlist}
\usepackage{mdwtab}

% Set pdftex paper size
\ifx\pdfoutput\undefined\else \ifx\pdfoutput\relax\else \ifcase\pdfoutput\else
    \setlength{\pdfpagewidth}{8.5in}
    \setlength{\pdfpageheight}{11in}
\fi \fi \fi

\usepackage[compact]{fancyvrb1}
\DefineShortVerb{\|}
\DefineVerbatimEnvironment{code}{Verbatim}{xleftmargin=\parindent}
\DefineVerbatimEnvironment{code2}{BVerbatim}{xleftmargin=\parindent,baseline=t}
\DefineVerbatimEnvironment{code3}{Verbatim}{xleftmargin=\parindent,baseline=t,fontsize=\small}

\DeclareMathOperator{\pe}{PE}
\DeclareMathOperator{\si}{SI}

% Make some Rel symbols into Bin symbols instead
\DeclareMathSymbol{\to}{\mathbin}{symbols}{"21}
\DeclareMathSymbol{:}{\mathbin}{operators}{"3A}

\hyphenation{meta-language meta-languages Meta-OCaml meta-cir-cu-lar type-case}

\newcommand{\jacques}[1]{{\it [Jacques says: #1]}}
\newcommand{\oleg}[1]{{\it [Oleg says: #1]}}
\newcommand{\ccshan}[1]{{\it [Ken says: #1]}}

\newcommand{\ZZ}{\mathbb{Z}}
\newcommand{\BB}{\mathbb{B}}

\newcommand{\fun}[1]{\mathopen{\lambda\mathord{#1}.\,}}
\newcommand{\fix}[1]{\mathopen{\mathrm{fix\,}\mathord{#1}.\,}}
\newcommand{\Forall}[1]{\mathopen{\forall\mathord{#1}.\,}}
\newcommand{\Exists}[1]{\mathopen{\exists\mathord{#1}.\,}}
\newcommand{\cond}[3]{\mathrm{if\ }#1\mathrm{\ then\ }#2\mathrm{\ else\ }#3}
\newcommand{\be}[1]{\mathrm{let\ }#1\mathrm{\ in\ }}
\newcommand{\True}{\mathinner{\mathrm{true}}}
\newcommand{\False}{\mathinner{\mathrm{false}}}
\newcommand{\Encode}[1]{\mathinner{\mathopen{\text{``}}#1\mathclose{\text{''}}}}
\newcommand{\encode}[1]{\mathinner{\mathopen{\text{`}}#1\mathclose{\text{'}}}}
\newcommand{\ident}[1]{\mathinner{\text{\textit{#1\/}}}}
\newcommand{\subst}[3]{#1\left\{#2\mapsto#3\right\}}

\newenvironment{floatrule}{}{}

\newcommand{\ourthanks}{We thank Martin Sulzmann and Walid Taha 
for helpful discussions.  Sam Staton, Pieter Hofstra, and Bart Jacobs
kindly provided some useful references on coalgebraic structures in the
$\lambda$\hyp calculus. We thank anonymous reviewers for pointing out
related work.}
