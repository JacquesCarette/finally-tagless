\pagenumbering{arabic}
\documentclass[ucs,professionalfont]{beamer}
\usepackage{concmath}
\usepackage{helvet}
%\renewcommand{\rmdefault}{cmss}
\usepackage[utf8x]{inputenc}
\usepackage[override]{cmtt}
\usepackage{mdwtab}
\usepackage{prooftree1} \proofrulebaseline=1.5ex \proofdotseparation=.75ex
\usepackage{microtype}
\usepackage{calc}

\usepackage{tikz}
\usepgflibrary{shapes}
\usepgflibrary{arrows}

\usepackage{textcomp}
\makeatletter
\uc@dclc{8249}{default}{\textlangle}
\uc@dclc{8250}{default}{\textrangle}
\makeatother

\usepackage[compact]{fancyvrb1}
\DefineVerbatimEnvironment{code}{Verbatim}{commandchars=\\\{\}}

\def\>{\leavevmode\rlap{\color{blue}$\blacktriangleright$}\ \ }
\newcommand\slide[2]{#2<#1>}

%\newcommand{\ZZ}{\mathbold{Z}}
%\newcommand{\BB}{\mathbold{B}}
\newcommand{\ZZ}{\textrm{int}}
\newcommand{\BB}{\textrm{bool}}
\newcommand{\fun}[1]{\mathopen{\lambda\mathord{#1}.\;}}
\newcommand{\fix}[1]{\mathopen{\mathop{\textrm{fix}}\mathord{#1}.\;}}
\newcommand{\cond}[2]{\mathopen{\textrm{if}}\;#1\mathrel{\textrm{then}}#2\mathrel{\textrm{else}}}
\newcommand{\ceil}[1]{\lceil#1\rceil}

\setbeamertemplate{section in toc}{{\leavevmode\llap{$\blacktriangleright$ }\bfseries\inserttocsection}\par}
\setbeamertemplate{section in toc shaded}{{\color{black}\inserttocsection}\par}
\setbeamertemplate{subsection in toc shaded}[default][50]

\AtBeginSection[]
{
  \begin{frame}<beamer>
    \frametitle{Outline}
    \tableofcontents[currentsection]
  \end{frame}
}

% To get the page numbers to stay constant on a frame [dpt]
\mode<presentation>
{
  \setbeamertemplate{sidebar left}{\thispdfpagelabel{\insertframenumber}}
}

\title{Finally tagless, partially evaluated}
\subtitle{Tagless staged interpreters for simpler typed languages}
\author{Jacques Carette and Oleg Kiselyov and Chung-chieh Shan}

\begin{document}

\newcommand{\authorstack}[3]{\begin{footnotesize}\begin{tabular}[t]{@{}c@{}}\begin{normalsize}#1\end{normalsize}\\[\jot]#2\\\texttt{#3}\end{tabular}\end{footnotesize}\ignorespaces}
\newsavebox\carette \sbox\carette{\authorstack{Jacques Carette}{McMaster University}{carette@mcmaster.ca}}
\newsavebox\oleg    \sbox\oleg   {\authorstack{Oleg Kiselyov}{FNMOC}{oleg@pobox.com}}
\newsavebox\ccshan  \sbox\ccshan {\authorstack{Chung-chieh Shan}{Rutgers University}{ccshan@rutgers.edu}}

\setbeamertemplate{sidebar right}{}

\author[Jacques Carette and Oleg Kiselyov and Chung-chieh Shan]{\usebox\carette \and \usebox\oleg \and \usebox\ccshan}
\date{APLAS\\30 November 2007}

\begin{frame}
    \titlepage
\end{frame}

\setbeamertemplate{background}{%
    \raisebox{0pt}[\paperheight][0pt]
        {\includegraphics[width=\paperwidth+1pt,trim=0 100 0 0]{White_pearl_necklace.jpg}}%
    \llap{\href{http://www.flickr.com/photos/28481088@N00/160781390/}
               {\scriptsize\color{gray}tanakawho on flickr\kern1pt}}}
\begin{frame}
\end{frame}
\setbeamertemplate{background}[default]

\setbeamertemplate{sidebar right}{\vfill\llap{\usebeamerfont{framesubtitle}\insertframenumber\hskip\jot}\vskip\jot}

\tikzstyle{lang}=[ellipse,draw,anchor=west,text depth=3pt,text height=9pt]
\tikzstyle{every picture}=[>=angle 60,semithick]

\setbeamertemplate{background}{%
    \raisebox{0pt}[\paperheight][0pt]
        {\includegraphics[width=\paperwidth,trim=0 35 0 180,clip]{FC0836213122.jpg}}}
\begin{frame}<1>[t,label=everywhere]{There's interpretation everywhere}
    A fold on an inductive data type is an interpreter of a domain-specific
    language.

    \bigskip
    \begin{tikzpicture}
        \draw node [lang] (lang0) {contract};
        \foreach \label/\prev/\cur in {grammar/0/1,music/1/2,$\lambda$-term/2/3}
            \draw (lang\prev.east)+(.3,0) node [lang] (lang\cur) {\label};
        \draw (lang3.east)+(.2,0) node [anchor=west] {\dots};
    \onslide<2>
        \draw (lang0)+(-110:2cm)   node (terp01) {value};
        \draw (lang0)+( -85:2.5cm) node (terp02) {schedule};
        \draw (lang1)+(-110:2cm)   node (terp11) {parse};
        \draw (lang1)+( -85:2.5cm) node (terp12) {pretty-print};
        \draw (lang2)+(-110:2cm)   node (terp21) {typeset};
        \draw (lang2)+( -85:2.5cm) node (terp22) {perform};
        \draw (lang3)+(-110:2cm)   node (terp31) {evaluate};
        \draw (lang3)+( -85:2.5cm) node (terp32) {compile};
        \foreach \lang in {0,...,3} {
            \draw (lang\lang)+(-65:1.5cm) node {\dots};
            \foreach \terp in {1,...,2}
                \draw [->] (lang\lang) -- (terp\lang\terp);
        }
    \onslide<3>
        \draw (lang3)+(-148:4.5cm)  node [inner xsep=0pt] (terp31) {evaluate};
        \draw (lang3)+(-133:4.25cm) node [inner xsep=0pt] (terp32) {compile};
        \draw (lang3)+(-113:4cm)    node [inner xsep=0pt] (terp33) {specialize};
        \draw (lang3)+( -93:4.4cm)  node [inner xsep=0pt] (terp34) {CPS transform};
        \draw (lang3)+( -73:4cm)    node [inner xsep=0pt] (terp35) {pretty-print};
        \draw (lang3)+( -58:3cm)    node {\dots};
        \foreach \terp in {1,...,5} \draw [->] (lang3) -- (terp3\terp);
    \end{tikzpicture}

    \only<2>{The same language can be interpreted in many useful ways.}%
    \only<3>{We focus on the $\lambda$-calculus as an example.}
\end{frame}
\setbeamertemplate{background}[default]
\againframe<2->[t]{everywhere}

\begin{frame}{Simple type preservation}
    \def\bool{\only<7-8>{\ \alert{bool}}}
    \begin{tikzpicture}
        \draw node [lang] (lang)
          {\begin{overprint}[12em]
               \onslide<1>\makebox[12em]{\alert{Typed} source language}
               \onslide<2>\makebox[12em]{$3 \le 4$}
               \onslide<3-7>\makebox[12em]{\texttt{LEQ (LIT 3, LIT 4):\makebox[4em][l]{\bool\ term}}}
               \onslide<8>\makebox[12em]{\texttt{\alert{leq} (\alert{lit} 3, \alert{lit} 4):\makebox[4em][l]{\bool}}}
               \onslide<9>\makebox[12em]{Simply typed $\lambda$-calculus}
           \end{overprint}};
        \draw [->] (lang) --
            node [above,midway,anchor=east]
              {\only<1>{\alert{Typed} metalanguage\hspace*{2em}}%
               \only<2-3>{evaluate}%
               \only<9>{\begin{tabular}{@{}r@{\qquad}}
                        Haskell constructor instances\\
                        or ML functors
                        \end{tabular}}}
            node [above,midway,anchor=east,text width=2.5in,inner xsep=0pt]
                {\only<-3,9>{\color{bg}}\ttfamily
                \alt<8->{\only<8>{\alert<8>{lit:\ int -> int}\\
                lit i = i\\[1ex]
                \alert<8>{leq:\ int * int -> bool}\\
                leq (i,j) = i <= j}}
                {\only<3-7>{\alert<7>{eval:\ \only<7>{'a }term -> \only<7>{'a }value}\\
                eval (LIT i) = INT i\\
                eval (LEQ (e1,e2)) =\\
                \quad \alert<5>{match (eval e1, eval e2) with\\
                \quad (INT i, INT j) ->} BOOL (i <= j)}}}
            +(-100:4cm)
            node [anchor=north,text depth=0pt,text height=9pt]
              {\llap{\onslide<3-7>{\texttt{BOOL }}%
                     \onslide<3-8>{\texttt{true:}}%
                     \only<1>{\rlap{\alert{Typed} target languages}}%
                     \only<2>{\rlap{\enspace\textrm{true}}}%
                     \only<9>{\rlap{evaluate}}}%
               \makebox[3em][l]{\texttt{\bool\only<3-7>{\ value}}}};
        \draw [->] (lang) -- +(-80:3.5cm)
            node [anchor=north west,inner xsep=0pt] {\only<9>{\hspace*{-16pt}compile}};
        \draw [->] (lang) -- +(-65:3cm)
            node [anchor=north west,inner xsep=0pt] {\only<9>{specialize}};
        \draw [->] (lang) -- +(-50:2.5cm)
            node [anchor=north west,inner xsep=0pt] {\only<9>{\hspace*{-3pt}CPS transform}};
        \draw      (lang)    +(-35:2cm) node {\dots};
    \end{tikzpicture}

    \bigskip

    \strut
    \only<-4>{It should be obvious in the metalanguage that interpreting
    a well-typed source term yields a well-typed target term.}%
    \only<5-6>{The term should be \alert{\alt<5>{well-typed}{closed}}, so
    \alert{\alt<5>{pattern matching}{environment lookup}} in the metalanguage should always \textbf{obviously} succeed.}%
    \only<7>{Previous solutions use (and motivate) fancier types:\\
    generalized abstract data types (GADT) and dependent types.}%
    \only<8>{Our simple solution is to be \textbf{finally tagless:}\\
    replace term constructors by cogen functions.}%
    \only<9>{The term accommodates \textbf{multiple interpretations}\\
    by abstracting over the cogen functions and their types.}%
    \strut
\end{frame}

\section{The object language}
\subsection{As a constructor class in Haskell}
\subsection{As a functor signature in ML}

\begin{frame}[t,fragile]{The object language\only<3-8>{ as a constructor class}\only<9-16>{ as a functor signature}}
\setbeamercovered{transparent=20}
\begin{overprint}
\onslide<1-2>
\medskip
\begin{proofrules}
    \[ \[ [x:t_1] \proofoverdots e:t_2 \] \justifies \fun{x}e:t_1\to t_2 \]
    \[ \[ [f:t_1\to t_2] \proofoverdots e:t_1\to t_2 \] \justifies \fix{f}e:t_1\to t_2 \]
    \[ e_1:t_1\to t_2 \quad e_2:t_1 \justifies e_1 e_2: t_2 \]
    \[ \text{$n$ is an integer} \justifies n:\ZZ \]
    \uncover<1>{\[ \text{$b$ is a boolean} \justifies b:\BB \]}
    \[ e_1:\ZZ \quad e_2:\ZZ \justifies e_1+e_2:\ZZ \]
    \uncover<1>{\[ e_1:\ZZ \quad e_2:\ZZ \justifies e_1 \times e_2:\ZZ \]}
    \uncover<1>{\[ e_1:\ZZ \quad e_2:\ZZ \justifies e_1 \le e_2:\BB \]}
    \[ e:\BB \quad e_1:t \quad e_2:t \justifies \cond{e}{e_1}{e_2}:t \]
\end{proofrules}
\onslide<3-8>
\begin{semiverbatim}
class Symantics \alert<3>{repr} where
     \alert<4,5>{int }\alert<5>{:\!: Int -> repr Int}
     \alert<4,6>{lam }\alert<6>{:\!: (repr a -> repr b) -> repr (a -> b)}
     \alert<4,6>{fix }\alert<6>{:\!: (repr a -> repr a) -> repr a}
     \alert<4,7>{app }\alert<7>{:\!: repr (a -> b) -> repr a -> repr b}
     \alert<4  >{add }:\!: repr Int -> repr Int -> repr Int
     \alert<4,8>{if_ }\alert<8>{:\!: repr Bool -> repr a -> repr a -> repr a}
\end{semiverbatim}
\onslide<9-16>
\begin{semiverbatim}
module type Symantics = sig type \alert<9>{\smash{('c,'a)}\,repr}
 val \alert<10,11>{int}\alert<11>{:\,int\,->\,('c,int)\,repr}
 val \alert<10,12>{lam}\alert<12>{:\,(('c,'a)\,repr\,->\,('c,'b)\,repr)\,->\,('c,'a->'b)\,repr}
 val \alert<10,12>{fix}\alert<12>{:\,('x\,->\,'x)\,->\,(('c,'a->'b)\,repr as 'x)}
 val \alert<10,13>{app}\alert<13>{:\,('c,'a\,->\,'b)\,repr\,->\,('c,'a)\,repr\,->\,('c,'b)\,repr}
 val \alert<10   >{add}:\,('c,int)\,repr\,->\,('c,int)\,repr\,->\,('c,int)\,repr
 val \alert<10,14>{if_}\alert<14>{:\,('c,bool)\,repr\,->\,(unit\,->\,'x)\,->\,(unit\,->\,'x)}
     \alert<10,14>{   }\alert<14>{ \,         \,    \,->\,(('c,'a)\,repr as 'x)}
end
\end{semiverbatim}
\end{overprint}

\onslide<1->
\begin{tabular}{@{}Mll@{}}
\only<5-8,11-14>{\text{\textcolor{structure.fg}{Object term}}}%
\only<16>{\rlap{\textcolor{structure.fg}{ML higher-order functor}}}
& \only<5-8,11-15>{\textcolor{structure.fg}{\only<-14>{\llap{\tikz\draw[->](0,0)--(1.5,0)node[anchor=mid,inner sep=0pt]{\vphantom{O}};\:}}\only<3-8>{Haskell term}\only<9-15>{ML higher-order functor}}} \\
\multicolumn{2}{@{}l@{}}{\only<16>{\alert{\texttt{module POWER (S:Symantics) = struct open S}}}} \\
\only<16>{\rlap{\alert{\texttt{ let term () =}}}}%
\visible<1-14>{\uncover<1-2,5-8,11-14>{\alert<6,12>{\fun{x} \fix{f} \fun{n}}}}
& \ttfamily
  \only<5-8>{\alert<6>{lam (\textbackslash x -> fix (\textbackslash f -> lam (\textbackslash n ->}}%
  \only<11-16>{\alert<12>{lam\,(fun x\,-> fix\,(fun f\,-> lam\,(fun n\,->}} \\
\visible<1-14>{\uncover<1-2,5-8,11-14>{\alert<8,14>{\cond{\textcolor{black}{n\le\alert<5,11>{0}}}{\textcolor{black}{1}}}}}
& \ttfamily
  \only<5-8>{\alert<8>{if\textunderscore} \alert<8>(leq n (\alert<5>{int 0})\alert<8>) \alert<8>(int 1\alert<8>)}%
  \only<11-16>{\alert<14>{if\textunderscore} \alert<14>(leq n (\alert<11>{int 0})\alert<14>) \alert<14>{(fun\,()\,->}\,int 1\alert<14>)} \\
\visible<1-14>{\uncover<1-2,5-8,11-14>{x\times \alert<7,13>{f(n-1)}}}
& \ttfamily
  \only<5-8>{\alert<8>(mul x (\alert<7>{app f (add n (int (-1)))})\alert<8>))))}%
  \only<11-16>{\alert<14>{(fun\,()\,->}\,mul x\,(\alert<13>{app f\,(add n\,(int\,(-1)\!)\!)}\!)\!\alert<14>)\!)\!)\!)} \\
\multicolumn{2}{@{}l@{}}{\only<16>{\alert{\texttt{end:\ functor (S:Symantics) -> sig}}}} \\
\only<16>{\rlap{\alert{\texttt{ val term:\ unit\,->}}}}%
\visible<1-14>{\uncover<1-2,5-8,11-14>{\mathord:\;\ZZ\to\ZZ\to\ZZ}}
& \ttfamily
  \only<5-8>{:\!:\ \!Symantics \!repr \!=>\,repr\,(Int\,->\,Int\,->\,Int)}%
  \only<11-16>{('c,\,int\,->\,int\,->\,int)\,\only<16>{\alert{S.}}repr} \\
\multicolumn{2}{@{}l@{}}{\only<16>{\alert{\texttt{end}}}}
\end{tabular}
\end{frame}

\begin{frame}[fragile]{Composing object programs as functors}
\[
    \visible<2->{\alert{\fun{x}}}
    \bigl( \fun{x} \fix{f} \fun{n} \cond{n\le0}{1}{x\times f(n-1)} \bigr)
    \visible<2->{\alert{\;x\;7}}
\]
\onslide<3->
\begin{semiverbatim}
module POWER7 (S:Symantics) = struct open S
 module P = POWER(S)
 let term () = \alert{lam (fun x -> app (app} (P.term ()) \alert{x)}
                                 \alert{(int 7))}
end: functor (S:Symantics) -> sig
 val term: unit -> ('c,\,int\,->\,int)\,S.repr
end
\end{semiverbatim}
\end{frame}

\section{Tagless interpretation}
\subsection{Evaluation}
\subsection{Compilation}

\begin{frame}[t,fragile]{Tagless interpretation: \alt<5->{Compilation}{Evaluation}}

No worry about pattern matching or environment lookup!

Well-typed source programs \textbf{obviously}
\alt<5->{\alert{translate to well-typed target programs.}}
        {don't go wrong.\\}%
\strut
\begin{semiverbatim}
module \alt<5->{C}{R} = struct
 \alert<1>{type ('c,'a) repr = \only<4->{\visible<5->{\alert<5>{('c,}}}'a\only<4->{\visible<5->{\alert<5>{) code}}}}
 let int (x:int) = \only<4->{\visible<5->{\alert<5>{‹}}}x\only<4->{\visible<5->{\alert<5>{›}}}
 let lam f       = \only<4->{\visible<5->{\alert<5>{‹}}}fun x -> \only<4->{\visible<5->{\alert<5>{~(}}}f \only<4->{\visible<5->{\alert<5>{‹}}}x\only<4->{\visible<5->{\alert<5>{›)›}}}
 let fix g       = \only<4->{\visible<5->{\alert<5>{‹}}}let rec f n = \only<4->{\visible<5->{\alert<5>{~(}}}g \only<4->{\visible<5->{\alert<5>{‹}}}f\only<4->{\visible<5->{\alert<5>{›)}}} n in f\only<4->{\visible<5->{\alert<5>{›}}}
 let app e1 e2   = \only<4->{\visible<5->{\alert<5>{‹}}}\only<4->{\visible<5->{\alert<5>{~}}}e1 \only<4->{\visible<5->{\alert<5>{~}}}e2\only<4->{\visible<5->{\alert<5>{›}}}
 let add e1 e2   = \only<4->{\visible<5->{\alert<5>{‹}}}\only<4->{\visible<5->{\alert<5>{~}}}e1 + \only<4->{\visible<5->{\alert<5>{~}}}e2\only<4->{\visible<5->{\alert<5>{›}}}
 let if_ e e1 e2 = \only<4->{\visible<5->{\alert<5>{‹}}}if \only<4->{\visible<5->{\alert<5>{~}}}e then \only<4->{\visible<5->{\alert<5>{~(}}}e1 ()\only<4->{\visible<5->{\alert<5>{)}}} else \only<4->{\visible<5->{\alert<5>{~(}}}e2 ()\only<4->{\visible<5->{\alert<5>{)›}}}
end
\only<2>{module POWER7R = \alert<2>{POWER7(R)}}\only<6->{module POWER7C = POWER7(C)}
\only<2>{\>POWER7R.term () 2}\only<6->{\>POWER7C.term ()}
  \only<2>{\rlap{128}}\onslide<6->{\alert{‹fun x -> (fun x -> let rec self = fun x ->
    (fun x -> if x <= 0 then 1 else x * self (x + (-1)))
    x in self) x 7›}}
\end{semiverbatim}
\end{frame}

\section{Type-indexed types}
\subsection{Partial evaluation}
\subsection{CPS transformation}

\begin{frame}[fragile]{\only<11->{Type-indexed types}\only<13->{: }\only<-10,13-15>{Partial evaluation}\only<16->{CPS transformation}}
\setbeamercovered{transparent=20}
\newcommand\remember[1]{\tikz[remember picture,baseline=0pt,text=structure.fg] \node[inner ysep=0pt,inner xsep=2pt,anchor=base] (#1) {#1};}
\newcommand\remark[1]{\textcolor{structure.fg}{\footnotesize(* #1 *)}}
\definecolor{dynamic}{rgb}{.7,0,.7}
\definecolor{static}{rgb}{0,.7,.7}
\newcommand\dynamic[1]{\textcolor{dynamic}{\texttt{‹#1›}}}
\newcommand\static[1]{\textcolor{static}{\texttt{#1}}}
\begin{overlayarea}{0pt}{3ex}
\onslide+<1,13-15>{\rlap{\hspace*{-1pc}\texttt{module P = struct}}}%
\onslide+<12>\vspace*{-1em}%
\begin{semiverbatim}
module type Symantics = sig type ('c,\alert<12>{'s},'a) repr
 val int: int -> ('c,\alert<12>{int},int) repr
 val lam: 'x\,-> ('c, \alert<12>{('c,'s,'a) repr ->
            \,        ('c,'t,'b) repr} as\,'x, 'a\,->\,'b) repr
\uncover<0>{ val fix: (('c, \alert<12>{('c,'s,'a) repr -> ('c,'t,'b) repr},
                'a -> 'b) repr as 'x -> 'x) -> 'x}
 val app: ('c, \alert<12>{('c,'s,'a) repr ->
               ('c,'t,'b) repr} as\,'x, 'a\,->\,'b) repr ->\,'x
\uncover<0>{ val add: 'x -> 'x -> (('c,\alert<12>{int},int) repr as\,'x)
 val if_: ('c,\alert<12>{bool},bool) repr -> (unit->'x) -> (unit->'x)
                              -> (('c,\alert<12>{'s},'a) repr as\,'x)}
end
\end{semiverbatim}
\end{overlayarea}
\begin{tabular}[L]{@{}>{\ttfamily\frenchspacing}l@{\hspace*{-.5em}}r@{ }cc@{}}
\onslide+<1-10,13->
type ('c,\alt<13->{\only<14->{\alert<14-15>{'s},}'a}{\alt<1>{ 'a}{\alert<2>{int}}}) repr
                    & \onslide+<3-10>\remember{source term}
                    & \onslide+<4-10>$3$
                    & \onslide+<5-10>$x$ \\
\onslide+<1-10,13->
\ \ \ = \alt<13->{\alt<16->{\makebox[14pc][l]{\alert{('s\,->\,ans)\,->\,ans}}\remark{CBN CPS evaluator}}{('c,'a) code}}
                 {\alt<1>{???}{('c,int) code}}
                    & \onslide+<3-10>\remember{dynamic part}
                    & \onslide+<4-10>\dynamic{3}
                    & \onslide+<5-10>\dynamic{x} \\
\onslide+<2-10,13->
\ \ \ \alt<16->{= \makebox[14pc][l]{\alert{('c, ('s\,->\,ans)\,->\,ans) code}}\remark{CBN CPS compiler}}{* \alt<13->{\alert{\alt<14->{'s}{('c,'a) static}}}{int} option}
                    & \onslide+<3-10>\remember{static part}
                    & \onslide+<4-10>\static{Some 3}
                    & \onslide+<5-10>\static{None} \\
\onslide+<14-15>...\\
\onslide+<14-15>\rlap{\hspace*{-1pc}end}%
\onslide+<6-10>
type ('c,\alert<6>{int->int}) repr
                    && \onslide+<8-10>\hspace*{-4em}$\fun{x}x$
                    & \onslide+<7-10>$f$ \\
\onslide+<6-10>
\ \ \ = ('c,int->int) code
                    && \onslide+<8-10>\hspace*{-4em}\dynamic{fun \!x->x}
                    & \onslide+<7-10>\dynamic{f} \\
\onslide+<15>\rlap{\hspace*{-1pc}module POWER7P = POWER7(P)}%
\onslide+<6-10>
\ \ \ * \alert<8>{(('c,int) repr ->}
                    && \onslide+<8-10>\hspace*{-4em}\static{Some (fun \!r->r)}
                    & \onslide+<7-10>\static{None} \\
\onslide+<15>\rlap{\hspace*{-1pc}\>POWER7P.term ()}%
\onslide+<6-10>
\ \ \ \ \ \ \alert<8>{('c,int) repr)} option \\
\onslide+<15>\rlap{\hspace*{-1pc}\ \ (‹fun x -> x\,*\,x\,*\,x\,*\,x\,*\,x\,*\,x\,*\,x›, Some <fun>)}%
\\
\onslide+<9-10>
type ('c,'a) \\
\ \ \ = ('c,'a) code \\
\ \ \ * \alert<9-10>{\alt<9>{???}{('c,'a) static}} option
\end{tabular}
\onslide+<10-13,14->
\texttt{\begin{tabular}[L]{@{}l@{ }l@{}}
    type ('c,\:int)        &static = int \\
    type ('c,\:bool)       &static = bool \\
    type ('c,\:'a\,->\,'b) &static = ('c,'a)\:repr -> ('c,'b)\:repr
\end{tabular}}

\onslide+<3-10>
\begin{tikzpicture}[remember picture,overlay,->,rounded corners=1ex,draw=structure.fg,text=structure.fg]
    \draw (source term) --
        +(-2.4,0) |- (dynamic part);
    \draw (source term) --
        node [midway,above,inner sep=0pt] {\footnotesize interpret}
        +(-2.4,0) |- (static part);
\end{tikzpicture}
\end{frame}

\begin{frame}{Conclusion}
    Early obvious safety

    Initial type-checking can be done

    Accommodates interpretations with type-indexed types

    Preserves sharing

    Acknowledge prior-work authors in slides
\end{frame}

\end{document}
