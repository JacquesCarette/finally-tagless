\begin{comment}
\authorinfo{Jacques Carette}
           {McMaster University}
           {carette@mcmaster.ca}
\authorinfo{Oleg Kiselyov}
           {FNMOC}
           {oleg@pobox.com}
\authorinfo{Chung-chieh Shan}
           {Rutgers University}
           {ccshan@cs.rutgers.edu}
\end{comment}

\begin{abstract}
We have built the first family of tagless interpretations for a
higher-order typed object
language in a typed metalanguage (Haskell or ML) that
require no dependent
types, generalized algebraic data types, or
postprocessing to eliminate tags.
The interpretations include
an evaluator, a compiler (or staged evaluator), a partial
evaluator, and call-by-name and
call-by-value CPS transformers.

Our main idea is to encode
% de Bruijn or higher-order abstract syntax
HOAS
using cogen functions rather than data constructors.
In other words, we represent object terms not in an initial algebra
but using the coalgebraic structure of the $\lambda$-calculus.
Our representation also simulates inductive maps from types to
types, which are required for typed partial evaluation and CPS transformations.

Our encoding of an object term abstracts over the various ways to
interpret it yet statically assures that the interpreters never get
stuck.  To achieve self\hyp interpretation and show Jones\hyp
optimality, we relate this exemplar of higher-rank and higher-kind
polymorphism (provided by ML functors and Haskell~98 constructor
classes) to plugging a term into a context of let\hyp polymorphic
bindings.
\end{abstract}

\begin{quote}
\small
    It should also be possible to define languages with a highly
    refined syntactic type structure. Ideally, such a treatment should
    be metacircular, in the sense that the type structure used in the
    defined language should be adequate for the defining language.
    \hfill\rm John Reynolds~\citep{reynolds-definitional}
\end{quote}

%\category{CR-number}{subcategory}{third-level}
%\terms term1, term2
%\keywords keyword1, keyword2

\section{Introduction}\label{intro}

A popular way to define and implement a language is to embed it in
another \citep{reynolds-definitional}.  Embedding means to represent
terms and values of the \emph{object language} as terms and values in the
\emph{metalanguage}.  Embedding is especially appropriate for domain\hyp
specific object languages because it supports rapid prototyping and integration
with the host environment \citep{hudak-building}.
If the metalanguage supports \emph{staging}, then
the embedding can compile object programs to the metalanguage and avoid the
overhead of interpreting them on the fly \citep{WalidICFP02}.  A staged
definitional interpreter is thus a promising way to build a domain\hyp specific
language.

\begin{figure}
    \begin{floatrule}
    \begin{proofrules}
        \[ \[ [x:t_1] \proofoverdots e:t_2 \] \justifies \fun{x}e:t_1\to t_2 \]
        \[ \[ [f:t_1\to t_2] \proofoverdots e:t_1\to t_2 \] \justifies \fix{f}e:t_1\to t_2 \]
        \[ e_1:t_1\to t_2 \quad e_2:t_1 \justifies e_1 e_2: t_2 \]
        \[ \text{$n$ is an integer} \justifies n:\ZZ \]
        \[ \text{$b$ is a boolean} \justifies b:\BB \]
        \[ e:\BB \quad e_1:t \quad e_2:t \justifies \cond{e}{e_1}{e_2}:t \]
        \[ e_1:\ZZ \quad e_2:\ZZ \justifies e_1+e_2:\ZZ \]
        \[ e_1:\ZZ \quad e_2:\ZZ \justifies e_1 \times e_2:\ZZ \]
        \[ e_1:\ZZ \quad e_2:\ZZ \justifies e_1 \le e_2:\BB \]
    \end{proofrules}
    \end{floatrule}
    \caption{Our typed object language}
    \label{fig:object}
\end{figure}

We focus on embedding a \emph{typed} object language into a 
\emph{typed} metalanguage.
The benefit of types in this setting is to rule out meaningless object terms,
thus enabling faster interpretation and assuring that our interpreters
do not get stuck.
To be concrete, we use the typed object language in
Figure~\ref{fig:object} throughout this paper.  We aim not just for
evaluation of object programs but also for
compilation, partial evaluation, and other processing.

\Citet{WalidICFP02} and \citet{xi-guarded} motivated interpreting
a typed object language in a typed metalanguage as an interesting
problem.  The known solutions to this problem store object terms and
values in the metalanguage in a universal type, a generalized algebraic
data type (GADT), or a dependent type.  In the remainder of this section,
we discuss these solutions, identify their drawbacks, then summarize our
proposal and contributions.  
%aplas
\begin{comment}
No matter how we represent the object language in the
metalanguage, the representation can be created either by hand (for example, by
entering object terms at a metalanguage interpreter's prompt) or
by a parser\slash type\hyp checker reading from a text string.
We leave aside the solved problem of writing such a parser\slash type\hyp checker,
whether using dependent types \cite{WalidICFP02} or not \cite{baars-typing}.
\end{comment}
%aplas-else
We leave aside the solved problem of writing a parser\slash type\hyp checker,
for embedding object language objects into the metalanguage
(whether using dependent types \cite{WalidICFP02} or not \cite{baars-typing}),
and just enter them by hand.
%aplas

\subsection{The tag problem}\label{tagproblem}

% see tagless\_interp1.ml, module Tagfull for the complete code.
% If you change the code in here, please adjust the .ml file
% accordingly. Let the paper and the accompanying code be in sync.
It is straightforward to create an algebraic data type, say in OCaml~%
%aplas-specific
Fig.~\ref{fig:tag-problem}(a)%
, to
represent object terms such as those in Figure~\ref{fig:object}.
% fig:tag-problem part (a) was here
For brevity, we elide treating integers, conditionals, and fixpoint in
this section.
We represent each variable using a unary de Bruijn index.
For example, we represent the object term $(\fun{x}x)\True$ as
%aplas-specific
\texttt{let test1 = A (L (V VZ), B true)}.
%aplas-else
\begin{comment}
\begin{code}
let test1 = A (L (V VZ), B true)
\end{code}
\end{comment}

\begin{figure}
%
(a) \begin{code2}
type var = VZ | VS of var
type exp = V of var | B of bool | L of exp | A of exp * exp
\end{code2}

(b) \begin{code2}
let rec lookup (x::env) = function | VZ   -> x
                                   | VS v -> lookup env v
let rec eval0 env = function
| V v       -> lookup env v
| B b       -> b 
| L e       -> fun x -> eval0 (x::env) e
| A (e1,e2) -> (eval0 env e1) (eval0 env e2) 
\end{code2}

(c) \begin{code2}
type u = UB of bool | UA of (u -> u)
\end{code2}

(d) \begin{code2}
let rec eval env = function
| V v       -> lookup env v
| B b       -> UB b
| L e       -> UA (fun x -> eval (x::env) e)
| A (e1,e2) -> match eval env e1 with UA f -> f (eval env e2)
\end{code2}
\caption{OCaml code illustrating the tag problem.}
\label{fig:tag-problem}
\end{figure}
% 
Following \cite{WalidICFP02},
we try to implement an interpreter function |eval0|%
%aplas-specific
, Fig.~\ref{fig:tag-problem}(b)%
. It takes
an object term such as |test1| above and gives us its value.
The first argument to |eval0| is the environment, initially empty,
which is the list of values bound to free variables in the
interpreted code.
% fig:tag-problem part (b) was here
If our OCaml-like metalanguage were untyped, the code above would be 
acceptable.
The |L e| line exhibits interpretive overhead:
|eval0| traverses the function body~|e| every time (the result of
evaluating) |L e| is applied. Staging can be used to remove this
interpretive overhead \citep[\S1.1--2]{WalidICFP02}.

However, the function |eval0| is ill-typed
if we use OCaml or some other typed language as the metalanguage.
The line |B b|
says that |eval0| returns a boolean, whereas the next line |L e| says
the result is a function, but all branches of a pattern-match form must
yield values of the same type. 
A related problem is the type of the environment |env|: a regular
OCaml list cannot hold both boolean and function values. 

The usual solution is to introduce a universal type \citep[\S1.3]
{WalidICFP02} containing both booleans and functions%
%aplas-specific
, Fig.~\ref{fig:tag-problem}(c)%
.
% fig:tag-problem part (c) was here
We can then write a typed interpreter%
%aplas-specific
, Fig.~\ref{fig:tag-problem}(d)
% fig:tag-problem part (d) was here
whose inferred type is |u list -> exp -> u|. Now we can evaluate
%aplas-specific
\texttt{eval [] test1} obtaining |UB true|.
%aplas-else
\begin{comment}
\begin{code}
let test1r = eval [] test1
val test1r : u = UB true 
\end{code}
\end{comment}
% fig:tag-problem part (f) was here
The unfortunate tag |UB| in the result reflects that |eval| is a partial
function.  
%aplas-specific
First, the pattern match |with UA f| in the line |A (e1,e2)| is not 
exhaustive, so |eval| can fail if we apply a boolean,
as in the ill-typed term |A (B true, B false)|.
Second, the |lookup|
function assumes a nonempty environment, so |eval| can fail if we
evaluate an open term |A (L (V (VS VZ)), B true)|.
After all, the type |exp| represents object
terms both well-typed and ill-typed, both open and closed.
%aplas-else
\begin{comment}
First, the pattern match |with UA f| in the line
|A (e1,e2)| is not exhaustive, so |eval| can fail if we apply a boolean,
as in the ill-typed term |A (B true, B false)|.
\begin{code}
let test2 = A (B true, B false)
let test2r = eval [] test2
Exception: Match_failure in eval
\end{code}
Second, the |lookup|
function assumes a nonempty environment, so |eval| can fail if we
evaluate an open term
\begin{code}
let test3 = A (L (V (VS VZ)), B true)
let test3r = eval [] test3
Exception: Match_failure in lookup
\end{code}
After all, the type |exp| represents object
terms both well-typed and ill-typed, both open and closed.
\end{comment}

If we evaluate only closed terms that have been type-checked, then
|eval| would never fail. Alas, this soundness is not obvious to the
metalanguage, whose type system we must still appease with the
nonexhaustive pattern matching in |lookup| and |eval| and the tags |UB|
and |UA| \cite[\S1.4]{WalidICFP02}.  In other words, the algebraic data
types above fail to express in the metalanguage that the object program
is well-typed.  This failure necessitates tagging and nonexhaustive
pattern\hyp matching operations that incur a performance penalty in
interpretation \cite{WalidICFP02} and impair optimality in partial evaluation
\cite{taha-tag}.  In short, the universal\hyp type solution is
unsatisfactory because it does not preserve typing.

%aplas-else
\begin{comment}
\subsection{Solutions using fancier types}
\end{comment}

It is commonly thought that to interpret a typed object language in
a typed metalanguage while preserving types is difficult and requires
GADTs or dependent types \citep{taha-tag}.  In fact, this problem
motivated much work on GADTs \cite{xi-guarded,peyton-jones-simple} and
on dependent types \cite{WalidICFP02,fogarty-concoqtion}.
%aplas-else
\begin{comment}
For a metalanguage's type system to allow the well-typed object term
|test1| but disallow the ill-typed object term |test2|, fancier types
such as GADTs or dependent types seem necessary.  
\end{comment}
Yet other type systems
have been proposed to distinguish closed terms like |test1| from open
terms 
%aplas-comment: like |test3|
%aplas \cite{WalidPOPL03,NanevskiICFP02,NanevskiJFP05,DaviesJACM01,nanevski-contextual},
\cite{WalidPOPL03,NanevskiJFP05,DaviesJACM01},
so that |lookup| never receives an empty environment.  We discuss these
proposals further in \S\ref{related}%
%aplas-specific
.
%aplas-else
\begin{comment}
; here we just note that many
advanced type systems have been devised to ensure statically that an
object term is well-typed and closed.
\end{comment}

\subsection{Our final proposal}\label{ourapproach}

We represent object programs using ordinary functions rather than
data constructors.  The interpreter below provides these functions
\begin{code4}
let varZ env    = fst env       let b (bv:bool) env = bv
let varS vp env = vp (snd env)  let app e1 e2 env   = (e1 env) (e2 env)
let lam e env   = fun x -> e (x,env)
\end{code4}
which comprise the entire interpreter.
We now represent our sample term $(\fun{x}x)\True$ as
%aplas-specific
\texttt{let testf1 = app (lam varZ) (b true)}
%aplas-else
\begin{comment}
\begin{code}
let testf1 = app (lam varZ) (b true)
\end{code}
\end{comment}
This representation is almost the same as in \S\ref{tagproblem}, only
written with lowercase identifiers. To evaluate an object term is to
apply its representation to the empty environment%
%aplas-specific
, |testf1 ()|, obtaining |true|.
%aplas-else
\begin{comment}
\begin{code}
let testf1r = testf1 ()
val testf1r : bool = true
\end{code}
\end{comment}
The result has no tags: the interpreter patently uses no tags and no
pattern matching. The term |b true| evaluates to a boolean and the term
|lam varZ| evaluates to a function, both untagged. The |app| function
applies |lam varZ| without pattern matching. What is more, evaluating an
%aplas-specific
open term such as \texttt{app (lam (varS varZ)) (b true)}
gives a type- rather than a run-time error.
%aplas-else
\begin{comment}
\begin{code}
open term such as |testf3| below gives a type error rather than
a run-time error
let testf3 = app (lam (varS varZ)) (b true)
let testf3r = testf3 ()
\end{code}
\end{comment}
The type error correctly complains
that the initial environment should be a tuple rather than |()|.
In other words, the term is open.

In sum, by Church\hyp encoding terms using ordinary functions, we
achieve a tagless evaluator for a typed object language in a
metalanguage with a simple 
%aplas
Hindley-Milner type system%
%aplas \cite{hindley-principal,milner-theory}
.  In this \emph{final} rather
than \emph{initial} approach, both kinds of run-time errors in
\S\ref{tagproblem} (applying a nonfunction and evaluating an open
term) are reported at compile time. Because the new interpreter
uses no universal type or pattern matching, it never results in a
run-time error (and is in fact total).  Because this safety is obvious
not just to us but also to the metalanguage implementation, we avoid
the serious performance penalty \cite{WalidICFP02} of error checking.
\Citet{Gluck-jones-optimality} explains deeper technical reasons that
inevitably lead to these performance penalties.

Our solution is emphatically distinct from Church-encoding of the
universal type. Church-encoding of the type |u|, \S\ref{tagproblem},
requires two continuations; the interpreter (see |app| above) would
have to provide both to the encoding of |e1|. The continuation
corresponding to the |UB| branch of |u| must either raise an error or
loop. For a well-typed object term that error continuation is never
invoked, yet it must be supplied. Our interpreter has no error
continuation at all.

The evaluator above is wired directly into the
functions |b|, |lam|, |app|, and so on.  %In the rest of this paper, 
We explain how to abstract the interpreter so as
to process the \emph{same} term in many other
ways: compilation, partial evaluation, CPS
conversion, and so forth.

\subsection{Contributions}\label{contributions}

The term ``constructor'' functions |b|, |lam|, |app|, and so on appear
free in the encoding of an object term such as |testf1| above.  Defining
these functions differently gives rise to different interpreters, that
is, different folds on object programs.  Given the same term
representation but varying the interpreter, we can
\begin{itemize*}
    \item evaluate the term to a value in the metalanguage;
    \item measure the size or depth of the term;
    \item compile the term, with staging support such as in MetaOCaml;
    \item even partially evaluate the term, online; and
    \item transform the term to continuation\hyp passing style (CPS),
        including call-by-name (CBN) CPS, so to isolate the evaluation
        order of the object language from that of the metalanguage
%aplas
\footnote{Due to serious lack of space, 
we refer the reader to the accompanying code for this.}%
.
\end{itemize*}
We have programmed our interpreters in OCaml (and, for staging,
\cite{metaocaml}) and standard Haskell. The complete code is available
\url{http://okmij.org/ftp/packages/tagless-final.tar.gz}
to supplement the paper. Our examples below switch between (Meta)OCaml
and Haskell even though we have implemented each example equivalently in
both metalanguages, because some of our claims are more obvious in one
metalanguage than the other.  For example, MetaOCaml provides
convenient, typed staging facilities.

We take the problem of tagless (staged) typed-preserving
interpretation exactly as it was posed in \cite{WalidICFP02} and
\cite{xi-guarded}.  We use their running examples and achieve the
result they call desirable.  Our contributions are as follows.

\begin{enumerate*}
\item We build interpreters that evaluate (\S\ref{language}),
    compile (or evaluate with staging) (\S\ref{compiler}), and partially evaluate (\S\ref{PE}) a typed higher-order object language
   in a typed metalanguage, in direct and continuation\hyp passing styles%
% aplas
%   (\S\ref{variations}).
.
\item All these interpreters use no type tags, patently never get stuck,
    and need no advanced type-system features such as GADTs, dependent types,
    or intentional type analysis.
\item The partial evaluator avoids polymorphic lift and delays binding-time
    ana-lysis.  It bakes a type-to-type map into the interpreter
    interface to eliminate the need for GADTs and thus remain portable
    across Haskell 98 and ML.
\item We use the type system of the metalanguage
    to check statically that an object program is well-typed and closed.
\item We show clean, comparable implementations in MetaOCaml and Haskell.
\item We specify a functor signature that encompasses all our
  interpreters, from evaluation and compilation (\S\ref{language}) 
   to partial evaluation%
% aplas: commented
%  and CPS transformation (\S\ref{PE}).
.
\item We point a clear way to extend the object language with more features
    such as state%
%aplas (\S\ref{state})%
%aplas
\footnote{again, see code}
.
\item We describe an approach to self\hyp interpretation compatible with the
  above%
%aplas: (\S\ref{selfinterp})
.  Self\hyp interpretation turned out
  harder than expected%
%aplas
\footnotemark[\value{footnote}]
.
\end{enumerate*}
Our code is surprisingly simple and obvious in hindsight, but
it has been an open problem to
interpret a typed object language in a typed metalanguage without
tagging or type\hyp system extensions.  For example, \citet{taha-tag}
say that ``expressing such an interpreter in a statically typed
programming language is a rather subtle matter. In fact, it is only
recently that some work on programming type-indexed values in ML
\cite{yang-encoding} has given a hint of how such a function can be
expressed.''  We discuss related work in~\S\ref{related}.

We reiterate that we specifically do not propose any new language
feature or technique. We use the existing, known approaches --
Hindley-Milner type system with inference-preserving module system or
constructor classes extensions, as realized in ML or Haskell98 -- to
solve the problem that was stated in published record as lacking
solution in ML or Haskell98 and most likely unimplementable in these
languages without extensions. The simplicity of our solution and our
use of only mainstream functional language features makes typed
embedded DSLs more practical to build.
\begin{comment}
We may claim some contribution about type-preserving CPS. Check
related work section in \cite{Guillemette-Monier-PLPV}, especially
check the work of Shao on type-preserving CPS in Flint. The PLPV paper
in related work shows other tasks, including closure conversion, which
we may tackle in our approach. We may be able to write a
type-preserving, assured compiler, whose properties and assured by HM.
\end{comment}


\section{The object language and its tagless interpreters}\label{language}

Figure~\ref{fig:object} shows our object language, a simply-typed
$\lambda$-calculus with fixpoint, integers, booleans, and comparison.
The language is close to \citets{xi-guarded}, without their polymorphic
lift but with more constants so as to express Fibonacci, factorial, and
power.
In contrast to \S\ref{intro}, we encode binding using higher-order
abstract syntax (HOAS) \cite{miller-manipulating,pfenning-higher-order}
rather than de Bruijn indices. This makes the encoding convenient and
also ensures that our object programs are closed.

\subsection{How to make encoding flexible: abstract the interpreter}
\label{encoding}
We embed our language in (Meta)OCaml and Haskell.  In Haskell,
the functions that construct object terms are methods in a type class
|Symantics| (with a parameter |repr| of kind |* -> *|)%
%aplas-specific
, Fig.~\ref{fig:symantics-haskell}(a)%
. The class is so named
because its interface gives the syntax of the object language and its
instances give the semantics.
\begin{figure}
%
(a) \begin{code2}
class Symantics repr where
  int  :: Int  -> repr Int;       bool :: Bool -> repr Bool
  lam :: (repr a -> repr b) -> repr (a -> b)
  app :: repr (a -> b) -> repr a -> repr b
  fix :: (repr a -> repr a) -> repr a

  add :: repr Int -> repr Int -> repr Int
  mul :: repr Int -> repr Int -> repr Int
  leq :: repr Int -> repr Int -> repr Bool
  if_ :: repr Bool -> repr a -> repr a -> repr a
\end{code2}

(b) \begin{code2}
testpowfix () = lam (\x -> fix (\self -> lam (\n ->
                 if_ (leq n (int 0)) (int 1)
                     (mul x (app self (add n (int (-1))))))))
\end{code2}

(c) \begin{code2}
testpowfix7 () = lam (\x -> app (app (testpowfix ()) x) (int 7))
\end{code2}
\caption{Symantics in Haskell.}
\label{fig:symantics-haskell}
\end{figure}
% fig:symantics-haskell part (a) was here
For example, we encode the term |test1|, or $(\fun{x}x)\True$, from
\S\ref{tagproblem} above as |app (lam (\x -> x)) (bool True)|,
whose inferred type is \texttt{Symantics repr => repr Bool}.
For another example, the classical $\mathit{power}$ function is
%aplas-specific
in Fig.~\ref{fig:symantics-haskell}(b)
% fig:symantics-haskell part (b) was here
and the partial application $\fun{x} \mathit{power}\;x\;7$ is
in Fig.~\ref{fig:symantics-haskell}(c).
% fig:symantics-haskell part (c) was here
The dummy argument |()| above is to avoid the monomorphism
restriction, to keep the type of |testpowfix| and |testpowfix7|
polymorphic in |repr|. Instead of supplying this dummy
argument, we could have given the terms explicit polymorphic
signatures.  We however prefer for
Haskell to infer the object types for us. We could also
avoid the dummy argument by switching off the monomorphism restriction
with a compiler flag.
The methods |add|, |mul|, and |leq| are quite similar, and so are
|int| and |bool|. Therefore, we will frequently show implementations of
only one method of each group and elide the rest, to save space. The
accompanying code has the complete implementations.

Comparing |Symantics| with Fig.~\ref{fig:object}
shows how to represent \emph{every} typed, closed object term in the
metalanguage. Moreover, the representation preserves types.
\begin{proposition}
If an object term has the object type~$t$, then its
representation in the metalanguage has the type 
|forall repr.| |Symantics repr => repr |$t$.
\end{proposition}
Conversely, the type system of the metalanguage statically checks that the
represented object term is well-typed and closed.
If we err, say replace |int 7| with |bool True| in
|testpowfix7|, Haskell will complain there that the expected type |Int| does not
match the inferred |Bool|.  Similarly, the object term $\fun{x}xx$ and its
encoding |lam (\x -> app x x)| both fail occurs-checks in type checking.
Haskell's type checker also flags syntactically invalid object terms, 
such as if we forget |app| somewhere above.

\begin{figure*}[t]
\begin{floatrule}
\begin{code}
module type Symantics = sig type ('c, 'dv) repr
  val int : int  -> ('c, int) repr
  val bool: bool -> ('c, bool) repr

  val lam : (('c, 'da) repr -> ('c, 'db) repr) -> ('c, 'da -> 'db) repr
  val app : ('c, 'da -> 'db) repr -> ('c, 'da) repr -> ('c, 'db) repr
  val fix : (('c, 'da -> 'db) repr -> ('c, 'da -> 'db) repr) -> ('c, 'da -> 'db) repr

  val add : ('c, int) repr -> ('c, int) repr -> ('c, int) repr
  val mul : ('c, int) repr -> ('c, int) repr -> ('c, int) repr
  val leq : ('c, int) repr -> ('c, int) repr -> ('c, bool) repr
  val if_ : ('c, bool) repr-> (unit-> ('c, 'da) repr)-> (unit-> ('c, 'da) repr)-> ('c, 'da) repr
end

module EX(S: Symantics) = struct open S
  let test1 () = app (lam (fun x -> x)) (bool true)
  let testpowfix () =  lam (fun x -> fix (fun self -> lam (fun n ->
                         if_ (leq n (int 0)) (fun () -> int 1)
                             (fun () -> mul x (app self (add n (int (-1))))))))
  let testpowfix7 =  lam (fun x -> app (app (testpowfix ()) x) (int 7))
end
\end{code}
\end{floatrule}
\caption{A simple (Meta)OCaml embedding of our object language}
\label{fig:ocaml-simple}
\end{figure*}
To embed the same object language in (Meta)OCaml, we replace the type
class |Symantics| and its instances by a module signature |Symantics|
and its implementations.  Figure~\ref{fig:ocaml-simple} shows a simple
signature that suffices until~\S\ref{PE}.  The two differences are:
the additional type parameter |c|, an
environment classifier \cite{WalidPOPL03} (required by MetaOCaml for
code generation in~\S\ref{compiler}); and the $\eta$-expanded type for
|fix| and thunk types in |if_| since OCaml is a call-by-value
language.

The functor |EX|, Fig.~\ref{fig:ocaml-simple} encodes 
our running examples |test1| and $\mathit{power}$.
The dummy argument to |test1| and |testpowfix| is an artifact of
MetaOCaml, related to monomorphism: in order for us to run a
piece of generated code, it must be polymorphic in its environment
classifier (the type variable |'c| in Figure~\ref{fig:ocaml-simple}).
The value restriction dictates that
the definitions of our object terms must look syntactically like
values. Alternatively, we could have used
the rank-2 record types of OCaml to maintain the necessary polymorphism.

Thus we represent an object expression in
OCaml as a functor from |Symantics| to an appropriate semantic domain. This
is essentially the same as the constraint |Symantics repr =>| in the
Haskell embedding.

\subsection{Two tagless interpreters}
\label{S:interpreter-RL}

Having abstracted our term representation over the interpreter, we are
now ready to present a series of interpreters.  Each interpreter is an
instance of the |Symantics| class in Haskell and a module implementing
the |Symantics| signature in MetaOCaml.

The first interpreter evaluates an object term to its value in the
metalanguage.  The module below interprets each object\hyp language
operation as the corresponding metalanguage operation
\vspace*{-0.8em}\begin{code3}
module R = struct type ('c,'dv) repr = 'dv (* no wrappers *)
  let int  (x:int)  = x         let bool (b:bool) = b
  let lam  f        = f         let app  e1 e2    = e1 e2
  let fix  f        = let rec self n = f self n in self
  let add  e1 e2    = e1 + e2   let mul  e1 e2    = e1 * e2
  let leq  x y      = x <= y
  let if_  eb et ee = if eb then et () else ee () end
\end{code3}
\vspace*{-0.8em}
%
As in~\S\ref{ourapproach},
this interpreter is patently tagless, using neither a universal type nor
any pattern matching: the operation |add| is really
OCaml's addition, and |app| is OCaml's application. To run our
examples, we instantiate the |EX| functor from~\S\ref{encoding} with |R|%
%aplas-specific
: \texttt{module EXR = EX(R)}%
.
%% \begin{code}
%% module EXR = EX(R)
%% \end{code}
Thus |EXR.test1 ()| evaluates to the untagged boolean value |true|.
% aplas: commented out
\begin{comment}
In Haskell, we define
\begin{code}
newtype R a = R {unR::a}
instance Symantics R where ...
\end{code}
Although |R| looks like a tag, it is only
a |newtype|.  The types |a| and |R a| are represented differently
only at compile time, not at run time.  Pattern matching against~|R|
cannot ever fail and is assuredly compiled away.
In OCaml, too, it is obvious to the compiler that
\end{comment}
%aplas: the following line is for aplas only
It is obvious to the compiler that
%
pattern matching cannot fail, because there is no
pattern matching. Evaluation can only fail to yield a value
due to interpreting |fix|.
%aplas-specific
(The source code shows a total interpreter |L| that measures
the size of each object term).
%aplas-specific (cont)
We can also generalize from~|R| to all interpreters; the proof
of these propositions follow immediately from the soundness of the
metalanguage's type system.
%end aplas-specific
\begin{proposition}
If an object term~$e$ encoded in the metalanguage has type~$t$,
then evaluating~$e$ in the interpreter~|R| either continues
indefinitely or terminates with a value of the same type~$t$.
\end{proposition}
% Generalizing from~|R| to all interpreters, we have the following
% broader and more useful, if also more obvious, proposition.
\begin{proposition}
  If an implementation of |Symantics| never gets stuck, then
  the type system of the object
  language is sound with respect to the dynamic semantics defined by
  that implementation.
\end{proposition}
% These propositions follow immediately from the soundness of the
% metalanguage's type system.
% aplas-commented
\begin{comment}
For variety, we show another interpreter, which measures the \emph{size}
of each object term, defined as the number of term
constructors. The following is slightly abbreviated code (see the
accompanying source code for the complete definition).
\begin{code}
module L = struct
  type ('c,'dv) repr = int
  let int (x:int)  = 1
  let lam f        = f 0 + 1
  let app e1 e2    = e1 + e2 + 1
  let fix f        = f 0 + 1
  let mul e1 e2    = e1 + e2 + 1
  let if_ eb et ee = eb + et () + ee () + 1
end
\end{code}
Now the expression
\begin{code}
let module E = EX(L) in E.test1 ()
\end{code}
evaluates to |3|. This interpreter is not only tagless but also
total. It ``evaluates'' even seemingly divergent terms like
\begin{code}
app (fix (fun self -> self)) (int 1)
\end{code}

%aplas \begin{comment}
module EX1(S: Symantics) = struct
 open S
 let tfix () = app (fix (fun self -> self)) (int 1)
end;;
let module E =EX1(R) in E.tfix ();;
let module E =EX1(L) in E.tfix ();;
%%aplas \end{comment}
\end{comment}

\section{A tagless compiler (or, a staged interpreter)}\label{compiler}
\vspace{-5pt}
Besides immediate evaluation, we can compile our object language
into OCaml code using MetaOCaml's staging facilities. MetaOCaml
represents future-stage expressions of type~$t$ at the
present stage as values of type |('c,|$t$|) code| where |'c| is the
environment classifier \cite{WalidPOPL03,calcagno-ml-like}. Code values are created
by a \emph{bracket} form |.<|$e$|>.|, which specifies that the expression~$e$ is to be
evaluated at a future stage. The \emph{escape} |.~|$e$ must occur
within a bracket and specifies that the expression~$e$ must be evaluated
at the current stage; its result, which must be a code value, is
spliced into the code being built by the enclosing bracket. The \emph{run} form |.!|$e$ evaluates
the future-stage code value~$e$ by compiling and linking it at run-time.
The bracket, escape, and run are akin to
quasi-quotation, unquotation, and |eval| of Lisp.

Inserting brackets and escapes appropriately into the
evaluator~|R| above yields the simple compiler~|C| below%
%aplas-specific
, Fig.~\ref{fig:interpreter-C}(a)
.
% fig:interpreter-C part (a) was here
%
\begin{figure}[t]
% this code does not have the 'sv parameter. It shows up later.
(a) \begin{code2}
module C = struct type ('c,'dv) repr = ('c,'dv) code
  let int (x:int)   = .<x>.           let bool (b:bool) = .<b>.
  let lam f         = .<fun x -> .~(f .<x>.)>.
  let app e1 e2     = .<.~e1 .~e2>.
  let fix f =  .<let rec self n = .~(f .<self>.) n in self>.
  let add e1 e2     = .<.~e1 + .~e2>. let mul e1 e2     = .<.~e1 * .~e2>.
  let leq x y       = .<.~x <= .~y>.
  let if_ eb et ee  = .<if .~eb then .~(et ()) else .~(ee ())>. end
\end{code2}

(b) \begin{code2}
let module E = EX(C) in E.test1 ()
\end{code2}

(c) \begin{code2}
let module E = EX(C) in E.testpowfix7
\end{code2}

(d) \begin{code2}
.<fun x_1 -> (fun x_2 -> let rec self_3 = fun n_4 ->
   (fun x_5 -> if x_5 <= 0 then 1  else x_2 * (self_3 (x_5 + (-1))))
   n_4 in self_3) x_1 7>.
\end{code2}
\caption{The tagless staged interpreter \texttt{C}.}
\label{fig:interpreter-C}
\end{figure}
%
This is a straightforward staging of
|module R|.
This compiler produces
unoptimized code. For example, interpreting our |test1|%
%aplas-specific
, Fig.~\ref{fig:interpreter-C}(b),
% fig:interpreter-C part (b) was here
gives the code value |.<(fun x_6 -> x_6) true>.|
of inferred type |('c, bool) C.repr|.  Interpreting |testpowfix7|
with
%aplas-specific
Fig.~\ref{fig:interpreter-C}(c),
% fig:interpreter-C part (c) was here
gives a code value with many apparent $\beta$-redexes
%aplas-specific
Fig.~\ref{fig:interpreter-C}(d).
% fig:interpreter-C part (d) was here
This compiler does not incur
any interpretive overhead: the
code produced for $\fun{x}x$ is simply |fun x_6 -> x_6| 
% aplas: and does not
% call the interpreter, unlike the recursive calls to |eval0| and
% |eval| in the |L e| lines in \S\ref{tagproblem}.
The resulting code obviously contains no tags and no pattern matching.
%aplas: the line is for aplas. The original paragraph commented out
The accompanying code shows the Haskell implementation. 
\begin{comment}
We have also implemented this compiler in Haskell. 
Since Haskell
has no (convenient, typed) staging facility, we had to emulate
it. To be precise, we defined a data type |ByteCode| with
constructors such as |Var|, |Lam|, |App|, |Fix|, and |INT|%
\footnote{\texttt{ByteCode} can be mapped to AST of Template Haskell
  (TH). The output of our compiler will then be assuredly type-correct TH.}
Whereas our representation of object terms uses HOAS,
our bytecode uses integer-named
variables to be realistic. 
We then define 
\begin{code}
newtype C t = C (Int -> (ByteCode t, Int)) 
\end{code}
where |Int| is the counter for creating fresh variable
names. We define the compiler by making |C| an instance of the
class |Symantics|.
The implementation\footnote{The implementation uses
GADTs because we also wanted to write a typed interpreter for 
the \texttt{ByteCode} \emph{data type}.} is quite similar (but slightly more
verbose) than the corresponding MetaOCaml code above. The
accompanying code gives the full details.
\end{comment}

\section{A tagless partial evaluator}\label{PE}

Surprisingly, we can write a partial evaluator using the idea above,
namely to build object terms using ordinary functions rather than data
constructors.  We present this partial evaluator in a sequence of four
attempts. It uses no universal type and no tags
for object types.  We then discuss residualization and binding-time
analysis.  Our partial evaluator is a modular extension of the evaluator
in~\S\ref{S:interpreter-RL} and the compiler in~\S\ref{compiler}, in
that it uses the former to reduce static terms and the latter to build
dynamic terms.

\subsection{Avoiding polymorphic lift}
\label{S:PE-lift}

Roughly, a partial evaluator interprets each object term to either
a static (present-stage) term (using~|R|) or
a dynamic (future-stage) term (using~|C|).  To
distinguish between static and dynamic terms, one might define
%aplas-specific
\texttt{data P0 t = S0 (R t) $\mid$ E0 (C t)}.
We then start to define |P0| as an instance of the |Symantics| class.
Integer and boolean literals are immediate, present-stage
values. Addition yields a static term (using~|R|) if and only if both operands
are static; otherwise we extract the dynamic terms from the operands and
add them using~|C|.
Whereas |mul| and |leq| are as easy to define as |add|, we encounter
a problem with |if_|.  If the first argument to |if_| is a dynamic term
(type |C Bool|) and the second and third arguments are a static term
(type |R a|) and a dynamic term (type |C a|), then we need to convert
the static term to dynamic, but there is no polymorphic ``lift''
function, of type |a -> C a|, to send a value to the future stage
\cite{xi-guarded,WalidPOPL03}.
%aplas-else
\begin{comment}
\begin{code}
data P0 t = S0 (R t) | E0 (C t)
\end{code}
To extract a dynamic term from this type, we create the functions
\begin{code}
abstrI :: P0 Int -> C Int
abstrI (S0 r) = int (unR r)
abstrI (E0 c) = c

abstrB :: P0 Bool -> C Bool
abstrB (S0 r) = bool (unR r)
abstrB (E0 c) = c
\end{code}
We then start to define |P0| as an instance of the |Symantics| class.
\begin{code}
instance Symantics P0 where
  int  x = S0 (int x)
  bool x = S0 (bool x)
  add (S0 e1) (S0 e2) = S0 (add e1 e2)
  add e1 e2 = E0 (add (abstrI e1) (abstrI e2))
\end{code}
Integer and boolean literals are immediate, present-stage
values. Addition yields a static term (using~|R|) if and only if both operands
are static; otherwise we extract the dynamic terms from the operands and
add them using~|C|.  Thus the two uses of |add| above refer to
|add| for~|R| and |add| for~|C|.

Whereas |mul| and |leq| are as easy to define as |add|, we encounter
a problem with |if_|.  If the first argument to |if_| is a dynamic term
(type |C Bool|) and the second and third arguments are a static term
(type |R a|) and a dynamic term (type |C a|), then we need to convert
the static term to dynamic, but there is no polymorphic ``lift''
function, of type |a -> C a|, to send a value to the future stage
\cite{xi-guarded,WalidPOPL03}.\footnote{We note in passing that, if we
were to add polymorphic \texttt{lift} to the type class
\texttt{Symantics repr}, then \texttt{repr} would become an instance of
\texttt{Applicative} and thus \texttt{Functor}:\quad\texttt{fmap
f = app (lift f)}}
\end{comment}

Our |Symantics| class only includes separate lifting methods |bool| and
|int|, not a parametrically polymorphic lifting method:
When compiling to a first-order target language such as machine code,
booleans, integers, and functions may well be represented differently.
Thus polymorphic lift cannot be compiled without intensional type
analysis.  To avoid this, we turn to
\citearound{'s technique}\citet[see also \citealp{sumii-hybrid}]{asai-binding-time}:
build a dynamic term
alongside every static term.

\subsection{Delaying binding-time analysis}
\label{S:PE-problem}

We switch to the Haskell data type
%aplas-specific
|data P1 t = P1 (Maybe (R t)) (C t)|
%aplas-else
\begin{comment}
\begin{code}
data P1 t = P1 (Maybe (R t)) (C t)
\end{code}
\end{comment}
so that a partially evaluated term |P1 t| always contains a dynamic
component and sometimes contains a static component.  The two
alternative constructors of a |Maybe| value, |Just| and |Nothing|,
tag each term with a phase: present or
future.  This tag is not an object type tag: all pattern matching below
is exhaustive.  Because the future-stage component is always present, we
can now define the polymorphic function
%aplas-specific
|abstr1 (P1 _ dyn) = dyn| of type
|abstr1 :: P1 t -> C t|
%aplas-else
\begin{comment}
\begin{code}
abstr1 :: P1 t -> C t
abstr1 (P1 _ dyn) = dyn
\end{code}
\end{comment}
to extract it without requiring polymorphic lift into~|C|.  We then try
to define the interpreter |P1|---and get as far as the first-order
constructs of our object language, including |if_|.
\vspace*{-0.7em}
\begin{code3}
instance Symantics P1 where
  int  x = P1 (Just (int x)) (int x)
  add (P1 (Just n1) _) (P1 (Just n2) _) = int (unR (add n1 n2))
  add e1 e2 = P1 Nothing (add (abstr1 e1) (abstr1 e2))
  if_ (P1 (Just s) _) et ef = if unR s then et else ef
  if_ eb et ef = P1 Nothing (if_ (abstr1 eb) (abstr1 et) (abstr1 ef))
\end{code3}
\vspace*{-0.7em}
When we come to functions, however, we stumble.  According to our
definition of~|P1|, a partially evaluated object function, such as the
identity $\fun{x}x$ embedded in Haskell as |lam (\x -> x)|\texttt{ ::
}|P1 (a -> a)|, consists of a dynamic part (type |C (a -> a)|) and
maybe a static part (type |R (a -> a)|).  The dynamic part is useful
when this function is passed to another function that is only
dynamically known, as in $\fun{k}k(\fun{x}x)$.  The static part is
useful when this function is applied to a static argument, as in
$(\fun{x}x)\True$.  Neither part, however, lets us \emph{partially}
evaluate the function, that is, compute as much as possible statically
when it is applied to a mix of static and dynamic inputs.  For example,
the partial evaluator should turn $\fun{n}(\fun{x}x)n$ into $\fun{n}n$
by substituting $n$ for~$x$ in the body of $\fun{x}x$ even though $n$ is
not statically known.  Even the same static function, applied to
different static arguments, can give both static and dynamic results: we
want to simplify $(\fun{y}x\times y)0$ to~$0$ but $(\fun{y}x\times y)1$
to~$x$.

To enable these simplifications, we delay binding-time analysis (BTA)
for a static function until it is applied, that is, until |lam f|
appears as the argument of |app|.  To do so, we have to incorporate |f|
as it is into the |P1| data structure: applying the type constructor
|P1| to a function type |a -> b| should yield one of
\vspace*{-0.7em}
\begin{code3}
data P1 (a -> b) = S1 (P1 a -> P1 b) | E1 (C (a -> b))
data P1 (a -> b) = P1 (Maybe (P1 a -> P1 b)) (C (a -> b))
\end{code3}
\vspace*{-0.7em}
That is, we need a nonparametric data type, something akin to
type-indexed functions and type-indexed types, which
\citet{oliveira-typecase} dub the \emph{typecase} design pattern.
%aplas-simplification
Thus typed partial evaluation, like typed CPS transformation,
inductively defines a map from source types to target types that
performs case distinction on the source type.
\begin{comment}

Thus typed partial evaluation, like typed CPS transformation,
inductively defines a map from source types to target types that
performs case distinction on the source type.  The connection with CPS
is not an accident, as we shall see in~\S\ref{S:CPS}.
\end{comment}

\subsection{Eliminating tags from typecase}
\label{S:PE-GADT}

Two common ways to provide typecase in Haskell are
GADTs and type-class functional dependencies
\cite{oliveira-typecase}.  These
methods are equivalent, and here we use GADTs; |incope1.hs|
in the accompanying source code shows the latter.
%aplas: the original text is in comments below. Here's a greatly
% abbreaviated text
\vspace*{-0.7em}
\begin{code3}
data P t where
  VI :: Int  -> P Int;                VB :: Bool -> P Bool
  VF :: (P a -> P b) -> P (a -> b);   E  :: C t -> P t
\end{code3}
\vspace*{-0.7em}
The constructors |VI|, |VB|, and |VF| build static terms (like |S0|
in~\S\ref{S:PE-lift}), and |E| builds dynamic terms (like |E0|).  However,
the type |P t| is no longer parametric in~|t|: the constructor |VF| takes an
operand of type |P a -> P b| rather than |a -> b|. As before, we define a
function |abstr| to extract a future-stage computation from a value of 
type |P t|. The implementations of |int|, |bool|, and |add| are like
in~\S\ref{S:PE-problem}.
The interpretation of |lam f = VF f| just wraps the HOAS function |f|. 
We can always compile |f| to a code value,
but we delay it to apply |f| to concrete arguments. The interpretation of
\vspace*{-0.7em}
\begin{code3}
app (VF f) ea = f ea;  app (E f)  ea = E (app f (abstr ea))
\end{code3}
\vspace*{-0.7em}
checks to see if |ef| is such a delayed
HOAS function |VF f|. If it is, we apply |f| to the
concrete argument |ea|, giving us a chance to perform static
computations (see example |testpowfix7| in~\S\ref{S:PE-solution}). If |ef| is a
dynamic value |E f|, we residualize.
This solution using GADTs works but is not quite satisfactory. First, it
cannot be ported to MetaOCaml.  Second,
the problem of nonexhaustive pattern\hyp matching reappears in
|app|. Although Coq or logical
frameworks with canonical forms could prove pattern-matching
exhaustive, GHC cannot and issues warnings. Recall, our goal is to
make this exhaustiveness \emph{syntactically} apparent.

\begin{comment}
We introduce a GADT with four data constructors.
\vspace*{-0.7em}
data P t where
  VI :: Int  -> P Int
  VB :: Bool -> P Bool
  VF :: (P a -> P b) -> P (a -> b)
  E  :: C t -> P t
\end{code3}
The constructors |VI|, |VB|, and |VF| build static terms (like |S0|
in~\S\ref{S:PE-lift}), and |E| builds dynamic terms (like |E0|).  However,
the type |P t| is no longer parametric in~|t|: the constructor |VF| takes an
operand of type |P a -> P b| rather than |a -> b|. As before, we define a
function to extract a future-stage computation from a value of type |P t|.
\begin{code}
abstr :: P t -> C t
abstr (VI i) = int i
abstr (VB b) = bool b
abstr (VF f) = lam (abstr . f . E)
abstr (E x)  = x
\end{code}
The cases of this function |abstr| are type-indexed.  In particular, the |VF f|
case uses the method |lam| of the |C| interpreter to compile~|f|.

We may now make |P| an instance of
|Symantics| and implement the partial evaluator as follows. We elide
|mul|, |leq|, |if_|, and |fix|.
\begin{code}
instance Symantics P where
  int x  = VI x
  bool b = VB b
  add (VI n1) (VI n2) = VI (n1 + n2)
  add e1 e2 = E (add (abstr e1) (abstr e2))
  lam = VF
  app (VF f) ea = f ea
  app (E f)  ea = E (app f (abstr ea))
\end{code}
The implementations of |int|, |bool|, and |add| are like
in~\S\ref{S:PE-problem}.
The interpretation of |lam f| just wraps the
HOAS function |f|. We can always compile |f| to a code value,
but we delay it to apply |f| to concrete arguments. The interpretation of
|app ef ea| checks to see if |ef| is such a delayed
HOAS function |VF f|. If it is, we apply |f| to the
concrete argument |ea|, giving us a chance to perform static
computations (see example |testpowfix7| in~\S\ref{S:PE-solution}). If |ef| is a
dynamic value |E f|, we residualize.

This solution using GADTs works but is not quite satisfactory. First, it
cannot be ported to MetaOCaml as GADTs are unavailable there.  Second,
the problem of nonexhaustive pattern\hyp matching reappears in |app|
above: the type |P t| has four constructors, of which pattern\hyp
matching in |app| uses only |VF| and~|E|. One may say that the
constructors |VI| and |VB| obviously cannot occur because they do not
construct values of type |P (a -> b)| as required by the type of |app|.
Indeed the metalanguage implementation could reason thus, 
but it may not; GHC for one issues warnings%
\footnote{If we use inductive families (as those in Coq) or logical
  frameworks with canonical forms, we can prove the pattern-matching
  to be exhaustive. The coverage checker of Twelf does such proof.}. 
Although this point may seem minor, it is the heart of
the tagging problem and the purpose of tag elimination. A typed tagged
interpreter contains many pattern\hyp matching forms that look partial
but never fail in reality. The goal is to make this safety
\emph{syntactically} apparent.
\end{comment}



\subsection{The ``final'' solution}
\label{S:PE-solution}
Let us re-examine the problem in~\S\ref{S:PE-problem}. What we
would ideally like is to write
%aplas: inlining
\texttt{data P t = P (Maybe (repr\_pe t)) (C t)}
\begin{comment}
\begin{code}
data P t = P (Maybe (repr_pe t)) (C t)
\end{code}
\end{comment}
where |repr_pe| is the type function defined
% inductively because P below depends on repr_pe
by\vspace*{-0.8em}
\begin{code3}
repr_pe Int      = Int          repr_pe Bool     = Bool
repr_pe (a -> b) = P a -> P b
\end{code3}
\vspace*{-0.8em}
Although we can use type classes to define this type function
in Haskell, that is not portable to MetaOCaml. However,
these three typecase alternatives are already present in existing
methods of |Symantics|.
A simple and portable solution thus emerges: we bake |repr_pe| 
into the signature |Symantics|. Switching to MetaOCaml to demonstrate this,
we recall from Figure~\ref{fig:ocaml-simple} in~\S\ref{encoding} that the |repr| type
constructor took two arguments |'c| and~|'dv|. We add an argument
|'sv| for the result of applying |repr_pe| to~|'dv|.
Figure~\ref{fig:ocaml} shows the new signature.
% aplas: abbreviated figure
\begin{figure*}
\begin{floatrule}
\begin{code}
module type Symantics = sig type ('c,'sv,'dv) repr
  val int : int  -> ('c,int,int) repr
  val lam : (('c,'sa,'da) repr -> ('c,'sb,'db) repr as 'x) -> ('c,'x,'da->'db) repr
  val app : ('c,'x,'da->'db) repr -> (('c,'sa,'da) repr -> ('c,'sb,'db) repr as 'x)
  val fix : ('x->'x) -> (('c, ('c,'sa,'da) repr -> ('c,'sb,'db) repr, 'da->'db) repr as 'x)
  val add : ('c,int,int) repr -> ('c,int,int) repr -> ('c,int,int) repr
  val if_ : ('c,bool,bool) repr -> (unit->'x) -> (unit->'x) -> (('c,'sa,'da) repr as 'x)
end
\end{code}
\end{floatrule}
\caption{A (Meta)OCaml embedding of our object language that supports
  partial evaluation (\texttt{bool}, \texttt{mul},
\texttt{leq} are elided)}
\label{fig:ocaml}
\end{figure*}

\begin{comment}
\begin{figure*}
\begin{floatrule}
\begin{code}
module type Symantics = sig
  type ('c,'sv,'dv) repr

  val int : int  -> ('c,int,int) repr
  val bool: bool -> ('c,bool,bool) repr

  val lam : (('c,'sa,'da) repr -> ('c,'sb,'db) repr as 'x) -> ('c,'x,'da->'db) repr
  val app : ('c,'x,'da->'db) repr -> (('c,'sa,'da) repr -> ('c,'sb,'db) repr as 'x)
  val fix : ('x -> 'x) -> (('c, ('c,'sa,'da) repr -> ('c,'sb,'db) repr, 'da->'db) repr as 'x)

  val add : ('c,int,int) repr -> ('c,int,int) repr -> ('c,int,int) repr
  val mul : ('c,int,int) repr -> ('c,int,int) repr -> ('c,int,int) repr
  val leq : ('c,int,int) repr -> ('c,int,int) repr -> ('c,bool,bool) repr
  val if_ : ('c,bool,bool) repr -> (unit -> 'x) -> (unit -> 'x) -> (('c,'sa,'da) repr as 'x)
end
\end{code}
\end{floatrule}
\caption{A (Meta)OCaml embedding of our object language that supports partial evaluation}
\label{fig:ocaml}
\end{figure*}
\end{comment}

% aplas: abberviated figure here. The original is in teh comments
% right down below
\begin{figure}
\begin{floatrule}
\begin{code}
module P = struct
  type ('c,'sv,'dv) repr = {st: 'sv option; dy: ('c,'dv) code}
  let abstr {dy = x} = x    let pdyn x = {st = None; dy = x}
  let int  (x:int)  = {st = Some (R.int x);  dy = C.int x}

  let add e1 e2 = match e1, e2 with
  | {st = Some 0}, e | e, {st = Some 0} -> e
  | {st = Some m}, {st = Some n} -> int (R.add m n)
  | _ -> pdyn (C.add (abstr e1) (abstr e2))

  let if_ eb et ee = match eb with
  | {st = Some b} -> if b then et () else ee ()
  | _ -> pdyn (C.if_ (abstr eb) (fun () -> abstr (et ()))
                                (fun () -> abstr (ee ())))
  let lam f = {st = Some f; dy = C.lam (fun x -> abstr (f (pdyn x)))}
  let app ef ea = match ef with {st = Some f} -> f ea
                  | _ -> pdyn (C.app (abstr ef) (abstr ea)) end
\end{code}
\end{floatrule}
\caption{Our partial evaluator (\texttt{bool}, \texttt{mul},
  \texttt{leq} and \texttt{fix} are elided)}
\label{fig:pe}
\end{figure}

\begin{comment}
\begin{figure}
\begin{floatrule}
\begin{code}
module P = struct
  type ('c,'sv,'dv) repr = {st: 'sv option; dy: ('c,'dv) code}
  let abstr {dy = x} = x
  let pdyn x = {st = None; dy = x}

  let int  (x:int)  = {st = Some (R.int x);  dy = C.int x}
  let bool (x:bool) = {st = Some (R.bool x); dy = C.bool x}

  let add e1 e2 = match e1, e2 with
  | {st = Some 0}, e | e, {st = Some 0} -> e
  | {st = Some m}, {st = Some n} -> int (R.add m n)
  | _ -> pdyn (C.add (abstr e1) (abstr e2))

  let if_ eb et ee = match eb with
  | {st = Some b} -> if b then et () else ee ()
  | _ -> pdyn (C.if_ (abstr eb) (fun () -> abstr (et ()))
                                (fun () -> abstr (ee ())))
  let lam f = {st = Some f; 
               dy = C.lam (fun x -> abstr (f (pdyn x)))}

  let app ef ea = match ef with
  | {st = Some f} -> f ea
  | _ -> pdyn (C.app (abstr ef) (abstr ea))

  let fix f = 
    let fdyn = C.fix (fun x -> abstr (f (pdyn x)))
    in let rec self = function
       | {st = Some _} as e -> app (f (lam self)) e
       | e -> pdyn (C.app fdyn (abstr e))
       in {st = Some self; dy = fdyn}
end
\end{code}
\end{floatrule}
\caption{Our partial evaluator (\texttt{mul} and \texttt{leq} are elided)}
\label{fig:pe}
\end{figure}
\end{comment}

The interpreters |R|, |L| and~|C| above only use the old
type arguments |'c| and~|'dv|, which are treated by the new signature
in the same way.  Hence all that needs to change in these interpreters
to match the new signature is to add a phantom type
argument~|'sv| to~|repr|.
For example, the compiler |C| now begins \texttt{module C = struct
  type ('c,'sv,'dv) repr = ('c,'dv) code} with the rest the same.
In contrast, the partial evaluator~|P| relies on the type argument |'sv|.

%aplas: here's the original text
\begin{comment}
For example, the compiler |C| now begins
\begin{code}
module C = struct
  type ('c,'sv,'dv) repr = ('c,'dv) code
  (* the rest is the same *)
\end{code}
In contrast, the partial evaluator~|P| relies on the type argument |'sv|.
\end{comment}

Figure~\ref{fig:pe} shows the partial evaluator~|P|.
Its type |repr| literally expresses the type equation for |repr_pe| above.
The function |abstr|, as in~\S\ref{S:PE-GADT},
extracts a future-stage code value from a result of
partial evaluation.  Conversely, the function |pdyn| injects a
code value into partial evaluation. As
in~\S\ref{S:PE-problem}, we build dynamic terms alongside
any static ones to avoid polymorphic lift.

To illustrate how to add optimizations, we improve |add| (and |mul|,
elided) to simplify the generated code using the monoid (and ring)
structure of~|int|: not only is addition performed statically
(using~|R|) when both operands are statically known, but it is
eliminated when one operand is statically~$0$; similarly for
multiplication by~$0$ or~$1$.  
\begin{comment}
Such algebraic simplifications are easy
to abstract over the specific domain (such as monoid or ring) where they
apply.  These simplifications and abstractions help a lot in a large
language with more base types and primitive operations.
\end{comment}
%aplas-specific
In large languages, this can be quite effective.

%aplas: abbreviated paragraph. The original is commented below
Any partial evaluator must decide how much to unfold recursion. The code in
Figure~\ref{fig:pe} na\"\i vely unfolds |fix| whenever the
argument is static.  In the accompanying source code is a conservative
alternative |P.fix| that unrolls recursion only once, then residualizes.
Many sophisticated approaches have been developed for this trade-off
\cite{jones-partial}, but this issue is orthogonal to our presentation.
\begin{comment}
Any partial evaluator must decide how much to unfold recursion
statically: unfolding too little can degrade the residual code, whereas
unfolding too much risks nontermination.  Our partial evaluator is no
exception, because our object language includes |fix|.  The code in
Figure~\ref{fig:pe} takes the na\"\i ve approach of ``going all the
way'', that is, unfold |fix| rather than residualize whenever the
argument is static.  In the accompanying source code is a conservative
alternative |P.fix| that unrolls recursion only once, then residualizes.
Many sophisticated approaches have been developed to make this trade-off
\cite{jones-partial}, but this issue is orthogonal to our presentation.
A separate concern in our treatment of |fix| is possible code bloat in
the residual program, which calls for let-insertion
\cite{SwadiTahaKiselyovPasalic2006}.
\end{comment}


%aplas: abbreviated paragraph. The original is in the comments below
Given this implementation of~|P|, our running example
\texttt{let module E = EX(P) in E.test1 ()} evaluates to
\texttt{\{P.st = Some true; P.dy = .<true>.\}}
of type |('a, bool, bool) P.repr|.  Unlike with~|C| in~\S\ref{compiler},
a $\beta$-reduction has been statically performed to yield |true|.  More
interestingly, whereas |testpowfix7| compiles to a code value with many
$\beta$-redexes in~\S\ref{compiler}, the partial evaluation
\texttt{let module E = EX(P) in E.testpowfix7}
gives the desired result
\vspace*{-0.7em}
\begin{code3}
{P.st = Some <fun>;
 P.dy = .<fun x_1 -> x_1 * (x_1 * (x_1 * (x_1 * (x_1 * (x_1 * x_1)))))>.}
\end{code3}
\vspace*{-0.7em}

\begin{comment}
Given this implementation of~|P|, our running example
\begin{code}
let module E = EX(P) in E.test1 ()
\end{code}
evaluates to
\begin{code}
{P.st = Some true; P.dy = .<true>.}
\end{code}
of type |('a, bool, bool) P.repr|.  Unlike with~|C| in~\S\ref{compiler},
a $\beta$-reduction has been statically performed to yield |true|.  More
interestingly, whereas |testpowfix7| compiles to a code value with many
$\beta$-redexes in~\S\ref{compiler}, the partial evaluation
\begin{code}
let module E = EX(P) in E.testpowfix7
\end{code}
gives the desired result
\begin{code}
{P.st = Some <fun>;
 P.dy = .<fun x_1 -> x_1 * (x_1 * (x_1 * (x_1 *
                    (x_1 * (x_1 * x_1)))))>.}
\end{code}
\end{comment}


Unlike the GADT approach in~\S\ref{S:PE-GADT}, all pattern\hyp matching
in~|P| is \emph{syntactically} exhaustive, so it is patent to the metalanguage
implementation that |P| never gets stuck.  Further, all pattern\hyp matching occurs
during partial evaluation, only to check if a value is known statically,
never what type it has.  In other words, our partial evaluator tags
phases (with |Some| and |None|) but not object types.
%aplas: commented out
\begin{comment}
Our typed partial evaluator is online and polyvariant.  It reuses the
compiler~|C| and the evaluator~|R| by composing them.  This situation is
simpler than \citets{SperberThiemann:TwoForOne} composition of a partial
evaluator and a compiler, but the general ideas are similar.
\end{comment}


% don't even have space for CPS
\begin{comment}
\section{Continuation\hyp passing style}\label{variations}
% aplas-specific
\label{state}
\label{S:CPS}

Our approach accommodates
several variants, including
a call-by-name CPS interpreter and a call-by-value CPS
transformation.
% Here's the text to replace the comment block below.
This lets us decouple the evaluation strategy of the object language
from that of the metalanguage. The accompanying code shows the CBN CPS
interpreter (module |RCN| implementing |Symantics|) and a CBV CPS
transformer |CPST|. The latter explicitly maps CPS interpretations to
(direct) interpretations performed by the base interpreter~|S|. All
these interpreters are typed, tagless and \emph{type-preserving} (as
well as fully polymorphic in the answer-type). The type preservation
is the consequence of the type soundness of the metalanguage.  We can
modify the CBV CPS transformation to pass a piece of state along with
the continuation. This technique lets us support mutable state.  Due
to the severe lack of space we cannot describe these interpreters and
refer the reader to the accompanying code.
\end{comment}

\begin{comment}
\subsection{Call-by-name CPS interpreters}\label{S:CPS}

The object language generally inherits the evaluation strategy from
the metalanguage---call-by-value (CBV) in OCaml, call-by-name (CBN) in
Haskell.  To represent a CBN object language in a CBV metalanguage,
\cite{reynolds-definitional,reynolds-relation} and \cite{PlotkinCBN}
introduce CPS to make the evaluation strategy of a definitional
interpreter indifferent to that of the metalanguage. To achieve the same
indifference in the typed setting, we build a CBN CPS interpreter for
our object language in OCaml.

The interpretation of an object term is a function
mapping a continuation~|k| to the answer
returned by~|k|.
\begin{code}
let int (x:int) = fun k -> k x
let add e1 e2 = fun k ->
  e1 (fun v1 -> e2 (fun v2 -> k (v1 + v2)))
\end{code}
In both |int| and |add|, the interpretation has type 
|(int -> 'w)|\texttt{ }|-> 'w|, where |'w| is the (polymorphic) answer type.

Unlike CBV CPS, the CBN CPS interprets
abstraction and application as follows:
\begin{code}
let lam f = fun k -> k f
let app e1 e2 = fun k -> e1 (fun f -> f e2 k)
\end{code}
Characteristic of CBN, |app e1 e2|
does not evaluate the argument~|e2| by applying it to the
continuation~|k|. Rather, it passes |e2| unevaluated to the abstraction.
Interpreting $\fun{x} x+1$ yields type
\begin{code}
((((int -> 'w1) -> 'w1) -> (int -> 'w1) -> 'w1)
 -> 'w2) -> 'w2
\end{code}

We would like to collect those interpretation functions into a module
with signature |Symantics|, to include the CBN CPS interpreter within our
general framework. Alas, as in~\S\ref{S:PE-problem}, the type of
an object term inductively determines the type of its interpretation:
the interpretation of an object term of type~$t$ may not have type
|(|$t$|->'w)->'w|, because $t$ may be a function type.  Again we
simulate a type function with a typecase distinction, using an extra
type argument to |repr|. Happily, the type function |repr_pe| needed for
the partial evaluator 
in~\S\ref{S:PE-solution} is precisely the same type function we
need for CBN CPS\@.
\begin{code}
module RCN = struct
  type ('c,'sv,'dv) repr = 
    {ko: 'w. ('sv -> 'w) -> 'w}
  let int (x:int) = {ko = fun k -> k x}
  let add e1 e2 = 
    {ko = fun k -> e1.ko (fun v1 -> 
                   e2.ko (fun v2 -> k (v1+v2)))}
  let if_ eb et ee = 
    {ko = fun k -> eb.ko 
         (fun vb -> if vb then (et ()).ko k 
                          else (ee ()).ko k)}
  let lam f = {ko = fun k -> k f}
  let app e1 e2 = 
    {ko = fun k -> e1.ko (fun f -> (f e2).ko k)}
  let fix f = 
    let rec fx f n = app (f (lam (fx f))) n in 
    lam (fx f)
  let run x = x.ko (fun v -> v)
end
\end{code}

This interpreter~|RCN| is fully polymorphic over the answer type,
using higher-rank polymorphism through OCaml record types.
It could also be a functor parameterized over
the answer type.

Because |RCN| has the signature |Symantics|, we can instantiate our previous
examples with it, and all works as expected.  More interesting
is the example $(\fun{x}1)\bigl((\fix{f}f)\mathinner2\bigr)$, which terminates
under CBN but not CBV\@.
\begin{code}
module EXS(S: Symantics) = struct open S
 let diverg () = app (lam (fun x -> int 1)) 
                  (app (fix (fun f->f)) (int 2))
end
\end{code}
Interpreting |EXS| with the |R| interpreter of
\S\ref{S:interpreter-RL} does not terminate.
\begin{code}
let module M = EXS(R) in M.diverg ()
\end{code}
In contrast, the CBN interpreter gives the result~|1|.
\begin{code}
let module M = EXS(RCN) in RCN.run (M.diverg ())
\end{code}

\subsection{CBV CPS transformers}

Changing one definition turns our CBN CPS interpreter into CBV\@.
\begin{code}
module RCV = struct include RCN
  let lam f = {ko = fun k -> k (fun e ->
    e.ko (fun v -> f {ko = fun k -> k v}))}
end
\end{code}
Now an applied abstraction
evaluates its argument before proceeding. This approach is in
line with \cite{reynolds-relation}, albeit typed. The
interpreter~|RCV| is useful for CBV evaluation of the object language
whether the metalanguage is CBV or CBN\@.

We turn to a more general approach to CBV CPS: a CPS transformer that
turns any implementation of |Symantics| into a CPS interpreter, whether
it is an evaluator.  This functor on interpreters performs a textbook
CPS transformation on the object language.
\begin{code}
module CPST(S: Symantics) = struct
  let int i = S.lam (fun k -> S.app k (S.int i))
  let add e1 e2 = S.lam (fun k ->
    S.app e1 (S.lam (fun v1 ->
    S.app e2 (S.lam (fun v2 ->
    S.app k (S.add v1 v2))))))
  let lam f = S.lam (fun k -> S.app k (S.lam
    (fun x -> f (S.lam (fun k -> S.app k x)))))
  let app e1 e2 = S.lam (fun k -> 
    S.app e1 (S.lam (fun f ->
    S.app e2 (S.lam (fun v ->
    S.app (S.app f v) k)))))
  let fix f = S.fix (fun self -> (f self))
end
\end{code}
This (abbreviated) code explicitly maps CPS interpretations to
(direct) interpretations performed by 
the base interpreter~|S|.

The module returned by |CPST| does not define |repr|
and thus does not have signature |Symantics|.
The reason is again the type of |lam f|. Whereas
|int| and |add| return the (abbreviated) type
|('c, ..., (int -> 'w) -> 'w) S.repr|,
the type of |lam (add (int 1))| is
\begin{code}
('c, ..., ((int -> (int -> 'w1) -> 'w1) -> 'w2)
          -> 'w2) S.repr
\end{code}
Hence, to write the type equation defining |CPST.repr| we again need
a type function with a typecase distinction, similar to |repr_pe|
in~\S\ref{S:PE-solution}. Alas, the type function we need is not
identical to |repr_pe|, so we need to add another type argument to
|repr| in the |Symantics| signature. As in~\S\ref{S:PE-solution}, the
terms in previous implementations of |Symantics| stay unchanged, but the
|repr| type equations in those implementations have to take a new
(phantom) type argument.

To save space, we just
use the module returned by |CPST| as is. Because it does not
match the signature |Symantics|, we cannot apply the |EX| functor to it.
Nevertheless, we can write the tests.
\begin{code}
module T = struct
 module M = CPST(C)
 open M
 let test1 () = (* same as before *)
   app (lam (fun x -> x)) (bool true)
 let testpowfix () = ... (* same as before *)
 let testpowfix7 = (* same as before *)
    lam (fun x -> app (app (testpowfix ()) x)
                      (int 7))
end
\end{code}
We instantiate |CPST| with the desired base interpreter~|C|,
then use the result |M| to
interpret object terms. Those terms are \emph{exactly} as before.
Having to textually copy the terms is the
price we pay for this simplified treatment.
Our discussion of self\hyp interpretation in~\S\ref{selfinterp} shows
that this copying is not frivolous but represents plugging a term into
a context, which is one of the many faces of polymorphism.

With 
|CPST| instantiated by the compiler~|C| above,
|T.test1| gives
\begin{code}
.<fun x_5 ->
  (fun x_2 -> x_2 (fun x_3 x_4 -> x_4 x_3))
  (fun x_6 -> (fun x_1 -> x_1 true)
              (fun x_7 -> x_6 x_7 x_5))>.
\end{code}
This output is a na\"{\i}ve CPS transformation of $(\fun{x}x)\True$,
containing several apparent $\beta$-redexes.  To eliminate these
redexes, we just change~|T| to instantiate |CPST| with |P| instead.
\begin{code}
{P.st = Some <fun>;
 P.dy = .<fun x_5 -> x_5 true>.}
\end{code}

\subsection{State and imperative features}
\label{state}

We can modify the CBV CPS transformation to pass a piece of state along
with the continuation. This technique lets us support mutable state. We
can also add mutable references to the object language using mutable
references of the metalanguage.  The accompanying code illustrates these
extensions.
\end{comment}


\begin{comment}
\section{Self-interpretation}\label{selfinterp}

We turn to interpreting the object language in the object language, to
clarify how expressive our typed object language can be and to argue that our
partial evaluator is Jones\hyp optimal.

Given an \emph{encoding} of each object term~$e$ as an \emph{object} term~$\Encode{e}$,
a \emph{self\hyp interpreter} is usually defined as an object
function~$\si$ such that any object term~$e$ is observationally
equivalent to the object application $\si\Encode{e}$
\cite{jones-partial,taha-tag,Danvy-tagging-encoding}.
A partial evaluator~$\pe$ maps object terms~$e$ to observationally
equivalent object terms~$\pe(e)$.  It is said to be
\emph{optimal} with respect to~|si| if $\pe(\si\Encode{e})$
is equal to~$e$ (up to $\alpha$\hyp conversion, or in some accounts, no
less efficient than~$e$).

Intuitively, self\hyp interpretation is straightforward in our
framework: the functions comprising the interpreters
in~\S\ref{S:interpreter-RL} may just as well be written in our object
language.  In particular, the following \emph{object} functions implement an
evaluator.  (We use the number~$0$ in lieu of a unit value.)
\begin{align*}
    \ident{int} &= \fun{x} x &
    \ident{lam} &= \fun{f} f \\
    \ident{add} &= \fun{x} \fun{y} x+y &
    \ident{app} &= \fun{f} \fun{x} fx \\
    \ident{if\_}&= \fun{b} \fun{t} \fun{e} \cond{b}{t\,0}{e\,0} &
    \ident{fix} &= \fun{g} \fix{f} \fun{x} gfx
\end{align*}
We thus map each object terms~$e$ to an object term~$\encode{e}$ as follows.
We call this mapping \emph{pre-encoding}.
\begin{align*}
    \encode{x} &= x &
    \encode{\fun{x}e} &= \ident{lam} (\fun{x} \encode{e}) \\
    \encode{n} &= \ident{int}\, n &
    \encode{fx} &= \ident{app} \encode{f} \encode{x} \\
    \encode{e_1 + e_2} &= \ident{add} \encode{e_1} \encode{e_2} &
    \encode{\fix{f}e} &= \ident{fix} (\fun{f} \encode{e}) \\
    \encode{\cond{b}{t}{e}} &= \rlap{$\ident{if\_} \encode{b}
        \left(\fun{\_}\encode{t}\right) \left(\fun{\_}\encode{e}\right)$}
\end{align*}
The metavariables $x$ and~$n$ stand for a variable and an integer,
respectively.
This pre-encoding is just like how we represent object terms in the
metalanguage in the preceding sections, but it produces
terms in the object language rather than the metalanguage.

To evaluate~$\encode{e}$, then, we
instantiate the free variables in~$\encode{e}$ such as $\ident{int}$,
$\ident{lam}$, and $\ident{add}$ by their definitions above.  For
example, the familiar object term $(\fun{x}x)\True$ pre-encodes to
\begin{equation*}
    \encode{(\fun{x}x)\True} = \ident{app}
    (\ident{lam} (\fun{x} x))\, (\ident{bool} \True),
\end{equation*}
and to evaluate this pre-encoded term is to evaluate the object term
\begin{equation*}
    (\fun{f} \fun{x} fx) \,
    ((\fun{f} f) (\fun{x} x))\, ((\fun{b} b) \True).
\end{equation*}
Because the evaluator above mostly consists of glorified identity
functions, we expect our simple partial evaluator to reduce the
result of this instantiation to~$e$.  In general, to
interpret~$\encode{e}$ using an interpreter is to instantiate its free
variables by that interpreter's definitions.

\subsection{Avoiding higher polymorphism}

Any approach to self\hyp interpretation needs to spell out first how to
encode object terms~$e$ to object terms~$\Encode{e}$, and then how to
interpret~$\Encode{e}$ in the object language.  For our approach, we
want to define encoding in terms of pre-encoding, and interpretation
using some notion of instantiation.  Unfortunately, the simple type
structure of our object language hinders both tasks.  To continue with
the example term above, we could try to define
\begin{equation*}
    \Encode{e} =
    \fun{\ident{app}} \fun{\ident{lam}} \fun{\ident{bool}} \encode{e},
\end{equation*}
in particular
\begin{equation*}
    \Encode{(\fun{x}x)\True} =
    \fun{\ident{app}} \fun{\ident{lam}} \fun{\ident{bool}}
    \ident{app} (\ident{lam} (\fun{x} x))\, (\ident{bool} \True).
\end{equation*}
To type-check this encoded term, we give the variable $\ident{lam}$ the
simple type $(\BB\to\BB)\to\BB\to\BB$.
We then define the self\hyp interpreter
\begin{equation*}
    \si = \fun{e} e
    (\fun{f} \fun{x} fx)
    (\fun{f} f)
    (\fun{b} b)
\end{equation*}
and apply it to the encoded term.  The result is the object term
\begin{multline*}
    \bigl(\fun{e} e (\fun{f} \fun{x} fx) (\fun{f} f) (\fun{b} b)\bigr)
\\
    \bigl(
    \fun{\ident{app}} \fun{\ident{lam}} \fun{\ident{bool}}
    \ident{app} (\ident{lam} (\fun{x} x))\, (\ident{bool} \True)
    \bigr),
\end{multline*}
which partially evaluates to~$\True$ easily.
However, encoding fails on an term with multiple $\lambda$\hyp
abstractions at different types.  For example, the pre-encoding
\begin{equation*}
    \encode{\fun{f}\fun{x}fx}
    = \ident{lam} (\fun{f} \ident{lam} (\fun{x} \ident{app} f x))
\end{equation*}
does not type-check in any typing environment, because $\ident{lam}$ needs
to take two incompatible types.  In sum, we need more polymorphism in the
object type system to type $\ident{lam}$, $\ident{app}$, $\ident{fix}$,
and~$\ident{if\_}$ (and $\Encode{e}$ and~$\si$).
(The polytypes in |Symantics| given by Haskell's type classes and OCaml's
modules supply this polymorphism.)  Moreover, we need to encode any
polymorphism of the object language \emph{into} the object language to achieve
self\hyp interpretation.

\subsection{Introducing let-bound polymorphism}

Instead of adding higher-rank and higher-kind polymorphism to our object
language (along with polymorphism over kinds!), we add let-bound polymorphism.
As usual,
we can add a new typing rule
\begin{equation*}
    \begin{prooftree}
        e_1:t_1 \quad \subst{e_2}{x}{e_1}:t_2
        \justifies \be{x=e_1} e_2 : t_2
    \end{prooftree}
    .
\end{equation*}
The pre-encoding of a let\hyp expression is trivial.
\begin{align*}
    \encode{\be{x=e_1}e_2} \quad &= \quad \be{x=\encode{e_1}} \encode{e_2}
\intertext{A \emph{context} is an object term with a hole~$[~]$.  The
hole may occur under a binder, so plugging a term into the context may
capture free variables of the term.  By pre-encoding a hole to a hole,
we extend pre-encoding from a translation on terms to one on
contexts.}
    \encode{[~]} \quad &= \quad [~]
\end{align*}
We define an interpreter in the object language to be not a term but
a context.  For example, the evaluator is the context
\begin{align*}
    \si[~] &=
    \begin{tabular}[t]{@{}Ml@{}>{{}}Ml@{}Ml@{}}
        \be{\ident{int} &= \fun{x} x&} \\
        \be{\ident{add} &= \fun{x} \fun{y} x+y&} \\
        \be{\ident{if\_}&= \fun{b} \fun{t} \fun{e} \cond{b}{t\,0}{e\,0}&} \\
        \be{\ident{lam} &= \fun{f} f&} \\
        \be{\ident{app} &= \fun{f} \fun{x} fx&} \\
        \be{\ident{fix} &= \fun{g} \fix{f} \fun{x} gfx&} [~],
    \end{tabular}
\intertext{and the size\hyp measurer is the context}
    \mathrm{SZ}[~] &=
    \begin{tabular}[t]{@{}Ml@{}>{{}}Ml@{}Ml@{}}
        \be{\ident{int} &= \fun{x} 1&} \\
        \be{\ident{add} &= \fun{x} \fun{y} x+y+1&} \\
        \be{\ident{if\_}&= b + t\,0 + e\,0&} \\
        \be{\ident{lam} &= \fun{f} f\,0 + 1&} \\
        \be{\ident{app} &= \fun{f} \fun{x} f + x + 1&} \\
        \be{\ident{fix} &= \fun{g} g\,0 + 1&} [~].
    \end{tabular}
\end{align*}
To interpret an object term~$e$ using an interpreter $I[~]$ is to
evaluate the object term~$I[\encode{e}]$.  $\si$ is
a self\hyp interpreter in the following sense.
\begin{proposition}
    $\si[\encode{e}]$ is observationally equivalent to~$e$.
\end{proposition}
As a corollary, we can pre-encode the self\hyp interpreter itself as
a context: the term $\si[\encode{\si[\encode{e}]}]$ is observationally
equivalent to $\si[\encode{e}]$, and in turn to~$e$.  In other words,
$\si$ can interpret itself.  Our partial evaluator is optimal with respect to
the self\hyp interpreter~$\si$.
\begin{proposition}
    Let $\pe$ be the partial evaluator~|P| in~\S\ref{S:PE-solution}.
    Then the object terms $\pe(\si[\encode{e}])$ and~$\pe(e)$ are
    either both undefined or both defined and
    equal up to $\alpha$\hyp conversion.
\end{proposition}

\subsection{Contexts clarify polymorphism}
\label{S:clarify}

We type-check a pre-encoded term~$\encode{e}$ and its interpreter~$I[~]$
only together, never separately.  This treatment has the drawback that
we must duplicate a pre-encoded term in order to interpret it in
multiple ways.  The meta-notions of contexts and plugging may seem ad
hoc, but in fact they just reflect the type-class and module
machinery that we have been using in this paper all along.

In the presence of let-bound polymorphism, we can understand a term
waiting to be plugged into a context as a higher-rank and higher-kind
abstraction over the context.  Whereas our object language does not support
higher abstraction, our metalanguages do, so they can type-check an object term
separately from its interpreter---either as a functor from a |Symantics| module
containing a type constructor
(in OCaml), or a value with a |Symantics| constraint over a type
constructor (in Haskell).  Thus
``context'' is a euphemism for a polymorphic argument, and ``plugging''
is a euphemism for application.

%aplas \begin{comment}
\jacques{But how 
to you create, in either Haskell or MetaOCaml, an untypechecked 
interpreter-with-a-hole [UIH] ?}
\oleg{Well, one can make an argument that we already have such an
interpreter with polymorphic let and the hole: incope. In Haskell, the
declaration of an instance of Symantics is like the sequence of
polymorphic lets. We construct terms where lam, add, etc, are free
variables. We apply the interpreter to the semantics (plug the hole)
by instantiating these terms (binding the free variables lam, etc. to
the particular instance of Symantics). The unRR construction does
this plugging in explicitly.}
Again, what is different from the above is that we can
typecheck terms separately, without inserting them first within the
hole of a particular interpreter. Rank-2 type of |repr| helps. It lets
enough of the type information out so the typechecking can
proceed. So, |repr| is the representation of the polymorphic
interpreter context with the hole, which permits separately
typecheckable terms (the evaluation still entails `duplication' so to
speak -- which is one way how polymorphism is resolved).

\oleg{Mention the RR interpreter: for any term E, (RR E) is equivalent to E.
RR is not an identity: it is an interpreter that encapsulates another
interpreter. The RR interpreter can be made self if we treat RR as a
special form and interpret it always with itself (similar to let and
hole below).}
%aplas \end{comment}

%aplas \begin{comment}

The crucial role of the higher-order type parameter r

The type constructor "r" above represents a particular interpreter.  The
meta-type "r tau" hides how the interpreter represents the object type
"tau" yet exposes enough of the type information so we can type-check
the encoding of an object term without knowing what "r" is.  The checked
term is then well-typed in any interpreter.  Each instance of Symantics
instantiates "r" to interpret terms in a particular way. The L
interpreter is quite illustrative. We need the above semantics to be
able to represent both R and L (the latter returns only Int as the
values) in the same framework.

We encode a term like |add 1 2| as
\texttt{app \_add (app \_int 1) (app \_int 2)} where |_add| and |_int| are just
`free variables'. Now, how to typecheck such a term? Some type should
be assigned to these free variables. The goal is to complete the work
without needing any type annotations (so we don't have to introduce any
type language), with all types inferred and all terms typed. It seems
the second-order type R neatly separates the typechecking part from
the representation of R: it hides aspects that depend on the
particular interpreter, and yet lets enough type information through
(via its type argument) to permit the typechecking of terms, and infer
all the types. 



The only approach that does seem to work
is the one in incope.hs or incope.ml. If we de-sugar away records and
type-classes, the type of a term L of the inferred type tau is
$$ 
  (\ZZ \rightarrow r \ZZ) \rightarrow
  (\BB \rightarrow r \BB) \rightarrow
  \forall \alpha \beta. (r \alpha \rightarrow r \beta)
      \rightarrow r (\alpha\rightarrow\beta) \rightarrow ... r \tau
$$
% (Int -> r Int) ->
% (Bool -> r Bool) ->
% (forall alpha beta. (r alpha -> r beta) -> r (alpha->beta)) -> ... r tau

or, if we denote the sequence of initial arguments as |S r|, terms have
the type |S r -> r tau|
The interpreter has the type
|(forall r. S r -> r tau) -> r' tau|

The higher-order type (variable) of kind |*->*| seems essential. So, at
least we need some fragment of Fw (somehow our OCaml code manages to
avoid the full Fw; probably because the module language is separated
from the term language). Thus, we seem to need a fragment of Fw. It
seems the inference is possible, as our Haskell and OCaml code
constructively illustrates. Perhaps we need to characterize our
fragment.

%aplas \end{comment}
\end{comment}

\section{Related work}\label{related}

Our initial motivation came from several papers
\citep{WalidICFP02,taha-tag,xi-guarded,peyton-jones-simple} that used
embedded interpreters to justify advanced type systems, in particular,
GADT. While we admire all this technical work, we were not convinced
that their motivating example actually needed all this machinery.
Although GADTs may indeed be more flexible and easier to use, they are
not available in mainstream ML, and their current implementation in
GHC has problems. We were also interested in discovering the minimal and
widely available set of language features necessary for tagless
type-preserving interpretation.

Even a simply typed $\lambda$-calculus obviously supports self\hyp
interpretation, provided we use universal types \cite{taha-tag}.  The
ensuing tagging overhead motivated \cite{taha-tag} to propose tag
elimination, which however does not statically guarantee that all tags
will be removed \cite{WalidICFP02}.

% aplas: this is the simplified paragraph. The original text is in the
% comment right below.
\Citet{WalidICFP02}, \citet{taha-tag}, \citet{xi-guarded}, and
\citet{peyton-jones-simple} seem to argue that a self\hyp
interpreter of a typed language cannot be tagless or Jones\hyp optimal:
(1) One needs to encode a typed language in a typed language based on
a sum type (at some level of the hierarchy);
(2) It is not possible to write a \emph{direct} interpreter 
for such an encoding of a typed language
in a typed language, without either an
advanced type system or the overhead of tagging and untagging;
(3) Thus an indirect interpreter is necessary, which needs a universal
  type (and hence tagging).
While the logic is sound, we showed that the premise in the very first step
was not valid.
\begin{comment}
\Citet{WalidICFP02}, \citet{taha-tag}, \citet{xi-guarded}, and
\citet{peyton-jones-simple} seem to argue as follows that a self\hyp
interpreter of a typed language cannot be tagless or Jones\hyp optimal.
\begin{enumerate*}
\item One needs to encode a typed language in a typed language based on
a sum type (at some level of the hierarchy).
\item It is not possible to write a \emph{direct} interpreter 
for such an encoding of a typed language
in a typed language, without either an
advanced type system or the overhead of tagging and untagging.
\item Thus an indirect interpreter is necessary, which needs a universal
  type (and hence tagging).
\item Thus any self-interpreter must have tags and cannot be 
  Jones-optimal.
\end{enumerate*}
While the logic is sound, we showed that the premise in the very first step
was not valid.
\end{comment}

\citet{Danvy-tagging-encoding} discuss Jones optimality at length and
apply HOAS to typed self\hyp interpretation.  However, their source
language is untyped.  Therefore, their object\hyp term encoding has
tags, and their interpreter can raise run-time errors.
Nevertheless, HOAS lets the partial
evaluator remove all the tags. In contrast, our object encoding and
interpreters do not have tags to start with and obviously cannot
raise run-time errors.

%aplas-else
\begin{comment}
  A lot of effort has gone into ``typing dynamic typing'': to statically
  type-check dynamically\hyp typed values
  \cite{baars-typing,haskell-list},
  using the host language's type system to varying extents.
  Our object terms are statically typed, so we would
  need one of these techniques to interpret dynamically\hyp typed
  terms such as those read from a file.
\end{comment}


Our partial evaluator establishes a bijection |repr_pe| between static
and dynamic types (the valid values of |'sv| and |'dv|), and between
static and dynamic terms.  It is customary to implement such a bijection
using an injection\hyp projection pair, as done for interpreters by
\cite{Ramsey-ML-module-mania,Benton-embedded-interpreters},
partial evaluation by \cite{Danvy-TDPE}, and type-level functions by
\cite{oliveira-typecase}.  As explained in~\S\ref{S:PE-solution}, we
avoid injection and projection at the type level by adding an argument
to |repr|.

At the term level, we also avoid converting between static and dynamic
terms by building them in parallel, using \citearound{'s
method}\citet{asai-binding-time}.
This method type-checks in Hindley-Milner once we
deforest the object term representation.  Put another way, we
manual apply type-level partial evaluation to our type
functions (see \S\ref{S:PE-solution}) to obtain simpler types 
acceptable to MetaOCaml.
\Citet{sumii-hybrid} also use Asai's method,
to combine online
and offline partial evaluation.  Like \cite{SperberThiemann:TwoForOne},
we strive for modularity by reusing interpreters for individual stages.  It would be interesting
to try to derive a \emph{cogen} \cite{Thiemann:cogeninsixlines}
in the same manner.

The idea of multiple interpretations of the host language pervasives
such as addition or application is quite common in implementing
embedded DSL. It is also common to use phantom types to prevent
forming of ill-typed DSL terms \cite{Lava,Rhiger-thesis}. However,
the universal type still remains (|Bit| and |NumSig| in Lava
\cite{Lava}, |Raw| and |Term| in \cite[Fig 2.2,
  Sec. 3]{Rhiger-thesis}), along with the attendant pattern-matching
overhead. These approaches are not tagless.  The universal type also
greatly complicates the soundness and completeness proofs of embedding
\cite{Rhiger-thesis}. In contrast, our proofs become trivial.
The approach of \Citet[Sec 3.3.4]{Rhiger-thesis} does not 
support typed CPS transformation.
\begin{comment}
Rhiger's But Fig 2.2, p33: universal type Raw.  He uses phantom type
upon the Exp datatype. But that is cheating: phantom type means
essentially we can easily do coerce. We use real types.  That's why he
had to do tedious proofs in Sec 2 of soundness and completeness of
embedding. Whereas our proofs are obvious.  His sec 3 is based on data
representation of terms. They have type tags.  We do nothing of that
kind: See Sec 3.1.2. See numerous "data Term" in Sec3, which is the U
type.  In Sec 3.3.4 (p76) Rhiger specifically says that his encoding
cannot do typed CPS transformation -- whereas our does. BTW, Rhiger
thesis contains the definitions of the interpreter and the compiler,
in the beginning. Use this in response to Rev1)
\end{comment}


We are not the first to implement a typed interpreter for a typed
language.  \Citet{laod93} use type classes to implement a metacircular
interpreter (rather than a self\hyp interpreter) of a
typed version of the SK language, which is quite different from our
object language.  Their interpreter
appears to be tagless, but they could not have implemented a
compiler or partial evaluator in the same way, since they rely
heavily on injection\hyp projection pairs.

\Citet{fiore:nbe-ppdp2002} and \citet{balat:tdpe-popl2004} also build
a tagless partial evaluator, using delimited control operators.  It is
type-directed, so the user must represent, as a term, the type of every
term to be partially evaluated.  We shift this work to the type checker
of the metalanguage.  By avoiding term-level type representations, our
approach makes it easier to perform algebraic simplifications (as
in~\S\ref{S:PE-solution}).


Our approach is based on encoding a term in its elimination form, as a
co-algebraic structure. That basic idea and its application to
meta-circular interpertation first appeared in \citet{Pfenning-Lee}.
Our approach however is implementable in mainstream ML and supports
type inference and typed CPS and partial evaluation. In contrast,
\cite{Pfenning-Lee} conclude that partial evaluation and program
transformations "do not seem to be expressible" even using their
extension to $F_{\omega}$, perhaps because their avoidance of general
recursive types compels them to include the polymorphic lift that we
avoid in \S\ref{S:PE-lift}.
\begin{comment}
It seems that Pfenning and Lee embed $F_2$ with type constructions in 
(pure) $F_3$.  We embed $F_1$ in (weak?) $F_2$, as I see it.  In a way, what 
we do is very similar to what they do (Figure 1, p.152), except that we 
do it in standard programming languges.  It is unclear if their work can 
be implemented (yet) in any language.  And we preserve type-inference, 
while their solution needs explcit types!
The following line of their conclusion is worth citing: "... this does 
not imply that the same language is also suitable for type 
metaprogramming. ... such as partial evaluation... do not seem to be 
expressible".
I suspect you're right, but I'm still reading the paper.  See also page
146: "for a term M in $F_1$ (a simply-typed term), the representation
$\bar{M}$ will be in $F_2$".  The move from $F_2$ to $F_3$ 
and beyond reminds me
strongly of our attempts at self-interpretation without the notion of a
syntactic hole.
\end{comment}

Our encoding of the type function |repr_pe| in \S\ref{S:PE-solution}
may be regarded as an emulation of type-indexed types and related to
the intensional type analysis
\cite{Morrisett-intensional,Generic-Haskell}. Our running HOAS
example however, includes |fix|, which cannot be handled by the
intensional type analysis \cite{xi-guarded}.  Our final approach
seems related to \citearound{'s approach to HOAS via catamorphism and
  anamorphism}\cite{Washburn-Weirich-boxes}.


We could not find work that establishes that
the \emph{typed} $\lambda$-calculus possesses a final coalgebra structure.
% aplas: abbreviated. The original text is in comments
(see \Citet{honsell99coinductive} for the untyped case).
% aplas: comment
\begin{comment}
\Citet{HonsellLenisa,honsell99coinductive}
investigate the untyped $\lambda$-calculus,
along this line.  
In particular, they use
contexts with a hole \cite[p.\,13]{honsell99coinductive} to define
\emph{observational equivalence}
(see \S\ref{selfinterp}).
\citearound{'s bibliography}\Citet{honsell99coinductive} refers to the
foundational work in this important area.  
Particularly intriguing is the link to the
coinductive aspects of B\"{o}hm trees, as pointed out by
\cite{berarducci-models} and \cite[Example 4.3.4]{jacobs-coalgebra}.
\end{comment}

%aplas
\begin{comment}
One way to understand our main idea is to eschew sum types and sum kinds
for their dual, record types and record kinds.
For the self\hyp interpreter, we then proceed to use a Church encoding for
recursive data types \cite{bohm-automatic}.
\end{comment}

%aplas: beforeAs \S\ref{S:clarify} observes, 
We observe that
higher-rank and higher-kind
polymorphism let us type-check and compile object terms separately from
interpreters.  This is consistent with the role of
polymorphism in the separate compilation of modules
\cite{shao-typed}.

\section{Conclusions}\label{conclusion}

We solve the problem of embedding a typed object language in a typed
metalanguage without using GADTs, dependent types, or a universal type.
Our family of interpreters include an evaluator, a compiler, a partial
evaluator, and CPS transformers.  It is patent that they never get stuck,
because we represent object types as metalanguage types.  This work
makes it safer and more efficient to embed %domain\hyp specific languages
DSLs
in practical metalanguages such as Haskell and ML\@.

Our main idea is to represent object programs not in an initial algebra
but using the existing coalgebraic structure of the $\lambda$-calculus.
More generally, to squeeze more invariants out of a type system as
simple as Hindley-Milner, we shift the burden of representation and
computation from consumers to producers: encoding object terms as calls
to metalanguage functions (\S\ref{ourapproach}); build dynamic terms
alongside static ones (\S\ref{S:PE-lift}); simulating type functions for
partial evaluation (\S\ref{S:PE-solution}) and CPS
transformation%
%aplas
.
%aplas-else ~(\S\ref{S:CPS}).
This shift also underlies fusion,
functionalization, and amortized complexity analysis.
%aplas-else
\begin{comment}
When the metalanguage does provide higher-rank and higher-kind
polymorphism, we can type-check and compile an object term separately
from any interpreters it may be plugged into.
\end{comment}

Our representation of object terms in elimination form encodes
primitive recursive folds over the terms. 
%aplas This makes operations like interpretation trivial to implement. 
We still
have to understand if and how non-primitively 
recursive operations can be supported.

% aplas: acknowledgments are at the beginning
% \section*{Acknowledgments}
% \ourthanks

\bibliographystyle{mcbride}
\bibsep=0pt
% aplas
\bibliography{aplas}
% \bibliography{tagless}
\end{document}
