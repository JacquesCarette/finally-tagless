\documentclass{llncs}

% Reduce display spacing
\divide\abovedisplayskip 2
\divide\belowdisplayskip 2
\divide\abovedisplayshortskip 2
\divide\belowdisplayshortskip 2

\usepackage{hyphenat}
\usepackage{comment}
\usepackage{amsmath,amssymb}
\usepackage{amstext}
\usepackage{url}
\usepackage[dvips]{color}


\usepackage{ifpdf}
\ifpdf
    \pdfpageheight=11in
    \pdfpagewidth=8.5in
\fi

%\usepackage{refrange}

%\usepackage[medium,compact]{titlesec}

\usepackage{fancyvrb}
\DefineShortVerb{\|}
\DefineVerbatimEnvironment{code}{Verbatim}{xleftmargin=\mathindent,commandchars=\\\{\},fontsize=\small}
\newcommand{\evalresult}[1]{\ensuremath{\Longrightarrow}\textcolor{red}{#1}}
% \setlength{\parskip}{0pt}
\newlength{\mathindent}
\setlength{\mathindent}{1em}

% \usepackage[backref,colorlinks,bookmarks=true]{hyperref}

% Reduce list spacing
%% \makeatletter
%% \renewcommand\@@listI{\leftmargin\leftmargini
%% \parsep \z@@
%% \topsep 3\p@@ \@@plus\p@@ \@@minus 2\p@@
%% \itemsep 2\p@@ \@@plus\p@@ \@@minus\p@@}
%% \let\@@listi\@@listI
%% \@@listi
%% \makeatother

\parskip 0pt plus 1pt minus 2pt
\textfloatsep 4pt plus 2pt minus 3pt % Less space around figures
\abovecaptionskip 0pt plus 0pt minus 2pt
\belowcaptionskip 0pt plus 0pt minus 2pt

\intextsep 2pt plus 0pt minus 1pt

\renewcommand\floatpagefraction{.9}
\renewcommand\topfraction{.9}
\renewcommand\bottomfraction{.9}
\renewcommand\textfraction{.1}   
\setcounter{totalnumber}{50}
\setcounter{topnumber}{50}
\setcounter{bottomnumber}{50}

\newcommand{\omitnow}[1]{}
\newcommand{\oleg}[1]{{\it [Oleg says: #1]}}
\newcommand{\jacques}[1]{{\it [Jacques says: #1]}}

\begin{document}
%\title{Functors, CPS and monads, or how to generate efficient
%code from abstract designs}
\title{Multi-stage programming with functors and monads:
eliminating abstraction overhead from generic code}
\author{Jacques Carette\inst{1}
 \and
Oleg Kiselyov\inst{2}
}
\institute{McMaster University,
1280 Main St. West, Hamilton, Ontario Canada L8S 4K1
\and
FNMOC, Monterey, CA 93943}

\maketitle

\begin{abstract}
With Gaussian Elimination as a representative family of numerical
and symbolic
algorithms, we use multi-stage programming, monads and Ocaml's
advanced module system to demonstrate the complete elimination of the
abstraction overhead while avoiding any inspection of the generated
code.  We parameterize our Gaussian Elimination code to a great extent
(over domain, matrix representations, determinant tracking, 
pivoting policies, result types, etc) at no run-time cost.  Because
the resulting code is generated just right and not changed afterwards,
we enjoy MetaOCaml's guaranty that the generated code is well-typed.
We further demonstrate that various abstraction parameters (aspects)
can be made orthogonal and compositional, even in the presence of
name-generation for temporaries and other bindings and
``interleaving'' of aspects.  We also show how to encode some
domain-specific knowledge so that ``clearly wrong'' compositions can
be statically rejected by the compiler when processing the generator
rather than the generated code.
\end{abstract}

% previous introduction is saved at the end
\section{Introduction}

In high-performance, symbolic, and numeric computing, there is a
well-known issue of balancing between maximal performance and the
level of abstraction at which code is written.  Furthermore, already
in linear algebra, there is a wealth of different aspects that
\emph{may} need to be addressed. For example, implementations of the
widely used Gaussian Elimination (GE) algorithm --- the running
example of our paper --- may need to account for the representation of
the matrix, whether to compute and return the determinant or rank, how
and whether search for pivot, etc. Furthermore, current architectures
demand more and more frequent tweaks which, in general, cannot be done by the
compiler because the tweaking often involves domain knowledge. 
A survey \cite{carette04} of
Gaussian elimination implementations in an industrial package Maple
found 6 different aspects and 35 different implementations of the
algorithm, as well as 45 implementations of directly related
algorithms.  We can manually write each of these implementations
optimizing for particular aspects and using cut-and-paste to ``share''
similar pieces of code.
We can write a very generic GE procedure that accounts for
all the aspects with appropriate abstractions \cite{Axiom}. The
abstraction mechanisms however -- be they procedure, method or a
function call -- have a significant cost, especially for
high-performance numerical computing \cite{carette04}. 

A more appealing approach is generative programming
\cite{Czarnecki,Veldhuizen:1998:ISCOPE,musser94algorithmoriented,BOOST,POOMA,ATLAS}.
The approach is not without problems, e.g., making sure that the
generated code is well-formed. This is a challenge in string-based
generation systems, which generally do not offer such guarantees and
therefore make it very difficult to determine which part of the
generator is at fault when the generated code cannot be parsed. Other
problems is preventing accidental variable capture (so-called hygiene
\cite{HygienicMacros}) and ensuring the generated code is
well-typed. Lisp-style macros, Scheme hygienic macros, camlp4
preprocessor \cite{camlp4}, C++ template meta-programming, Template
Haskell \cite{conf/dagstuhl/CzarneckiOST03} solve some of the above
problems. Of the widely available maintainable languages, only
MetaOCaml \cite{CTHL03,metaocaml-org}  solves all the above problems
including the well-typing of the generated code \cite{TahaSheard97,TahaThesis}.

But more difficult problems remain. Is the generated code optimal? Do
we still need post-processing to eliminate common subexpressions and
fold constants, remove redundant bindings? Is the generator readable,
resembling the original algorithm, and extensible? Are the aspects
truly modular? Can we add another aspect to it or another instance of
the existing aspect without affecting the existing ones? Finally, can
we express domain-specific knowledge, e.g., one should not attempt to
use full division when dealing with matrices of exact integers, nor is
it worthwhile to use full pivoting on a matrix over $\mathbb Q $.

MetaOCaml is \emph{generative}: generated code can only be treated as
a black box: it cannot be inspected and it cannot be post-processed
(i.e., no intensional analysis). This approach gives a stronger
equational theory \cite{Taha2000}, and avoids the danger of creating
unsoundness \cite{TahaThesis}. Furthermore, intensional code analysis
essentially requires one to insert both an optimizing compiler and an
automated theorem proving system into the code generating system
\cite{Pueschel:05,Kennedy01Telescoping,dongarra7,Veldhuizen:2004}.
While this is potentially extremely powerful and an exciting area of
research, it is also extremely complex, which means that it is
currently more error-prone and difficult to ascertain the correctness
of the resulting code.

Therefore, in MetaOCaml, code must be generated just right (see
\cite{TahaThesis} for many simple examples).  For more complex
examples, new techniques are necessary, e.g., abstract interpretation
\cite{KiselyovTaha}.  But more problems remain
\cite{Padua:MetaOcaml:04}: generating binding statements (``names''),
especially when generating loop bodies or conditional branches; making
continuation-passing style (CPS) code clear.  Many authors
understandably shy away from CPS code as it quickly becomes
unreadable.  But this is needed for proper name generation.
The problems of compositionality of code generators, expressing
dependencies among them and domain-specific knowledge remain.

In this paper we report on progress of solving these problems using GE
as our running example. Specifically, our contributions:
\vspace*{-2pt}
\begin{itemize}
    \item Extending monad of \cite{KiselyovTaha} for generating
      binding statements when generating control structures such as
      loops and conditionals. We argue that code generation is an
      effect and so requires something like a monad for proper
      serialization.
    \item Implementation of the |mdo|-notation (patterned after
      |do|-notation of Haskell) to make monadic code readable.
    \item Use of functors (including higher-order functors) to
      modularize the generator, express aspects (including results of
      various types) and \emph{assure composability of aspects} even
      for aspects that use state and have to be accounted in many
      places in the generated code.
    \item Use functor type sharing constraints to encode domain-specific
      knowledge.
\end{itemize}

The rest of this paper is structured as follows: The next section
introduces code generation in MetaOCaml, the problem of name
generation, and continuation-passing style (CPS) as a general
solution.  We also introduce the monad and the issues of generating
control statements. Section~\ref{functors} describes the use of
parametrized modules of OCaml to encode all of the aspects of the
Gaussian Elimination algorithm family in completely separate,
independent modules.  We briefly discuss related work in
section~\ref{related}. We then outline the future work and conclude.
Appendices give samples of the generated code (which is available in
full at \cite{metamonadsURL}).

We wish to thank Cristiano Calgano for his help in adapting camlp4 for
use with MetaOCaml.


\section{Generating binding statements, CPS, and monad}\label{CPS}
%%%%%%%%%%%%%%%%%%%%%%%%%%%%%%%%%%%%%%%%%%%%%%%

We build code generators out of primitive ones using code generation 
combinators. MetaOCaml, as an instance of a multi-stage
programming system \cite{TahaThesis}, provides exactly the needed
features: to construct a code expression, to combine them, and to
execute them. Figure~\ref{easycode} shows the simplest code generator |one|,
as well as more complex generators.

\begin{figure}\label{easycode}
\begin{code}
let one = .<1>. and plus x y = .<.~x + .~y>.
let simplest_code = let gen x y = plus x (plus y one) in
  .<fun x y -> .~(gen .<x>. .<y>.)>.
\evalresult{.<fun x_1 -> fun y_2 -> (x_1 + (y_2 + 1))>.}
let simplest_param_code plus one = let gen x y = plus x (plus y one) in
  .<fun x y -> .~(gen .<x>. .<y>.)>.
let param_code1 plus one =
  let gen x y = plus (plus y one) (plus x (plus y one)) in
  .<fun x y -> .~(gen .<x>. .<y>.)>.
let param_code1' plus one =
  let gen x y = let ce = (plus y one) in  plus ce (plus x ce) in
  .<fun x y -> .~(gen .<x>. .<y>.)>.
param_code1' plus one
\evalresult{.<fun x_1 -> fun y_2 -> ((y_2 + 1) + (x_1 + (y_2 + 1)))>.}
\end{code}
\caption{Code generation and combinators. $\Longrightarrow$ under an
  expression shows the result of its evaluation}
\end{figure}

We use MetaOCaml brackets |.<...>.| to generate code expressions,
i.e., to construct future-stage computations. We use escapes |.~| to
perform an immediate code generating computation \emph{while} we are
building the future-stage computation. The immediate computation in
|simplest_code| is the evaluation of the function |gen|, which in turn
applies |plus|. The function |gen| receives code expressions |.<x>.|
and |.<y>.| as arguments. At the generating stage, we can manipulate
code expressions as (opaque) values. The function |gen| returns a code
expression, which is inlined in the place of the escape. MetaOCaml can
print out code expressions, so we can see the final generated code. It
has no traces of |gen| and |plus|: their applications are done at the
generation stage.

The final MetaOCaml feature, |.!| (pronounced ``run'') 
executes the code expression: |.! simplest_code| is a function of two
integers, which we can apply: |(.! simplest_code) 1 2|. The original
|simplest_code| is not a function on integers -- it is a code
expression.

To see the benefit of code generation, we notice that we can easily
parameterize our code, |simplest_param_code|, and use it to generate
code that operates on integers, floating point numbers or booleans --
in general, any domain that implements |plus| and |one|.

The generator |param_code1| has two occurrences of |plus y one|,
which may be quite a complex computation and so we would rather not do
it twice. We may be tempted to rely on the compiler's
common-subexpression elimination optimization. When the generated code is
very complex, however, the compiler may overlook common subexpressions.  Or the
subexpressions may occur in such an imperative context where the compiler
might not be able to determine if lifting them is sound. So, being
conservative, the optimizer will leave the duplicates as they are. We
may attempt to eliminate subexpressions as in |param_code1'|. However,
the result of |param_code1' plus one| still exhibits duplicate
sub-expressions.  Our |let|-insertion optimization saved the
computation at the generating stage.  We need a combinator that
inserts the |let| expression in the generat\emph{ed} code. We need a
combinator |letgen| to be used as\\
|let ce = letgen (plus y one) in plus ce (plus x ce)|
yielding the code like |.<let t = y + 1 in t + (x + t)>.| 
But that seems impossible because |letgen exp| has to generate
the expression |.<let t = exp in body>.| but |letgen| does not
have the |body| yet. The body needs a temporary identifier |.<t>.|
that is supposed to be the result of |letgen| itself.  Certainly
|letgen| cannot generate only part of a let-expression, without the
|body|, as all generated expressions in MetaOCaml are well-formed and
complete.

The key is to use continuation-passing style (CPS). Its benefits were first
pointed out by \cite{Bondorf:92} in the context
of partial evaluation, and extensively used by \cite{KiselyovTaha} for code
generation. Now, |param_code2 plus one| gives us the desired code. 

\begin{code}
let letgen exp k = .<let t = .~exp in .~(k .<t>.)>.
let param_code2 plus one =
  let gen x y k = letgen (plus y one) (fun ce -> k (plus ce (plus x ce)))
  and k0 x = x
  in .<fun x y -> .~(gen .<x>. .<y>. k0)>.
param_code2 plus one
\evalresult{.<fun x_1 -> fun y_2 -> let t_3 = (y_2 + 1) in (t_3 + (x_1 + t_3))>.}
\end{code}

\noindent Comparison of the code that did let-insertion at the generating stage\\
|let ce = (plus y one) in  plus ce (plus x ce)|\\
with the corresponding code inserting let at the generated code stage\\
|letgen (plus y one) (fun ce -> k (plus ce (plus x ce)))|\\
clearly shows the difference between  direct-style and CPS code.
What was\\ |let ce = init in ...| in direct style became
|init' (fun ce -> ...)| in CPS. For one thing, |let| became
``inverted''. For another, what used to be an expression that yields
a value, |init|, became an expression that takes an extra argument,
the continuation, and invokes it. The differences look negligible in
the above example. In larger expressions with many let-forms, the
number of parentheses around |fun| increases, the need to add and
then invoke the |k| continuation argument become increasingly annoying. The
inconvenience is great enough for some people to explicitly avoid CPS
or claim that numerical programmers (our users) cannot or will not
program in CPS. Clearly a better notation is needed.

The |do|-notation of Haskell \cite{Haskell98Report} shows that it is possible
to write CPS code in a conventional-looking style. The
|do|-notation is the notation for monadic code \cite{moggi-notions}.
Not only can monadic code represent CPS \cite{Filinski:Representing},
it also helps in composability by offering to add different
layers of effects (state, exception, non-determinism, etc) to the
basic monad \cite{liang-interpreter} in a controlled way.

A monad \cite{moggi-notions} is an abstract data type representing
computations that yield a value and may have an \emph{effect}.
The data type must have at least two operations, |return| to build
trivial effect-less computations and |bind| for combining
computations. These operations must satisfy \emph{monadic laws}:
|return| being the left and the right unit of |bind| and |bind| being
associative. Figure~\ref{ourmonad} defines the monad used throughout
the present paper and shows its implementation.

\begin{figure}\label{ourmonad}
\begin{code}
type ('v,'s,'w) monad = 's -> ('s -> 'v -> 'w) -> 'w
let ret a = fun s k -> k s a
let bind a f = fun s k -> a s (fun s' b -> f b s' k)
let fetch s k = k s s  and  store v s k = k (v::s) ()

let k0 s v = v
let runM m = m [] k0

let l1 f = fun x -> mdo \{ t <-- x; f t\}
let l2 f = fun x y -> mdo \{ tx <-- x; ty <-- y; f tx ty\}

let retN a = fun s k -> .<let t = .~a in .~(k s .<t>.)>.

let ifL test th el = ret .< if .~test then .~th else .~el >.
let ifM test th el = fun s k ->
  k s .< if .~(test s k0) then .~(th s k0) else .~(el s k0) >.
\end{code}
\caption{Our monad}
\end{figure}

Our monad represents two kinds of computational effects: reading and
writing a computation-wide state, and control effects. The latter are
normally associated with exceptions, forking of computations, etc. --
in general, whenever a computation ends with something other than
invoking its natural continuation in the tail position. In our case
the control effects manifest themselves as code generation.

In Figure~\ref{ourmonad}, the monad is implemented as a function of two
arguments: the state (of type |s|) and the continuation. The
continuation receives the current state, the value (of the type |v|) and
yields the answer of the type |w|.  The monad is polymorphic over the
three type parameters.  Other implementations are
possible. Except for the code in Figure~\ref{ourmonad}, the rest of our code
treats the monad as a truly abstract data type. The implementation of
the basic
monadic operations |ret| and |bind| are conventional. It is easy to
see that the monadic laws are satisfied.  Other monadic operations
construct computations that do have specific effects.  Operations |fetch| and
|store v| construct computations that read and write the state. In our
case the state is a list (of polymorphic variants), which models an
open discriminated union, as we shall see later.

The operation |retN a| is the let-insertion operation, whose simpler
version we called |letgen| earlier. It is the first computation with
a control effect: indeed, the result of |retN a| is \emph{not} the
result of invoking its continuation |k|. Rather, its result is a |let|
code expression. Such a behavior is symptomatic of control operators
(in particular, |abort|).

Finally, |runM| runs our monad, that is, performs the computation of
the monad and returns its result, which in our case is the code
expression. We run the monad by passing it the initial state and the
initial continuation |k0|. We can now re-write our |param_code2|
example of the previous section as |param_code3|.

\begin{figure}
\begin{code}
let param_code3 plus one =
  let gen x y = bind (retN (plus y one)) (fun ce -> 
                ret (plus ce (plus x ce)))
  in .<fun x y -> .~(runM (gen .<x>. .<y>.))>.
let param_code4 plus one =
  let gen x y = mdo \{ ce <-- retN (plus y one);
                      ret (plus ce (plus x ce)) \}
  in .<fun x y -> .~(runM (gen .<x>. .<y>.))>.
let ifM' test th el = mdo \{
  testc <-- test; thc <-- th; elc <-- el;
  ifL testc thc elc\}
let gen a i = ifM' (ret .<(.~i) >= 0>.) 
                   (retN .<Some (.~a).(.~i)>.) (ret .<None>.)
 in .<fun a i -> .~(runM (gen .<a>. .<i>.))>.
\evalresult{.<fun a_1 i_2 ->  
let t_3 = (Some a_1.(i_2)) in if (i_2 >= 0) then t_3 else None>.}
let gen a i = ifM (ret .<(.~i) >= 0>.) 
                  (retN .<Some (.~a).(.~i)>.) (ret .<None>.)
 in .<fun a i -> .~(runM (gen .<a>. .<i>.))>.
\evalresult{.<fun a_1 i_2 -> 
if (i_2 >= 0) then let t_3 = (Some a_1.(i_2)) in t_3 else None>.}
\end{code}
\end{figure}
% param_code3 plus one;;

That does not seem like much of an improvement. With the help of
camlp4 pre-processor, we introduce the |mdo|-notation \cite{metamonadsURL},
patterned after the |do|-notation of Haskell. The function
|param_code4|, written in the |mdo|-notation, is equivalent to
|param_code3| -- in fact, the camlp4 preprocessor will convert the
former into the latter. And yet, |param_code4| looks far more
conventional, as if it were indeed in direct style.

We can write operations that generate code other than let-statements,
e.g., conditionals: see |ifL| in Figure~\ref{ourmonad}. The function |ifL|, 
albeit straightforward, is not as general as we wish: its arguments are
already generated pieces of code rather than monadic values. We
``lift it'', see |ifM'|. We define functions |l1|,
|l2|, |l3| (analogues of |liftM|, |liftM2|, |liftM3| of Haskell) 
to make such a lifting generic. However we also need
another |ifM| function, with the same
interface (see Figure~\ref{ourmonad}). The difference between them is
apparent: in the code above with |ifM'|, the let-insertion
happened \emph{before} the if-expression, that is, before the test that
the index |i| is positive. If |i| turned out
negative, |a.(i)| would generate an out-of-bound array access
error. On the other hand, the code with |ifM| accesses the array only
when we have verified that the index is non-negative. This example
makes it clear that the code generation (such as the one in |retN|) is 
truly an effect and we have to be clear about the sequencing of
effects when generating control constructions such as conditionals.
The form |ifM| handles such effects correctly. We
need similar operators for other Ocaml control forms: for generating
case-matching statements and |for|- and |while|-loops.

\section{Aspects and Functors}\label{functors}

The monad represents finer-scale code generation. We need tools for
larger-scale modularization; we can use any abstraction
mechanisms we want to structure our code generators, as long as none
of those abstractions infiltrate the generated code.

While the Object-Oriented Design community has acquired an extensive
vocabulary for describing modularity ideas, the guiding principles for
modular designs has not changed since they were first articulated by
Parnas~\cite{journals/cacm/parnas72a} and Dijkstra~
\cite{EWD:EWD447}: information hiding and separation of concerns.  To
apply these principles to the study of Gaussian Elimination, we need
to understand what are the changes between different implementations, and 
what concerns need to be addressed.  We also need to study the degree
to which these concerns are independent.
A study of Gaussian Elimination \cite{carette04} shows that
the following variations occur:
\begin{enumerate}
\vspace*{-6pt}
	\item \textbf{Domain}: In which (algebraic) domain do the
	  matrix elements belong to.  Sometimes the domains are very
	  specific (e.g., $\mathbb{Z}, \mathbb{Q}, \mathbb{Z}_p$ and
	  floating point numbers), while in other cases the domains
	  were left generic, e.g., multivariate polynomials over a
	  field.  In the roughly 85 pieces of code surveyed
	  \cite{carette04} 20 different domains were encountered.
	\item \textbf{Container}: Whether the matrix is represented as
	  an array of arrays, a one-dimensional array, a hash table, a
	  sparse matrix, etc., and whether indexing is done in C or
	  Fortran style.  Additionally, if a particular representation
	  had a special mechanism for efficient row exchanges.
	\item \textbf{Output choices}: Whether just the reduced
	  matrix, or additionally the rank, the determinant, and the
	  pivoting matrix are to be returned. In the larger algorithm
	  family, routines like Maple's
	  \texttt{LinearAlgebra:-LUDecomposition} have up to $2^6 +
	  2^5 + 2^2 = 100$ outputs.
	\item \textbf{Fraction-free}: Whether the Gaussian Elimination
		algorithm is allowed to use unrestricted division, or only
		exact (remainder-free) division.
	\item \textbf{Pivoting}: Whether to use no pivoting, 
		column-wise pivoting, or full pivoting.
	\item \textbf{Augmented Matrices}: Whether we are doing GE on
		a full matrix, or only a restricted number of columns, while
		doing elimination on the full matrix.  We currently
		do not treat this aspect in our code.
\end{enumerate}
\noindent In addition to the above variations, there are two aspects that 
recur frequently:
\vspace*{-6pt}
\begin{enumerate}
	\item \textbf{Length measure}:  For stability reasons
		(numerical or coefficient growth), if a domain
	  possesses
		an appropriate length measure, this is sometimes used to choose
		an ``optimal'' pivot.
	\item \textbf{Normalization and zero-equivalence}: Whether the 
		arithmetic operations of the domain at hand gives results in 
		normalized form, and whether a specialized zero-equivalence 
		routine needs to be used.
\end{enumerate}
\noindent These are separated out from the others are they are cross-cutting
concerns: in the case of the length measure, a property of the domain
will influence the method used to do pivoting \emph{if} pivoting is to be
performed.

The simplest parametrization is to make the domain abstract. As it
turns out, we need the following to exist in our domains: $0$, $1$,
$+$, $*$, (unary and binary) $-$, at least \emph{exact} division,
normalization, and potentially a relative size measure. The simplest
case of such domain abstraction is |param_code1| in
Fig.~\ref{easycode}. There, code-generators such as |plus| and |one|
were passed as arguments. We need far more than
two parameters, so we have to group them. Instead of the grouping
offered by regular records, we use Ocaml \emph{structures} (i.e.,
modules)
so we can take advantage of extensibility, type abstraction and constraints,
and especially parameterized structures (\emph{functors}).
We define the type of the domain, the signature |DOMAIN| which
different domains must satisfy:
\begin{figure}[h]
\begin{code}
module type DOMAIN = sig
  type v    type 'a vc = ('a,v) code
  type kind (* Field or Ring ? *)
  val zero : 'a vc   val one : 'a vc
  val plus : 'a vc -> 'a vc -> ('a vc, 's, 'w) monad
  (* times, minus, uminus, div elided for brevity *)
  val better_than : ('a vc -> 'a vc -> 
      (('a,bool) code, 's, 'w) monad) option
  val normalizerf : (('a,v -> v) code ) option
end 
module IntegerDomain : DOMAIN = struct
  type v = int  type kind = domain_is_ring
  type 'a vc = ('a,v) code
  let zero = .< 0 >.  and one = .< 1 >. 
  let plus x y = ret .<.~x + .~y>. 
  let better_than = Some (fun x y -> ret .<abs .~x > abs .~y >. )
  let normalizerf = None
  ...
end
\end{code}
\end{figure}

\noindent  The types above are
generally lifted twice: once from the value domain |v| to the code
domain |'a vc|, and once more from values to monadic computations
|('a vc, 's, 'w) monad|. 

One particular domain instance is |IntegerDomain|. The notation\\
|module IntegerDomain : DOMAIN| makes the compiler verify that our
|IntegerDomain| is indeed a |DOMAIN|, that is, satisfies the required
signature. The constraint |DOMAIN| may be omitted; in that case, the
compiler will verify the type when we try to use that structure as a
|DOMAIN|. In any case, the errors such as missing ``methods'' or
methods with incorrect types will be caught statically, even
\emph{before} any code generation takes place. The abstract type
|domain_is_ring| encodes a semantic constraint that the full division
is not available. While the |DOMAIN| type may have looked daunting to
some, the implementation is quite straightforward.  Other domains such
as |float| and arbitrary precision exact rational numbers |Num.num|
are equally simple.

Parameterizing by the kind of container representing a matrix is
almost as straightforward.  Our containers are  parametric
over a |DOMAIN|, i.e., functors from a |DOMAIN| module
to the actual implementation of a container. The functor signature
|CONTAINER2D| specifies that a container must provide functions |dim1|
and |dim2| to extract the dimensions, functions |get| and |set| to
generate container getter and setters, the cloning generator |copy|
and functions that generate code for row and column swapping. The
inclusion of these functions in the signature of all containers makes
it simpler to optimize the relevant functions depending on the actual
representation of the container while not burdening the users of
containers with efficiency details. 

The use of a |functor| for making a container parametric is fairly
straightforward.  More interesting is the aspect of what to return
from the GE algorithm.  One could create an algebraic data type (as
was done in \cite{carette04}) to encode the various choices: the
matrix, the matrix and the rank, the matrix and the determinant, the
matrix, rank and determinant, and so on. This is wholly unsatisfying
as we know that for any single use, only one of the choices is ever
possible, yet any routine which calls the generated code must deal
with these unreachable options.  Instead we use a module type with an
\emph{abstract} type |res| for the result type; different instances of
the signature set the result type differently. Given below is this
module type and one instantiation, which specifies the output of a GE
algorithm as a 3-tuple |contr * Det.outdet * int| of the U-factor, the
determinant and the rank.

\begin{code}
module type OUTPUT = sig
  type contr  type res
  module D : DETERMINANT   module R : RANK   module P : TRACKPIVOT
  val make_result : ('a,contr) code -> 
    (('a,res) code,
     [> `TDet of 'a D.lstate | `TRan of 'a R.lstate | `TPivot of 'a P.lstate]
       list, ('a,'w) code) monad
end
module OutDetRank(Dom:DOMAIN)(C: CONTAINER2D)
    (Det : DETERMINANT with type indet = Dom.v and type outdet = Dom.v)
    (Rank : RANK) = struct
  module Ctr = C(Dom)
  type contr = Ctr.contr
  type res = contr * Det.outdet * int
  module D = Det   module R = Rank   module P = DiscardPivot
  let make_result b = mdo \{ det  <-- D.fin ();  rank <-- R.fin ();
    ret .< ( .~b, .~det, .~rank ) >. \}
end
\end{code}

As is apparent from the output choices, several different quantities
\emph{may} need to be tracked in a particular GE implementation.  We
therefore need to be able to conditionally generate variables
representing the tracking state, and weave in corresponding tracking
code. We may need to (independently) keep track of the rank, the
determinant and the permutation list.  The tracking state variables
then become part of the \emph{state} that is tracked by our monad.  To
have all this choice when needed, and yet have our code be modular and
composable as well as ensuring that the generated code does not
contain any abstraction artifacts, it is important to make this state
modular.  For example,
\begin{code}
module type DETERMINANT = sig
  type indet  type outdet  type 'a lstate
  type tdet = outdet ref   
  val decl : unit -> 
    (unit, [> `TDet of 'a lstate ] list, ('a,'b) code) monad
  val upd_sign : unit -> 
    (('a,unit) code, [> `TDet of 'a lstate ] list, ('a,'b) code) monad
  ...
end
\end{code}
\noindent  determinant tracking aspects involves being able to
generate code that defines variables used for tracking (|decl|),
generate code that updates the sign or the absolute
value of the determinant, and finally, converting the tracking state
to the final determinant value of the type |outdet|. GE of a
floating-point matrix with no determinant tracking uses the
instantiation of |DETERMINANT| where |outdet| is |unit| and all the
functions of that module generate no code. For integer matrices, we
have to track some aspects of the determinant, even if we don't output
it. The determinant tracking aspect is complex because tracking
variables, if any, are to be declared at the beginning of GE; the sign
of the determinant has to be updated on each row or column
permutation; the value of the determinant should be updated per each
pivoting. We use |lstate| to pass the tracking state, e.g., a piece of
code for the value of the type |Dom.v ref|, among
various determinant-tracking functions. The |lstate| is a part of the
overall monadic state. Other aspects, e.g., rank tracking, may use the
monadic state for passing of rank tracking variables. To be able to
compose determinant and rank tracking functors -- each of which may
(or may not) use the monadic state for passing its own data -- we make
extensive use of open records (a list of polymorphic variants
appeared to be the easiest way to implement such a union, in a purely
functional way). This lets us freely compose determinant-tracking,
rank-tracking, and other aspects.

\begin{figure}
\begin{code}
module Gen(Dom: DOMAIN)(C: CONTAINER2D)(PivotF: PIVOT)
          (Update: UPDATE with type baseobj = Dom.v and type ctr = C(Dom).contr)
          (Out: OUTPUT with type contr = C(Dom).contr and type D.indet = Dom.v 
                        and type 'a D.lstate = 'a Update.D.lstate) = struct
    module Ctr = C(Dom)
    module Pivot = PivotF(Dom)(C)(Out.D)
    let gen =
      let zerobelow b r c m n brc =
        let innerbody i = mdo \{
            bic <-- Ctr.get b i c;
            whenM (l1 LogicCode.not (LogicCode.equal bic Dom.zero ))
                (seqM (retLoopM (Idx.succ c) (Idx.pred m)
                          (fun k -> Update.update b r c i k) )
                      (Ctr.set b i c Dom.zero)) \} in 
        mdo \{
              seqM (retLoopM (Idx.succ r) (Idx.pred n) innerbody) 
                   (Update.update_det brc) \} in
      let dogen a = mdo \{
          r <-- Out.R.decl ();
          c <-- retN (liftRef Idx.zero);
          b <-- retN (Ctr.mapper Dom.normalizerf (Ctr.copy a));
          m <-- retN (Ctr.dim1 a);
          n <-- retN (Ctr.dim2 a);
          () <-- Update.D.decl ();
          () <-- Out.P.decl ();
          seqM 
            (retWhileM (LogicCode.and_ (Idx.less (liftGet c) m)
                                       (Idx.less (liftGet r) n) )
               ( mdo \{
               rr <-- retN (liftGet r);
               cc <-- retN (liftGet c);
               pivot <-- l1 retN (Pivot.findpivot b rr m cc n);
               seqM (retMatchM pivot (fun pv -> 
                        seqM (zerobelow b rr cc m n pv)
                             (Out.R.succ ()) )
                        (Update.D.zero_sign () ))
                    (Code.update c Idx.succ) \} ))
            (Out.make_result b) \} in
    .<fun a -> .~(runM (dogen .<a>.)) >.
end
\end{code}
\end{figure}

The GE generator functor itself is 
parameterized by the domain, container, pivoting policy (full, row,
nonzero, no pivoting), update policy (that use either ``fraction-less'
or full division), and by what to yield as the result. Some of the
argument modules such as |PIVOT| are functors themselves (parameterized
by the domain, the container, and the determinant functor). The sharing
constraints express obvious constraints on the instantiation of |Gen|,
for example, pivoting, determinant etc. components all use the same
domain. It must be stressed that all structures (i.e., module
instances) are stateless, and so we never have to worry that different
aspect functors (such as |CONTAINER2D| and |PIVOT|) are instantiated
with different but type-compatible instances of |DOMAIN|. That is, we
are not concerned at all about the value sharing. Aspects such as
determinant tracking may be stateful so that the determinant update
code have access to the determinant tracking variables declared
previously. But that state is handled via the monadic state. As we
have shown, open unions makes the overall monadic state compositional
with respect to the state of various aspects.

In addition to the ``regular'' type sharing constraints shown in the
|Gen| functor, there are also ``semantic'' sharing constraints, shown
in the following structure of the |UPDATE| signature:
\vspace*{-5pt}\begin{code}
module DivisionUpdate
  (Dom:DOMAIN with type kind = domain_is_field)
  (C:CONTAINER2D)
  (Det:DETERMINANT with type indet=Dom.v) = struct ... end
\end{code}
\vspace*{-4pt} 
This structure implements an update policy of using
|Dom.div| operation without restrictions -- which is possible only if
the domain has such unrestricted operation. A domain such as the integer domain
may still provide |Dom.div| of the same type, but that operation may
only be used when we are sure that the division is exact. Our type
sharing constraint expresses such domain-specific knowledge:
instantiating |DivisionUpdate| with |IntegerDomain| leads to a
compile-time error, when compiling the \emph{generator} code. Thus, in
some cases we can use module types for ``semantic'' constraints
that cannot normally be expressed via the types of module members.
\vspace*{-5pt}
\begin{code}
module GenIV5 = Gen(IntegerDomain)
   (GenericVectorContainer)(FullPivot)
   (FractionFreeUpdate(IntegerDomain)(GenericVectorContainer)(IDet))
   (OutDetRank(IntegerDomain)(GenericVectorContainer)(IDet)(Rank))
module GenFA1 = Gen(FloatDomain)
   (GenericArrayContainer)(RowPivot)
   (DivisionUpdate(FloatDomain)(GenericArrayContainer)(NoDet(FloatDomain)))
   (OutJustMatrix(FloatDomain)(GenericArrayContainer)(NoDet(FloatDomain)))
\end{code}
\vspace*{-5pt}
We can instantiate the |Gen| functor as shown above and inspect the generated
code, e.g., by printing |GenFA1.gen|. The code can then be ``compiled'' as 
|!. GenFA1.gen| or with off-shoring. The code for |GenIV5| (Appendix A) shows
full pivoting, determinant and rank tracking. The code for all these aspects is
fully inlined; no extra functions are invoked and no tests other than those
needed by the GE algorithm itself are performed. The GE function returns a
triple |int array * int * int| of the U-factor, determinant and the rank. The
code generated by |GenFA1| (Appendix B) shows absolutely no traces of
determinant tracking: no declaration of spurious variables, no extra tests,
etc. The code appears as if the determinant tracking aspect did not exist
at all. The generated code for the above and other instantiations of
|Gen| can be examined at \cite{metamonadsURL}.

\section{Related and future work}\label{related}

The monad in this paper is similar to the one used in \cite{KiselyovTaha}.
However, the latter work used only |retN| of all monadic operations,
and fixpoints (for performing iterations at generation time).
This paper does not involve monadic fixpoints (because the generator
is not recursive) makes extensive use of monadic operations for
generating conditional and looping operations.

|Blitz++| \cite{Veldhuizen:1998:ISCOPE} and in general template
meta-programming in C++ achieve similar results of eliminating levels
of abstraction.  Using traits and concepts, some domain-specific
knowledge can also be encoded.  However overhead elimination
critically depends on the compiler's fully inlining of all methods,
which has been reported to be challenging to insure. Furthermore, all
errors (such as type errors and concept violation errors, i.e.,
composition errors) are detected only when compiling the generated
code. It is immensely difficult to correlate errors (e.g., line
numbers) to the ones in the generator itself.

ATLAS \cite{ATLAS} is another successful project in this area.  However
they use much simpler weaving technology, which leads them to note
that \emph{generator complexity tends to go up along with flexibility, 
so that these routines become almost insurmountable barriers to 
outside contribution}.  Our results show how to surmount this barrier,
by building modular, composable generators.
SPIRAL \cite{Pueschel:05} is another such even more ambitious project.
But SPIRAL does intentional code analysis, relying on a set of code
transformation ``rules'' which make sense, but which are not proved 
to be either complete or confluent.  The strength of both of these
project relies on their platform-specific optimizations performed
via search techniques, something we have not attempted here.

The highly parametric version of our Gaussian Elimination is directly
influenced by the generic implementations available in Axiom
\cite{Axiom} and Aldor \cite{Watt:2002:HCA}.  Even though the Aldor
compiler frequently can optimize away a lot of abstraction overhead, 
it does not provide any guarantees that it will do so, unlike our
approach.

To the best of our knowledge, nobody has yet used functors to
abstract code generators, or even mixed functors and 
multi-stage programming.

We plan to further investigate the connection between delimited
continuations and our implementations of code generators like
|ifM|.  As well, by using some additional syntactic sugar
(for |ifM|, |whileM|, etc.), the available notation should be
even more direct-style, and potentially clearer.

There are many more aspects which can also be handled:
Input variations (augmented
matrices), error reporting (i.e. asking for the determinant of a 
non-square matrix), memory hierarchy issues, loop-unrolling
\cite{Padua:MetaOcaml:04},
warnings when zero-testing is undecidable and
a value is only probabilistically non-zero, etc.  The larger program
family of LU decompositions contains more aspects still.

Finally, we would like to be able to make the |Gen| functor applicable
to runnable functors in Ocaml, as well as to code generation functors
in MetaOCaml - for ease of debugging.  The current use of abstract
types should make this straightforward.

\section{Conclusion}\label{conclusion}
In this paper we demonstrate numerical code extensively parameterized
by complex aspects at no run-time overhead.  The combination of
stateless functors and structures, and our monad with the
compositional state makes aspects freely composable without having to
worry about value aliasing. The only constraints to compositionality
are the typing constraints plus the constraints we specifically
impose, including semantic constraints (e.g., rings do not have full
division).

There is an interesting relation with aspect-oriented code
\cite{kiczales97aspectoriented}: in AspectJ, 
aspects are (comparatively) lightly typed, and are post-facto extensions of an
existing piece of code.  Here aspects are weaved together ``from scratch'' to
make up a piece of code/functionality.  One can understand previous work to be
more akin to dynamically typed aspect weaving, while we have started
investigating statically typed aspect weaving.

\bibliography{metamonads}
\bibliographystyle{plain}
\section{Appendix A}
The code generated for |GenIV5|, fraction-free GE of the integer matrix
represented by a flat vector, full pivoting, returning the |U|-factor,
the determinant and the rank.
\begin{code}
# val resIV5 :
  ('a,
   Funct4.GenIV5.Ctr.contr ->
   Funct4.OutDetRank(Funct4.IntegerDomain)(Funct4.GenericVectorContainer)
                    (Funct4.IDet)(Funct4.Rank).res)
  code =
  .<fun a_405 ->
   let t_406 = (ref 0) in
   let t_407 = (ref 0) in
   let t_408 = {arr = (Array.copy a_405.arr)} (a_405) in
   let t_409 = a_405.m in
   let t_410 = a_405.n in
   let t_411 = (ref 1) in
   let t_412 = (ref 1) in
   while (((! t_407) < t_409) && ((! t_406) < t_410)) do
    let t_413 = (! t_406) in
    let t_414 = (! t_407) in
    let t_415 = (ref (None)) in
    let t_435 =
     begin
      for j_431 = t_413 to (t_410 - 1) do
       for j_432 = t_414 to (t_409 - 1) do
        let t_433 = (t_408.arr).((j_431 * t_408.m) + j_432) in
        if (not (t_433 = 0)) then
         (match (! t_415) with
          | Some (i_434) ->
             if ((abs (snd i_434)) > (abs t_433)) then
              (t_415 := (Some ((j_431, j_432), t_433)))
             else ()
          | None -> (t_415 := (Some ((j_431, j_432), t_433))))
        else ()
       done
      done;
      (match (! t_415) with
       | Some (i_416) ->
          if ((snd (fst i_416)) <> t_414) then begin
           let a_424 = t_408.arr
           and nm_425 = (t_408.n * t_408.m)
           and m_426 = t_408.m in
           let rec loop_427 =
            fun i1_428 ->
             fun i2_429 ->
              if (i2_429 < nm_425) then
               let t_430 = a_424.(i1_428) in
               a_424.(i1_428) <- a_424.(i2_429);
               a_424.(i2_429) <- t_430;
               (loop_427 (i1_428 + m_426) (i2_429 + m_426))
              else () in
           (loop_427 t_414 (snd (fst i_416)));
           (t_412 := (~- (! t_412)))
          end else ();
          if ((fst (fst i_416)) <> t_413) then begin
           let a_417 = t_408.arr
           and n_418 = t_408.n
           and m_419 = t_408.m in
           let i1_420 = (t_413 * m_419)
           and i2_421 = ((fst (fst i_416)) * m_419) in
           for i_422 = 0 to (m_419 - 1) do
            let t_423 = a_417.(i1_420 + i_422) in
            a_417.(i1_420 + i_422) <- a_417.(i2_421 + i_422);
            a_417.(i2_421 + i_422) <- t_423
           done;
           (t_412 := (~- (! t_412)))
          end else ();
          (Some (snd i_416))
       | None -> (None))
     end in
    (match t_435 with
     | Some (i_436) ->
        begin
         for j_437 = (t_413 + 1) to (t_410 - 1) do
          if (not ((t_408.arr).((j_437 * t_408.m) + t_414) = 0)) then begin
           for j_438 = (t_414 + 1) to (t_409 - 1) do
            (t_408.arr).((j_437 * t_408.m) + j_438) <-
             ((((t_408.arr).((j_437 * t_408.m) + j_438) *
                 (t_408.arr).((t_413 * t_408.m) + t_414)) -
                ((t_408.arr).((t_413 * t_408.m) + j_438) *
                  (t_408.arr).((j_437 * t_408.m) + t_413))) / (! t_411))
           done;
           (t_408.arr).((j_437 * t_408.m) + t_414) <- 0
          end else ()
         done;
         (t_411 := i_436)
        end;
        (t_406 := ((! t_406) + 1))
     | None -> (t_412 := 0));
    (t_407 := ((! t_407) + 1))
   done;
   (t_408,
    if ((! t_412) = 0) then 0
    else if ((! t_412) = 1) then (! t_411)
    else (~- (! t_411)), (! t_406))>.
\end{code}
\section{Appendix B}
The code generated for |GenFA1|, GE of the floating point matrix
represented by a 2D array, row pivoting, returning just the |U|-factor.
\begin{code}
#   val resFA1 :
  ('a,
   Funct4.GenFA1.Ctr.contr ->
   Funct4.OutJustMatrix(Funct4.FloatDomain)(Funct4.GenericArrayContainer)
                       (Funct4.NoDet(Funct4.FloatDomain)).res)
  code =
  .<fun a_1 ->
   let t_2 = (ref 0) in
   let t_3 = (ref 0) in
   let t_5 = (Array.map (fun x_4 -> (Array.copy x_4)) (Array.copy a_1)) in
   let t_6 = (Array.length a_1.(0)) in
   let t_7 = (Array.length a_1) in
   while (((! t_3) < t_6) && ((! t_2) < t_7)) do
    let t_8 = (! t_2) in
    let t_9 = (! t_3) in
    let t_10 = (ref (None)) in
    let t_16 =
     begin
      for j_13 = t_8 to (t_7 - 1) do
       let t_14 = (t_5.(j_13)).(t_9) in
       if (not (t_14 = 0.)) then
        (match (! t_10) with
         | Some (i_15) ->
            if ((abs_float (snd i_15)) < (abs_float t_14)) then
             (t_10 := (Some (j_13, t_14)))
            else ()
         | None -> (t_10 := (Some (j_13, t_14))))
       else ()
      done;
      (match (! t_10) with
       | Some (i_11) ->
          if ((fst i_11) <> t_8) then begin
           let t_12 = t_5.(t_8) in
           t_5.(t_8) <- t_5.(fst i_11);
           t_5.(fst i_11) <- t_12;
           ()
          end else ();
          (Some (snd i_11))
       | None -> (None))
     end in
    (match t_16 with
     | Some (i_17) ->
        begin
         for j_18 = (t_8 + 1) to (t_7 - 1) do
          if (not ((t_5.(j_18)).(t_9) = 0.)) then begin
           for j_19 = (t_9 + 1) to (t_6 - 1) do
            (t_5.(j_18)).(j_19) <-
             ((t_5.(j_18)).(j_19) -.
               (((t_5.(j_18)).(t_9) /. (t_5.(t_8)).(t_9)) *.
                 (t_5.(t_8)).(j_19)))
           done;
           (t_5.(j_18)).(t_9) <- 0.
          end else ()
         done;
         ()
        end;
        (t_2 := ((! t_2) + 1))
     | None -> ());
    (t_3 := ((! t_3) + 1))
   done;
   t_5>.
\end{code}
\end{document}

%%%%%%%%%%%%%%%%%%%%%%%%%%%%%%%%%%%%%%%%%%%%%%%%%%%%%%%%%%%%%%%%%%%%%%
% all the text that used to be here is now in unused.tex
% same with any text in an \omitnow

