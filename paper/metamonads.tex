\documentclass[11pt]{llncs}

% Reduce display spacing
\divide\abovedisplayskip 2
\divide\belowdisplayskip 2
\divide\abovedisplayshortskip 2
\divide\belowdisplayshortskip 2

\usepackage{hyphenat}
\usepackage{comment}
\usepackage{amsmath}
\usepackage{amstext}
\usepackage{url}

\usepackage{ifpdf}
\ifpdf
    \pdfpageheight=11in
    \pdfpagewidth=8.5in
\fi

%\usepackage{refrange}

\usepackage{fancyvrb}
\DefineShortVerb{\|}
\DefineVerbatimEnvironment{code}{Verbatim}{xleftmargin=\mathindent}
\setlength{\parskip}{0pt}
\newlength{\mathindent}
\setlength{\mathindent}{1em}

% Reduce list spacing
%% \makeatletter
%% \renewcommand\@@listI{\leftmargin\leftmargini
%% \parsep \z@@
%% \topsep 3\p@@ \@@plus\p@@ \@@minus 2\p@@
%% \itemsep 2\p@@ \@@plus\p@@ \@@minus\p@@}
%% \let\@@listi\@@listI
%% \@@listi
%% \makeatother

\newcommand{\oleg}[1]{{\it [Oleg says: #1]}}

\begin{document}
\title{Functors, CPS and monads, or how to generate efficient
code from abstract designs}
\author{Jacques Carette\inst{1}
 \and
Oleg Kyselyov\inst{2}
}
\institute{McMaster University,
1280 Main St. West, Hamilton, Ontario Canada L8S 4K1
\and
Monterey, CA 93943}


\maketitle

\begin{abstract}
We use monads and Ocaml's advanced module system to
demonstrate that, in a generative context, it is
possible to totally eliminate the overhead of abstraction.
This lets one use extreme forms of \textit{information hiding}
at no run-time cost. Furthermore the typed nature of
the generative context provide static guarantees about the
generated code. The various extensions
(aspects) can be made completely orthogonal and compositional,
even in the presence of name-generation for temporaries and
other bindings and ``interleaving'' of aspects. We also
show how to encode some domain-specific knowledge so that
``clearly wrong'' compositions can be statically rejected by the
compiler.
\end{abstract}

\section{Introduction}

In high-performance and numeric computing, there is a well-known
issue of balancing between maximal performance and the level of
abstraction at which code is written.  Furthermore, already in
linear algebra, there is a wealth of different concerns that 
\emph{may} need to be addressed.  While it is possible to manually
inline and simplify such code, this is labour-intensive and 
error-prone.  Furthermore the resulting code is frequently unreadable,
and thus unmaintainable.

There are a number of projects that have attempted to address these issues
\cite{Czarnecki,Veldhuizen:1998:ISCOPE,Veldhuizen:1998:OO98,musser94algorithmoriented,BOOST,POOMA,ATLAS}, 
and they have generally succeeded in generating
efficient code, but the cost was frequently the loss of modularity
of the design.  In other words, to allow the various aspects to
weave with each other \emph{efficiently}, information which should
have been kept hidden leaked from module to module.  This seriously
affects maintainability and scalability.  Furthermore, current architectures
demand more and more frequent tweaks which, in general, cannot be done by the
compiler because the tweaking often involves domain knowledge.  

Our goal was to implement a completely modular, textbook-style 
modular design \cite{journals/cacm/parnas72a}, yet 
have our code be very efficient.  Gaussian elimination, as it is 
well-known, already contains many different aspects \cite{carette04},
was chosen as a representative example.

Whether one prefers to code in an imperative, object-oriented, functional, 
or even in a mixed style, all abstraction mechanisms in current
languages have a non-zero run-time cost.  Whether it is procedure,
method or function call, these have a cost; worse still, higher level
and/or higher order abstractions (like Functors or AspectJ's aspects)
frequently have a larger cost still.  Of course, if one is faced with
writing 60 different implementations of the ``same'' algorithm, this
abstraction cost starts to look cheap!  Although bad programmers have
been known to resort to cut\&paste programming, no self-respecting
programmer would be caught doing that\footnote{Sadly, there are still
many programmers who do not see the harm of cut\&paste programming}.
The common solution is to turn to code generation.

While many people still make do with string-based code generation,
this is not much different than writing code in a dynamically typed
interpreted scripting language.  It works, it is convenient, but
it does not scale well, it is a debugging nightmare, and usually
suffers most from Software Aging \cite{parnas_aging}.  As with
other tasks in programming, the answer is to turn to statically
typed languages, which provide strong guarantees about the soundness
of the resulting code.  MetaOCaml in particular stands out: it
guarantees that both the code generator and the generated
code are well-typed.  While Template programming in C++ and 
Template Haskell do offer some amount of static typing 
\cite{conf/dagstuhl/CzarneckiOST03}, MetaOCaml\footnote{and MetaML.}
offers stronger guarantees.  As well, since MetaOCaml is implemented
on top of the very solid Ocaml language, one accrues additional benefits.

While it is theoretically feasible in a strings-based
generative system to re-parse the generative code to inspect it, such
an approach rapidly degenerates into complete chaos.  In languages
that natively offer reflection and reification (either at no cost, 
such as in LISP and Scheme, or at a modest cost in languages like 
Maple), this is actually feasible, even easy.  But this is fragile
and error-prone, since the languages offer little support for catching
errors early.

In MetaOCaml, generated code cannot be examined and optimized at all. It must
be generated just right.  

- The FFT paper
showed one way: abstract interpretation. But more problems remain
(cite Padua et al paper at Haskell Workshop04): generating names in
the presence of `if', 'while' etc. control operators (simple problem
of generating names solved in the FFT paper); making CPS code clear
(note that many authors are afraid of CPS code because it quickly
becomes unreadable; we need CPS for name generation); compositionality
of various (often interleaving: explain: determinant computation
requires attention at several places) aspects and expressing
dependencies among them; expressing more static guarantees about the
generated code (e.g., not attempting full (fractional) division for
the integer domain; generating full-pivoting code for the rational
domain without |better_than| operation). In short: can we achieve all
of the abstractions needed for the flexible Gaussian Elimination? Can
we eliminate all of that abstraction's overhead in the generated code
(without using intensional code analysis -- that is, maintaining the
guarantees of MetaOCaml and using pure generative approach).

[Note why it is important to maintain generativity: stronger
equational theory: Walid's paper. Also, intensional code analysis
(e.g., SPIRAL project) essentially requires one to insert an
optimizing compiler into the code generating system. That is extremely
complex, quite error-prone, and very difficult to ascertain the
correctness.]

\section{CPS and the problem of generating names}

In this paper, we are concerned with building and composing
generators. We start with the code expressing the algorithm (in
OCaml), and, adding MetaOCaml annotations, turn that code into a
\emph{code expression} that will generate the code when runs. We then
modify that code expression to add various parameters and aspects, to
yield the complete generator.

Here we illustrate the MetaOCaml constrcutions and code generators.

Turning code into the code expression is relatively straightforward
until we come to binding constructs such as |let|. For example, here
is a piece of code that defines a function that searches a given array
|arr| for the maximal (in absolute value) element, and the index of
that element and the found absolute value. This code is representative
of pivot determination. The code uses mutations and destructive
updates -- partly for illustrative purposes and mainly because many
linear algebra algorithms are expressed via an imperative code (so it
is less error-prone to translate the ``canonical'' algorithms with
minimal changes).

\begin{code}
let fp = fun arr n ->
   let pv = ref (0,arr.(0)) in
   for i = 1 to n-1 do
     if abs_float arr.(i) > snd !pv then
        pv := (i, abs_float arr.(i))
   done;
   !pv
\end{code}

This function has the type |float array -> int -> int * float|. By
enclosing the code into MetaOCaml brackets, we obtain the simplest
generator expression.

\oleg{later on, we can write the following just let fpc = .<...>.}

\begin{code}
let fpc = .<fun arr n ->
   let pv = ref (0,arr.(0)) in
   for i = 1 to n-1 do
     if abs_float arr.(i) > snd !pv then
        pv := (i, abs_float arr.(i))
   done;
   !pv>.
\end{code}

This value has the type |('a, float array -> int -> int * float) code|.
Here |'a| type parameter is needed (ask Walid for the reference).
The main difference between |fp| and |fpc| is that |fp| is a function,
which we can immediately use as \verb^fp [|1.0; -4.; 3.|] 3^. In
contrast, |fpc| is not a function: it is a \emph{code} value that denotes
that function. We would say that the code expression is the
``second-stage'' value. The ability to manipulate code expressions as
regular values is the distinguished fature of MetaOCaml.
We can ``compile'' the code by using MetaOCaml |run| command, or |.!|:
|.! fpc|. The result is equivalent is |fp|. Although |fpc| doesn't
seem to do much, it is already useful as we can use offshoring (cite)
to translate |fpc| into C or Fortran code (which we can dynamically
link into the running MetaOCaml code -- or use separately).

The code |fpc| obviously lacks generality: the name |abs_float|
implies that that code deals with |float| arrays. We wish to handle
arrays in various domains, integer, rationals, etc. We can make the
code generic by parametrizing over the |abs| function:

\begin{code}
let fpca absf = .<fun arr n ->
   let pv = ref (0,arr.(0)) in
   for i = 1 to n-1 do
     if absf arr.(i) > snd !pv then
        pv := (i, absf arr.(i))
   done;
   !pv>.
\end{code}

and so |fpca abs_float| will manipulate |float array| and |fpca abs|
will do for the integer arrays. Here we used the value from the first
stage, |absf|, to incorporate into the code at the second stage:
croos-stage persistent values.

We should now consider the function |fpca| in a browder context: of a
function that computes the pivot and then swaps the pivot withe 0-th
element of the array:

\begin{code}
let upv = .<fun arr n ->
   let pv = .~(fpca abs_float) arr n in
   let i = fst pv in
   (if not (i = 0) then
      let t = arr.(0) in
      begin arr.(0) <- arr.(i); arr.(i) <- t end);
   arr>.
\end{code}

Here, we see the third MetaOCaml operator, escape |.~|. It does ...
MetaOcaml can print out the code, which looks like
|.<fun arr n -> let pv = (fun arr n -> fpcs code) arr n in ... >.|
That is, the compiler inlined the code for fpca -- which is a
functional expression. The |upv| code shows the functional
application, |(fun arr n -> ...)| applied to arguments |arr n|. That
application will be executed at run-time, unless a sufficiently smart
compiler can inline it. As we stated, our aim is to not to rely on
post-generation optimizations and sufficiently smart compiler. We
strive to generate the efficient code ourselves. We can do that by
modifying the definition of |fpca| as

\begin{code}
let fpca absf arr n = .<
   let pv = ref (0,(.~arr).(0)) in
   for i = 1 to .~n-1 do
     if absf (.~arr).(i) > snd !pv then
        pv := (i, absf (.~arr).(i))
   done;
   !pv>.
\end{code}
\begin{code}
let upv = .<fun arr n ->
   let pv = .~(fpca abs_float .<arr>. .<n>.) in
   let i = fst pv in
   (if not (i = 0) then
      let t = arr.(0) in
      begin arr.(0) <- arr.(i); arr.(i) <- t end);
   arr>.
\end{code}
% (.! upv) [|1.0; -4.; 2.|] 3;;
We see that we pass code fragment |.<arr>.| as arguments to the
function. Now when we look at the code for |upv|, we see
|.<fun arr n -> let pv1 = let pv = ref (0,arr.(0)) in ...>.|
So the fpca code got really inlined.

The inlining is not perfect however. We would like that our code
expression to look like
\begin{code}
 fun arr n ->
   let pv = ref (0,arr.(0)) in ... the rest of fpca ...
   in let i = fst !pv in ...
\end{code}

We would like that the inlining of |.~(fpca abs_float .<arr>. .<n>.)|
generated us the |let pv = ref ... | statement. But this seems to
quite a challenge. We wish the resulting code to look
|let pv = ... in body| where |pv = ...| part come from |fpca| and
|body| come from |upv|. At first blush, this seems impossible. A code
expression may contain only complete expressions, such as
|let v =... in ...|. A code expression cannot contain only the binding part of
|let|, which is not an expression per se.

The solution: CPS. And here where I go to sleep.



\section{Monadic notation, making CPS code clear}

What is our monad: what is the action and what is the value.
ifM and ifL: show example. Mention 'ifM' and `reset'. Investigating
the connection with the delimited continuations: for future work.

What is state and why we need it: making aspects interleave: use det
as an example.

\section{Functors}

Simple parameterization: by the domain, by the container
 
Functors and the generated return type: the Out structure.

Making state modular and composable: several aspects can contribute
to the state, without interference (show: Rank, Det, Permutation
aspects).

Expressing domain constraints in functors (Div and integer domain).
Show the type error that occurs in improper instantiation.

Show several examples (e.g., code with NullPivot and nodet: no traces
of either det nor pivoting). Then the code with RowPivoting, Det and
float domain.

\section{Related work}

\section{Future work}
connections with delimited continuations: making notation
more direct-style and potentially clearer.

More aspects that can be handled: Input variations (augmented
matrices). Fit into the larger program family, where there are
more aspects still.

\section{Conclusion}

\bibliography{metamonads}
\bibliographystyle{plain}

\end{document}
