\documentclass{llncs}

% Reduce display spacing
\divide\abovedisplayskip 2
\divide\belowdisplayskip 2
\divide\abovedisplayshortskip 2
\divide\belowdisplayshortskip 2

\usepackage{hyphenat}
\usepackage{comment}
\usepackage{amsmath,amssymb}
\usepackage{amstext}
\usepackage{url}
\usepackage[dvips]{color}


\usepackage{ifpdf}
\ifpdf
    \pdfpageheight=11in
    \pdfpagewidth=8.5in
\fi

%\usepackage{refrange}

\usepackage[medium,compact]{titlesec}

\usepackage{fancyvrb}
\DefineShortVerb{\|}
\DefineVerbatimEnvironment{code}{Verbatim}{xleftmargin=\mathindent,commandchars=\\\{\},fontsize=\small}
\newcommand{\evalresult}[1]{\ensuremath{\Longrightarrow}\textcolor{red}{#1}}
% \setlength{\parskip}{0pt}
\newlength{\mathindent}
\setlength{\mathindent}{1em}

% \usepackage[backref,colorlinks,bookmarks=true]{hyperref}

% Reduce list spacing
%% \makeatletter
%% \renewcommand\@@listI{\leftmargin\leftmargini
%% \parsep \z@@
%% \topsep 3\p@@ \@@plus\p@@ \@@minus 2\p@@
%% \itemsep 2\p@@ \@@plus\p@@ \@@minus\p@@}
%% \let\@@listi\@@listI
%% \@@listi
%% \makeatother

\parskip 0pt plus 1pt minus 2pt
\textfloatsep 4pt plus 2pt minus 3pt % Less space around figures
\abovecaptionskip 0pt plus 0pt minus 2pt
\belowcaptionskip 0pt plus 0pt minus 2pt

\intextsep 2pt plus 0pt minus 1pt

\renewcommand\floatpagefraction{.9}
\renewcommand\topfraction{.9}
\renewcommand\bottomfraction{.9}
\renewcommand\textfraction{.1}   
\setcounter{totalnumber}{50}
\setcounter{topnumber}{50}
\setcounter{bottomnumber}{50}

\newcommand{\omitnow}[1]{}
\newcommand{\oleg}[1]{{\it [Oleg says: #1]}}
\newcommand{\jacques}[1]{{\it [Jacques says: #1]}}

\begin{document}
\title{Functors, CPS and monads, or how to generate efficient
code from abstract designs}
\author{Jacques Carette\inst{1}
 \and
Oleg Kyselyov\inst{2}
}
\institute{McMaster University,
1280 Main St. West, Hamilton, Ontario Canada L8S 4K1
\and
Monterey, CA 93943}

\maketitle

\begin{abstract}
With Gaussian Elimination as a representative family of numerical
and symbolic
algorithms, we use multi-stage programming, monads and Ocaml's
advanced module system to demonstrate the complete elimination of the
abstraction overhead while avoiding any inspection of the generated
code.  We parameterize our Gaussian Elimination code to a great extent
(over domain, matrix representations, determinant tracking, 
pivoting policies, result types, etc) at no run-time cost.  Because
the resulting code is generated just right and not changed afterwards,
we enjoy MetaOCaml's guaranty that the generated code is well-typed.
We further demonstrate that various abstraction parameters (aspects)
can be made orthogonal and compositional, even in the presence of
name-generation for temporaries and other bindings and
``interleaving'' of aspects.  We also show how to encode some
domain-specific knowledge so that ``clearly wrong'' compositions can
be statically rejected by the compiler when processing the generator
rather than the generated code.
\end{abstract}

\section{Introduction}

In high-performance, symbolic, and numeric computing, there is a well-known
issue of balancing between maximal performance and the level of
abstraction at which code is written.  Furthermore, already in
linear algebra, there is a wealth of different concerns that 
\emph{may} need to be addressed.  While it is possible to manually
inline and simplify such code, this is labour-intensive and 
error-prone.  The resulting code is frequently unreadable,
and thus unmaintainable.

There are a number of projects that have attempted to address these issues
\cite{Czarnecki,Veldhuizen:1998:ISCOPE,musser94algorithmoriented,BOOST,POOMA,ATLAS}, 
and they have generally succeeded in generating
efficient code, but the cost was frequently the loss of modularity
of the design.  In other words, to allow the various aspects to
weave with each other \emph{efficiently}, information which should
have been kept hidden leaked from module to module.  This seriously
affects maintainability and scalability.  Furthermore, current architectures
demand more and more frequent tweaks which, in general, cannot be done by the
compiler because the tweaking often involves domain knowledge.  

Our goal was to implement a textbook-style modular design 
\cite{journals/cacm/parnas72a}, yet 
have our code be very efficient.  Gaussian elimination, as a  
well-known algorithm with many different aspects 
\cite{carette04}, was chosen as a representative example.

Whether one prefers to code in an imperative, object-oriented, functional, 
or even in a mixed style, all abstraction mechanisms in current
languages have a non-zero run-time cost.  Whether it is procedure,
method or function call, these have a cost; worse still, higher level
and/or higher order abstractions (like Functors or AspectJ's aspects)
frequently have a larger cost still.  Of course, if one is faced with
writing 35 different implementations of the ``same'' algorithm
\cite{carette04}, this
abstraction cost starts to look cheap!  Although bad programmers have
been known to resort to cut\&paste programming, no self-respecting
programmer would be caught doing that.
The common solution is to turn to code generation.

While many people still make do with string-based code generation,
this is not much different than writing code in a dynamically typed
interpreted scripting language.  It works, it is convenient, but it
does not scale well, it is a debugging nightmare, and usually suffers
most from Software Aging \cite{parnas_aging}.  As with other tasks in
programming, the answer is to turn to statically typed languages,
which provide strong guarantees about the soundness of the resulting
code.  MetaOCaml \cite{CTHL03,metaocaml-org} in particular stands
out: not only it guarantees well-formedness and the absense of problems 
such as accidental variable capture \cite{HygienicMacros}. MetaOcaml also
statically ensures that both the code generator and the generated code
are well-typed \cite{TahaSheard97,TahaThesis}.  While Template
programming in C++ and Template Haskell do offer some amount of static
typing \cite{conf/dagstuhl/CzarneckiOST03}, MetaOCaml
offers stronger guarantees.  As well, since MetaOCaml is
implemented on top of the very solid Ocaml language, one accrues
additional benefits.

While it is theoretically feasible in a strings-based
generative system to re-parse the generative code to inspect it, such
an approach rapidly degenerates into complete chaos.
% \oleg{is there a good citation?}
% \jacquesP{none that I am aware of}
In languages
that natively offer reflection and reification (either at no cost, 
such as in LISP and Scheme, or at a modest cost in languages like 
Maple), inspection is actually feasible, even easy.  But this is fragile
and error-prone, since the languages offer little support for catching
errors early.

In MetaOCaml, generated code cannot be examined and optimized at all. It must
be generated just right.  For simple code, this is relatively easy
(see \cite{TahaThesis} for many examples).  For more complex examples,
new techniques are necessary.  \cite{KiselyovTaha} introduces one
such tool: abstract interpretation.  But more problems remain
\cite{Padua:MetaOcaml:04}:
\begin{itemize}
    \item Generating names in the presence of ``if'', ``while'', and other
        control operators. \cite{KiselyovTaha} showed how to solve
        this problem in simple cases.
    \item Making continuation-passing style (CPS) code clear.  Many
        authors understandably shy away from CPS code as it quickly
        becomes unreadable.  But this is needed for proper name
        generation.
    \item Compositionality of various interleaving aspects, as well
        as expressing dependencies among them.  For example, if we
        choose to compute the determinant of a matrix while performing
        Gaussian Elimination, this requires inserting code in 
        several specific places in the algorithm.  The method to
        compute this determinant crucially depends on whether 
        a division-based or a fraction-free Gaussian Elimination is
        used.
    \item As mentionned above, there are dependencies between
        certain algorithmic choices and properties of the
        underlying domain, and we would like to statically guarantee
        that inappropriate combinations cannot be chosen.  For instance,
        one should not attempt to use full division when dealing
        with matrices of exact integers, nor is it worthwhile to use
        full pivoting on a matrix of rational integers ($\mathbb Q $).
\end{itemize}
The questions we want to answer is:
Can we achieve all of the abstractions needed for a flexible 
Gaussian Elimination, and can
we eliminate all of that abstraction's overhead in the generated code?
Of course, we want to do this without using
intensional code analysis -- that is, maintaining the
guarantees of MetaOCaml and using a purely generative approach.

Maintaining generativity is not just an interesting exercise:
one gets a stronger equational theory \cite{Taha2000}, and avoids
the danger of creating unsoundness \cite{TahaThesis}.  It should
also be remarked that in general, intensional code analysis
essentially requires one to insert both an optimizing compiler and 
an automated theorem proving system into the code generating system
\cite{Pueschel:05,Kennedy01Telescoping,dongarra7,Veldhuizen:2004}.
While this is
potentially extremely powerful and an exciting area of research,
it is also extremely complex, which means that it is currently more
error-prone and difficult to ascertain the correctness of the 
resulting code.

This work makes the following contributions:
\vspace*{-2pt}
\begin{itemize}
    \item use of generative functors for meta-programming
    \item orderly aspect weaving
    \item mdo notation for greater monadic code clarity
    \item use of monads for meta-programming
    \item use of type constraints to encode domain knowledge and 
		ensure correctness
\end{itemize}

The rest of this paper is structured as follows:
The next section introduces code generation in MetaOCaml, the problem
of name generation, and how continuation-passing style is a general
tool for solving that problem.  We also introduce
the monad that greatly simplifies the task of dealing with
possibly effect-inducing CPS code combinators.  Section~\ref{functors}
shows how one can use the advanced module system of OCaml, namely its
parametrized modules, to encode all of the aspects of the Gaussian
Elimination algorithm family in completely separate, independent modules.
We then briefly discuss related work in section~\ref{related}, outline
some work still to do in section~\ref{future}, and finally draw some 
conclusions in section~\ref{conclusion}.  As an appendix, we show
the details of the syntax extension for monadic programming in Ocaml
and MetaOCaml.

The first author wishes to thank Cristiano Calgano for his help in
adapting camlp4 for use with MetaOCaml.

\section{The problem of generating names, CPS, and monad}\label{CPS}
%%%%%%%%%%%%%%%%%%%%%%%%%%%%%%%%%%%%%%%%%%%%%%%
%% And now, a better -- and shorter -- write-up.

The essence of our approach is assembling the code of the program by
combining instances of various aspects together. In the case of
Gaussian Elimination, the algorithmic aspects are pivot determination, 
swapping of
rows and columns, determinant tracking, etc. A primitive code
generator yields a code expression.  A combinator takes (complete and
well-typed) code expressions and combines them into a composite expression
(which is also statically guaranteed to be well-formed and well-typed).

We will be using MetaOcaml which, as an instance of a multi-stage
programming system \cite{TahaThesis}, provides exactly the needed
features: to construct a code expression, to combine them, and to
execute them. Figure~\ref{easycode} shows the simplest code generator |one|,
as well as more complex generators.

\begin{figure}\label{easycode}
\begin{code}
let one = .<1>. and plus x y = .<.~x + .~y>.
let simplest_code = let gen x y = plus x (plus y one) in
  .<fun x y -> .~(gen .<x>. .<y>.)>.
\evalresult{.<fun x_1 -> fun y_2 -> (x_1 + (y_2 + 1))>.}
let simplest_param_code plus one = let gen x y = plus x (plus y one) in
  .<fun x y -> .~(gen .<x>. .<y>.)>.
let plus x y = .<.~x +. .~y>. and one = .<1.0>. in
  simplest_param_code plus one
let param_code1 plus one =
  let gen x y = plus (plus y one) (plus x (plus y one)) in
  .<fun x y -> .~(gen .<x>. .<y>.)>.
let param_code1' plus one =
  let gen x y = let ce = (plus y one) in  plus ce (plus x ce) in
  .<fun x y -> .~(gen .<x>. .<y>.)>.
param_code1' plus one
\evalresult{.<fun x_1 -> fun y_2 -> ((y_2 + 1) + (x_1 + (y_2 + 1)))>.}
\end{code}
\caption{Code generation and combinators. $\Longrightarrow$ shows the
  result of evaluating the preceeding expression}
\end{figure}

We use MetaOCaml brackets |.<...>.| to generate code expressions -- in
other words, to construct future-stage computations. We use escapes
|.~| to perform an immediate code generating computation while we are
building the future-stage computation (code expression). The immediate
computation in |simplest_code| includes the evaluation of the function
|gen|, which in turn applies the function |plus|. The function
receives code expressions |.<x>.| and |.<y>.| as arguments. At the
generating stage, we can manipulate code expressions as (opaque)
values. The function |gen| returns a code expression, which will be
inlined in the place of the escape. MetaOCaml can print out code
expressions, so we can see the final generated code, It has
no traces of the functions |gen| and |plus|: their applications are
done at the code generation stage. 

The final MetaOCaml feature, |.!| (pronounced ``run'') 
executes the code expression: |.! simplest_code| is a function of two
integers, which we can apply: |(.! simplest_code) 1 2|. The original
|simplest_code| is not the function on integers -- it is a code
expression.

To see the benefit of code generation, we notice that we can easily
parameterize our code, |simplest_param_code|, and use it to generate
code that operates on integers, floating point numbers or booleans --
in general, any domain that implements |plus| and |one|.

A more complex expression |param_code1| has two occurrences of 
|plus y one|. Depending on the implementation of |plus|, this may be quite a
complex computation, and so we would rather not do it twice. We may be
tempted to rely on the compiler's common-subexpression elimination
optimization. We would rather not rely on a ``sufficiently smart
compiler'': when the generated code is very complex, the compiler may
overlook common subexpressions.  Or the subexpressions may occur in an
imperative context where the compiler might not be able to determine
if lifting them is sound. So, being conservative, the optimizer will
leave the duplicates as they are. We may attempt to eliminate
subexpressions as in |param_code1'|. However, the result of
|param_code1' plus one| still exhibits duplicate sub-expressions. 
Our |let|-insertion
optimization saved the computation at the generating stage. 
We need a combinator that inserts the |let| expression in the
generat\emph{ed} code. We need a combinator |letgen| to be used
as\\|let ce = letgen (plus y one) in plus ce (plus x ce)| 
yielding the code like |.<let t = y + 1 in t + (x + t)>.|
But that seems impossible because |letgen exp| has to generate
the expression |.<let t = exp in body>.| although |letgen| does not have
the |body| yet. The body needs a temporary identifier |.<t>.| that
is supposed to be the result of |letgen| itself. 
Certainly |letgen| cannot generate only part of a let-expression,
without the |body|,  as all generated expressions in
MetaOCaml are well-formed and complete.

The key is to use continuation-passing style (CPS). Its benefits were first
pointed out by \cite{Bondorf:92} in the context
of partial evaluation, and extensively used by \cite{KiselyovTaha} for code
generation. Now, |param_code2 plus one| gives us the desired code. 

\begin{code}
let letgen exp k = .<let t = .~exp in .~(k .<t>.)>.
let param_code2 plus one =
  let gen x y k = letgen (plus y one) (fun ce -> k (plus ce (plus x ce)))
  and k0 x = x
  in .<fun x y -> .~(gen .<x>. .<y>. k0)>.
param_code2 plus one
\evalresult{.<fun x_1 -> fun y_2 -> let t_3 = (y_2 + 1) in (t_3 + (x_1 + t_3))>.}
\end{code}

\noindent Comparing the code that did let-insertion at the generating stage\\
|let ce = (plus y one) in  plus ce (plus x ce)|\\
with the corresponding code inserting let at the generated code stage\\
|letgen (plus y one) (fun ce -> k (plus ce (plus x ce)))|\\
clearly shows the difference between the direct-style and CPS code.
What was |let ce = init in ...| in direct style became
|init' (fun ce -> ...)| in CPS. For one thing, |let| became
inverted. For another, what used to be an expression that yields
a value, |init|, now became an expression that takes an extra argument,
the continuation, and invokes it. The differences look negligible in
the above example. In larger expressions with many let-forms, the
number of parentheses around |fun| increases, the need to add and
then invoke the |k| continuation argument become increasingly annoying. The
inconvenience is great enough for some people to explicitly avoid CPS
or claim that numerical programmers (our users) cannot or will not
program in CPS. Clearly a better notation is needed.

The |do|-notation of Haskell \cite{Haskell98Report} shows that it is possible
to write CPS code in conventional-looking style. The
|do| notation is the notation for writing monadic code \cite{moggi-notions}.
The benefit of monadic notation is not that it can represent CPS \cite{Filinski:Representing}, but it helps in composability by offering to add different
layers of effects (state, exception, non-determinism, etc) to the
basic monad \cite{liang-interpreter} in a controlled way.

A monad \cite{moggi-notions} is an abstract datatype representing
computations that yield a value and may have an \emph{effect}.
The datatype must have at least two operations, |return| to build
trivial effect-less computations and |bind| for combining
computations. These operations must satisfy \emph{monadic laws}:
|return| being the left and the right unit of |bind| and |bind| being
associative. Figure~\ref{ourmonad} defines the monad used throughout
the present paper and shows its implementation.

\begin{figure}\label{ourmonad}
\begin{code}
type ('v,'s,'w) monad = 's -> ('s -> 'v -> 'w) -> 'w
let ret a = fun s k -> k s a
let bind a f = fun s k -> a s (fun s' b -> f b s' k)
let fetch s k = k s s
let store v s k = k (v::s) ()

let k0 s v = v
let runM m = m [] k0

let l1 f = fun x -> mdo { t <-- x; f t}
let l2 f = fun x y -> mdo { tx <-- x; ty <-- y; f tx ty}

let retN a = fun s k -> .<let t = .~a in .~(k s .<t>.)>.

let ifL test th el = ret .< if .~test then .~th else .~el >.
let ifM test th el = fun s k ->
  k s .< if .~(test s k0) then .~(th s k0) else .~(el s k0) >.
\end{code}
\caption{Our monad}
\end{figure}

Our monad represents two kinds of computational effects: reading and
writing a computation-wide state, and control effects. The latter are
normally associated with exceptions, forking of computations, etc. --
in general, whenever a computation ends with something other than
invoking its natural continuation in the tail position. In our case
the control effects manifest themselves as code generation.

In Figure~\ref{ourmonad}, the monad is implemented as a function of two
arguments: the state (of type |s|) and the continuation. The
continuation receives the current state, the value (of the type |v|) and
yields the answer of the type |w|.  The monad is polymorphic over all
these three type parameters.  Other implementations are
possible. Except for the code in Figure~\ref{ourmonad}, the rest of our code
treats the monad as a truly abstract data type. The implementation of basic
monadic operations |ret| and |bind| are conventional. It is easy to
see that the monadic laws are satisfied.  Other monadic operations
construct computations that do have effects.  Operations |fetch| and
|store v| construct computations that read and write the state. In our
case the state is a list (of polymorphic variants), which models an
open discriminated union, as we shall see later.

The operation |retN a| is the let-insertion operation, whose simpler
version we called |letgen| earlier. It is the first computation with
a control effect: indeed, the result of |retN a| is \emph{not} the
result of invoking its continuation |k|. Rather, its result is a |let|
code expression. Such a behavior is symptomatic of control operators
(in particular, |abort|).

Finally, |runM| runs our monad, that is, performs the computation of
the monad and gets the result, which in our case is the code
expression. We run the monad by passing it the initial state and the
initial continuation |k0|. We can now re-write our |param_code2|
example of the previous section as |param_code3|.

\begin{figure}
\begin{code}
let param_code3 plus one =
  let gen x y = bind (retN (plus y one)) (fun ce -> 
                ret (plus ce (plus x ce)))
  in .<fun x y -> .~(runM (gen .<x>. .<y>.))>.
let param_code4 plus one =
  let gen x y = mdo \{ ce <-- retN (plus y one);
                      ret (plus ce (plus x ce)) \}
  in .<fun x y -> .~(runM (gen .<x>. .<y>.))>.
let ifM' test th el = mdo \{
  testc <-- test; thc <-- th; elc <-- el;
  ifL testc thc elc\}
let gen a i = ifM' (ret .<(.~i) >= 0>.) 
                   (retN .<Some (.~a).(.~i)>.) (ret .<None>.)
 in .<fun a i -> .~(runM (gen .<a>. .<i>.))>.
\evalresult{.<fun a_1 i_2 ->  
let t_3 = (Some (a_1.(i_2))) in if (i_2 >= 0) then t_3 else (None)>.}
let gen a i = ifM (ret .<(.~i) >= 0>.) 
                  (retN .<Some (.~a).(.~i)>.) (ret .<None>.)
 in .<fun a i -> .~(runM (gen .<a>. .<i>.))>.
\evalresult{.<fun a_1 i_2 -> 
if (i_2 >= 0) then let t_3 = (Some (a_1.(i_2))) in t_3 else (None)>.}
\end{code}
\end{figure}
% param_code3 plus one;;

That does not seem like much of an improvement. With the help of
camplp4 pre-processor, we introduce the do notation (see
Appendix A), patterned after the
do-notation of Haskell. The function |param_code4|, written in
the do-notation, is equivalent to |param_code3| -- in fact, the camlp4
preprocessor will convert the former into the latter. And yet,
|param_code4| looks far more conventional, as if it were indeed in
direct style.

We can write operations that generate code other than let-statements,
e.g., conditionals: see |ifL| in Figure~\ref{ourmonad}. The function |ifL|, 
albeit straightforward, is not as general as we wish: its argument are
already generated pieces of code rather than monadic values. So, we
may want to ``lift it'', see |ifM'|. We can define functions |l1|,
|l2|, |l3| (analogues of |liftM|, |liftM2|, |liftM3| of Haskell) 
to make such a lifting generic.
We soon notice the need for another |ifM| function, with the same
interface (see Figure~\ref{ourmonad}). The difference between them is
apparent: in the code above with |ifM'|, the let-insertion
happened \emph{before} the if-expression, that is, before the test that
the index |i| is positive. If |i| turned out
negative, |a.(i)| would generate an out-of-bound array access
error. On the other hand, the code with |ifM| accesses the array only
when we have verified that the index is non-negative. This example
makes it clear that the code generation (such as the one in |retN|) is 
truly an effect and we have to be clear about the sequencing of
effects when generating control constructions such as conditionals.
The form |ifM| handles such effects correctly. We
need similar operators for other Ocaml control forms: for generating
case-matching statements and for- and while-loops.

\section{Functors}\label{functors}

Now that we know that we have the appropriate basic techniques for
writing our code generators, we need to worry about modularizing them
appropriately.  Here we can really take advantage of the fact that
MetaOCaml is a staged language: it is not really relevant how much
computation happens at staging time, as long as the generated code
is correct and efficient.  In other words, we can use any abstraction
mechanisms we want to structure our code generators, as long as none
of those abstractions infiltrate the generated code.

While the Object-Oriented Design community has acquired an extensive
vocabulary for describing modularity ideas, the guiding principles for
modular designs has not changed since they were first articulated by
Parnas~\cite{journals/cacm/parnas72a} and Dijkstra~
\cite{EWD:EWD447}: information hiding and separation of concerns.  To
apply these principles to the study of Gaussian Elimination, we need
to understand what changes between different implementations, and 
what concerns need to be addressed.  We also need to study the degree
to which these concerns are independent.

A careful study of Gaussian Elimination \cite{carette04} shows that
(at least) the following variations occur:
\begin{enumerate}
	\item \textbf{Domain}: In which (algebraic) domain do the matrix
		elements belong to.  Sometimes the domains are very specific
		($\mathbb{Z}, \mathbb{Q}, \mathbb{Z}_p, 
		\mathbb{Z}_p\left[\alpha_1,\ldots,\alpha_n\right], 
		\mathbb{Z}\left[x\right]$, $\mathbb{Q}\left(x\right)$, 
		$\mathbb{Q}\left[\alpha\right]$, and floating point numbers for 
		example), while in other cases the domains were left generic,
		namely for elements of a field,
		multivariate polynomials over a field, elements of a formal Ring
		with possibly undecidable zero-equivalence, or elements of a 
		Division Ring.  In the roughly 85 pieces of code
		surveyed \oleg{citation?},
		20 different domains were encountered.
	\item \textbf{Container}: Whether the matrix
		is represented as an array of arrays, a one-dimensional array,
		a hash table, a sparse matrix, etc., and
		whether indexing is done in C or Fortran style.  Additionally,
		if a particular representation had a special mechanism for efficient
		row exchanges, this is sometimes used.  
	\item \textbf{Output choices}:  Whether just the reduced matrix, or
		additionally the rank, the determinant, and the
		pivoting matrix are to be returned.
		It is worthwhile noting that in the larger algorithm family,
		routines like Maple's \texttt{LinearAlgebra:-LUDecomposition} have
		up to $2^6 + 2^5 + 2^2 = 100$ possible outputs.
	\item \textbf{Fraction-free}: Whether the Gaussian Elimination
		algorithm is allowed to use unrestricted division, or only
		exact (remainder-free) division.
	\item \textbf{Pivoting}: Whether to use no pivoting at all, only
		column-wise pivoting, or full pivoting.
	\item \textbf{Augmented Matrices}: Whether we are doing GE on
		a full matrix, or only a restricted number of columns, while
		doing elimination on the full matrix.  Note that we currently
		do not treat this aspect in our code.
\end{enumerate}
\noindent In addition to the above variations, there are two aspects that 
recur frequently:
\begin{enumerate}
	\item \textbf{Length measure}:  For stability reasons
		(whether numerical or coefficient growth), if a domain
	  possesses
		an appropriate length measure, this is sometimes used to choose
		an ``optimal'' pivot for row exchanges.
	\item \textbf{Normalization and zero-equivalence}: Whether the 
		arithmetic operations of the domain at hand gives results in 
		normalized form, and whether a specialized zero-equivalence 
		routine needs to be used.
\end{enumerate}
\noindent These are separated out from the others are they are cross-cutting
concerns: in the case of the length measure, then a property of the domain
will influence the method used to do pivoting \textbf{if} pivoting is to be
performed.

The simplest parametrization is to make the domain abstract. As it 
turns out, we need the following to exist in our domains:  $0$, $1$,
$+$, $*$, (unary and binary) $-$, some kind of division, normalization,
and potentially a relative size measure.  We definitely need at least
\emph{exact} division to exist, but if full division is available, than
can be used as well.  Normalization is frequently trivial for domains
where elements have unique representations, but in many applications
we have domains (like polynomials in sum-of-product form) where this
is not the case.  The simplest case of such an abstraction was given
in Section XXX, |param_code1|. In the latter, code-generators such as
|plus| and |one| were passed as arguments. We see however that we 
need far more than two parameters, so we have to group them. Instead
of the grouping offered by regular records, we use Ocaml
\emph{structures} (i.e., modules) (cite Ocaml module system tutorial?)
so we can take advantage extensions, type abstraction and constraints,
and especially parameterized structures (i.e., \emph{functors}).
We first define the type of the domain, the signature |DOMAIN| to which
different domain must satisfy:
\begin{figure}
\begin{code}
module type DOMAIN = sig
  type v    type 'a vc = ('a,v) code
  type kind (* Field or Ring ? *)
  val zero : 'a vc   val one : 'a vc
  val plus : 'a vc -> 'a vc -> ('a vc, 's, 'w) monad
  (* times, minus, uminus, div elided for brevity *)
  val better_than : ('a vc -> 'a vc -> 
      (('a,bool) code, 's, 'w) monad) option
  val normalizerf : (('a,v -> v) code ) option
end 
module IntegerDomain : DOMAIN = struct
  type v = int  type kind = domain_is_ring
  type 'a vc = ('a,v) code
  let zero = .< 0 >.  and one = .< 1 >. 
  let plus x y = ret .<.~x + .~y>. 
  let better_than = Some (fun x y -> ret .<abs .~x > abs .~y >. )
  let normalizerf = None
  ...
end
\end{code}
\end{figure}
\noindent  The first item to notice is that the types above are
generally lifted twice: once from the value domain |v| to the code
domain |'a vc|, and once more from values to monadic computations
|('a vc, 's, 'w) monad|.  Since there are times where we do not want
to redundantly walk a whole matrix to apply normalization when 
normalization is not necessary, |normalizerf| is optional.

One particular instantiation of the domain (a particular structure of
|DOMAIN| signature) is the |int| domain. The notattion
|module IntegerDomain : DOMAIN| makes the compiler
verify that our |IntegerDomain| is indeed a |DOMAIN|, that is,
satisfies the required signature. The constraint |DOMAIN| may be
omitted; in that case, the compiler will verify the type when we try
to use that structure as a |DOMAIN|. In any case, the errors such as
missing ``methods'' or methods of wrong types will be caught
statically, even \emph{before} any code generation takes place. The
abstract type |domain_is_ring| encodes a semantic constraint that the
full division is not available. While the |DOMAIN| type may have
looked daunting to some, the implementation is quite straightforward.
It is equally simple to make implementations for |float| and for
|Num.num| (arbitrary precision exact rational numbers).  With scarcely
more work (as these are not built-in and the arithmetic must be
implemented), $\mathbb{Z}_p$, $\mathbb{Z}\left[x\right]$,
$\mathbb{Q}\left(x\right)$ and more complex domains can be done as
well.

Parametrizing by the kind of container used to represent a matrix is
almost as straightforward.  We choose to make the container parametric
over a |DOMAIN| by making containers functors from a |DOMAIN| module
to the actual implementation of a container. The functor signature
|CONTAINER2D| specifies that a container must provide functions |dim1|
and |dim2| to extract the dimensions, functions |get| and |set| to
generate container getter and setters, the cloning generator |copy|
and functions that generate code for row and column swapping. The
inclusion of these functions in the signature of all containers makes
it simpler to optimize the relevant functions depending on the actual
representation of the container while not burdening the users of
containers with efficiency details. 
\omitnow{ As an example, below is
  an implementation of a matrix as an array of rows (themselves
  arrays).  This implementation clearly shows how swapping rows is
  more efficient than swapping columns in this representation.  We
  have not (yet) implemented the further potential optimization
  wherein if it is known that part of two rows/columns are already
  known to be equal (in fact to |Dom.zero| in the case of GE) and
  row/column exchange needs to be performed element-wise, then certain
  swaps are unecessary.

\begin{code}
module GenericArrayContainer(Dom:DOMAIN) =
  struct
  type contr = Dom.v array array (* Array of rows *)
  type 'a vc = ('a,contr) code
  type 'a vo = ('a,Dom.v) code
  let get' x n m = .< (.~x).(.~n).(.~m) >.
  let get x n m = ret (get' x n m)
  let set' x n m y = .< (.~x).(.~n).(.~m) <- .~y >.
  let set x n m y = ret (set' x n m y)
  let dim2 x = .< Array.length .~x >.       (* number of rows *)
  let dim1 x = .< Array.length (.~x).(0) >. (* number of cols *)
  let mapper g a = match g with
      | Some f -> .< Array.map (fun x -> Array.map .~f x) .~a >.
      | None   -> a
  let copy = (fun a -> .<Array.map (fun x -> Array.copy x) 
                       (Array.copy .~a) >. )
  (* this can be optimized with a swap_rows_from if it is known that
     everything before that is already = Dom.zero *)
  let swap_rows_stmt a r1 r2 =
      .< let t = (.~a).(.~r1) in
         begin 
             (.~a).(.~r1) <- (.~a).(.~r2);
             (.~a).(.~r2) <- t
         end >.
  let swap_cols_stmt a c1 c2 = .< 
      for r = 0 to .~(dim2 a)-1 do
          let t = (.~a).(r).(.~c1) in
          begin 
              (.~a).(r).(.~c1) <- (.~a).(r).(.~c2);
              (.~a).(r).(.~c2) <- t
          end
      done  >.
end
\end{code}
\noindent  Implementing other containers based on flattened vectors,
Fortran-style access, |Bigarray| or |Bigarray.Array2| is also possible.
Naturally, for proper use of |Bigarray| with specific types, it is 
important that the |DOMAIN| type matches -- but luckily the compiler 
will prevent us from making such a mistake.}

The use of a |functor| for making a container parametric is fairly 
straightforward.  It gets more interesting when one turns to the aspect
of what to return from the GE algorithm.  One could create an algebraic
data type (as was done in \cite{carette04}) to encode the various
choices; but this is wholly unsatisfying as we know that for any single
use, only one of the choices is ever possible, yet any routine which
calls the generated code must deal with these unreachable options.
Instead we use a module type with an \emph{abstract} type |res| for the
result type; the actual result type is then made by instantiating
this module type with different modules that decide what will be 
output: the matrix, the matrix and the rank, the matrix and the
determinant, the matrix, rank and determinant, and so on.  More
precisely, below we show this module type and one instantiation,
which specifies the output of a  GE algorithm as a 3-tuple
|contr * Det.outdet * int| of the U-factor, the determinant and the rank.

\begin{code}
module type OUTPUT = sig
  type contr  type res
  module D : DETERMINANT   module R : RANK
  module P : TRACKPIVOT
  val make_result : ('a,contr) code -> 
    (('a,res) code,
     [> `TDet of 'a D.lstate | `TRan of 'a R.lstate | `TPivot of 'a P.lstate]
       list, ('a,'w) code) monad
end
module OutDetRank(Dom:DOMAIN)(C: CONTAINER2D)
    (Det : DETERMINANT with type indet = Dom.v and type outdet = Dom.v)
    (Rank : RANK) = struct
  module Ctr = C(Dom)
  type contr = Ctr.contr
  type res = contr * Det.outdet * int
  module D = Det   module R = Rank
  module P = DiscardPivot
  let make_result b = mdo \{ det  <-- D.fin ();  rank <-- R.fin ();
    ret .< ( .~b, .~det, .~rank ) >. \}
end
\end{code}

As is apparent from the output choices, several different quantities
\emph{may} need to be tracked in a particular GE implementation.
We therefore need to be able to conditionally generate code-variables,
and weave in corresponding tracking code for each of these quantities.
To be precise, we may need to keep track (independently) of the rank,
the determinant and the permutation list.  These pieces of data then
become part of the \emph{state} that is tracked by our monad.  To have
all this choice when needed, and yet have our code be modular and
composable as well as ensuring that the generated code does not 
contain any abstraction artifacts, it is important to make this 
state modular.  As this discussion is rather abstract, an example
will likely illustrate our point more clearly: we will show the 
code for tracking the determinant.

\begin{code}
module type DETERMINANT = sig
  type indet  type outdet  type 'a lstate
  type tdet = outdet ref   
  val decl : unit -> 
    (unit, [> `TDet of 'a lstate ] list, ('a,'b) code) monad
  val upd_sign : unit -> 
    (('a,unit) code, [> `TDet of 'a lstate ] list, ('a,'b) code) monad
  ...
end
\end{code}
\noindent  Determinant tracking aspects involves being able to
generate code that defines variables used for tracking (|decl|),
generate code that updates the sign or the absolute
value of the determinant, and finally, converting the tracking state
to the final determinant value of the type |outdet|. GE of a
floating-point matrix with no determinant tracking uses the
instantiation of |DETERMINANT| where |outdet| is |unit| and all the
functions of that module generate no code. For integer matrices, we
have to track some aspects of the determinant even if we don't output
it. The determinant tracking aspect is complex becasue tracking
variables, if any, are to be declared at the beginning of GE; the sign
of the determinant has to be updated on each row or column
permutation; the value of the determinant should be updated per each
pivoting. We use |lstate| to pass the tracking state, which may be, for
example, a piece of code for the value of the type |float ref| among
various determinant-tracking functions. The |lstate| is a part of the
overall monadic state. Other aspects, e.g., rank tracking, may use the
monadic state for passing of rank tracking variables. To be able to
compose determinant and rank tracking functors -- each of which may
(or may not) use the monadic state for passing its own data -- we make
an extensive use of open records (list of polymorphic variants
appeared to be the easiest way to implement such a union, in a pure
functional way). This lets us freely compose determinant-tracking,
rank-tracking, and other aspects.

The GE generator functor is as follows. It is
parameterized by the domain, container, pivoting policy (full, row,
nonzero, nopivoting), update policy (that use either ``fraction-less'
or full division), and by what to yield as the result. Some of the
argument modules such as |PIVOT| are functors themselves (parameterized
by the domain, the container, and the determinant functor). The sharing
constraints express obvious constraints on the instantiation of |Gen|,
for example, pivoting, determinant etc. components all use the same
domain. It must be stressed that all structures (i.e., module
instances) are stateless, and so we never have to worry that different
aspect functors (such as |CONTAINER2D| and |PIVOT|) are instantiated
with different but type-compatible instances of |DOMAIN|. That is, we
are not concerened at all about the value sharing. Aspects such as
determinant tracking may be stateful so that the determinant update
code have access to the determinant tracking variables declared
previsouly. But that state is handled via the monadic state. As we
have shown, open unions makes the overall monadic state compositional
with respect to the state of various aspects.

\begin{figure}
\begin{code}
module Gen(Dom: DOMAIN)(C: CONTAINER2D)(PivotF: PIVOT)
          (Update: UPDATE with type baseobj = Dom.v and type ctr = C(Dom).contr)
          (Out: OUTPUT with type contr = C(Dom).contr and type D.indet = Dom.v 
                        and type 'a D.lstate = 'a Update.D.lstate) = struct
    module Ctr = C(Dom)
    module Pivot = PivotF(Dom)(C)(Out.D)
    let gen =
      let zerobelow b r c m n brc =
        let innerbody i = mdo \{
            bic <-- Ctr.get b i c;
            whenM (l1 LogicCode.not (LogicCode.equal bic Dom.zero ))
                (seqM (retLoopM (Idx.succ c) (Idx.pred m)
                          (fun k -> Update.update b r c i k) )
                      (Ctr.set b i c Dom.zero)) \} in 
        mdo \{
              seqM (retLoopM (Idx.succ r) (Idx.pred n) innerbody) 
                   (Update.update_det brc) \} in
      let dogen a = mdo \{
          r <-- Out.R.decl ();
          c <-- retN (liftRef Idx.zero);
          b <-- retN (Ctr.mapper Dom.normalizerf (Ctr.copy a));
          m <-- retN (Ctr.dim1 a);
          n <-- retN (Ctr.dim2 a);
          () <-- Update.D.decl ();
          () <-- Out.P.decl ();
          seqM 
            (retWhileM (LogicCode.and_ (Idx.less (liftGet c) m)
                                       (Idx.less (liftGet r) n) )
               ( mdo \{
               rr <-- retN (liftGet r);
               cc <-- retN (liftGet c);
               pivot <-- l1 retN (Pivot.findpivot b rr m cc n);
               seqM (retMatchM pivot (fun pv -> 
                        seqM (zerobelow b rr cc m n pv)
                             (Out.R.succ ()) )
                        (Update.D.zero_sign () ))
                    (Code.update c Idx.succ) \} ))
            (Out.make_result b) \} in
    .<fun a -> .~(runM (dogen .<a>.)) >.
end
\end{code}
\end{figure}

In addition to the ``regular'' type sharing constraints shown in the
|Gen| functor, there are also ``semantic'' sharing constraints, shown
in the following structure of the |UPDATE| signature:

\begin{code}
module DivisionUpdate
  (Dom:DOMAIN with type kind = domain_is_field)
  (C:CONTAINER2D)
  (Det:DETERMINANT with type indet=Dom.v) = struct ... end
\end{code}

The structure implements update policy that can use |Dom.div| operation without
restrictions. That is possible only if the domain has such an
operation. A domain such as integer domain may still provide |Dom.div|
of the same type, but that operation can only be used when we are sure
that the division is exact. Using the integer domain with
|DivisionUpdate| will lead to the compile-time error, when compiling
the \emph{generator} code. Thus, in some cases we can use module types
to express ``semantic'' constraints that cannot normally be expressed
via the types of module members.

\begin{code}
module GenFV4 = Gen(FloatDomain)
   (GenericVectorContainer)(FullPivot)
   (DivisionUpdate(FloatDomain)(GenericVectorContainer)(FDet))
   (OutDetRank(FloatDomain)(GenericVectorContainer)(FDet)(Rank))
module GenIA1 = Gen(IntegerDomain)
   (GenericArrayContainer)(RowPivot)
   (FractionFreeUpdate(IntegerDomain)(GenericArrayContainer)(IDet))
   (OutJustMatrix(IntegerDomain)(GenericArrayContainer)(IDet))
\end{code}

We can instantiate the |Gen| functor as shown above and inspect the
generated code, e.g., by printing |GenIA1.gen|. The code can then be
``compiled'' as |!. GenIA1.gen| or with off-shiring (cite). The code
for |GenFV4| shows full pivoting, determinant and rank tracking. The
code for all these aspects is fully inlined; no extra functions are
invoked and no tests other than those needed by the GE algorithm
itself are performed. The GE function returns a triple 
|float array array * float * int| of the U-factor, determinant and 
the rank. The
code generated by |GenIA1| shows absolutely no traces of determinant
tracking: no declaratrion of spurious variables, no extra tests,
etc. The code appears as if determinant tracking aspect did not exist
at all. The generated code for the above and other instantiations of
|Gen| can be examined at (cite).

\section{Related work}\label{related}

Note that the monad is similar to the one used in \cite{KiselyovTaha}.
However, the latter work used only |retN| of all monadic operations,
and used fixpoints (for performing iterations at the generation time).
In this paper we do not use monadic fixpoints (because the generator
is not recursive) but we make extensive use of monadic operations for
generating conditional and looping operations.

Mention |Blitz++| and template meta-programming. Can do similar
eliminating of abstraction. Can even encode some domain-specific
knowledge in the form of concepts. But inlining critically depends on
the compiler's fully inlining of all methods. Furthermore, all errors
(such as type errors and concept violation errors, that is,
composition errors) are detected only when compiling the generated
code. It is immensely difficult to correlate errors (e.g., line numbers) 
to the ones in the generator itself. Furthermore, I think templates
can't be local, right? So, a lot of code generation is spread around
file-level entities. 

SPIRAL does the intentional code analysis. Probably ATLAS as well.

\jacques{What do we consider \emph{related work} here?  Do we essentially gloss 
over related work in numerical software, code generation in general,
aspect-oriented stuff?  We don't want this section to be too long, but
not so short that we make it seem like we're inventing more than we
really are}
\oleg{Yes, let's gloss over ATLAS, SPIRAL, Axiom. I wonder what would
  be the use of functors? Something by David McQueen or R. Harper.
  In any way, mention that nobody before used functors to abstract
  code generators -- or mix functors and multi-stage programming.}


\section{Future work}\label{future}
connections with delimited continuations: making notation
more direct-style and potentially clearer.  More syntactic sugar
would help a lot here.

More aspects that can be handled: Input variations (augmented
matrices). Fit into the larger program family, where there are
more aspects still.

Making all of the algorithmic code be completely independent of 
MetaOCaml markup and isolate all the code generation to a select
few 'base' modules and functors.  This should allow for simpler
debugging, where one can choose to instantiate either a direct-style
set of base modules for debugging, or a code-generation style set
of base modules for generating optimized code.

\section{Conclusion}\label{conclusion}
Mention the mdo notation. Mention it into the introduction.

Combination of stateless modules (functors and structures) with
the monad with the compositional state greatly makes the aspects
freely composable without having to worry about presence or absense of
value aliasing. The only constraints to compositionality are the
typing constraints plus the constraints we specifically impose,
including semantic constraints (e.g., rings do not have the full
division).



Relation with aspect-oriented code: in AspectJ, aspects are (comparatively)
lightly typed, and are post-facto extensions of an existing piece of
code.  Here aspects are weaved together ``from scracth'' to make up a
piece of code/functionality.  One can understand previous work to be
more akin to dynamically typed aspect weaving, while we have started
investigating statically typed aspect weaving.

\bibliography{metamonads}
\bibliographystyle{plain}
\end{document}

% It seems no appendices are allowed. Check with Walid?

\section{Appendix A}
The do notation for Ocaml.
To implement the |mdo| notation, we use the Ocaml pre-processor,
camplp4 (cite?). Informally, our notation is
\begin{code}
mdo { exp }
mdo { exp1; exp2 }
mdo { x <-- exp; exp }
mdo { let x = foo in; exp }
\end{code}
which matches closely the Haskell |do| notation, modulo |do|/|mdo| and 
|<-|/|<--|.

Grammar formally:
\begin{code}
mdo { <do-body> }
<do-body> :: =
    "let" var = EXP ("and" var = EXP)* "in" ";" <do-body>
    | EXP
    | (pat <--)? EXP ";" <do-body>
\end{code}
The semantics is given by the following rules re-writing the |mdo|
form into the regular Ocaml code. We assume that the variable |bind|
must be in scope.
\begin{code}
mdo { exp } ===> exp
mdo { pat <-- exp; rest } ===> bind exp (fun pat -> mdo { rest })
mdo { exp; rest } ===> bind exp (fun _ -> mdo { rest })
mdo { let pat = exp in; rest } ===> let pat = exp in mdo { rest }
\end{code}
We also introduce |odo| form, which is equivalent to |mdo| but uses 
a bind `method' rather than a bind `value'.

The major difficulty with the |do| notation is that it cannot truly be
parsed by an LR(n)-grammar for any finite |n| because many |pat| can also
be parsed as an |exp|. 

Perhaps give the command line how to load the camlp4 module into
ocaml. Point to the code.

\omitnow{
In this paper, we are concerned with building and composing
generators. We start with the code expressing the algorithm (in
OCaml), and, adding MetaOCaml annotations, turn that code into a
\emph{code expression} that will generate the code when runs. We then
modify that code expression to add various parameters and aspects, to
yield the complete generator.

Here we illustrate the MetaOCaml constrcutions and code generators.

Turning code into the code expression is relatively straightforward
until we come to binding constructs such as |let|. For example, here
is a piece of code that defines a function that searches a given array
|arr| for the maximal (in absolute value) element, and the index of
that element and the found absolute value. This code is representative
of pivot determination. The code uses mutations and destructive
updates -- partly for illustrative purposes and mainly because many
linear algebra algorithms are expressed via an imperative code (so it
is less error-prone to translate the ``canonical'' algorithms with
minimal changes).

\begin{code}
let fp = fun arr n ->
   let pv = ref (0,arr.(0)) in
   for i = 1 to n-1 do
     if abs_float arr.(i) > snd !pv then
        pv := (i, abs_float arr.(i))
   done;
   !pv
\end{code}

This function has the type |float array -> int -> int * float|. By
enclosing the code into MetaOCaml brackets, we obtain the simplest
generator expression.

\begin{code}
let fpc = .<fun arr n ->
   let pv = ref (0,arr.(0)) in
   for i = 1 to n-1 do
     if abs_float arr.(i) > snd !pv then
        pv := (i, abs_float arr.(i))
   done;
   !pv>.
\end{code}

This value has the type |('a, float array -> int -> int * float) code|.
Here |'a| type parameter is needed (ask Walid for the reference).
The main difference between |fp| and |fpc| is that |fp| is a function,
which we can immediately use as \verb^fp [|1.0; -4.; 3.|] 3^. In
contrast, |fpc| is not a function: it is a \emph{code} value that denotes
that function. We would say that the code expression is the
``second-stage'' value. The ability to manipulate code expressions as
regular values is the distinguished fature of MetaOCaml.
We can ``compile'' the code by using MetaOCaml |run| command, or |.!|:
|.! fpc|. The result is equivalent is |fp|. Although |fpc| doesn't
seem to do much, it is already useful as we can use offshoring (cite)
to translate |fpc| into C or Fortran code (which we can dynamically
link into the running MetaOCaml code -- or use separately).

The code |fpc| obviously lacks generality: the name |abs_float|
implies that that code deals with |float| arrays. We wish to handle
arrays in various domains, integer, rationals, etc. We can make the
code generic by parametrizing over the |abs| function:

\begin{code}
let fpca absf = .<fun arr n ->
   let pv = ref (0,arr.(0)) in
   for i = 1 to n-1 do
     if absf arr.(i) > snd !pv then
        pv := (i, absf arr.(i))
   done;
   !pv>.
\end{code}

and so |fpca abs_float| will manipulate |float array| and |fpca abs|
will do for the integer arrays. Here we used the value from the first
stage, |absf|, to incorporate into the code at the second stage:
croos-stage persistent values.

We should now consider the function |fpca| in a browder context: of a
function that computes the pivot and then swaps the pivot withe 0-th
element of the array:

\begin{code}
let upv = .<fun arr n ->
   let pv = .~(fpca abs_float) arr n in
   let i = fst pv in
   (if not (i = 0) then
      let t = arr.(0) in
      begin arr.(0) <- arr.(i); arr.(i) <- t end);
   arr>.
\end{code}

Here, we see the third MetaOCaml operator, escape |.~|. It does ...
MetaOcaml can print out the code, which looks like
|.<fun arr n -> let pv = (fun arr n -> fpcs code) arr n in ... >.|
That is, the compiler inlined the code for fpca -- which is a
functional expression. The |upv| code shows the functional
application, |(fun arr n -> ...)| applied to arguments |arr n|. That
application will be executed at run-time, unless a sufficiently smart
compiler can inline it. As we stated, our aim is to not to rely on
post-generation optimizations and sufficiently smart compiler. We
strive to generate the efficient code ourselves. We can do that by
modifying the definition of |fpca| as

\begin{code}
let fpca absf arr n = .<
   let pv = ref (0,(.~arr).(0)) in
   for i = 1 to .~n-1 do
     if absf (.~arr).(i) > snd !pv then
        pv := (i, absf (.~arr).(i))
   done;
   !pv>.
\end{code}
\begin{code}
let upv = .<fun arr n ->
   let pv = .~(fpca abs_float .<arr>. .<n>.) in
   let i = fst pv in
   (if not (i = 0) then
      let t = arr.(0) in
      begin arr.(0) <- arr.(i); arr.(i) <- t end);
   arr>.
\end{code}
% (.! upv) [|1.0; -4.; 2.|] 3;;
We see that we pass code fragment |.<arr>.| as arguments to the
function. Now when we look at the code for |upv|, we see
|.<fun arr n -> let pv1 = let pv = ref (0,arr.(0)) in ...>.|
So the fpca code got really inlined.

The generator |fpca| shows one particular way of searching for pivot.
In the domain of exact rational numbers, for example, we would be
satisfied with the first non-zero element as a pivot, and so we would
write a different pivot-searching generator. The body of |upv| moves
the found pivot to the ``canonical'' location. Again, depending on the
container (a dense matrix, a sparse matrix, etc), there are different
ways of doing that. So, we would like to be able to ``compose'' pivot
generating function with the swap generating function. We would like
to write something like this

\begin{code}
let fpca absf arr n pv = .<
   for i = 1 to .~n-1 do
     if absf (.~arr).(i) > snd ! (.~pv) then
        (.~pv) := (i, absf (.~arr).(i))
   done>.
let swapper arr n pv = .<
   let i = fst !(.~pv) in
   if not (i = 0) then
      let t = (.~arr).(0) in
      begin (.~arr).(0) <- (.~arr).(i); (.~arr).(i) <- t end>.
let seqv a b c = .<begin .~a; .~b; .~c end>.
let pv_swap absf arr n =
   let pv = .<ref (0,(.~arr).(0))>. in
   seqv (fpca absf arr n pv)
        (swapper arr n pv)
        arr
let top =  .<fun arr n -> .~(pv_swap abs_float .<arr>. .<n>.)>.
\end{code}

The function |pv_swap| shows our ideal -- composing code gerenators
|fpca| and |swapper|, using a higher-order combinator `seqv'
(sequence of code fragments). We lifted the pivot declaration out of
old version of |fpca| so that both |fpca| and |swapper| could be
parameterized by the pivot (and |fpca| and similar function will deal
only with pivot determination). |top| puts it all together. The above
code represents the pattern we will be following for the rest of the
paper.

And it \emph{almost} works. It is type correct.  When we look at the
generated code for |top|, we notice the |(ref (0, arr_1.(0)))| is
inlined wherever pivot was expected. For example, the swapping part of
the code begins with |let i_3 = (fst (! (ref (0, arr_1.(0))))) in ...|. 
That is certainly not what we want. We meant for pivot searching
and swapping parts of the code to communicate via a shared mutable
state, variable |pv|. In the generated code, each piece of code gets
its own copy of |pv|. Rather then inline |(ref (0, arr_1.(0)))|
everywhere, we should have generated |let pv = ref (0,(.~arr).(0)) in ...|
and inlined |pv|. Incidentally, even in a pure functional code we
often wish to generate such a |let| statement so we can compute a
complex expression only once and use it value in several other
places. Again, we would rather not leave the job of detecting such
common subexpressions to the compiler (because in the complex code the
compiler may fail to detect the common subexpressions). The problem
with generating such |let| expressions is that we wish our geenrator
to return just |.<let pv = ref (0,(.~arr).(0)) in>.|, without the `body'
of the |let|, which we would fill in later. In languages which
generate code in the string form this is easy. We generate the string
|''let pv = ref (0,(.~arr).(0)) in''| and then append to it the string
representting the body, wcich would give us the complete |let|
statement. In MetaOCaml, we can't generate incomplete expressions, for
a good reason. 

To solve the problem of writing a generator for a |let| expression 
when the generated body is not yet known, we use CPS.
}


%%% Old, detailed version is here

\section{CPS and the problem of generating names}\label{CPS}
%%%%%%%%%%%%%%%%%%%%%%%%%%%%%%%%%%%%%%%%%%%%%%%
%% And now, a better -- and shorter -- write-up.

The essence of our approach is assembling the code of the program by
combining instances of various aspects together. In the case of
Gaussian Elimination, the algorithmic aspects are pivot determination, 
swapping of
rows and columns, determinant tracking, etc. A primitive code
generator yields a code expression.  A combinator takes (complete and
well-typed) code expressions and combines them into a composite expression
(which is also statically guaranteed to be well-formed and well-typed).

We will be using MetaOcaml which, as an instance of a multi-stage
programming system \cite{TahaThesis}, provides exactly the needed
features: to construct a code expression, to combine them, and to
execute them.

For example, the following is probably the simplest code generator:
\begin{code}
let one = .<1>.
\end{code}

In the floating-point instance of that aspect, we would have
|.<1.0>.| instead. We use MetaOCaml brackets |.<...>.| to generate
code expressions -- in other words, to construct future-stage
computations. The simplest combination is inlining:
\begin{code}
let plus x y = .<.~x + .~y>.
\end{code}
Here we see the second component of MetaOCaml -- escapes |.~|. They
let us perform an immediate code generating computation while we are
building the future-stage computation (code expression). In the
example above, the immediate computation |.~x| is trivial -- obtaining
the value of the variable |x|, which should be a code expression, and
inlining it. The following shows an example of using the simple
generators:

\begin{code}
let simplest_code =
  let gen x y = plus x (plus y one) in
  .<fun x y -> .~(gen .<x>. .<y>.)>.
\end{code}

Here, the immediate computation |gen .<x>. .<y>.| is less trivial --
it includes the evaluation of the function |gen|, which in turn
applies the function |plus|. The function receives code expressions
|.<x>.| and |.<y>.| as arguments. At the generating stage, we can
manipulate code expressions as (opaque) values. The function |gen| returns a
code expression, which will be inlined in the place of the
escape. MetaOCaml can print out code expressions, so we can see the
final generated code:\\
  |.<fun x_1 -> fun y_2 -> (x_1 + (y_2 + 1))>.|

We must point out that |gen x y| looks like a combinator; its body 
involves several applications of the function |plus|. When
we look at the generated code however, we see no traces of the function
|plus| and no such administrative applications. Instead, we see the
fully inlined expression. These administative applications are done at
the code generation stage. 

The final MetaOCaml feature, |.!| (pronounced ``run'') can be used to
execute the code expression: |.! simplest_code| is a function of two
integers, which we can apply: |(.! simplest_code) 1 2|. The original
|simplest_code| is not the function on integers -- it is a code
expression.

To see the benefit of code generation, we notice that we can easily
parameterize our code:

\begin{code}
let simplest_param_code plus one =
  let gen x y = plus x (plus y one) in
  .<fun x y -> .~(gen .<x>. .<y>.)>.
\end{code}

\noindent and use it to generate code that operates on integers, floating point
numbers or booleans -- in general, any domain that implements |plus|
and |one|:
\begin{code}
let plus x y = .<.~x +. .~y>. and one = .<1.0>. in
  simplest_param_code plus one
let plus x y = .<.~x || .~y>. and one = .<true>. in
  simplest_param_code plus one
\end{code}
Running the former expression yields the function on |float|s, whereas
the latter expression is the code expression for a boolean function.

This clearly shows the separation of concerns, namely of that for domain
operations.

Let us consider a more complex expression:
\begin{code}
let param_code1 plus one =
  let gen x y = plus (plus y one) (plus x (plus y one)) in
  .<fun x y -> .~(gen .<x>. .<y>.)>.
\end{code}

We notice two occurrences of |plus y one|. Depending on the
implementation of |plus| that may be quite a complex computation, and
so we would rather not do it twice. We may be tempted to rely on the
compiler's common-subexpression elimination optimization. We would
rather not rely on a ``sufficiently smart compiler'': when the
generated code is very complex, the compiler may overlook common
subexpressions.  Or the subexpressions may occur in an imperative
context where the compiler might not be able to determine if lifting
them is sound. So, being conservative, the optimizer will leave the
duplicates as they are. We may attempt to write

\begin{code}
let param_code1 plus one =
  let gen x y = 
     let ce = (plus y one) in  plus ce (plus x ce) in
  .<fun x y -> .~(gen .<x>. .<y>.)>.
\end{code}

However, the result of |param_code1 plus one|, which prints as\\
|.<fun x_1 -> fun y_2 -> ((y_2 + 1) + (x_1 + (y_2 + 1)))>.|,
still exhibits duplicate sub-expressions. Our |let|-insertion
optimization saved the computation at the generating stage. 
We need a combinator that inserts the |let| expression in the
generat\emph{ed} code. We need a combinator |letgen| to be used
as\\|let ce = letgen (plus y one) in plus ce (plus x ce)| 
yielding the code like |.<let t = y + 1 in t + (x + t)>.|
But that seems impossible because |letgen exp| has to generate
the expression\\|.<let t = exp in body>.| although |letgen| does not have
the |body| yet. The body needs a temporary identifier |.<t>.| that
is supposed to be the result of |letgen| itself. 
Certainly |letgen| cannot generate only part of a let-expression,
without the |body|,  as all generated expressions in
MetaOCaml are well-formed and complete.

The key is to use continuation-passing style (CPS). Its benefits were first
pointed out by \cite{Bondorf:92} in the context
of partial evaluation, and extensively used by \cite{KiselyovTaha} for code
generation.

\begin{code}
let letgen exp k = .<let t = .~exp in .~(k .<t>.)>.
let param_code2 plus one =
  let gen x y k = letgen (plus y one) (fun ce -> k (plus ce (plus x ce)))
  and k0 x = x
  in .<fun x y -> .~(gen .<x>. .<y>. k0)>.
\end{code}

Now, |param_code2 plus one| gives us the desired code\\
|.<fun x_1 -> fun y_2 -> let t_3 = (y_2 + 1) in (t_3 + (x_1 + t_3))>.|.

\section{Monadic notation, making CPS code clear}\label{monadicnotation}

Comparing the code that did let-insertion at the generating stage\\
|let ce = (plus y one) in  plus ce (plus x ce)|\\
with the corresponding code inserting let at the generated code stage\\
|letgen (plus y one) (fun ce -> k (plus ce (plus x ce)))|\\
clearly shows the difference between the direct-style and CPS code.
What was |let ce = init in ...| in direct style became
|init' (fun ce -> ...)| in CPS. For one thing, |let| became
inverted. For another, what used to be an expression that yields
a value, |init|, now became an expression that takes an extra argument,
the continuation, and invokes it. The differences look negligible in
the above example. In larger expressions with many let-forms, the
number of parentheses around |fun| increases, the need to add and
then invoke the |k| continuation argument become increasingly annoying. The
inconvenience is great enough for some people to explicitly avoid CPS
or claim that numerical programmers (our users) cannot or will not
program in CPS. Clearly a better notation is needed.

The |do|-notation of Haskell \cite{Haskell98Report} shows that it is possible
to write essentially CPS code in conventional-looking style. The
|do| notation is the notation for writing monadic code \cite{moggi-notions}.
The benefit of monadic notation is not that it can represent CPS \cite{Filinski:Representing}, but it helps in composability by offering to add different
layers of effects (state, exception, non-determinism, etc) to the
basic monad \cite{liang-interpreter} in a controlled way.

A monad \cite{moggi-notions} is an abstract datatype representing
computations that yield a value and may have an \emph{effect}.
The datatype must have at least two operations, |return| to build
trivial effect-less computations and |bind| for combining
computations. These operations must satisfy \emph{monadic laws}:
|return| being the left and the right unit of |bind| and |bind| being
associative. Figure~\ref{ourmonad} defines the monad used throughout
the present paper and shows its implementation.

\begin{figure}\label{ourmonad}
\begin{code}
type ('v,'s,'w) monad = 's -> ('s -> 'v -> 'w) -> 'w
let ret a = fun s k -> k s a
let bind a f = fun s k -> a s (fun s' b -> f b s' k)
let fetch s k = k s s
let store v s k = k (v::s) ()

let k0 s v = v
let runM m = m [] k0

let l1 f = fun x -> mdo { t <-- x; f t}
let l2 f = fun x y -> mdo { tx <-- x; ty <-- y; f tx ty}

let retN a = fun s k -> .<let t = .~a in .~(k s .<t>.)>.

let ifL test th el = ret .< if .~test then .~th else .~el >.
let ifM test th el = fun s k ->
  k s .< if .~(test s k0) then .~(th s k0) else .~(el s k0) >.
\end{code}
\caption{Our monad}
\end{figure}

Our monad represents two kinds of computational effects: reading and
writing a computation-wide state, and control effects. The latter are
normally associated with exceptions, forking of computations, etc. --
in general, whenever a computation ends with something other than
invoking its natural continuation in the tail position. In our case
however the control effects manifest themselves as code generation.

In Figure~\ref{ourmonad}, the monad is implemented as a function of two
arguments: the state (of type |s|) and the continuation. The
continuation receives the current state, the value (of the type |v|) and
yields the answer of the type |w|.  The monad is polymorphic over all
these three type parameters.  Other implementations are
possible. Except for the code in Figure~\ref{ourmonad}, the rest of our code
treats the monad as a truly abstract data type. The implementation of basic
monadic operations |ret| and |bind| are conventional. It is easy to
see that the monadic laws are satisfied.  Other monadic operations
construct computations that do have effects.  Operations |fetch| and
|store v| construct computations that read and write the state. In our
case the state is a list (of polymorphic variants), which models an
open discriminated union, as we shall see later.

The operation |retN a| is the let-insertion operation, whose simpler
version we called |letgen| earlier. It is the first computation with
a control effect: indeed, the result of |retN a| is \emph{not} the
result of invoking its continuation |k|. Rather, its result is a |let|
code expression. Such a behavior is symptomatic of cntrol operators
(in particular, |abort|).

Finally, |runM| runs our monad, that is, performs the computation of
the monad and gets the result, which in our case is the code
expression. We run the monad by passing it the initial state and the
initial continuation |k0|. We can now re-write our |param_code2|
example of the previous section as

\begin{code}
let param_code3 plus one =
  let gen x y = bind (retN (plus y one)) (fun ce -> 
                ret (plus ce (plus x ce)))
  in .<fun x y -> .~(runM (gen .<x>. .<y>.))>.
\end{code}
% param_code3 plus one;;

That does not seem like much of an improvement. But here we can introduce the
do notation (Appendix A), which is patterned after the do-notation of Haskell.

\begin{code}
let param_code4 plus one =
  let gen x y = mdo { ce <-- retN (plus y one);
                      ret (plus ce (plus x ce)) }
  in .<fun x y -> .~(runM (gen .<x>. .<y>.))>.
\end{code}

The function |param_code4| is totally equivalent to |param_code3| --
the |mdo|-notation is just syntactic sugar. In fact, the camlp4
preprocessor (with our extension) will transform |param_code4|
into exactly |param_code3|. And yet, |param_code4| looks far more
conventional, as if it were indeed in direct style.

We can write operations that generate code other than let-statements,
e.g., conditionals: see |ifL| in Figure~\ref{ourmonad}. The function |ifL|, 
albeit straightforward, is not as general as we wish: its argument are
already generated pieces of code rather than monadic values. So, we
may want to ``lift it'':
\begin{code}
let ifM' test th el = mdo {
  testc <-- test;
  thc   <-- th;
  elc   <-- el;
  ifL testc thc elc}
\end{code}

\noindent We can define functions |l1|, |l2|, |l3| (analogues of |liftM|,
|liftM2|, |liftM3| of Haskell) to make such a lifting generic.
We soon notice the need for another |ifM| function, with the same
interface (see |ifM| in Figure~\ref{ourmonad}). The difference between them is
apparent from the following:
\begin{code}
let gen a i = ifM' (ret .<(.~i) >= 0>.) (retN .<Some (.~a).(.~i)>.)
                           (ret .<None>.)
 in .<fun a i -> .~(runM (gen .<a>. .<i>.))>.
;;
.<fun a_1 -> fun i_2 ->
    let t_3 = (Some (a_1.(i_2))) in if (i_2 >= 0) then t_3 else (None)>.
\end{code}
\noindent whereas if we replace |ifM'| with |ifM|, the result will be
\begin{code}
.<fun a_1 -> fun i_2 ->
    if (i_2 >= 0) then let t_3 = (Some (a_1.(i_2))) in t_3 else (None)>.
\end{code}

The difference is apparent: in the code with |ifM'|, the let-insertion
happend \emph{before} the if-expression, that is, before the test that
the index |i| is positive. That is, if |i| turned out
negative, |a.(i)| will generate an out-of-bound array access
error. On the other hand, the code with |ifM| accesses the array only
when we have verified that the index is non-negative. This example
makes it clear that the code generation (such as the one in |retN|) is 
truly an effect and we have to be clear about the sequencing of
effects, when generating control constructions such as conditionals.
The form |ifM| handles the such effects correctly. In our code, we
need similar operators for other Ocaml control forms: for generating
case-matching statements and for- and while-loops.
