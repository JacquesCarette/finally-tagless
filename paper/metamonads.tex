\documentclass[11pt]{llncs}

\title{Declarative Assembler}
\author{ Jacques Carette\inst{1}
 \and
 Oleg Kyselyov\inst{2}
} 
\institute{McMaster University,
1280 Main St. West, Hamilton, Ontario Canada L8S 4K1
\and
Monterey, CA 93943}

\usepackage{amsmath}
\usepackage{fancyvrb}
\usepackage{comment}

\begin{document}
\maketitle

\begin{abstract}
We use monads and Ocaml's advanced module system to
demonstrate that, in a generative context, it is 
possible to totally eliminate the overhead of abstraction.
This lets one use extreme forms of \textit{information hiding}
at no run-time cost.  Furthermore the typed nature of
the generative context provide static guarantees about the 
generated code.  The various extensions
(aspects) can be made completely orthogonal and compositional,
even in the presence of name-generation for temporaries and
other bindings and ``interleaving'' of aspects.  We also
show how to encode some domain-specific knowledge so that
``clearly wrong'' compositions can be statically rejected by the
compiler.
\end{abstract}

\section{Introduction}

	- problem of high-performance and numeric computing:
performance versus abstraction. Manual inlining and optimizations is
error-prone and makes code unreadable. Furthermore, the current
architectures demand more and more frequent tweaks (which, in general,
cannot be done by the compiler because the tweaking often involves domain
knowledge).

	- Gaussian elimination as a representative example.

	- Generating code. MetaOCaml. Static guarantees: the generated
code is well-typed.

	- Problems: generativity: generated code cannot be examined
and optimized at all. It must be generated just right. The FFT paper
showed one way: abstract interpretation. But more problems remain
(cite Padua et al paper at Haskell Workshop04): generating names in
the presence of `if', 'while' etc. control operators (simple problem
of generating names solved in the FFT paper); making CPS code clear
(note that many authors are afraid of CPS code because it quickly
becomes unreadable; we need CPS for name generation); compositionality
of various (often interleaving: explain: determinant computation
requires attention at several places) aspects and expressing
dependencies among them; expressing more static guarantees about the
generated code (e.g., not attempting full (fractional) division for
the integer domain; generating full-pivoting code for the rational
domain without `better\_than' operation). In short: can we achieve all
of the abstractions needed for the flexible Gaussian Elimination? Can
we eliminate all of that abstraction's overhead in the generated code
(without using intensional code analysis -- that is, maintaining the
guarantees of MetaOCaml and using pure generative approach).

[Note why it is important to maintain generativity: stronger
equational theory: Walid's paper. Also, intensional code analysis
(e.g., SPIRAL project) essentially requires one to insert an
optimizing compiler into the code generating system. That is extremely
complex, quite error-prone, and very difficult to ascertain the
correctness.]

\section{CPS and the problem of generating names}

[Oleg to write.]

\section{Monadic notation, making CPS code clear}

What is our monad: what is the action and what is the value.
ifM and ifL: show example. Mention 'ifM' and `reset'. Investigating
the connection with the delimited continuations: for future work.

What is state and why we need it: making aspects interleave: use det
as an example.

\section{Functors}

Simple parameterization: by the domain, by the container
 
Functors and the generated return type: the Out structure.

Making state modular and composable: several aspects can contribute
to the state, without interference (show: Rank, Det, Permutation
aspects).

Expressing domain constraints in functors (Div and integer domain).
Show the type error that occurs in improper instantiation.

Show several examples (e.g., code with NullPivot and nodet: no traces
of either det nor pivoting). Then the code with RowPivoting, Det and
float domain. 

\section{Related work}

\section{Future work}
connections with delimited continuations: making notation
more direct-style and potentially clearer.

More aspects that can be handled: Input variations (augmented 
matrices).  Fit into the larger program family, where there are
more aspects still.

\section{Conclusion}

\bibliography{metamonads} 
\bibliographystyle{plain}

\end{document}
