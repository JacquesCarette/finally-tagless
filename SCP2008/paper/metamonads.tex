\documentclass[draft]{elsart}

% Reduce display spacing
\divide\abovedisplayskip 2
\divide\belowdisplayskip 2
\divide\abovedisplayshortskip 2
\divide\belowdisplayshortskip 2

\usepackage{hyphenat}
\usepackage{comment}
\usepackage{amsmath,amssymb}
\usepackage{amstext}
\usepackage{url}
\usepackage[dvips]{color}

\usepackage{ifpdf}
\ifpdf
    \pdfpageheight=11in
    \pdfpagewidth=8.5in
\fi

%\usepackage{refrange}

%\usepackage[medium,compact]{titlesec}

\usepackage{fancyvrb}
\DefineShortVerb{\|}
\DefineVerbatimEnvironment{code}{Verbatim}{xleftmargin=\ourmathindent,commandchars=\\\{\},fontsize=\small}
\DefineVerbatimEnvironment{code2}{Verbatim}{xleftmargin=\ourmathindent,commandchars=\\\{\},fontsize=\small}
\newcommand{\evalresult}[1]{\ensuremath{\Longrightarrow}\textcolor{red}{#1}}
% \setlength{\parskip}{0pt}
\newlength{\ourmathindent}
\setlength{\ourmathindent}{1em}

% \usepackage[backref,colorlinks,bookmarks=true]{hyperref}

% Reduce list spacing
%% \makeatletter
%% \renewcommand\@@listI{\leftmargin\leftmargini
%% \parsep \z@@
%% \topsep 3\p@@ \@@plus\p@@ \@@minus 2\p@@
%% \itemsep 2\p@@ \@@plus\p@@ \@@minus\p@@}
%% \let\@@listi\@@listI
%% \@@listi
%% \makeatother

% \parskip 0pt plus 1pt minus 2pt
\textfloatsep 4pt plus 2pt minus 3pt % Less space around figures
%\abovecaptionskip 0pt plus 0pt minus 2pt
%\belowcaptionskip 0pt plus 0pt minus 2pt

% \intextsep 2pt plus 0pt minus 1pt

\renewcommand\floatpagefraction{.95}
\renewcommand\topfraction{.95}
\renewcommand\bottomfraction{.95}
\renewcommand\textfraction{.05}   
\setcounter{totalnumber}{50}
\setcounter{topnumber}{50}
\setcounter{bottomnumber}{50}

\newcommand{\omitnow}[1]{}
\newcommand{\oleg}[1]{{\it [Oleg says: #1]}}
\newcommand{\jacques}[1]{{\it [Jacques says: #1]}}

%
% Useful abbreviations
%
\newcommand{\floats}{\mathbb{F}}
\newcommand{\reals}{\mathbb{R}}

\journal{Science of Computer Programming}

\begin{document}
%\title{Functors, CPS and monads, or how to generate efficient
%code from abstract designs}
\begin{frontmatter}
\title{Multi-stage programming with functors and monads:
eliminating abstraction overhead from generic code}
\author{Jacques Carette\thanksref{1}}
\address{McMaster University,
1280 Main St. West, Hamilton, Ontario Canada L8S 4K1}
\ead{carette@mcmaster.ca}
\ead[url]{http://www.cas.mcmaster.ca/\textasciitilde carette}
\author{Oleg Kiselyov}
\address{FNMOC, Monterey, CA 93943}
\ead{oleg@pobox.com}
\ead[url]{http://pobox.com/\textasciitilde oleg/ftp/}

\thanks[1]{Supported in part by NSERC Discovery Grant RPG262084-03.}

\begin{abstract}
We use multi-stage programming, monads and OCaml's
advanced module system to demonstrate how to eliminate all
abstraction overhead while avoiding any inspection of the resulting
code.  We demonstrate this clearly with LU decomposition as a 
representative family of symbolic and numeric algorithms, and also
include an application to solving ordinary differential equations 
(via a Runge-Kutta algorithm).
We parameterize our code to a great extent
(over domain, matrix representations, determinant tracking, 
pivoting policies, result types, etc) at no run-time cost.  Because
the resulting code is generated just right and not changed afterwards,
MetaOCaml guarantees that the generated code is well-typed.
We further demonstrate that various abstraction parameters (aspects)
can be made orthogonal and compositional, even in the presence of
name-generation for temporaries, and 
``interleaving'' of aspects.  We also show how to encode some
domain-specific knowledge so that ``clearly wrong'' compositions can
be rejected at or before generation time rather than during
the compilation or running of the generated code.
\end{abstract}

\begin{keyword}
MetaOCaml \sep linear algebra \sep genericity \sep generative \sep staging
\sep Functor \sep symbolic.
\end{keyword}
% Should use either PACS or MSC scheme
\end{frontmatter}

\section{Introduction}

In high-performance symbolic and numeric computing, there is a
well-known issue of balancing between maximal performance and the
level of abstraction at which code is written.  Furthermore, already
in linear algebra, there is a wealth of different aspects that
\emph{may} need to be addressed. Implementations of the
widely used LU decomposition algorithm (which subsumes Gaussian 
Elimination (GE)) --- the running
example of our paper --- may need to account for the representation of
the matrix, whether to compute and return the determinant or rank, how
and whether search for a pivot, etc. Furthermore, current architectures
demand more and more frequent tweaks which, in general, cannot be done by the
compiler because the tweaking often involves domain knowledge. 

A survey \cite{Carette06} of
Gaussian elimination implementations in the industrial package Maple
found 6 clearly identifiable aspects and 35 different implementations of the
algorithm, as well as 45 implementations of directly related algorithms such as
LU decomposition, Cholesky decomposition, and so on.  We could
manually write each of these implementations optimizing for particular aspects
and using cut-and-paste to ``share'' similar pieces of code.
Or we can write a very generic procedure that accounts for
all the aspects with appropriate abstractions \cite{Axiom}. The
abstraction mechanisms however -- be they procedure, method or a
function call -- have a significant cost, especially for
high-performance numerical computing \cite{Carette06}. 

A more appealing approach is generative programming
\cite{Czarnecki,Veldhuizen:1998:ISCOPE,musser89generic,musser94algorithmoriented,BOOST,POOMA,ATLAS}.
The approach is not without problems, e.g., making sure that the
generated code is well-formed. This is a challenge in string-based
generation systems, which generally do not offer any guarantees and
therefore make it very difficult to determine which part of the
generator is at fault when the generated code cannot be parsed. Other
problems is preventing accidental variable capture (so-called hygiene
\cite{HygienicMacros}) and ensuring the generated code is
well-typed. Lisp-style macros, Scheme hygienic macros, the camlp4
preprocessor \cite{camlp4}, C++ template meta-programming, and Template
Haskell \cite{conf/dagstuhl/CzarneckiOST03} solve some of the above
problems. Of the widely available maintainable languages, only
MetaOCaml \cite{CTHL03,metaocaml-org}  solves all the above problems
including the well-typing of both the generator and 
the generated code \cite{TahaSheard97,TahaThesis}.

But more difficult problems remain. Is the generated code optimal? Do
we still need post-processing to eliminate common subexpressions,
fold constants, and remove redundant bindings? Is the generator readable,
resembling the original algorithm? Is the generator extensible? Are the aspects
truly modular? Can we add another aspect to it or another instance of
the existing aspect without affecting the existing ones? Finally, can
we express domain-specific knowledge, e.g., one should not attempt to
use full division when dealing with matrices of exact integers, nor is
it worthwhile to use full pivoting on a matrix over $\mathbb Q$.

MetaOCaml is \emph{generative}: generated code can only be treated as
a black box: it cannot be inspected and it cannot be post-processed
(i.e., no intensional analysis). This approach gives a stronger
equational theory \cite{Taha2000}, and avoids the danger of creating
unsoundness \cite{TahaThesis}. Furthermore, intensional code analysis
essentially requires one to insert both an optimizing compiler and an
automated theorem proving system into the code generating system
\cite{Pueschel:05,Kennedy01Telescoping,dongarra7,Veldhuizen:2004}.
While this is potentially extremely powerful and an exciting area of
research, it is also extremely complex, which means that it is
currently more error-prone and difficult to ascertain the correctness
of the resulting code.

Therefore, in MetaOCaml, code must be generated just right (see
\cite{TahaThesis} for many simple examples).  For more complex
examples, new techniques are necessary, e.g., abstract interpretation
\cite{KiselyovTaha}.  But more problems remain
\cite{Padua:MetaOcaml:04}: generating binding forms (``names'')
when generating loop bodies or conditional branches; making
continuation-passing style (CPS) code clear.  Many authors
understandably shy away from CPS code as it quickly becomes
unreadable.  But this is needed for proper name generation.
The problems of compositionality of code generators, expressing
dependencies among them and domain-specific knowledge remain.

In this paper, we report on our continued progress \cite{CaretteKiselyov05}
\footnote{We describe here an improved version of our generator
  dealing with the complete LU decomposition algorithm, as well as
  linear solving. We worked out previously missing aspects of in-place
  updates, representing permutation matrices, dealing with augmented 
  input matrix, and back-propagation. We have also changed the representation
  of domain-specific knowledge about permissible compositions of aspects.
  Also included is a careful description of all the aspects involved, as
  well as documenting our development methodology for highly parametric
  scientific software.}
in using code generation for scientific (both numeric and symbolic)
software.  We will use the algorithm family of LU decomposition 
and linear system solving as our running examples to demonstrate our
techniques.  Specifically, our contributions:
\begin{itemize}
    \item Extending a let-insertion, memoizing monad of
      \cite{MSP:PADL04,KiselyovTaha} for generating control structures
      such as loops and conditionals. The extension is non-trivial
      because of control dependencies and because
      let-insertion, as we argue, is a control effect on its own.
      : e.g.,\\
      |let x = exp in ...| has a different \emph{effect} within a
      conditional branch.
    \item Implementation of the |perform|-notation (patterned after
      the |do|-notation of Haskell) to make monadic code readable.
    \item Use of functors (including higher-order functors) to
      modularize the generator, express aspects (including results of
      various types) and \emph{insure composability of aspects} even
      for aspects that use state and have to be accounted for in many
      places in the generated code.
    \item Encode domain-specific knowledge in the generators which 
      will catch domain-specific instantiation errors at generation
      time.
    \item Provide a thorough breakdown of the family of LU decomposition
      algorithms.
\end{itemize}

The rest of this paper is structured as follows: The next section
gives an overview of the design space of LU decomposition algorithms,
as well as the methodology we follow.  Then section~\ref{CPS}
introduces code generation in MetaOCaml, the problem of name
generation, and continuation-passing style (CPS) as a general
solution.  We also present a key monad and the issues of generating
control statements. Section~\ref{functors} describes the use of
parametrized modules of OCaml to encode all of the aspects of the
LU decomposition algorithm family in completely separate,
independent modules.  Section~\ref{s:ode} describes generating
Runge-Kutta solvers for ordinary differential equations (ODE).
We briefly discuss related work in
section~\ref{related}. We then outline the future work and conclude.
Appendices give samples of the generated code (which is available in
full at \cite{metamonadsURL}).

\section{The design space}\label{design}

Before studying the details of the technologies involved in the 
implementation, it is worthwhile to carefully study the design space involved.
A preliminary study~\cite{Carette06} revealed a number of aspects of the 
family of Gaussian Elimination algorithms.  In the present work, we outline
a number of additional aspects involved in the faimily of LU decomposition 
algorithms.  These will first be presented in a somewhat ad hoc manner, 
roughly coresponding to the order in which they were ``discovered''.  These
will then be reorganized into sets of semantically related aspects, which will
form the basis of our design.  Finally, we will extract a methodology from
our experience.  

Throughout, we assume that the reader is familiar with the basic LU
decomposition algorithm, which factors a invertible matrix $A$ into a unit
lower triangular matrix $L$ and (usually) an upper triangular matrix $U$,
such that $A = LU$.  When pivoting is used, we get a unitary matrix $P$ such
that the factorization is now $A = PLU$.  The case of numeric matrices is well
covered in~\cite{Golub-vanLoan}.  When $A$ is singular, one can still get
a $PLU$ decomposition, with $L$ remaining unit lower-triangular but with a
$U$ which is no longer upper triangular but instead ``staggered''
in the upper triangle.

\subsection{Aspects}

We reuse the english word ``aspect'' for the various facets of the family
of LU decomposition algorithms.  While this use if very much in the same 
spirit as in aspect-oriented programming (AOP)~\cite{kiczales97aspectoriented},
our implementation methodology is radically different\footnote{However it seems
that we are closer to the original ideas on AOP~\cite{709568,mendhekar97rg}
which were also concerned with scientific software}.  We firmly believe that
our typed generative methodogy is much better suited to the functional
programming paradigm than attempting to graft the program-trace-based
methodology of object-oriented versions of AOP.  

At this point in time, it is better to think of aspects as purely
design-time entities.  Here we are firmly influenced by Parnas' original
view of modules and information hiding~\cite{journals/cacm/parnas72a} as well
as his view of product families \cite{journals/tse/Parnas76}, and by
Dijkstra's ideas on separation of concerns \cite{EWD:EWD447}.
To apply these principles to the study of LU decomposition, we need
to understand what are the changes between different implementations, and 
what concerns need to be addressed.  We also need to study the degree
to which these concerns are independent.

The various aspects listed below all come from variations found in actual
implementations (in various languages and settings).

\newcounter{naspects}
\begin{enumerate}
	\item \textbf{Domain}: In which (algebraic) domain do the matrix
		elements belong to.  Some implementations were very specific
		($\mathbb{Z}, \mathbb{Q}, \mathbb{Z_p}, 
		\mathbb{Z}_p\left[\alpha_1,\ldots,\alpha_n\right], 
		\mathbb{Z}\left[x\right]$, $\mathbb{Q}\left(x\right)$, 
		$\mathbb{Q}\left[\alpha\right]$, and floating point numbers 
		($\floats$) for 
		example), while others were generic for elements of a field,
		multivariate polynomials over a field, or elements of a division ring
		with possibly undecidable zero-equivalence.  In the roughly 85 pieces
		of code we surveyed, 20 different domains were encountered.
	\item \textbf{Representation of the matrix}: Whether the matrix
		was represented as an array of arrays, a one-dimensional array,
		a hash table, etc.  For the case of a one-dimensional array,
		whether indexing was done in C or Fortran style.  Additionally,
		if a particular representation had a special mechanism for efficient
		row exchanges, this was sometimes used.
	\item \textbf{Fraction-free}: Whether the 
		algorithm is allowed to use unrestricted division, or only
		exact (remainder-free) division.
	\item \textbf{Length measure (for pivoting)}:  For stability reasons
		(whether numerical or coefficient growth), if a domain possesses
		an appropriate length measure, this was sometimes used to choose
		an ``optimal'' pivot.  Not all domains have such a measure.
	\item \textbf{Full division}: Whether the input domain supports full
		division (i.e. is a \emph{field} or pretends to be ($\floats$)) 
		or only exact division (i.e. a \emph{division ring}).
	\item \textbf{Domain normalization}: Whether the arithmetic operations
		of the base domain keep the results in normal form, or whether
		an extra normalization step is required.  For example, some 
		representations of polynomials require an extra step for
		zero-testing.
	\item \textbf{Output choices}:  Whether just the reduced matrix
		(the `U' factor), both L and U, as well as
		the rank, the determinant, and/or the sequence of 
		pivots is to be returned.  For example, Maple's
		\texttt{LinearAlgebra:-LUDecomposition} routine has
		$2^6 + 2^5 + 2^2 = 100$ possible outputs, depending on whether
		one chooses a $PLU$, $PLUR$ or \emph{Cholesky} 
		decomposition.  We chose to only consider $PLU$ for now.
	\item \textbf{Rank}: Whether to explicitly track the rank of the matrix
		as the algorithm proceeds.
	\item \textbf{Determinant}:  Whether to explicitly track the determinant
		of the matrix as the algorithm proceeds.
	\item \textbf{Code representation}: Whether we are actually generating
		code or we are in fact representation the algorithm in ``direct
		style'', albeit with a lot of abstraction overhead.
	\item \textbf{Zero-equivalence}: Whether the 
		arithmetic operations require a specialized zero-equivalence 
		routine needs to be used.  It turns out that for certain classes
		of \emph{expressions}, it is convenient to use a zero-equivalence
		test which is separate from the domain normalization.  This is
		usually the case when zero-equivalence is formally undecidable
		but semi-algorithms or probabilistic algorithms do exist.
		See~\cite{ZhCaJeMo06a} for an example.
    \item \textbf{Pivoting}: Whether to use no, 
        column-wise, or full pivoting.
    \item \textbf{Augmented Matrices}: Whether all or only some
      columns of the matrix participate in elimination.
	\item \textbf{Pivot representation}: Whether the pivot is represented
	  as a list of row (and/or column) exchanges, as a unitary matrix,
	  or as a permutation vector.
  \item \textbf{Lower matrix}: whether the matrix $L$ should be tracked
	  as the algorithm proceeds, reconstructed at the end of the
	  algorithm, or not tracked at all.
  \item \textbf{Input choices}: Whether the input matrix is an augmented
	  matrix or not, and if so, where is the end of the main matrix.
  \item \textbf{Packed}: Whether the output matrices $L$ and $U$ are
	  packed into a single matrix for output.
\setcounter{naspects}{\value{enumi}}
\end{enumerate}

There are other aspects which we have not (yet) implemented, but that have
been seen in different implementations.  In particular,
\begin{enumerate}
\setcounter{enumi}{\value{naspects}}
\item \textbf{Logging}: This is a classical cross-cutting concern, which
is also useful in this context.  Of course, there are many different
parts of the algorithm which could be ``logged'', and ensuring fine control
over this is a thorny issue.
\item \textbf{Sparsity}: If a matrix is known to be sparse, the traversal
(at least) should be sparse.  Additionally, it would better still if
care was taken to minimize fill-in.
\item \textbf{Other structure}: For example, if a matrix is known in
advance to be real symmetric tri-diagonal, then LU decomposition can be
done in $O(n^2)$ time (instead of $O(n^3)$; the main difficulty in 
implementing this kind of reduction is the fact that this version of 
LU requires $O(n)$ extra storage (unlike all other versions).
\item \textbf{Warnings}: In the case of LU decomposition of a matrix over
a domain for which we only have a heuristic algorithm for zero
testing, it is customary to issue a warning (or otherwise log) when 
a potentially zero pivot is chosen.
\item \textbf{In-place}: When it is possible, it would be good to have the
option to re-use the input matrix as the storage space for the output.
\item \textbf{Error-on-singular}: There are instances where we are only
interested in an LU decomposition for a non-singular matrix, and would
prefer to get an exception when a singular matrix is encountered.
\setcounter{naspects}{\value{enumi}}
\end {enumerate}

Clearly the above $\thenaspects$ aspects have a rather complex 
dependency graph.  While some aspects are purely orthogonal to others, 
most are inter-dependent.  For example, if one wants the determinant of
the matrix as part of the output, then the determinant has to be tracked
during the algorithm; a priori, it is possible to make incompatible choices,
and our implementation has to insure that this is in fact impossible.
More precisely, our goal is to make sure that this is detected long
before the algorithm could be run.

There are other kinds of dependencies: for example, whether or not a 
domain has a length measure will influence pivoting, \emph{if} pivoting
is to be performed.  While there are no circular dependencies, there is
no clear directed graph of dependencies.  In other words, one cannot 
simply use product, co-products, and compositions of the various aspects 
to build the resulting algorithm.  We will get back to this point when
we describe the details of the implementation.

\subsection{Organizing aspects}

\subsection{Methodology}

\section{Generating binding statements, CPS, and monad}\label{CPS}
%%%%%%%%%%%%%%%%%%%%%%%%%%%%%%%%%%%%%%%%%%%%%%%

We build code generators out of primitive ones using code generation 
combinators. MetaOCaml, as an instance of a multi-stage
programming system \cite{TahaThesis}, provides exactly the needed
features: to construct a code expression, to combine them, and to
execute them. The following shows the simplest code generator |one|,
and the simplest code combinators\footnote{%
$\Longrightarrow$ under an expression shows the result of its evaluation}:

\begin{code}
let one = .<1>. and plus x y = .<.~x + .~y>.
let simplest_code = let gen x y = plus x (plus y one) in
  .<fun x y -> .~(gen .<x>. .<y>.)>.
\evalresult{.<fun x_1 -> fun y_2 -> (x_1 + (y_2 + 1))>.}
\end{code}

We use MetaOCaml brackets |.<...>.| to generate code expressions,
i.e., to construct future-stage computations. MetaOCaml provides only
one mechanism of combining code expressions, by splicing one
piece of code into
another. The power of that operation, called escape |.~|, comes from
the fact that the expression to be spliced in (inlined) can be
computed: escape lets us perform an arbitrary immediate code-generating
computation \emph{while} we are
building the future-stage computation. The immediate computation in
|simplest_code| is the evaluation of the function |gen|, which in turn
applies |plus|. The function |gen| receives code expressions |.<x>.|
and |.<y>.| as arguments. At the generating stage, we can manipulate
code expressions as (opaque) values. The function |gen| returns a code
expression, which is inlined in the place of the escape. MetaOCaml can
print out code expressions, so we can see the final generated code. It
has no traces of |gen| and |plus|: their applications are done at the
generation stage.

The final MetaOCaml feature, |.!| (pronounced ``run'') 
executes the code expression: |.! simplest_code| is a function of two
integers, which we can apply: |(.! simplest_code) 1 2|. The original
|simplest_code| is not a function on integers -- it is a code
expression.

To see the benefit of code generation, we notice that we can easily
parameterize our code:

\begin{code}
let simplest_param_code plus one =
  let gen x y = plus x (plus y one) in
  .<fun x y -> .~(gen .<x>. .<y>.)>.
\end{code}
and use it to generate code that operates on integers, floating point
numbers or booleans -- in general, any domain that implements |plus|
and |one|:
\begin{code}
let plus x y = .<.~x +. .~y>. and one = .<1.0>. in
  simplest_param_code plus one
let plus x y = .<.~x || .~y>. and one = .<true>. in
  simplest_param_code plus one
\end{code}
Running the former expression yields the function on |float|s, whereas
the latter expression is the code expression for a boolean function.
This clearly shows the separation of concerns, namely of that for domain
operations.

Let us consider a more complex expression:
\begin{code}
let param_code1 plus one =
  let gen x y = plus (plus y one) (plus x (plus y one)) in
  .<fun x y -> .~(gen .<x>. .<y>.)>.
\end{code}
with two occurrences of |plus y one|,
which may be quite a complex computation and so we would rather not do
it twice. We may be tempted to rely on the compiler's
common-subexpression elimination optimization. When the generated code is
very complex, however, the compiler may overlook common subexpressions.  Or the
subexpressions may occur in such an imperative context where the compiler
might not be able to determine if lifting them is sound. So, being
conservative, the optimizer will leave the duplicates as they are. 
We may attempt to eliminate subexpressions as follows: 
\begin{code}
let param_code1' plus one =
  let gen x y = let ce = (plus y one) in  plus ce (plus x ce) in
  .<fun x y -> .~(gen .<x>. .<y>.)>.
param_code1' plus one
\evalresult{.<fun x_1 -> fun y_2 -> ((y_2 + 1) + (x_1 + (y_2 + 1)))>.}
\end{code}
However,
the result of |param_code1' plus one| still exhibits duplicate
sub-expressions.  Our |let|-insertion optimization saved the
computation at the generating stage.  We need a combinator that
inserts the |let| expression in the generat\emph{ed} code. We need a
combinator |letgen| to be used as
\begin{code}
let ce = letgen (plus y one) in plus ce (plus x ce)
\end{code}
yielding the code like 
\begin{code}
.<let t = y + 1 in t + (x + t)>.
\end{code}
But that seems impossible because |letgen exp| has to generate
the expression |.<let t = exp in body>.| but |letgen| does not
have the |body| yet. The body needs a temporary identifier |.<t>.|
that is supposed to be the result of |letgen| itself.  Certainly
|letgen| cannot generate only part of a let-expression, without the
|body|, as all generated expressions in MetaOCaml are well-formed and
complete.

The key is to use continuation-passing style (CPS). Its benefits were
first pointed out by \cite{Bondorf:92} in the context of partial
evaluation, and extensively used by \cite{MSP:PADL04,KiselyovTaha} for
code generation. Like \cite{conf/pepm/BondorfD94}, we use this in the 
context of writing a cogen by hand.  Now, |param_code2 plus one| gives us the
desired code.

\begin{code}
let letgen exp k = .<let t = .~exp in .~(k .<t>.)>.
let param_code2 plus one =
  let gen x y k = letgen (plus y one)
                         (fun ce -> k (plus ce (plus x ce)))
  and k0 x = x
  in .<fun x y -> .~(gen .<x>. .<y>. k0)>.
param_code2 plus one
\evalresult{.<fun x_1 -> fun y_2 -> let t_3 = (y_2 + 1) in (t_3 + (x_1 + t_3))>.}
\end{code}

\subsection{Monadic notation, making CPS code clear}\label{monadicnotation}

Comparison of the let-insertion in the generator
\begin{code}
let ce = (plus y one) in  plus ce (plus x ce)
\end{code}
with the corresponding code generating let-insertion for the future
stage
\begin{code}
letgen (plus y one) (fun ce -> k (plus ce (plus x ce)))
\end{code}
clearly shows the difference between  direct-style and CPS code.
What was |let ce = init in ...| in direct style became
|init' (fun ce -> ...)| in CPS. For one, |let| became
``inverted''. For another, what used to be an expression that yields
a value, |init|, became an expression that takes an extra argument,
the continuation, and invokes it. The differences look negligible in
the above example. In larger expressions with many let-forms, the
number of parentheses around |fun| increases, the need to add and
then invoke the |k| continuation argument become increasingly annoying. The
inconvenience is great enough for some people to explicitly avoid CPS
or claim that numerical programmers (our users) cannot or will not
program in CPS. Clearly a better notation is needed.

The |do|-notation of Haskell \cite{Haskell98Report} shows that it is possible
to write CPS code in a conventional-looking style. The
|do|-notation is the notation for monadic code \cite{moggi-notions}.
Not only can monadic code represent CPS \cite{Filinski:Representing},
it also helps in composability by offering to add different
layers of effects (state, exception, non-determinism, etc) to the
basic monad \cite{liang-interpreter} in a controlled way.

A monad \cite{moggi-notions} is an abstract data type representing
computations that yield a value and may have an \emph{effect}.
The data type must have at least two operations, |return| to build
trivial effect-less computations and |bind| for combining
computations. These operations must satisfy \emph{monadic laws}:
|return| being the left and the right unit of |bind| and |bind| being
associative. Figure~\ref{ourmonad} defines the monad used throughout
the present paper and shows its implementation.

\begin{figure}
\begin{code}
type ('p,'v) monad = 's -> ('s -> 'v -> 'w) -> 'w
    constraint 'p = <state : 's; answer : 'w; ..>

let ret (a :'v) : ('p,'v) monad = fun s k -> k s a
let bind a f = fun s k -> a s (fun s' b -> f b s' k)
let fetch s k = k s s  and  store v _ k = k v ()

let k0 _ v = v
let runM m = fun s0 -> m s0 k0 

let l1 f = fun x     -> perform t <-- x; f t
let l2 f = fun x y   -> perform tx <-- x; ty <-- y; f tx ty

let retN a = fun s k -> .<let t = .~a in .~(k s .<t>.)>.

let ifL test th el = ret .< if .~test then .~th else .~el >.
let ifM test th el = fun s k -> 
  k s .< if .~test then .~(th s k0) else .~(el s k0) >.
\end{code}
\caption{Our monad}\label{ourmonad}
\end{figure}

Our monad represents two kinds of computational effects: reading and
writing a computation-wide state, and control effects. The latter are
normally associated with exceptions, forking of computations, etc. --
in general, whenever a computation ends with something other than
invoking its natural continuation in the tail position. In our case
the control effects manifest themselves as code generation.

In Figure~\ref{ourmonad}, the monad (yielding values of the type |v|)
is implemented as a function of two
arguments: the state (of type |s|) and the continuation. The
continuation receives the current state and the value, and
yields the answer of the type |w|.  The monad is polymorphic over the
three type parameters, which would require |monad| to be a type
constructor with three arguments. When we use this monad for code
generation, we will need yet another type variable, environment
classifier \oleg{cite Walid03?}. With type constructors taking more
and more arguments, it becomes more difficult to read and write
types -- which we will be doing extensively when writing module
signatures in Section XXX. The fact that OCaml renames all type
variables when printing out types confuses matters further. An elegant
solution to these kinds of problems has been suggested by 
Garrigue on the Caml mailing list 
(cited from
\url{http://groups.google.com/group/fa.caml/msg/e80b1245702d6b24}
no date, no exact ref). We use a single type parameter |'p| to
represent all parameters of our monad (all parameters but the type of
the monadic value |'v|). The type variable |'p| is constrained to be
the type of an object with methods (fields) |state| and |answer|. The
object may include more fields, represented by |..|. Values of that
type are not part of our computations and need not exist. We merely
use the object type, as an convenient way to specify extensible
\emph{type-level} records in OCaml.     

Our monad could be implemented in other ways. Except for the code in
Figure~\ref{ourmonad}, the rest of our code treats the monad as a
truly abstract data type. The implementation of the basic monadic
operations |ret| and |bind| is conventional and clearly satisfies the
monadic laws. Other monadic operations construct computations that do
have specific effects.  Operations |fetch| and |store v| construct
computations that read and write the state.

The operation |retN a| is the let-insertion operation, whose simpler
version we called |letgen| earlier. It is the first computation with
a control effect: indeed, the result of |retN a| is \emph{not} the
result of invoking its continuation |k|. Rather, its result is a |let|
code expression. Such a behavior is symptomatic of control operators
(in particular, |abort|).

Finally, |runM| runs our monad, that is, given the initial state,
performs the computation of
the monad and returns its result, which in our case is the code
expression. We run the monad by passing it the initial state and the
initial continuation |k0|. We can now re-write our |param_code2|
example of the previous section as |param_code3|.
\begin{code}
let param_code3 plus one =
  let gen x y = bind (retN (plus y one)) (fun ce -> 
                ret (plus ce (plus x ce)))
  in .<fun x y -> .~(runM (gen .<x>. .<y>.) ())>.
\end{code}
% param_code3 plus one;;
%
That does not seem like much of an improvement. With the help of
camlp4 pre-processor, we introduce the |perform|-notation \cite{metamonadsURL},
patterned after the |do|-notation of Haskell (see App.~\ref{app:perform}).
\begin{code}
let param_code4 plus one =
  let gen x y = perform ce <-- retN (plus y one);
                        ret (plus ce (plus x ce))
  in .<fun x y -> .~(runM (gen .<x>. .<y>.) ())>.
\end{code}
The function
|param_code4|, written in the |perform|-notation, is equivalent to
|param_code3| -- in fact, the camlp4 preprocessor will convert the
former into the latter. And yet, |param_code4| looks far more
conventional, as if it were indeed in direct style.

\subsection{Generating control statements}
We can write operations that generate code other than let-statements,
e.g., conditionals: see |ifL| in Figure~\ref{ourmonad}. The function |ifL|, 
albeit straightforward, is not as general as we wish: its arguments are
already generated pieces of code rather than monadic values. We
``lift it'':
\begin{code}
let ifM' test th el = perform
  testc <-- test; thc <-- th; elc <-- el;
  ifL testc thc elc
\end{code}
We define functions |l1|,
|l2|, |l3| (analogues of |liftM|, |liftM2|, |liftM3| of Haskell) 
to make such a lifting generic. However we also need
another |ifM| function, with the same
interface (see Figure~\ref{ourmonad}).
The difference between them is
apparent from the following example:
\begin{code}
let gen a i = ifM' (ret .<(.~i) >= 0>.) 
                   (retN .<Some (.~a).(.~i)>.) (ret .<None>.)
 in .<fun a i -> .~(runM (gen .<a>. .<i>.) ())>.
\evalresult{.<fun a_1 i_2 ->}
\textcolor{red}{      let t_3 = (Some a_1.(i_2)) in if (i_2 >= 0) then t_3 else None>.}
let gen a i = ifM (ret .<(.~i) >= 0>.) 
                  (retN .<Some (.~a).(.~i)>.) (ret .<None>.)
 in .<fun a i -> .~(runM (gen .<a>. .<i>.) ())>.
\evalresult{.<fun a_1 i_2 ->}
\textcolor{red}{      if (i_2 >= 0) then let t_3 = (Some a_1.(i_2)) in t_3 else None>.}
\end{code}
%
If we use |ifM'| to generate guarded array access code, the let-insertion
happened \emph{before} the if-expression, that is, before the test that
the index |i| is positive. If |i| turned out
negative, |a.(i)| would generate an out-of-bound array access
error. On the other hand, the code with |ifM| accesses the array only
when we have verified that the index is non-negative. This example
makes it clear that the code generation (such as the one in |retN|) is 
truly an effect and we have to be clear about the sequencing of
effects when generating control constructions such as conditionals.
The form |ifM| handles such effects correctly. 

We need similar operators for other OCaml control forms: for
generating sequencing, case-matching statements and |for|- and |while|-loops.
\begin{code}
let seqM a b = fun s k -> 
  k s .< begin .~(a s k0) ; .~(b s k0) end >.

let whileM cond body = fun s k -> 
  k s .< while .~(cond) do .~(body s k0) done >.

let matchM x som non = fun s k -> k s .< match .~x with
           | Some i -> .~(som .<i>. s k0)
           | None   -> .~(non s k0) >.

let genrecloop gen rtarg = fun s k -> 
  k s .<let rec loop j = .~(gen .<loop>. .<j>. s k0) in loop .~rtarg>.
\end{code}



\section{Aspects and Functors}\label{functors}

The monad represents fine-scale code generation. We need tools for
larger-scale modularization; we can use any abstraction
mechanisms we want to structure our code generators, as long as none
of those abstractions infiltrate the generated code.


\jacques{This whole section needs to be edited to take into account
the different organization of the material}.

\subsection{Domains}
Reviewing the various aspects outlined in section~\ref{design}, the
simplest parametrization is to make the domain abstract.  As it turns
out, we need the following to exist in our domains: $0$, $1$, $+$,
$*$, (unary and binary) $-$, at least \emph{exact} division,
normalization, and potentially a relative size measure. The simplest
case of such domain abstraction is |param_code1|.  There,
code-generators such as |plus| and |one| were passed as arguments. We
need far more than two parameters, so we have to group them. Instead
of the grouping offered by regular records, we use OCaml
\emph{structures} (i.e., modules) so we can take advantage of
extensibility, type abstraction and constraints, and especially
parameterized structures (\emph{functors}).  We define the type of the
domain, the signature |DOMAIN|, which different domains must satisfy:

\begin{code}
type domain_kind = Domain_is_Ring | Domain_is_Field

module type DOMAIN = sig
  type v
  val kind : domain_kind
  val zero : v
  val one : v
  val plus : v -> v -> v
  val times : v -> v -> v
  val minus : v -> v -> v
  val uminus : v -> v
  val div : v -> v -> v
  val better_than : (v -> v -> bool) option
  val normalizer : (v -> v) option
end 

module IntegerDomain : DOMAIN = struct
    type v = int
    let kind = Domain_is_Ring
    let zero = 0 and one = 1
    let plus x y = x + y  and  minus x y = x - y
    let times x y = x * y and  div x y = x / y
    let uminus x = -x
    let normalizer = None
    let better_than = Some (fun x y -> abs x > abs y)
end
\end{code}

One particular domain instance is |IntegerDomain|. The notation\\
|module IntegerDomain : DOMAIN| makes the compiler verify that our
|IntegerDomain| is indeed a |DOMAIN|, that is, satisfies the required
signature. The constraint |DOMAIN| may be omitted; in that case, the
compiler will verify the type when we try to use that structure as a
|DOMAIN|. In any case, the errors such as missing ``methods'' or
methods with incorrect types will be caught statically, even
\emph{before} any code generation takes place. The variant
|Domain_is_Ring| of |IntegerDomain.domain_kind| encodes a semantic constraint 
that the full division
is not available. While the |DOMAIN| type may have looked daunting to
some, the implementation is quite straightforward.  Other domains such
as |float| and arbitrary precision exact rational numbers |Num.num|
are equally simple.

A more complex domain is |Zp| of the field of integers in prime
modulus:
\begin{code}
module ZpMake(P:sig val p:int end) = struct
    type v = int
    let kind = Domain_is_Field
    let zero = 0 and one = 1
    let plus x y = (x + y) mod P.p
    let times x y = (x * y) mod P.p
    \dots
    let normalizer = None and better_than = None
    let () = assert (is_prime P.p)
end
\end{code}
This domain is parametrized by an integer number |p|, the
(modulus?). To be more precise, the structure |ZpMake| is
parameterized over another structure of the type described by
signature P; that argument structure has one field, the int value |p|.
Such a parameterized structure (or, a function from structures to
structures) is a \emph{functor}. The result of |ZpMake| is a domain that is a
field with no defined order. Hence
|normalizer| and |better_than| are set to |None|. |Zp| forms
a field only when |p| is prime. Therefore, we must make this check,
in the last line of the above code. That line differs from the other 
bindings in |ZpMake| in that it neither defines a function, such as
|plus|, nor binds a value, such as |zero|. Therefore, the non-value
expression |assert (is_prime P.p)| will be evaluated
when the corresponding module is instantiated, like the following
\begin{code}
module Z19 = ZpMake(struct let p = 19 end)
\end{code}
We will call such an expression an initializing expression.  if we
replace |p = 19| with |p = 9| above, we receive an error, a run-time
error. It is raised however as we instantiate and combine modules that
will eventually make the generator.  Although the error is reported at
run-time rather than compile time as one might have hoped, the error
is raised when \emph{generating the generator} -- well before the
generation of the target code could begin. In our code we make
extensive use of these so-called preflight checks performed as part of
module initialization. The checks seem to offer a good compromise:
they are dynamic and so do not require complicated type systems; on
the other hand, the checks are performed quite early, when buiding
code generators, and so ensure that no code violating the
corresponding \emph{semantic} constraints will be generated. Although
some may frown on the use of module initializing expressions (since
that requires careful attention to sharing and multiple instantiations
of a module), these concerns do not apply in our case: our preflight
tests are all idempotent and maintain no state.

\subsection{Abstract code and lifted domains}
We will be using MetaOCaml operations to build code expressions, values
of the type |('a,'v) code|. However, we wish to make our generators
abstract over the particular code representations. Therefore, in the
following Sections (in the rest of the paper?) rather than 
MetaOCaml brackets and escapes, we will be using code combinators
described below. These combinators are functions that produce and
combine values of the type of abstract code. 

This abstraction lets us vary the output form of our generated LU
programs. We may use thunks rather than MetaOCaml |('a,'v) code|
values, for benchmarking purposes. We could use code combinators that
produce C or Fortran code.

Our code combinator library includes operations to create, dereference
and mutate reference cells, compare values, manipulate values used
for indexing arrays. The following is the implementation of a sample
combinators, in the case of the abstract code being MetaOCaml code values. 
\begin{code}
let lift x = .< x >.

let cunit = .< () >.
let unitL = fun s k -> k s .< () >.

let liftRef x = .< ref .~x >. 
let liftGet x = .< ! .~x >. 
let liftPair x = (.< fst .~x >., .< snd .~x >.)

module Logic = struct
  let notL a        = .< not .~a >.
  let equalL a b    = .< .~a = .~ b >.
  let notequalL a b = .< .~a <> .~ b >.
  let andL a b     = .< .~a && .~b >. 
end

module Idx = struct
  let zero = .< 0 >. and one = .< 1 >. and minusone = .< -1 >.
  let succ a = .< .~a + 1 >.
  let pred a = .< .~a - 1 >.
  let less a b = .< .~a < .~b >.
  let uminus a = .< - .~a >.
  let add a b = .< .~a + .~b >.

let update a f = let b = f (liftGet a) in .< .~a := .~b >.
let assign a b = .< .~a := .~b >.
let apply  f x = .< .~f .~x >.
let updateM a f = ret (update a f)
let assignM a b = ret (assign a b)
let applyM  f x = ret (apply f x)
\end{code}


\noindent  The types above are
generally lifted twice: once from the value domain |v| to the code
domain |'a vc|, and once more from values to monadic computations
|('p,'a vc) monad|. 

We now convert the domains of values from teh previous section into
lifted domains, of abstract code values. We define the signature
|DOMAINL| 
\begin{code}
module S(T: sig type ('a, 'b) rep  end) = struct
open T
module type DOMAINL = sig
  include DOMAIN
  type 'a vc = ('a,v) rep
  val zeroL : 'a vc
  val oneL : 'a vc
  val ( +^ ) : 'a vc -> 'a vc -> 'a vc
  val ( *^ ) : 'a vc -> 'a vc -> 'a vc
  val ( -^ ) : 'a vc -> 'a vc -> 'a vc
  val uminusL : 'a vc -> 'a vc
  val divL : 'a vc -> 'a vc -> 'a vc
  val better_thanL : ('a vc -> 'a vc -> ('a,bool) rep) option
  val normalizerL : ('a vc -> 'a vc) option
end 
\end{code}
(part of the larger module |Domains_sig|) which is parameterised by
the type of the abstract code, |('a, 'b) rep|. The line |include DOMAIN|
says that lifted domains include all members of non-lifted domains,
including the initializing expressions with pre-flight checks.
non-lifted 

The following is a particular instance of |DOMAINL|, the lifted
version of |IntegerDomain| of the previous section, in the particular
case of abstract code being MetaOCaml code values |('a, 'b) code|:
\begin{code}
module T = Domains_sig.S(struct type ('a, 'b) rep = ('a, 'b) code end)
open T
module IntegerDomainL = struct
    include IntegerDomain
    type 'a vc = ('a,v) code
    let zeroL = .< 0 >.  
    let oneL = .< 1 >. 
    let (+^) x y = .<.~x + .~y>. 
    let ( *^ ) x y = .<.~x * .~y>.
    let ( -^ ) x y = .<.~x - .~y>.
    let uminusL x = .<- .~x>.
    let divL x y = .<.~x / .~y>. 
    let normalizerL = None
    let better_thanL = Some (fun x y -> .<abs .~x > abs .~y >. )
end
\end{code}


\subsection{Containers}

The signature |CONTAINER2D| specifies that a container must provide
functions |dim1| and |dim2| to extract the dimensions, functions |getL|
to generate container getters, the cloning
generator |copy| and functions that generate code for row and column
swapping. The inclusion of these functions in the signature of all
containers makes it simpler to optimize the relevant functions
depending on the actual representation of the container while not
burdening the users of containers with efficiency details.

\begin{code}
module type CONTAINER2D = sig
  module Dom:DOMAINL
  type contr
  type 'a vc = ('a,contr) rep
  type 'a vo = ('a,Dom.v) rep
  val getL : 'a vc -> ('a,int) rep -> ('a,int) rep -> 'a vo
  val dim1 : 'a vc -> ('a,int) rep
  val dim2 : 'a vc -> ('a,int) rep
  val mapper : ('a vo -> 'a vo) option -> 'a vc -> 'a vc
  val copy : 'a vc -> 'a vc
  val init : ('a,int) rep -> ('a, int) rep -> 'a vc
  val augment : 'a vc -> ('a,int) rep -> ('a, int) rep -> 'a vc ->
                ('a, int) rep -> 'a vc
  val identity : ('a,int) rep -> ('a, int) rep -> 'a vc
  val swap_rows_stmt : 'a vc -> ('a, int) rep -> ('a, int) rep -> 
                       ('a,unit) rep
  val swap_cols_stmt : 'a vc -> ('a, int) rep -> ('a, int) rep -> 
                       ('a,unit) rep
  val row_head : 'a vc -> ('a, int) rep -> ('a, int) rep -> 'a vo
  val col_head_set : 'a vc -> ('a,int) rep -> ('a,int) rep -> 'a vo -> 
            ('a,unit) rep
end
\end{code}

The type of our containers include the lifted domain |Dom| as one of
the component. This is quite convenient since operations on containers
are usually accompanied by the operations on the retrieved and stored
values. Our containers are parametric over a |DOMAINL|, i.e., functors
from a |DOMAINL| module to the actual implementation of a
container. For example, the following functor defines a matrix
container represented as arrays of rows.

\begin{code}
module GenericArrayContainer(Dom:DOMAINL) =
  struct
  module Dom = Dom
  type contr = Dom.v array array (* Array of rows *)
  type 'a vc = ('a,contr) code
  type 'a vo = ('a,Dom.v) code
  let getL x n m = .< (.~x).(.~n).(.~m) >.
  let dim2 x = .< Array.length .~x >.       (* number of rows *)
  let dim1 x = .< Array.length (.~x).(0) >. (* number of cols *)
  \dots
\end{code}
%
The accompanying code includes containers whose
elements are stored in a 1D array, in a
row-wise (C-like) or column-wise (Fortran-like) modes.


\subsection{Maintaining the extensible state}

Various aspects of our LU generator such as determinant
computation, rank, permutation list may need to keep a state during
the code generation. For example, the initializing section of the LU
generator will invoke the |decl| of the determinant aspect so that the
latter may generate let-binding for reference cells that track the
magnitude and the sign of the determinant. The generator of the
row-swapping code will notify the determinant aspect so that the
latter may insert the code that flips the sign of the tracked
determinant. Finally, the final section of the generator will ask the
determinant aspect to generator code producing the final, signed value
of the determinant. Clearly the determinant aspect needs to maintain
state from one invocation of its methods to another; this state should
include the `names' of mutable variables accumulating the sign and the
magnitude of the determinant.

The simplest method of keeping the state of the generator is by using
mutable variables private to the Deteminant module (aspect). That,
however, makes our aspects stateful. Although we are generating
imperative code, we would like to keep our generators stateless and
pure functional, for ease of comprehension and reasoning. Also, if a
module is stateless, we do not have to worry about sharing, aliasing
and multiple instances. Our main program may include several LU
generators, which may refer to the Determinant aspect -- which may be
present in one shared instance or in several. It is irrelevant for us
which is which. With stateful modules, the issues of aliasing or
separate instantiation are the source of very subtle problems.

We chose therefore a different way of maintaining the generator state,
using a monad. We need monad anyway for let-insertion; we add state as
an additional monad parameter (Fig.~\ref{ourmonad}) and monadic
actions |fetch| and |store| to access that monadic state. The state is
threaded throughout the entire code-generation computation. 

This monadic state has to accommodate several pieces of state, for
various aspects. We should be able to add a new aspect, which may need
to keep its own state as part of teh overall monadic state, without
modifying or even recompiling the rest of the code. Thus our monadic
state should be extensible. One way of achieving is via an extensible
record, or OCaml object. Each aspect will have its own field; record
subtyping will assure modularity. Alas, this approach makes it
difficult to create the initial state, to pass to the monad's |runM|
method. We should know the names of all fields and should know the
proper initial value for these fields. This breaks the modularity.

A better approach of property list has been suggested my the MLTon
team \oleg{cite MLTon}.  Encoding an object by its dual, list of
polymorphic variants.  Property list represents an extensible
object. The initial object is just the empty list. Alas, the MLton's
approach does not literally apply to us. First of all, due to
the environment classifiers, we need universes parameterized by
classifiers. Also, the property names are generated automatically
in MLton approach, based on generativity of exceptions or reference
cells. That means if some particular aspect happens to be included
twice, the state will be incompatible. We will have to worry about
sharing or the lack of sharing of our aspects. We would rather
prefer our aspects to be stateless and work the same way whether two
instances of the aspect are shared or not. Fortunately, OCaml offers a
way to build open unions using polymorphic variants. Each variant is
identified by a manifest tag, e.g., `Tdet, and may include a
value. The tags are not generative and provide the form of manifest
naming (somewhat similar to symbols of Lisp and Scheme).

Below is the implementation of our open records with manifest
naming, functions |orec_store| to add a new field to the open record
and |orec_find| to obtain the value associated with a particular field
name. Each `field' is characterized by a triple: injection, 
projection functions and the string name. The latter is used for 
printing error messages. For example, for the
determinant tracking aspect, this triple has the form
\begin{code}
  let ip = (fun x -> `TDet x), 
           (function `TDet x -> Some x | _ -> None), 
           "Det"
\end{code}
We combine these functions with monadic
actions to access monadic state and so obtain 
|mo_extend| and |mo_lookup| to store and retrieve one particular value
of the monadic state.
\begin{code}
let rec lookup ((_,prj,_) as ip) = 
   function [] -> raise Not_found
   | (h::t) -> (match prj h with Some x -> x | _ -> lookup ip t)

let orec_store ((inj,_,name) as ip) v s =
  let () = 
    try let _ = lookup ip s in 
        failwith ("The field of an open record is already present: " ^ name)
    with Not_found -> () in
  (inj v)::s

let orec_find ((_,_,name) as ip) s =
  try lookup ip s 
  with Not_found -> failwith ("Failed to locate orec field: " ^ name)

let mo_extend ip v = 
  perform s <-- fetch; store (orec_store ip v s)

let mo_lookup ip =
  perform s <-- fetch; ret (orec_find ip s)
\end{code}

Currently, we check at generator-time that one cannot add a slot which
already exists (already the part of the state) nor we can obtain the
value of a non-existing slot. These problems may occur if our genetor
is wrong (calls fin method before decl method of an aspect). It is
possible to make these checks static (TFP2007, state-variable monad?)
state-changing monad...




\subsection{Determinant aspect}

To track determinant we should be able to generate code
for: defining variables used for tracking (|decl|),
updating the sign or the absolute
value of the determinant, converting the tracking state
to the final determinant value of the type |outdet|. LU of a
floating-point matrix with no determinant tracking uses the
instantiation of |DETERMINANT| where |outdet| is |unit| and all the
functions of that module generate no code. For integer matrices, we
have to track some aspects of the determinant, even if we don't output
it. The determinant tracking aspect is complex because tracking
variables, if any, are to be declared at the beginning of LU; the sign
of the determinant has to be updated on each row or column
permutation; the value of the determinant should be updated per each
pivoting. We use |lstate| to pass the tracking state, e.g., a piece of
code for the value of the type |Dom.v ref|, among
various determinant-tracking functions. The |lstate| is a part of the
overall monadic state. 

Other aspects follow a similar outline

\begin{code}
module type DETERMINANT = sig
  type tdet = C.Dom.v ref
  type 'a lstate
  type ('pc,'v) lm = ('pc,'v) cmonad
    constraint 'pc = <state : [> `TDet of 'a lstate ]; classif : 'a; ..>
  type ('pc,'v) om = ('pc,'v) omonad
    constraint 'pc = <state : [> `TDet of 'a lstate ]; classif : 'a; ..>
  val decl : unit -> ('b,unit) lm (* could be unit rather than unit code...*)
  val upd_sign  : unit -> ('b,unit) om
  val zero_sign : unit -> ('b,unit) lm
  val acc       : ('a,C.Dom.v) abstract -> (<classif : 'a; ..>,unit) lm
  val get       : unit -> ('b,tdet) lm
  val set       : ('a,C.Dom.v) abstract -> (<classif : 'a; ..>,unit) lm
  val fin       : unit -> ('b,C.Dom.v) lm
end
\end{code}

We have two instances of |Det|. The first one is used when the user
requested that no determinant tracking is needed, and so no code is
generated. The methods just return without executing any
code-generation action.
\begin{code}
module NoDet =
  struct
  type tdet = C.Dom.v ref
  type 'a lstate = unit
  let decl () = unitL
  let upd_sign () = ret None
  let zero_sign () = unitL
  let acc _ = unitL
  let get () = ret (liftRef C.Dom.zeroL)
  let set _ = unitL
  let fin () = failwith "Determinant is needed but not computed"
  type ('pc,'v) lm = ('pc,'v) cmonad
    constraint 'pc = <state : [> `TDet of 'a lstate ]; classif : 'a; ..>
  type ('pc,'v) om = ('pc,'v) omonad
    constraint 'pc = <state : [> `TDet of 'a lstate ]; classif : 'a; ..>
end
\end{code}

The following instance does track the determinant. The |decl| method
generates two let-bindings for mutable variables tracking the
magnitude and the sign of the determinant, and places the names of
these variables in the monadic state. The |upd_sign| method, invoked
from row- or column-swap generators, retrieves the name of the
sign-accumulating variable and generates the code to update the
sign. The |fin| method retrieves the names of both accumulators and
generates code to compute the final, signed determinant value.
\begin{code}
module AbstractDet =
  struct
  open C.Dom
  type tdet = v ref
  (* the first part of the state is an integer: which is +1, 0, -1:
     the sign of the determinant *)
  type 'a lstate = ('a,int ref) abstract * ('a,tdet) abstract
  let decl () = perform
      ddecl <-- retN (liftRef oneL);
      dsdecl <-- retN (liftRef Idx.one);
      mo_extend ip (dsdecl,ddecl);
      unitL
  let upd_sign () = perform
      det <-- mo_lookup ip;
      det1 <-- ret (fst det);
      ret (Some (assign det1 (Idx.uminus (liftGet det1))))
  let fin = fun () -> perform
      (det_sign,det) <-- mo_lookup ip;
      ifM (Logic.equalL (liftGet det_sign) Idx.zero) (ret zeroL)
      (ifM (Logic.equalL (liftGet det_sign) Idx.one) (ret (liftGet det))
          (ret (uminusL (liftGet det))))
  \dots
end
\end{code}


\subsection{Output}

More interesting is the aspect of what to return
from the LU algorithm.  One could create an algebraic data type (as
was done in \cite{Carette06}) to encode the various choices: the
matrix, the matrix and the rank, the matrix and the determinant, the
matrix, rank and determinant, and so on. This is wholly unsatisfying
as we know that for any single use, only one of the choices is ever
possible, yet any routine which calls the generated code must deal
with these unreachable options.  Instead we use a module type with an
\emph{abstract} type |res| for the result type; different instances of
the signature set the result type differently. Given below is this
module type and one instantiation, which specifies the output of a LU
algorithm as a 3-tuple |contr * Det.outdet * int| of the U-factor, the
determinant and the rank.

\begin{code2}
module type OUTPUTDEP = sig 
    module PivotRep : PIVOTKIND 
    module Det      : DETERMINANT
end
(* The `keyword' list of all the present internal features *)
module type INTERNAL_FEATURES = sig
  module R      : TrackRank.RANK
  module P      : TRACKPIVOT
  module L      : LOWER
end

module type OUTPUT = functor(OD : OUTPUTDEP) -> sig
  module IF : INTERNAL_FEATURES
  type res
  val make_result : 'a wmatrix -> (\dots,res) monad
end
module OutDetRank(OD : OUTPUTDEP) = struct
  module IF = struct
      module R   = Rank
      module P   = DiscardPivot
      module L   = NoLower end
  type res = C.contr * C.Dom.v * int
  let make_result m = perform
    det  <-- OD.Det.fin ();
    rank <-- IF.R.fin ();
    ret (Tuple.tup3 m.matrix det rank)
  (* Initialization: check the preconditions of instantiation of this struct*)
  let _ = OD.Det.fin ()
  let _ = IF.R.fin ()
end
\end{code2}

The initialization expressions |OD.Det.fin ()| and |IF.R.fin ()| are
the pre-flight tests, to be explained in Section XXX.

Both instances of |Det| contain the |fin ()| function to generate the
code representing the computed determinant. Only the |NoDet| does not
track any determinant, and so |fin ()| raises an error. The code 
|let _ = OD.Det.fin ()| in |OutDetRank| invokes that |fin| function, which will
produce the monadic code generating action, or raise the error. We do
not run the action at that time -- we only make sure there is an
action to run. This is the preflight check again, to rule out the
semantic error when the user specified that determinant should be computed
and returned and yet specified the |NoDet| aspect which tracks no
determinant at all.

\subsection{Main generation}
We combine all aspects in a record. This record is like a keyword
argument list to the |GenGE| functor below.
\begin{code}
module type FEATURES = sig
  module Det       : DETERMINANT
  module PivotF    : PIVOT
  module PivotRep  : PIVOTKIND
  module Update    : UPDATE
  module Input     : INPUT
  module Output    : OUTPUT
end
\end{code}

\oleg{describe how generation proceeds}
We first instantiate the container |GVC_I| for the vector over an
integer domain. We pass the container to the main functor |GenLA|,
since all the linear algebra aspects are parameterized by the
container and the domain included in it. We open the |G_GVC_I| module
to make all its components easily available. One of thos components is
the GE generrator, |GE| (which we open as well). The module |G_GVC_I|
also inlcudes the generator for GE solvers. The module |GE| includes
the functor |GenGE| that is the generator. We instantiate the
generator by passing the functor the `record' of various aspects (the
order is irrelevant, but all aspects like |Det|, etc. must be
specified). In the code below, we request generation with full pivot,
fraction-free update, operate on non-augmented matrix and return the
|U| factor, determinant and the rank. We run the monad passing the
initial state |[]| and obtain the code in App A.

\begin{code}
module GVC_I = GenericVectorContainer(IntegerDomainL)
module G_GVC_I = GenLA(GVC_I)
open G_GVC_I
open G_GVC_I.GE
module GenIV5 = GenGE(struct 
    module Det = AbstractDet
    module PivotF = FullPivot
    module PivotRep = PermList
    module Update = FractionFreeUpdate
    module Input = InpJustMatrix
    module Output = OutDetRank end)
let instantiate gen =
    .<fun a -> .~(runM (gen .<a>.) []) >.;;
let resIV5 = instantiate GenIV5.gen ;;
let rIV5 = .! resIV5;;

module GAC_F = GenericArrayContainer(FloatDomainL)
module G_GAC_F = GenLA(GAC_F)
open G_GAC_F
open G_GAC_F.GE
module GenFA9 = GenGE(struct 
    module Det = NoDet
    module PivotF = RowPivot
    module PivotRep = PermList
    module Update = DivisionUpdate
    module Input = InpJustMatrix
    module Output = Out_LU_Packed end)
\end{code}


\begin{code2}
module GenGE(F : FEATURES) = struct
    module O = F.Output(F)

    let wants_pack = O.IF.L.wants_pack
    let can_pack   = 
        let module U = F.Update(F.Det) in
        (U.upd_kind = DivisionBased)
    (* some more pre-flight tests *)
    let _ = ensure ((not wants_pack) || can_pack) 
           "Cannot return a packed L in this case"

    let zerobelow mat pos = 
        let module IF = O.IF in
        let module U = F.Update(F.Det) in
        let innerbody j bjc = perform
            whenM (Logic.notequalL bjc C.Dom.zeroL ) (perform
                det <-- F.Det.get ();
                optSeqM (Iters.col_iter mat.matrix j (Idx.succ pos.p.colpos) 
               (Idx.pred mat.numcol) C.getL
                      (fun k bjk -> perform
                      brk <-- ret (C.getL mat.matrix pos.p.rowpos k);
                      U.update bjc pos.curval brk bjk 
                          (fun ov -> C.col_head_set mat.matrix j k ov) det) UP )
                      (IF.L.updt mat.matrix j pos.p.colpos C.Dom.zeroL 
                          (* this makes no sense outside a field! *)
                          (C.Dom.divL bjc pos.curval))) in
        perform
              seqM (Iters.row_iter mat.matrix pos.p.colpos
              (Idx.succ pos.p.rowpos)
              (Idx.pred mat.numrow) C.getL innerbody UP) 
                   (U.update_det pos.curval)

   let init input = perform
        let module IF = O.IF in
          (a,rmar,augmented) <-- F.Input.get_input input;
          r <-- IF.R.decl ();
          c <-- retN (liftRef Idx.zero);
          b <-- retN (C.mapper C.Dom.normalizerL (C.copy a));
          m <-- retN (C.dim1 a);
          rmar <-- retN rmar;
          n <-- if augmented then retN (C.dim2 a) else ret rmar;
          F.Det.decl ();
          IF.P.decl rmar;
          _ <-- IF.L.decl (if wants_pack then b else C.identity rmar m);
          let mat = {matrix=b; numrow=n; numcol=m} in
          ret (mat, r, c, rmar)

   let forward_elim (mat, r, c, rmar) = perform
        let module IF = O.IF in
          whileM (Logic.andL (Idx.less (liftGet c) mat.numcol)
                              (Idx.less (liftGet r) rmar) )
             ( perform
             rr <-- retN (liftGet r);
             cc <-- retN (liftGet c);
             let cp  = {rowpos=rr; colpos=cc} in
             let module Pivot = F.PivotF(F.Det)(IF.P) in
             pivot <-- l1 retN (Pivot.findpivot mat cp);
             seqM (matchM pivot (fun pv -> 
                      seqM (zerobelow mat {p=cp; curval=pv} )
                           (IF.R.succ ()) )
                      (F.Det.zero_sign () ))
                  (updateM c Idx.succ) )

   let gen input = perform
          (mat, r, c, rmar) <-- init input;
          seqM 
            (forward_elim (mat, r, c, rmar))
            (O.make_result mat)
end
\end{code2}

The LU generator functor itself is 
parameterized by the domain, container, pivoting policy (full, row,
nonzero, no pivoting), update policy (with either `fraction-less'
or full division), and the result specification. Some of the
argument modules such as |PIVOT| are functors themselves (parameterized
by the domain, the container, and the determinant functor). 

There are more pre-flight checks for various ``semantic'' 
constraints, shown
in the following structure of the |UPDATE| signature:
\vspace*{-5pt}\begin{code}
module DivisionUpdate(Det:DETERMINANT) =
  struct
  open C.Dom = struct 
  let _ = assert (C.Dom.kind = Domains_sig.Domain_is_Field)
end
\end{code}
%
This structure implements an update policy of using
|Dom.div| operation without restrictions -- which is possible only if
the domain has such an unrestricted operation. A domain such as the integer
domain may still provide |Dom.div| of the same type, but that operation may
only be used when we are sure that the division is exact. 


We can instantiate the |Gen| functor as shown above and inspect the generated
code, e.g., by printing |resIV5|. The code can then be ``compiled'' as 
|!. resIV5| or with off-shoring. The code for |resIV5| (Appendix A) shows
full pivoting, determinant and rank tracking. The code for all these aspects is
fully inlined; no extra functions are invoked and no tests other than those
needed by the LU algorithm itself are performed. The GE?? function returns a
triple |int array * int * int| of the U-factor, determinant and the rank. The
code generated by |GenFA9| (Appendix B) shows absolutely no traces of
determinant tracking: no declaration of spurious variables, no extra tests,
etc. The code appears as if the determinant tracking aspect did not exist
at all. The generated code for the above and other instantiations of
|Gen| can be examined at \cite{metamonadsURL}. The website also 
contains benchmark code and timing comparisons.

\oleg{We should mention somewhere that we have implemented not only GE
  but Solvers as well.}

\section{Runge-Kutta solvers}
\label{s:ode}
\oleg{Drop this? no space}
\jacques{We should drop this as a whole section, but I still think we 
should mention it as something that we have done.  Mostly to say that
things ``extend'' to other families too, and that no new difficulties 
were encountered.}

\section{Related and future work}\label{related}

The monad in this paper is similar to the one described in
\cite{MSP:PADL04,KiselyovTaha}.  However the latter papers used only
|retN| and fixpoints (for generation-time iterations).  This paper
does not involve monadic fixpoints because the generator is not
recursive, but heavily relies on monadic operations for generating
conditionals and loops.

|Blitz++| \cite{Veldhuizen:1998:ISCOPE} and {C++} template
meta-programming in general similarly eliminate levels
of abstraction.  With traits and concepts, some domain-specific
knowledge can also be encoded.  However overhead elimination
critically depends on full inlining of all methods by the compiler,
which has been reported to be challenging to insure. Furthermore, all
errors (such as type errors and concept violation errors, i.e.,
composition errors) are detected only when compiling the generated
code. It is immensely difficult to correlate errors (e.g., line
numbers) to the ones in the generator itself.

ATLAS \cite{ATLAS} is another successful project in this area.
However they use much simpler weaving technology, which leads them to
note that \emph{generator complexity tends to go up along with
  flexibility, so that these routines become almost insurmountable
  barriers to outside contribution}. Our results show how to surmount
this barrier, by building modular, composable generators. A
significant part of ATLAS' complexity is that the generator is
extremely error-prone and difficult to debug.  Indeed, when generating
C code in C using |printf|, nothing prevents producing code that
misses semicolons, open or close parentheses or variable
bindings. MetaOCaml gives us assurance that these errors, and more
subtle type errors, shall not occur in the generated code.  SPIRAL
\cite{Pueschel:05} is another such even more ambitious project.  But
SPIRAL does intentional code analysis, relying on a set of code
transformation ``rules'' which make sense, but which are not proved to
be either complete or confluent.  The strength of both of these
project relies on their platform-specific optimizations performed via
search techniques, something we have not attempted here.

The highly parametric version of our Gaussian Elimination is directly
influenced by the generic implementations available in Axiom
\cite{Axiom} and Aldor \cite{Watt:2002:HCA}.  Even though the Aldor
compiler frequently can optimize away a lot of abstraction overhead, 
it does not provide any guarantees that it will do so, unlike our
approach.

We should also mention early work \cite{Gluck95} on automatic
specialization of mathematical algorithms. Although it can eliminate
some overhead from a very generic implementation (e.g., by inlining
aspects implemented as higher-order functions), specialization cannot
change the type of the function and cannot efficiently handle aspects
that communicate via a private shared state.

The paper \cite{GluckJ97} describes early simple experiments in
\emph{automatic} and manual staging, and the multi-level language
based on an annotated subset of Scheme (which is untyped and has no
imperative features). The generated code requires post-processing to
attain efficiency.  

We are looking into encapsulating staging
annotations into just a few functors, so that the rest of the code (in
particular, the |Gen| functor that puts it all together) should be
annotation-free and thus can be used as is in a one-stage environment
(pure OCaml) as well as in a multi-stage environment (generating
extensions). The one-stage code is a good baseline for benchmarks and
regression tests. Obtaining a generating extension from properly
modularized OCaml code (along the lines of our |Gen|) is an exciting
area of our future research.

To the best of our knowledge, nobody has yet used functors to
abstract code generators, or even mixed functors and 
multi-stage programming.

We plan to further investigate the connection between delimited
continuations and our implementations of code generators like
|ifM|.  As well, by using some additional syntactic sugar
(for |ifM|, |whileM|, etc.), the available notation should be
even more direct-style, and potentially clearer.
We also would like to extend our monad to a monad transformer.

There are many more aspects which can also be handled:
error reporting (i.e. asking for the determinant of a 
non-square matrix), memory hierarchy issues, loop-unrolling
\cite{Padua:MetaOcaml:04},
warnings when zero-testing is undecidable and
a value is only probabilistically non-zero, etc.  The larger program
family of LU decompositions contains more aspects still.

\section{Conclusion}\label{conclusion}
In this paper we have demonstrated numerical code extensively parameterized
by complex aspects at no run-time overhead.  The combination of
stateless functors and structures, and our monad with the
compositional state makes aspects freely composable without having to
worry about value aliasing. The only constraints to compositionality
are the typing ones plus the constraints we specifically
impose, including semantic constraints (e.g., rings do not have full
division).

There is an interesting relation with aspect-oriented code
\cite{kiczales97aspectoriented}: in AspectJ, 
aspects are (comparatively) lightly typed, and are post-facto extensions of an
existing piece of code.  Here aspects are weaved together ``from scratch'' to
make up a piece of code/functionality.  One can understand previous work to be
more akin to dynamically typed aspect weaving, while we have started
investigating statically typed one.

\subsection*{Acknowledgments}
We wish to thank Cristiano Calgano for his help in adapting camlp4 for
use with MetaOCaml. Many helpful discussions with Walid Taha are very
appreciated. The implementation of the monadic notation, |perform|,
was the joint work with Lydia van Dijk.

\bibliography{metamonads}
\bibliographystyle{elsart-num}
\section{Appendix 0}
\label{app:perform}
Grammar of our perform monad
We support four different constructs to introduce a monadic
expressions.

\begin{code}
  perform exp
  perform exp1; exp2
  perform x <-- exp1; exp2
  perform let x = foo in exp
\end{code}

which is almost literally the grammar of the Haskell's "do"-notation,
with the differences that Haskell uses |do| and |<-| where we use
|perform| and |<--|.

We support not only |let x = foo in ...|  expressions but arbitrarily
complex let-expressions, including |let rec| and |let module|.

The actual bind function of the monad defaults to |bind| and the
pattern-match--failure function to |failwith| (only used for refutable
patterns).  To select a different function, use the
extended forms of |perform|. For example, use the function named 
|bind| from module |Mod|.  In
addition use the module's |failwith|-function in refutable patterns.
\begin{code}
        perform with module Mod in exp2
\end{code}
The code has full explanation...

\section{Appendix A}
The code generated for |GenIV5|, fraction-free LU of the integer matrix
represented by a flat vector, full pivoting, returning the |U|-factor,
the determinant and the rank.
\begin{code2}
val resIV5 : ('a, GVC_I.contr -> GenIV5.O.res) code =
  .<fun a_1 ->
   let t_2 = (ref 0) in
   let t_3 = (ref 0) in
   let t_4 = (a_1) \{arr = (Array.copy a_1.arr)\} in
   let t_5 = a_1.m in  (* magnitude of det *)
   let t_6 = a_1.n in  (* sign of the det *)
   let t_7 = (ref 1) in
   let t_8 = (ref 1) in
   while (((! t_3) < t_5) && ((! t_2) < t_6)) do
    let t_13 = (! t_2) in
    let t_14 = (! t_3) in
    let t_15 = (ref (None)) in
    let t_34 =
     begin (* full pivoting, search for the pivot *)
      for j_30 = t_13 to (t_6 - 1) do
       for j_31 = t_14 to (t_5 - 1) do
        let t_32 = (t_4.arr).((j_30 * t_4.m) + j_31) in
        if (t_32 <> 0) then
         (match (! t_15) with
          | Some (i_33) ->
             if ((abs (snd i_33)) > (abs t_32)) then
              (t_15 := (Some ((j_30, j_31), t_32)))
             else ()
          | None -> (t_15 := (Some ((j_30, j_31), t_32))))
        else ()
       done
      done;
      (match (! t_15) with
       | Some (i_16) -> (* swapping of columns *)
          if ((snd (fst i_16)) <> t_14) then begin
           let a_23 = t_4.arr
           and nm_24 = (t_4.n * t_4.m)
           and m_25 = t_4.m in
           let rec loop_26 =
            fun i1_27 ->
             fun i2_28 ->
              if (i2_28 < nm_24) then
               let t_29 = a_23.(i1_27) in
               a_23.(i1_27) <- a_23.(i2_28);
               a_23.(i2_28) <- t_29;
               (loop_26 (i1_27 + m_25) (i2_28 + m_25))
              else () in
           (loop_26 t_14 (snd (fst i_16)));
           (t_8 := (~- (! t_8))) (* adjust the sign of det *)
          end else ();
          if ((fst (fst i_16)) <> t_13) then begin (* swapping of rows *)
           let a_17 = t_4.arr
           and m_18 = t_4.m in
           let i1_19 = (t_13 * m_18)
           and i2_20 = ((snd (fst i_16)) * m_18) in
           for i_21 = 0 to (m_18 - 1) do
            let t_22 = a_17.(i1_19 + i_21) in
            a_17.(i1_19 + i_21) <- a_17.(i2_20 + i_21);
            a_17.(i2_20 + i_21) <- t_22
           done;
           (t_8 := (~- (! t_8)))
          end else ();
          (Some (snd i_16))
       | None -> (None))
     end in
    (match t_34 with
     | Some (i_35) ->
        begin (* elimination loop *)
         for j_36 = (t_13 + 1) to (t_6 - 1) do
          let t_37 = (t_4.arr).((j_36 * t_4.m) + t_14) in
          if (t_37 <> 0) then begin
           for j_38 = (t_14 + 1) to (t_5 - 1) do
            (t_4.arr).((j_36 * t_4.m) + j_38) <-
             ((((t_4.arr).((j_36 * t_4.m) + j_38) * i_35) -
                ((t_4.arr).((t_13 * t_4.m) + j_38) * t_37)) / (! t_7))
           done;
           (t_4.arr).((j_36 * t_4.m) + t_14) <- 0
          end else ()
         done;
         (t_7 := i_35)
        end;
        (t_2 := ((! t_2) + 1)) (* advance the rank *)
     | None -> (t_8 := 0));
    (t_3 := ((! t_3) + 1))
   done;
   (t_4, (* matrix with the U factor *)
    if ((! t_8) = 0) then 0 (* adjust the sign of the determinant *)
    else if ((! t_8) = 1) then (! t_7)
    else (~- (! t_7)), (! t_2))>.
\end{code2}
\section{Appendix B}
The code generated for |GenFA9|, LU of the floating point
non-augmented matrix
represented by a 2D array, row pivoting, returning the complete
factorization: |L| and |U|
factors packed in a single matrix and the permutation matrix
represented as the list of row number exchanges.
\begin{code2}
val resFA9 : ('a, GAC_F.contr -> GenFA9.O.res) code =
  .<fun a_1 ->
   let t_2 = (ref 0) in
   let t_3 = (ref 0) in
   let t_5 = (Array.map (fun x_4 -> (Array.copy x_4)) (Array.copy a_1)) in
   let t_6 = (Array.length a_1.(0)) in
   let t_7 = (Array.length a_1) in
   let t_8 = (ref ([])) in  (* accumulate permutations in a list *)
   while (((! t_3) < t_6) && ((! t_2) < t_7)) do
    let t_9 = (! t_2) in
    let t_10 = (! t_3) in
    let t_11 = (ref (None)) in
    let t_17 =
     begin   (* row pivoting *)
      for j_14 = t_9 to (t_7 - 1) do
       let t_15 = (t_5.(j_14)).(t_10) in
       if (t_15 <> 0.) then
        (match (! t_11) with
         | Some (i_16) ->
            if ((abs_float (snd i_16)) < (abs_float t_15)) then
             (t_11 := (Some (j_14, t_15)))
            else ()
         | None -> (t_11 := (Some (j_14, t_15))))
       else ()
      done;
      (match (! t_11) with  (* swapping of rows *)
       | Some (i_12) ->
          if ((fst i_12) <> t_9) then begin
           let t_13 = t_5.(t_9) in
           t_5.(t_9) <- t_5.(fst i_12);
           t_5.(fst i_12) <- t_13;  (* and accumulate permutations *)
           (t_8 := ((RowSwap ((fst i_12), t_9)) :: (! t_8)))
          end else ();
          (Some (snd i_12))
       | None -> (None))
     end in
    (match t_17 with  (* elimination loop *)
     | Some (i_18) ->
        begin
         for j_19 = (t_9 + 1) to (t_7 - 1) do
          let t_20 = (t_5.(j_19)).(t_10) in
          if (t_20 <> 0.) then
           for j_21 = (t_10 + 1) to (t_6 - 1) do
            (t_5.(j_19)).(j_21) <-
             ((t_5.(j_19)).(j_21) -. ((t_20 /. i_18) *. (t_5.(t_9)).(j_21)))
           done
          else ()
         done;
         ()
        end;
        (t_2 := ((! t_2) + 1))
     | None -> ());
    (t_3 := ((! t_3) + 1))
   done;
   (t_5, (! t_8))>. (* return both L and U factors, list permutations *)
\end{code2}

\end{document}

%%%%%%%%%%%%%%%%%%%%%%%%%%%%%%%%%%%%%%%%%%%%%%%%%%%%%%%%%%%%%%%%%%%%%%
% all the text that used to be here is now in unused.tex
% same with any text in an \omitnow

