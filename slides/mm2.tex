\documentclass{beamer}
\setcounter{errorcontextlines}{100}
\usepackage[T1]{fontenc}
\usepackage{charter}
\usepackage{eulervm}
% \usepackage{comment}
\usepackage{amssymb,amsmath}
\usepackage{fancyvrb}
% \DefineShortVerb{}
%\DefineVerbatimEnvironment{code}{Verbatim}{commandchars=\\\{\},fontsize=\small}
\DefineVerbatimEnvironment{code}{Verbatim}{commandchars=\\\{\},fontsize=\small}
\DefineVerbatimEnvironment{code2}{Verbatim}{commandchars=\\\{\},fontsize=\footnotesize}

\usefonttheme{serif}
\makeatletter
\setbeamertemplate{frametitle}
{\vskip-2pt
  \@tempdima=\textwidth%
  \advance\@tempdima by\beamer@leftmargin%
  \advance\@tempdima by\beamer@rightmargin%
  \begin{beamercolorbox}[sep=0.3cm,left,wd=\the\@tempdima]{frametitle}
    {\usebeamerfont{frametitle}\smash[b]{\insertframetitle}\par}%
    {%
      \ifx\insertframesubtitle\@empty%
      \else%
      {\usebeamerfont{framesubtitle}\usebeamercolor[fg]{framesubtitle}\smash[b]{\insertframesubtitle\par}}%
      \fi
    }%
  \end{beamercolorbox}%
}
\setbeamertemplate{block begin}
{
  \par\vskip\medskipamount%
  \ifx\insertblocktitle\@empty\else
    \begin{beamercolorbox}[colsep*=.75ex]{block title}
      \usebeamerfont*{block title}\smash[b]{\insertblocktitle}%
    \end{beamercolorbox}%
    {\parskip0pt\par}%
    \ifbeamercolorempty[bg]{block title}
      {}
      {\ifbeamercolorempty[bg]{block body}{}{\nointerlineskip\vskip-0.5pt}}%
  \fi
  \usebeamerfont{block body}%
  \begin{beamercolorbox}[colsep*=.75ex,vmode]{block body}%
    \ifbeamercolorempty[bg]{block body}{\vskip-.25ex}{\vskip-.75ex}\vbox{}%
}
\makeatother
\setbeamertemplate{sidebar right}{\vfill\llap{\usebeamerfont{framesubtitle}\insertframenumber\hskip\jot}\vskip\jot}
\usecolortheme{crane}
\setbeamercolor{alerted text}{fg=red!80!craneorange!80!fg}
%%%\setbeamercolor{highlight}{bg=yellow!80!craneorange!80!bg}
%\newcommand<>{\highlight}[1]{\text{\fboxsep=0pt \only#2{\colorbox{yellow}}{\strut$#1$}}}
%\newcommand<>{\Overbrace}[2]{\alt#3{\mathord{\overbrace{#2}^{\text{\small #1}}}}{\vphantom{\overbrace{#2}^{\text{\small #1}}}#2}}
%\newcommand<>{\Underbrace}[2]{\alt#3{\mathord{\underbrace{#2}_{\text{\small #1}}}}{\vphantom{\underbrace{#2}_{\text{\small #1}}}#2}}

\setbeamersize{text margin left=2mm,text margin right=2mm}
% \def\Forall#1\circ{\forall#1.\:}

\newtheorem{defn}{Definition}
\newtheorem{thm}{Theorem}
\newtheorem{Ex}{Example}

\DeclareMathOperator{\len}{length}
\DeclareMathOperator{\theory}{theory}

\title{Multi-stage programming with functors and monads:
eliminating abstraction overhead from generic code}
\author{Jacques Carette\inst{1} \and
Oleg Kiselyov\inst{2}}
\institute{\inst{1}McMaster University \inst{2} FNMOC}
\date{GPCE 2005}
\keywords{MetaOCaml, functors, monads, CPS, Gaussian Elimination, abstraction}

\begin{document}
%\pagestyle{empty}
\divide\abovedisplayskip 2
\divide\belowdisplayskip 2
\divide\abovedisplayshortskip 2
\divide\belowdisplayshortskip 2
%\blanklineskip\abovedisplayskip

\newcommand{\seq}[1]{{\langle #1 \rangle}}
\newcommand{\set}[1]{{\{ #1 \}}}
\newcommand{\tuple}[1]{{( #1 )}}
\newcommand{\mname}[1]{\mbox{\sf #1}}
\newcommand{\Nat}{\mathbb N}

\begin{frame}
\titlepage
\end{frame}

\begin{frame}
\frametitle{Overview}
\begin{enumerate}
    \item Problem
    \item Design Points
    \item Tools: CPS, Monads, Extensible Grammar, Functors, \dots
    \item Solution \& Results
\end{enumerate}
\end{frame}

\begin{frame}
\frametitle{Problem (Origins)}
    \begin{itemize}
        \item<1-> 35+ implementations of Gaussian Elimination in Maple
        \item<2-> Efficiency!
            \begin{itemize}
                \item[+] get from specialized versions 
                \item[-] generic versions are slow
            \end{itemize}
        \item<3-> Problem: how to unify without losing speed?
    \end{itemize}
\end{frame}

\begin{frame}
    \frametitle{Problem (as solved)}
    \begin{itemize}
        \item[]<1->\textbf{Goal:} Type-safe, generic, efficient algorithms
        \item[]<2->\textbf{Issues:}
    \begin{itemize}
        \item Untyped $\Rightarrow$ lots of debugging, no certainty
        \item Abstraction $\Rightarrow$ overhead 
            (unless you're the compiler writer)
        \item Efficiency requires \emph{no} overhead
        \item Output type varies
    \end{itemize}
    \end{itemize}
    \begin{block}{conclusion}<3->
    $\Longrightarrow$ implement as a \emph{typed} meta-program
\end{block}
\end{frame}

\begin{frame}
\frametitle{Design Points \emph{(aspects)}}
\begin{tabular}{@{}l|l|l@{}} \hline
    Design Dimension & Abstracts & Depends on \\ \hline
    Domain & Matrix values & -- \\ \hline
    Representation & Matrix representation & Domain \\ \hline
    Fraction-free & use of division & Domain \\ \hline
    Pivoting Strategy & length measure \& user & Domain \\ \hline
    Output & choice of output & -- \\ \hline
    Normalization & domain needs normalization & Domain \\ \hline
    Determinant & determinant tracking & Output \& Fraction-free \\ \hline
    ZeroEquivalence & decidability of $=0$ & -- \\ \hline
    UserInformation & user-feedback & ZeroEquivalence\\ \hline
    Augmented & matrix is augmented & -- \\ \hline
\end{tabular}

\vspace{5mm}
Outputs: new matrix, rank, determinant, permutation matrix.
\end{frame}

\begin{frame}[fragile]
\frametitle{MetaOCaml example}
\begin{code}
let one = .<1>. and plus x y = .<.~x + .~y>.
let simplestcode = let gen x y = plus x (plus y one) in
  .<fun x y -> .~(gen .<x>. .<y>.)>.
The function above is generic (can change to float easily)
# val .<fun x1 -> fun y2 -> (x1 + (y2 + 1))>.
let paramcode1' plus one =
  let gen x y = let ce = (plus y one) in  plus ce (plus x ce) in
  .<fun x y -> .~(gen .<x>. .<y>.)>.
paramcode1' plus one
# val .<fun x1 -> fun y2 -> ((y2 + 1) + (x1 + (y2 + 1)))>.
\end{code}
Problem: \verb=y2+1= repeated.  But obvious solution doesn't work!
\end{frame}

\begin{frame}[fragile]
    \frametitle{Tools: Continuation passing Style (CPS)}
    But if you pass your future around\dots
\begin{code}
let letgen exp k = .<let t = .~exp in .~(k .<t>.)>.
let paramcode2 plus one =
  let gen x y k = letgen (plus y one) 
                  (fun ce -> k (plus ce (plus x ce)))
  and k0 x = x
  in .<fun x y -> .~(gen .<x>. .<y>. k0)>.
paramcode2 plus one
# val .<fun x1 -> fun y2 -> 
           let t3 = (y2 + 1) in (t3 + (x1 + t3))>.
\end{code}
From \textit{A Methodology for Generating Verified Combinatorial Circuits} by 
Kiselyov, Swadi and Taha.
\end{frame}

\begin{frame}[fragile]
    \frametitle{Tools: Monads}
    \begin{itemize}
        \item[]<1->Method to encapsulate ``computations''.
        \item[]<2->We need \emph{state} and \emph{continuations}
        \item[]<3->
\begin{code}
type ('v,'s,'w) monad = 's -> ('s -> 'v -> 'w) -> 'w
let ret a = fun s k -> k s a
let bind a f = fun s k -> a s (fun s' b -> f b s' k)

let k0 s v = v
let runM m = m [] k0

let retN a = fun s k -> .<let t = .~a in .~(k s .<t>.)>.
\end{code}
\item[]<4->\textit{state} is encoded as a list of
    \emph{polymorphic variants} for type-safety.
    \end{itemize}
\end{frame}

\begin{frame}[fragile]
    \frametitle{Tools: \textit{New} control structures}
\begin{code}
let ifL test th el = ret .< if .~test then .~th else .~el >.

(* Note the implicit `reset` *)
let ifM test th el = fun s k -> k s
  .< if .~(test s k0) then .~(th s k0) else .~(el s k0) >.

let seqM a b = fun s k -> k s 
  .< begin .~(a s k0) ; .~(b s k0) end >.

let whenM test th  = fun s k -> k s 
  .< if .~(test s k0) then .~(th s k0) else () >.
\end{code}
\end{frame}

\begin{frame}[fragile]
    \frametitle{Extensible Grammar}
    In all programming languages, a lot of \emph{expressible abstractions}
    have atrocious syntax.  Definitely a problem of languages with
    just a few powerful primitives. \texttt{camlp4} for Ocaml allows one to add
    to (or redefine completely) the grammar of the language.  For monads:
\begin{code}
let param_code3 plus one =
  let gen x y = bind (retN (plus y one)) (fun ce -> 
                ret (plus ce (plus x ce)))
  in .<fun x y -> .~(runM (gen .<x>. .<y>.))>.

let param_code4 plus one =
  let gen x y = doM ce <-- retN (plus y one);
                    ret (plus ce (plus x ce))
  in .<fun x y -> .~(runM (gen .<x>. .<y>.))>.
\end{code}
\end{frame}

\begin{frame}[fragile]
    \frametitle{Extensible Grammar (cont) }
A multi-monad version also exists
\begin{code}
let param_code4 plus one =
  let gen x y = doM Monad in
                    ce <-- retN (plus y one);
                    ret (plus ce (plus x ce))
  in .<fun x y -> .~(runM (gen .<x>. .<y>.))>.
\end{code}
which will use \verb+Monad.bind+ instead of \verb+bind+.
\end{frame}

\begin{frame}[fragile]
    \frametitle{Tools: Modules \& Functors}
\begin{code}
module IntegerDomain = 
  struct
    type v = int
    type kind = domain_is_ring
    type 'a vc = ('a,v) code
    let zero = .< 0 >.  
    let one = .< 1 >. 
    let plus x y = ret .<.~x + .~y>. 
    let div x y = ret .<.~x / .~y>.
    ...
    let better_than = Some (fun x y -> ret .<abs .~x > abs .~y >. )
    let normalizerf = None 
end
\end{code}
\end{frame}

\begin{frame}[fragile]
    \frametitle{Tools: Modules \& Functors}
\begin{code}
module type DOMAIN = sig
  type v
  type kind (* Field or Ring ? *)
  type 'a vc = ('a,v) code
  val zero : 'a vc
  val one : 'a vc
  val plus : 'a vc -> 'a vc -> ('a vc, 's, 'w) monad
  val div : 'a vc -> 'a vc -> ('a vc, 's, 'w) monad
  val better_than : ('a vc -> 'a vc -> 
                    (('a,bool) code, 's, 'w) monad) option
  val normalizerf : (('a,v -> v) code ) option
end 
\end{code}
\end{frame}

\begin{frame}[fragile]
    \frametitle{Tools: Modules \& Functors}
\begin{code}
module GenericArrayContainer(Dom:DOMAIN) = struct
  type contr = Dom.v array array (* Array of rows *)
  type 'a vc = ('a,contr) code
  type 'a vo = ('a,Dom.v) code
  let get' x n m = .< (.~x).(.~n).(.~m) >.
  let set' x n m y = .< (.~x).(.~n).(.~m) <- .~y >.
  let dim2 x = .< Array.length .~x >.       (* number of rows *)
  let dim1 x = .< Array.length (.~x).(0) >. (* number of cols *)
  let copy = ...
  let swap_rows_stmt a r1 r2 =
      .< let t = (.~a).(.~r1) in
         begin 
             (.~a).(.~r1) <- (.~a).(.~r2);
             (.~a).(.~r2) <- t
         end >.
end
\end{code}
\end{frame}

\begin{frame}[fragile]
    \frametitle{Solution}
    Have Modules and (higher-order) Functors, all using the CPS Monad, of
    \emph{code generators} that assemble the solution from base \emph{code
    combinators}.
\begin{code}
module GenFA1 = 
    Gen(FloatDomain)
       (GenericArrayContainer)
       (RowPivot)
       (DivisionUpdate(FloatDomain)(GenericArrayContainer)
                      (NoDet(FloatDomain)))
       (OutJustMatrix(FloatDomain)(GenericArrayContainer)
                     (NoDet(FloatDomain)))
\end{code}
\end{frame}

\begin{frame}[fragile]
    \frametitle{Solution}
\begin{code2}
module Gen(Dom: DOMAIN)(C: CONTAINER2D)(PivotF: PIVOT)
          (Update: UPDATE with type baseobj = Dom.v 
                          and type ctr = C(Dom).contr)
          (Out: OUTPUT with type contr = C(Dom).contr 
                       and type D.indet = Dom.v 
                       and type 'a D.lstate = 'a Update.D.lstate) =
   struct
    module Ctr = C(Dom) module Pivot = PivotF(Dom)(C)(Out.D)
    type v = Dom.v
    let gen =
      let zerobelow b r c m n brc = 
        let innerbody i = doM
            bic <-- Ctr.get b i c;
            whenM (l1 LogicCode.not (LogicCode.equal bic Dom.zero ))
                (seqM (retLoopM (Idx.succ c) (Idx.pred m)
                          (fun k -> Update.update b r c i k) )
                      (Ctr.set b i c Dom.zero)) in 
        doM seqM (retLoopM (Idx.succ r) (Idx.pred n) innerbody) 
                   (Update.update_det brc) in
\end{code2}
\end{frame}

\begin{frame}[fragile]
    \frametitle{Solution (cont)}
\begin{code2}
  let dogen a = doM
      r <-- Out.R.decl ();
      c <-- retN (liftRef Idx.zero);
      b <-- retN (Ctr.mapper Dom.normalizerf (Ctr.copy a));
      m <-- retN (Ctr.dim1 a);
      n <-- retN (Ctr.dim2 a);
      () <-- Update.D.decl ();
      () <-- Out.P.decl ();
      seqM 
        (retWhileM (LogicCode.and_ (Idx.less (liftGet c) m)
                                   (Idx.less (liftGet r) n) )
           ( doM
           rr <-- retN (liftGet r);
           cc <-- retN (liftGet c);
           pivot <-- l1 retN (Pivot.findpivot b rr m cc n);
           seqM (retMatchM pivot (fun pv -> 
                    seqM (zerobelow b rr cc m n pv) (Out.R.succ ()) )
                    (Update.D.zero_sign () ))
                (Code.update c Idx.succ) ))
        (Out.make_result b)
in .<fun a -> .~(runM (dogen .<a>.)) >.  end
\end{code2}
\end{frame}

\begin{frame}
    \frametitle{Results}
    \begin{enumerate}
        \item Defined aspects:
    \begin{itemize}
        \item $8$ domains (of an infinite family)
        \item $5$ container types
        \item $3$ pivoting strategies 
        \item $5$ output choices 
        \item $2$ elimination strategies
    \end{itemize}
Which leads to 800 (of 1200) type-safe possibilities.

\item Resulting code is \textbf{completely isomorphic} to hand-written code.

\item Using \emph{offshoring}, can generate C (and sometimes Fortran) code
for results.

\item Using (new) native code support, can generate (and run) code
    natively on some platforms.
    \end{enumerate}
\end{frame}

\begin{frame}
\frametitle{Timings (Bytecode)}
Machine: P4 2.00Ghz, FreeBSD 4.9-RELEASE-p1, 1Gig memory
\begin{figure}
\begin{tabular}{l|l|l|l}
    size & generated & generation & direct \\ \hline
    10 & 0.28 msec & 9.1 & 2.8 \\
    30 & 6.9  & 9.1 & 66.8 \\
    50 & 31.4 & 9.1 & 84.7 \\
    70 & 84.7 & 9.1 & 816.6 \\
\end{tabular}
\caption{Floating-point version, matrix-only return, row-pivoting}
\end{figure}

\begin{figure}
\begin{tabular}{l|l|l|l}
    size & generated & generation & direct \\ \hline
    10 & 0.18 & 9.0 & 3.5 \\
    30 & 2.0  & 9.0 & 41.8 \\
    50 & 11.3 & 9.0 & 244.3 \\
    70 & 50.6 & 9.0 & 1102.5 \\
\end{tabular}
\caption{Integer, Matrix-rank-determinant return, fraction-free, row-pivoting}
\end{figure}
\end{frame}

\begin{frame}
\frametitle{Timings (Native)}

\begin{figure}
\begin{tabular}{l|l|l|l}
    size & generated & generation & direct \\ \hline
    10 & 0.017 msec & 46.8 & 1.25 \\
    30 & 0.36       & 58.6 & 29.4 \\
    50 & 1.82       & 64.3 & 130.2 \\
    70 & 4.86       & 69.7 & 358.5 \\
\end{tabular}
\caption{Floating-point version, matrix-only return, row-pivoting}
\end{figure}

\begin{figure}
\begin{tabular}{l|l|l|l}
    size & generated & generation & direct \\ \hline
    10 & 0.022 msec & 46.6 & 1.48 \\
    30 & 0.25       & 58.8 & 17.2 \\
    50 & 1.51       & 66.0 & 102.9 \\
    70 & 6.85       & 71.1 & 458.2 \\
\end{tabular}
\caption{Integer, Matrix-rank-determinant return, fraction-free, row-pivoting}
\end{figure}
\end{frame}

\begin{frame}[allowframebreaks=.99,fragile]
    \frametitle{Sample Output}
\begin{code2}
  ('a, Funct4.GenFA1.Ctr.contr ->
   Funct4.OutJustMatrix(Funct4.FloatDomain)(Funct4.GenericArrayContainer)(Funct4.NoDet(Funct4.FloatDomain)).res)
  code =
  .<fun a_1 -> let t_2 = (ref 0) in let t_3 = (ref 0) in
   let t_5 = (Array.map (fun x_4 -> (Array.copy x_4)) (Array.copy a_1)) in
   let t_6 = (Array.length a_1.(0)) in let t_7 = (Array.length a_1) in
   while (((! t_3) < t_6) && ((! t_2) < t_7)) do
    let t_8 = (! t_2) in let t_9 = (! t_3) in
    let t_10 = (ref (None)) in
    let t_16 =
     begin
      for j_13 = t_8 to (t_7 - 1) do
       let t_14 = (t_5.(j_13)).(t_9) in
       if (not (t_14 = 0.)) then
        (match (! t_10) with
         | Some (i_15) ->
            if ((abs_float (snd i_15)) < (abs_float t_14)) then
             (t_10 := (Some (j_13, t_14)))
            else ()
         | None -> (t_10 := (Some (j_13, t_14))))
       else () done;
      (match (! t_10) with
       | Some (i_11) ->
          if ((fst i_11) <> t_8) then begin
           let t_12 = t_5.(t_8) in
           t_5.(t_8) <- t_5.(fst i_11);
           t_5.(fst i_11) <- t_12;
          end else ();
          (Some (snd i_11))
       | None -> (None))
     end in
    (match t_16 with
     | Some (i_17) ->
        begin
         for j_18 = (t_8 + 1) to (t_7 - 1) do
          if (not ((t_5.(j_18)).(t_9) = 0.)) then begin
           for j_19 = (t_9 + 1) to (t_6 - 1) do
            (t_5.(j_18)).(j_19) <- ((t_5.(j_18)).(j_19) -.
               (((t_5.(j_18)).(t_9) /. (t_5.(t_8)).(t_9)) *.
                 (t_5.(t_8)).(j_19)))
           done;
           (t_5.(j_18)).(t_9) <- 0.
          end else () done;
        end;
        (t_2 := ((! t_2) + 1))
     | None -> ());
    (t_3 := ((! t_3) + 1))
   done;
   t_5>.

  ('a, Funct4.GenIA4.Ctr.contr ->
   Funct4.OutDetRank(Funct4.IntegerDomain)(Funct4.GenericArrayContainer)(Funct4.IDet)(Funct4.Rank).res) code =
  .<fun a_280 -> let t_281 = (ref 0) in let t_282 = (ref 0) in
   let t_284 = (Array.map (fun x_283 -> (Array.copy x_283)) (Array.copy a_280)) in
   let t_285 = (Array.length a_280.(0)) in let t_286 = (Array.length a_280) in
   let t_287 = (ref 1) in let t_288 = (ref 1) in
   while (((! t_282) < t_285) && ((! t_281) < t_286)) do
    let t_289 = (! t_281) in let t_290 = (! t_282) in
    let t_291 = (ref (None)) in
    let t_297 =
     begin
      for j_294 = t_289 to (t_286 - 1) do
       let t_295 = (t_284.(j_294)).(t_290) in
       if (not (t_295 = 0)) then
        (match (! t_291) with
         | Some (i_296) ->
            if ((abs (snd i_296)) > (abs t_295)) then
             (t_291 := (Some (j_294, t_295)))
            else ()
         | None -> (t_291 := (Some (j_294, t_295))))
       else ()
      done;
      (match (! t_291) with
       | Some (i_292) ->
          if ((fst i_292) <> t_289) then begin
           let t_293 = t_284.(t_289) in
           t_284.(t_289) <- t_284.(fst i_292);
           t_284.(fst i_292) <- t_293;
           (t_288 := (~- (! t_288)))
          end else ();
          (Some (snd i_292))
       | None -> (None))
     end in
    (match t_297 with
     | Some (i_298) -> begin
         for j_299 = (t_289 + 1) to (t_286 - 1) do
          if (not ((t_284.(j_299)).(t_290) = 0)) then begin
           for j_300 = (t_290 + 1) to (t_285 - 1) do
            (t_284.(j_299)).(j_300) <-
             ((((t_284.(j_299)).(j_300) * (t_284.(t_289)).(t_290)) -
                ((t_284.(t_289)).(j_300) * (t_284.(j_299)).(t_289))) /
               (! t_287))
           done;
           (t_284.(j_299)).(t_290) <- 0
          end else () done;
         (t_287 := i_298)
        end;
        (t_281 := ((! t_281) + 1))
     | None -> (t_288 := 0));
    (t_282 := ((! t_282) + 1))
   done;
   (t_284, if ((! t_288) = 0) then 0
    else if ((! t_288) = 1) then (! t_287)
    else (~- (! t_287)), (! t_281))>.
\end{code2}
\end{frame}

\begin{frame}
\frametitle{Conclusions and future work}
\textbf{Conclusions}
\begin{itemize}
    \item This is not Gaussian Elimination specific -- this should scale
easily to all of Linear Algebra, and with a bit more work, to
most algebraic code.
    \item Need more syntactic sugar\dots
\end{itemize}

\textbf{Future work}
\begin{itemize}
    \item Implement more Linear Algebra algorithms
    \item Implement all aspects
    \item More complete code combinator library
    \item Add some abstract interpretation for algorithm selection
\end{itemize}

\end{frame}

\end{document}
