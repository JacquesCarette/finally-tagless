\documentclass[landscape]{slides}
\usepackage[dvipsnames,usenames]{color}
\usepackage{slidesec}
\usepackage{background}
\usepackage{pause}
\usepackage{amssymb,amsmath}

\newtheorem{defn}{Definition}
\newtheorem{thm}{Theorem}
\newtheorem{Ex}{Example}

\DeclareMathOperator{\len}{length}
\DeclareMathOperator{\theory}{theory}

\begin{document}
\pagestyle{empty}

\setlength{\topmargin}{0in}
\setlength{\headheight}{0.0in}
\setlength{\headsep}{0in}
\setlength{\topskip}{0in}
\setlength{\footskip}{0in}
%\setlength{\textwidth}{8.5in}
%\setlength{\textheight}{11in}

\newcommand{\vsi}{\vspace{-5mm}}
\newcommand{\vsii}{\vspace{-7mm}}
\newcommand{\vsiii}{\vspace{-11mm}}

\newcommand{\seq}[1]{{\langle #1 \rangle}}
\newcommand{\set}[1]{{\{ #1 \}}}
\newcommand{\tuple}[1]{{( #1 )}}
\newcommand{\mname}[1]{\mbox{\sf #1}}
\newcommand{\Nat}{\mathbb N}

\definecolor{bgblue}{rgb}{0.04,0.39,0.53}
\definecolor{blue1}{rgb}{0.95,0.95,1}
\definecolor{blue2}{rgb}{0.80,0.80,1}
\vpagecolor[blue1]{blue2}

\begin{slide}
\title{Multi-stage programming with functors and monads:
eliminating abstraction overhead from generic code}
\author{Jacques Carette\\
McMaster University\\
and\\
Oleg Kiselyov\\
FNMOC}
\date{June 6, 2005}
\maketitle
\end{slide}

\begin{slide}
\slideheading{Overview}
\begin{enumerate}
    \item Problem
    \item Design Points
    \item Tools: Functors, CPS, Monads, Extensible Grammar
    \item Solution \& results
\end{enumerate}

% What is simpler?\\
% \begin{tabular}{l|l}
% $0$ & $(x+3)^3 - x^3 - 9x^2 -27x -27$\pause \\
% $1+\sqrt{2}$ & $\sqrt[3]{7+5\sqrt{2}}$\pause \\
% $9$ & $4+2+1+1+1$\pause \\
% $2^{2^{20}}-1$ & \texttt{...} \pause \\
% $\texttt{ChebyshevT}(10000, x)$ & \texttt{...} \pause \\
% $\texttt{ChebyshevT}(5, x)$ & $16x^5 - 20x^3 + 5x$ \pause \\
% $1$ & $\frac{x-1}{x-1}$ \\
% \end{tabular}
\end{slide}

\begin{slide}
    \slideheading{Problem (Origins)}
    \begin{itemize}
        \item 35+ implementations of Gaussian Elimination in Maple
        \item Efficiency:
            \begin{itemize}
                \item[+] specialized versions 
                \item[-] generic slow
                \item[-] not enough specialization
            \end{itemize}
        \item Problem: how to unify without losing speed?
    \end{itemize}
\end{slide}

\begin{slide}
    \slideheading{Problem (as solved)}
    \textbf{Goal:} Type-safe, generic, efficient algorithms\\
    \textbf{Issues:}
    \begin{itemize}
        \item Untyped $\Rightarrow$ lots of debugging, no certainty\vsii
        \item Abstraction $\Rightarrow$ overhead 
            (unless you're the compiler writer)\vsii
        \item Efficiency requires \emph{no} overhead\vsi
    \end{itemize}
    $\Longrightarrow$ implement as a \emph{typed} meta-program
\end{slide}

\begin{slide}
    \slideheading{Design Points \emph{(aspects)}}
    \hspace*{-2cm}
        \begin{tabular}{|l|l|l|} \hline
        Design Dimension & Abstracts & Depends on \\ \hline
        Domain & Matrix values & -- \\ \hline
        Representation & Matrix representation & -- \\ \hline
        Fraction-free & use of division & Domain \\ \hline
        Pivoting & length measure & Domain \\ \hline
        Output & choice of output & -- \\ \hline
        Normalization & domain needs normalization & Domain \\ \hline
        Determinant & determinant tracking & Output \& Fraction-free \\ \hline
        ZeroEquivalence & decidability of $=0$ & -- \\ \hline
        UserInformation & user-feedback & ZeroEquivalence\\ \hline
        Augmented & matrix is augmented & -- \\ \hline
    \end{tabular}
\end{slide}

\begin{slide}
    \slideheading{let insertion}
\begin{verbatim}
let even n = (n mod 2) == 0;;
let square x = .<let z=.~x in z*z>. ;;

let rec power n x =
      if n=0 then .< 1 >.
      else if n=1 then x
      else if even n then square(power (n/2) x )
      else .< .~x * .~(power (n-1) x ) >. ;;

let p1 = .< fun y -> .~(power 5 .<y>.) >. ;;
# val p1 : ('a, int->int) code = 
  .< fun y1 -> (y1* let y3= let y2=y1 in (y2*y2) in (y3*y3)) >.
\end{verbatim}
\end{slide}

\begin{slide}
    \slideheading{Modules \& Functors}
\begin{verbatim}
module IntegerDomain = 
  struct
    type v = int
    type kind = domain_is_ring
    type 'a vc = ('a,v) code
    let zero = .< 0 >.  
    let one = .< 1 >. 
    let plus x y = ret .<.~x + .~y>. 
    let times x y = ret .<.~x * .~y>.
    let minus x y = ret .<.~x - .~y>.
    let uminus x = ret .< - .~x>.
    let div x y = ret .<.~x / .~y>. 
    let better_than = Some (fun x y -> ret .<abs .~x > abs .~y >. )
    let normalizerf = None 
    let normalizerg = fun x -> x
end
\end{verbatim}
\end{slide}

\begin{slide}
    \slideheading{Modules \& Functors}
\begin{verbatim}
module type DOMAIN = sig
  type v
  type kind (* Field or Ring ? *)
  type 'a vc = ('a,v) code
  val zero : 'a vc
  val one : 'a vc
  val plus : 'a vc -> 'a vc -> ('a vc, 's, 'w) monad
  val times : 'a vc -> 'a vc -> ('a vc, 's, 'w) monad
  val minus : 'a vc -> 'a vc -> ('a vc, 's, 'w) monad
  val uminus : 'a vc -> ('a vc, 's, 'w) monad
  val div : 'a vc -> 'a vc -> ('a vc, 's, 'w) monad
  val better_than : ('a vc -> 'a vc -> (('a,bool) code, 's, 'w) monad) option
  val normalizerf : (('a,v -> v) code ) option
  val normalizerg : 'a vc -> 'a vc
end 
\end{verbatim}
\end{slide}

\begin{slide}
    \slideheading{Modules \& Functors}
\begin{verbatim}
module GenericArrayContainer(Dom:DOMAIN) = struct
  type contr = Dom.v array array (* Array of rows *)
  type 'a vc = ('a,contr) code
  type 'a vo = ('a,Dom.v) code
  let get' x n m = .< (.~x).(.~n).(.~m) >.
  let get x n m = ret (get' x n m)
  let set' x n m y = .< (.~x).(.~n).(.~m) <- .~y >.
  let set x n m y = ret (set' x n m y)
  let dim2 x = .< Array.length .~x >.       (* number of rows *)
  let dim1 x = .< Array.length (.~x).(0) >. (* number of cols *)
  let mapper g a = match g with
      | Some f -> .< Array.map (fun x -> Array.map .~f x) .~a >.
      | None   -> a
  let copy = (fun a -> .<Array.map (fun x -> Array.copy x) 
                       (Array.copy .~a) >. )
  let swap_rows_stmt a r1 r2 =
      .< let t = (.~a).(.~r1) in
         begin 
             (.~a).(.~r1) <- (.~a).(.~r2);
             (.~a).(.~r2) <- t
         end >.
end
\end{verbatim}
\end{slide}

\begin{slide}
    \slideheading{Continuation-passing Style (CPS)}
\begin{verbatim}
let one = .<1>. and plus x y = .<.~x + .~y>.
let simplest_code = let gen x y = plus x (plus y one) in
  .<fun x y -> .~(gen .<x>. .<y>.)>.
# val .<fun x_1 -> fun y_2 -> (x_1 + (y_2 + 1))>.
let param_code1' plus one =
  let gen x y = let ce = (plus y one) in  plus ce (plus x ce) in
  .<fun x y -> .~(gen .<x>. .<y>.)>.
param_code1' plus one
# val .<fun x_1 -> fun y_2 -> ((y_2 + 1) + (x_1 + (y_2 + 1)))>.
\end{verbatim}
Problem: \verb=y_2+1= repeated.  But obvious solution doesn't work!
\end{slide}

\begin{slide}
    \slideheading{Continuation-passing Style (CPS)}
    But if you pass your future around\dots
\begin{verbatim}
let letgen exp k = .<let t = .~exp in .~(k .<t>.)>.
let param_code2 plus one =
  let gen x y k = letgen (plus y one) 
                  (fun ce -> k (plus ce (plus x ce)))
  and k0 x = x
  in .<fun x y -> .~(gen .<x>. .<y>. k0)>.
param_code2 plus one
#val .<fun x_1 -> fun y_2 -> 
           let t_3 = (y_2 + 1) in (t_3 + (x_1 + t_3))>.
\end{verbatim}
From \textit{A Methodology for Generating Verified Combinatorial Circuits} by 
Kiselyov, Swadi and Taha.
\end{slide}

\begin{slide}
    \slideheading{Monads}
    Method to encapsulate ``computations'', including state.\\
    We need \emph{state} and \emph{continuations}
\begin{verbatim}
type ('v,'s,'w) monad = 's -> ('s -> 'v -> 'w) -> 'w
let ret a = fun s k -> k s a
let bind a f = fun s k -> a s (fun s' b -> f b s' k)

let k0 s v = v
let runM m = m [] k0

(* not monadic -- idiom ? *)
let retN a = fun s k -> .<let t = .~a in .~(k s .<t>.)>.
\end{verbatim}
\end{slide}

\begin{slide}
    \slideheading{\textit{New} control structures}
\begin{verbatim}
let ifL test th el = ret .< if .~test then .~th else .~el >.

(* Note the implicit `reset` *)
let ifM test th el = fun s k ->
  k s .< if .~(test s k0) then .~(th s k0) else .~(el s k0) >.

let seqM a b = fun s k -> k s .< begin .~(a s k0) ; .~(b s k0) end >.

let rshiftM cf = fun s k -> k s (cf s)

let whenM test th  = rshiftM (fun s -> 
  .< if .~(test s k0) then .~(th s k0) else () >.)
\end{verbatim}
\end{slide}

\begin{slide}
    \slideheading{Extensible Grammar}
    In all programming languages, a lot of \emph{expressible abstractions}
    have atrocious syntax.  Definitely a problem of languages with
    just a few powerful primitives. \texttt{camlp4} for Ocaml allows one to add
    to (or redefine completely) the grammar of the language.  For monads:
\begin{verbatim}
let param_code3 plus one =
  let gen x y = bind (retN (plus y one)) (fun ce -> 
                ret (plus ce (plus x ce)))
  in .<fun x y -> .~(runM (gen .<x>. .<y>.))>.

let param_code4 plus one =
  let gen x y = doM ce <-- retN (plus y one);
                    ret (plus ce (plus x ce))
  in .<fun x y -> .~(runM (gen .<x>. .<y>.))>.
\end{verbatim}
\end{slide}

\begin{slide}
    \slideheading{Extensible Grammar (cont) }
A multi-monad version also exists
\begin{verbatim}
let param_code4 plus one =
  let gen x y = doM Monad in
                    ce <-- retN (plus y one);
                    ret (plus ce (plus x ce))
  in .<fun x y -> .~(runM (gen .<x>. .<y>.))>.
\end{verbatim}
which will use \verb+Monad.bind+ instead of \verb+bind+.
\end{slide}

\begin{slide}
    \slideheading{Solution}
    Have Modules and (higher-order) Functors, all using the CPS Monad, of
    \emph{code generators} that assemble the solution from base \emph{code
    combinators}.
\begin{verbatim}
module GenFA1 = 
    Gen(FloatDomain)
       (GenericArrayContainer)
       (RowPivot)
       (DivisionUpdate(FloatDomain)(GenericArrayContainer)
                      (NoDet(FloatDomain)))
       (OutJustMatrix(FloatDomain)(GenericArrayContainer)
                     (NoDet(FloatDomain)))
\end{verbatim}
\end{slide}

\begin{slide}
    \slideheading{Solution}
    \begin{small}
\begin{verbatim}
module Gen(Dom: DOMAIN)(C: CONTAINER2D)(PivotF: PIVOT)
          (Update: UPDATE with type baseobj = Dom.v 
                          and type ctr = C(Dom).contr)
          (Out: OUTPUT with type contr = C(Dom).contr 
                       and type D.indet = Dom.v 
                       and type 'a D.lstate = 'a Update.D.lstate) =
   struct
    module Ctr = C(Dom) module Pivot = PivotF(Dom)(C)(Out.D)
    type v = Dom.v
    let gen =
      let zerobelow b r c m n brc =
        let innerbody i = mdo {
            bic <-- Ctr.get b i c;
            whenM (l1 LogicCode.not (LogicCode.equal bic Dom.zero ))
                (seqM (retLoopM (Idx.succ c) (Idx.pred m)
                          (fun k -> Update.update b r c i k) )
                      (Ctr.set b i c Dom.zero)) } in 
        mdo { seqM (retLoopM (Idx.succ r) (Idx.pred n) innerbody) 
                   (Update.update_det brc) } in
      let dogen a = mdo {
          r <-- Out.R.decl ();
          c <-- retN (liftRef Idx.zero);
          b <-- retN (Ctr.mapper Dom.normalizerf (Ctr.copy a));
          m <-- retN (Ctr.dim1 a);
          n <-- retN (Ctr.dim2 a);
          () <-- Update.D.decl ();
          () <-- Out.P.decl ();
          seqM 
            (retWhileM (LogicCode.and_ (Idx.less (liftGet c) m)
                                       (Idx.less (liftGet r) n) )
               ( mdo {
               rr <-- retN (liftGet r);
               cc <-- retN (liftGet c);
               pivot <-- l1 retN (Pivot.findpivot b rr m cc n);
               seqM (retMatchM pivot (fun pv -> 
                        seqM (zerobelow b rr cc m n pv)
                             (Out.R.succ ()) )
                        (Update.D.zero_sign () ))
                    (Code.update c Idx.succ) } ))
            (Out.make_result b) } 
    in .<fun a -> .~(runM (dogen .<a>.)) >.
end
\end{verbatim}
\end{small}
\end{slide}

\begin{slide}
    \slideheading{Results}
$8$ domains, $5$ container types, $3$ pivoting strategies, 
$5$ output choices, $2$ elimination strategies\\
$\Longrightarrow$ over 800 (of 1200) type-safe possibilities.

Resulting code is \textbf{identical} to hand-written code.

Using \emph{offshoring}, can generate C (and sometimes Fortran) code
for results.
\end{slide}


\begin{slide}
    \slideheading{Sample Output}
\begin{small}
\begin{verbatim}
  ('a, Funct4.GenFA1.Ctr.contr ->
   Funct4.OutJustMatrix(Funct4.FloatDomain)(Funct4.GenericArrayContainer)(Funct4.NoDet(Funct4.FloatDomain)).res)
  code =
  .<fun a_1 ->
   let t_2 = (ref 0) in
   let t_3 = (ref 0) in
   let t_5 = (Array.map (fun x_4 -> (Array.copy x_4)) (Array.copy a_1)) in
   let t_6 = (Array.length a_1.(0)) in
   let t_7 = (Array.length a_1) in
   while (((! t_3) < t_6) && ((! t_2) < t_7)) do
    let t_8 = (! t_2) in let t_9 = (! t_3) in
    let t_10 = (ref (None)) in
    let t_16 =
     begin
      for j_13 = t_8 to (t_7 - 1) do
       let t_14 = (t_5.(j_13)).(t_9) in
       if (not (t_14 = 0.)) then
        (match (! t_10) with
         | Some (i_15) ->
            if ((abs_float (snd i_15)) < (abs_float t_14)) then
             (t_10 := (Some (j_13, t_14)))
            else ()
         | None -> (t_10 := (Some (j_13, t_14))))
       else ()
      done;
      (match (! t_10) with
       | Some (i_11) ->
          if ((fst i_11) <> t_8) then begin
           let t_12 = t_5.(t_8) in
           t_5.(t_8) <- t_5.(fst i_11);
           t_5.(fst i_11) <- t_12;
           ()
          end else ();
          (Some (snd i_11))
       | None -> (None))
     end in
    (match t_16 with
     | Some (i_17) ->
        begin
         for j_18 = (t_8 + 1) to (t_7 - 1) do
          if (not ((t_5.(j_18)).(t_9) = 0.)) then begin
           for j_19 = (t_9 + 1) to (t_6 - 1) do
            (t_5.(j_18)).(j_19) <-
             ((t_5.(j_18)).(j_19) -.
               (((t_5.(j_18)).(t_9) /. (t_5.(t_8)).(t_9)) *.
                 (t_5.(t_8)).(j_19)))
           done;
           (t_5.(j_18)).(t_9) <- 0.
          end else ()
         done;
         ()
        end;
        (t_2 := ((! t_2) + 1))
     | None -> ());
    (t_3 := ((! t_3) + 1))
   done;
   t_5>.

  ('a,
   Funct4.GenIA4.Ctr.contr ->
   Funct4.OutDetRank(Funct4.IntegerDomain)(Funct4.GenericArrayContainer)(Funct4.IDet)(Funct4.Rank).res)
  code =
  .<fun a_280 ->
   let t_281 = (ref 0) in
   let t_282 = (ref 0) in
   let t_284 =
    (Array.map (fun x_283 -> (Array.copy x_283)) (Array.copy a_280)) in
   let t_285 = (Array.length a_280.(0)) in
   let t_286 = (Array.length a_280) in
   let t_287 = (ref 1) in
   let t_288 = (ref 1) in
   while (((! t_282) < t_285) && ((! t_281) < t_286)) do
    let t_289 = (! t_281) in
    let t_290 = (! t_282) in
    let t_291 = (ref (None)) in
    let t_297 =
     begin
      for j_294 = t_289 to (t_286 - 1) do
       let t_295 = (t_284.(j_294)).(t_290) in
       if (not (t_295 = 0)) then
        (match (! t_291) with
         | Some (i_296) ->
            if ((abs (snd i_296)) > (abs t_295)) then
             (t_291 := (Some (j_294, t_295)))
            else ()
         | None -> (t_291 := (Some (j_294, t_295))))
       else ()
      done;
      (match (! t_291) with
       | Some (i_292) ->
          if ((fst i_292) <> t_289) then begin
           let t_293 = t_284.(t_289) in
           t_284.(t_289) <- t_284.(fst i_292);
           t_284.(fst i_292) <- t_293;
           (t_288 := (~- (! t_288)))
          end else ();
          (Some (snd i_292))
       | None -> (None))
     end in
    (match t_297 with
     | Some (i_298) ->
        begin
         for j_299 = (t_289 + 1) to (t_286 - 1) do
          if (not ((t_284.(j_299)).(t_290) = 0)) then begin
           for j_300 = (t_290 + 1) to (t_285 - 1) do
            (t_284.(j_299)).(j_300) <-
             ((((t_284.(j_299)).(j_300) * (t_284.(t_289)).(t_290)) -
                ((t_284.(t_289)).(j_300) * (t_284.(j_299)).(t_289))) /
               (! t_287))
           done;
           (t_284.(j_299)).(t_290) <- 0
          end else ()
         done;
         (t_287 := i_298)
        end;
        (t_281 := ((! t_281) + 1))
     | None -> (t_288 := 0));
    (t_282 := ((! t_282) + 1))
   done;
   (t_284,
    if ((! t_288) = 0) then 0
    else if ((! t_288) = 1) then (! t_287)
    else (~- (! t_287)), (! t_281))>.
\end{verbatim}
\end{small}
\end{slide}

\begin{slide}
    \slideheading{Timings}
	P4 2.00Ghz, FreeBSD 4.9-RELEASE-p1, 1Gig memory\\
	insert numbers here.
\end{slide}

\begin{slide}
    \slideheading{Conclusion}
    This is not Gaussian Elimination specific -- this should scale
    easily to all of Linear Algebra, and with a bit more work, to
    most algebraic code.

    Need more syntactic sugar\dots
\end{slide}

\end{document}

Hello!

	Here are some numbers, for byte- and native compilation.

My computer is FreeBSD 4.9-RELEASE-p1
hw.machine: i386
hw.model: Intel(R) Pentium(R) 4 CPU 2.00GHz
hw.ncpu: 1
hw.byteorder: 1234
hw.physmem: 1070080000
hw.usermem: 927322112
hw.pagesize: 4096
hw.floatingpoint: 1
hw.machine_arch: i386

BTW, I added an assertion to the files exper.ml and exper2.ml

Bytecode:

~/langs/ML/num-math/trunk/code> metaocamlc -verbose -pp "camlp4o pa_extend.cmo q_MLast.cmo ../camlp4/pa_monad.cmo" -I +camlp4 ~/Cache/ometa-cvs/lib/ocaml/nums.cma funct4.ml direct2.ml exper.ml

~/langs/ML/num-math/trunk/code> ./a.out 
__ generated ________________ 4096x avg= 2.865859E-01 msec
__ generation time ___________ 128x avg= 9.146503E+00 msec
__ higher-order ______________ 512x avg= 2.834229E+00 msec
__ generated _________________ 256x avg= 6.868903E+00 msec
__ generation time ___________ 128x avg= 9.111401E+00 msec
__ higher-order _______________ 16x avg= 6.680728E+01 msec
__ generated __________________ 32x avg= 3.144436E+01 msec
__ generation time ___________ 128x avg= 9.137982E+00 msec
__ higher-order ________________ 4x avg= 3.045543E+02 msec
__ generated __________________ 16x avg= 8.469101E+01 msec
__ generation time ___________ 128x avg= 9.035031E+00 msec
__ higher-order ________________ 2x avg= 8.165625E+02 msec


~/langs/ML/num-math/trunk/code> metaocamlc -verbose -pp "camlp4o pa_extend.cmo q_MLast.cmo ../camlp4/pa_monad.cmo" -I +camlp4 ~/Cache/ometa-cvs/lib/ocaml/nums.cma funct4.ml direct2.ml exper2.ml

~/langs/ML/num-math/trunk/code> ./a.out 
__ generated ________________ 8192x avg= 1.842597E-01 msec
__ generation time ___________ 128x avg= 9.033271E+00 msec
__ higher-order ______________ 512x avg= 3.551199E+00 msec
__ generated _________________ 512x avg= 2.001440E+00 msec
__ generation time ___________ 128x avg= 9.044719E+00 msec
__ higher-order _______________ 32x avg= 4.186380E+01 msec
__ generated _________________ 128x avg= 1.135904E+01 msec
__ generation time ___________ 128x avg= 9.038421E+00 msec
__ higher-order ________________ 8x avg= 2.443530E+02 msec
__ generated __________________ 32x avg= 5.056813E+01 msec
__ generation time ___________ 128x avg= 8.988767E+00 msec
__ higher-order ________________ 1x avg= 1.102511E+03 msec


Niative code. That gave me some headache. Direct metaocamlopt with -pp
option didn't work, some kind of fatal error. So, I had to expand
manually:

~/langs/ML/num-math/trunk/code> camlp4o -I ../camlp4/ ../camlp4/pa_monad.cmo pr_o.cmo funct4.ml > /tmp/funct4.ml
~/langs/ML/num-math/trunk/code> camlp4o -I ../camlp4/ ../camlp4/pa_monad.cmo pr_o.cmo direct2.ml > /tmp/direct2.ml

Alas, the file /tmp/funct4.ml needed a lot of adjustments: inserting
parenetheses in many places. The pretty-printer pr_o.cmo misjudged the
precedences of .~ operations and dropped the parentheses. It's so good
to have static typing: OCaml helped me to put those parentheses back.

Here are the results:

~/langs/ML/num-math/trunk/code> metaocamlopt -verbose -I . -I /tmp ~/Cache/ometa-cvs/lib/ocaml/nums.cmxa /tmp/funct4.ml /tmp/direct2.ml exper.ml
~/langs/ML/num-math/trunk/code> ./a.out 
__ generated _______________ 65536x avg= 1.753359E-02 msec
__ generation time ____________ 32x avg= 4.680109E+01 msec
__ higher-order _____________ 1024x avg= 1.247749E+00 msec
__ generated ________________ 4096x avg= 3.586502E-01 msec
__ generation time ____________ 32x avg= 5.856777E+01 msec
__ higher-order _______________ 64x avg= 2.944830E+01 msec
__ generated ________________ 1024x avg= 1.817178E+00 msec
__ generation time ____________ 16x avg= 6.433709E+01 msec
__ higher-order ________________ 8x avg= 1.302654E+02 msec
__ generated _________________ 256x avg= 4.856711E+00 msec
__ generation time ____________ 16x avg= 6.970411E+01 msec
__ higher-order ________________ 4x avg= 3.585474E+02 msec


~/langs/ML/num-math/trunk/code> metaocamlopt -verbose -I . -I /tmp ~/Cache/ometa-cvs/lib/ocaml/nums.cmxa /tmp/funct4.ml /tmp/direct2.ml exper2.ml
__ generated _______________ 65536x avg= 2.217953E-02 msec
__ generation time ____________ 32x avg= 4.666901E+01 msec
__ higher-order _____________ 1024x avg= 1.483538E+00 msec
__ generated ________________ 4096x avg= 2.571623E-01 msec
__ generation time ____________ 32x avg= 5.880562E+01 msec
__ higher-order _______________ 64x avg= 1.722143E+01 msec
__ generated ________________ 1024x avg= 1.513102E+00 msec
__ generation time ____________ 16x avg= 6.604590E+01 msec
__ higher-order _______________ 16x avg= 1.029261E+02 msec
__ generated _________________ 256x avg= 6.858643E+00 msec
__ generation time ____________ 16x avg= 7.118375E+01 msec
__ higher-order ________________ 4x avg= 4.581764E+02 msec
