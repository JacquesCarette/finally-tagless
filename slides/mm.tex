\documentclass[landscape]{slides}
\usepackage[dvipsnames,usenames]{color}
\usepackage{slidesec}
\usepackage{background}
\usepackage{pause}
\usepackage{amssymb,amsmath}

\newtheorem{defn}{Definition}
\newtheorem{thm}{Theorem}
\newtheorem{Ex}{Example}

\DeclareMathOperator{\len}{length}
\DeclareMathOperator{\theory}{theory}

\begin{document}
\pagestyle{empty}

\setlength{\topmargin}{0in}
\setlength{\headheight}{0.0in}
\setlength{\headsep}{0in}
\setlength{\topskip}{0in}
\setlength{\footskip}{0in}
%\setlength{\textwidth}{8.5in}
%\setlength{\textheight}{11in}

\newcommand{\vsi}{\vspace{-5mm}}
\newcommand{\vsii}{\vspace{-7mm}}
\newcommand{\vsiii}{\vspace{-11mm}}

\newcommand{\seq}[1]{{\langle #1 \rangle}}
\newcommand{\set}[1]{{\{ #1 \}}}
\newcommand{\tuple}[1]{{( #1 )}}
\newcommand{\mname}[1]{\mbox{\sf #1}}
\newcommand{\Nat}{\mathbb N}

\definecolor{bgblue}{rgb}{0.04,0.39,0.53}
\definecolor{blue1}{rgb}{0.95,0.95,1}
\definecolor{blue2}{rgb}{0.80,0.80,1}
\vpagecolor[blue1]{blue2}

\begin{slide}
\title{Functors, CPS and Monads,\\
or how to generate efficient algebraic code from abstract designs}
\author{Jacques Carette\\
McMaster University\\
and\\
Oleg Kiselyov}
\date{June 6, 2005}
\maketitle
\end{slide}

\begin{slide}
\slideheading{Overview}
\begin{enumerate}
    \item Problem
    \item MetaOCaml
    \item Design Points
    \item Functors, CPS, Monads, Extensible Grammar
    \item Solution \& results
\end{enumerate}

% What is simpler?\\
% \begin{tabular}{l|l}
% $0$ & $(x+3)^3 - x^3 - 9x^2 -27x -27$\pause \\
% $1+\sqrt{2}$ & $\sqrt[3]{7+5\sqrt{2}}$\pause \\
% $9$ & $4+2+1+1+1$\pause \\
% $2^{2^{20}}-1$ & \texttt{...} \pause \\
% $\texttt{ChebyshevT}(10000, x)$ & \texttt{...} \pause \\
% $\texttt{ChebyshevT}(5, x)$ & $16x^5 - 20x^3 + 5x$ \pause \\
% $1$ & $\frac{x-1}{x-1}$ \\
% \end{tabular}
\end{slide}

\begin{slide}
    \slideheading{Problem (Origins)}
    \begin{itemize}
        \item 35+ implementations of GE in Maple
        \item Efficiency:
            \begin{itemize}
                \item[+] specialized versions 
                \item[-] generic slow
                \item[-] not enough specialization
            \end{itemize}
        \item Problem: how to unify without losing speed?
    \end{itemize}
\end{slide}

\begin{slide}
    \slideheading{Problem (as solved)}
    \textbf{Goal:} Type-safe, generic, efficient algorithms\\
    \textbf{Issues:}
    \begin{itemize}
        \item Untyped $\Rightarrow$ lots of debugging, no certainty\vsii
        \item Abstraction $\Rightarrow$ overhead 
            (unless you're the compiler writer)\vsii
        \item Efficiency requires \emph{no} overhead\vsi
    \end{itemize}
    $\Longrightarrow$ implement as a \emph{typed} meta-program
\end{slide}

\begin{slide}
    \slideheading{www.metaocaml.org}
    Version of \textbf{Ocaml} with typed metaprogramming facilities\\
    \begin{tabular}{lll}\hline
        \verb=.< x + y >.= & \emph{code} of $x+y$ & code expression\\
        \verb=.~z= & \emph{code-value} of (code) $z$ & splice \\
        \verb=.!z= & \emph{value} of (code) $z$ & run \\
    \end{tabular}\\
    Essentially: a form of \emph{typed} symbolic computation!\\
    Except: \textit{closed} code cannot be manipulated other than being
    spliced into other code.
\end{slide}

\begin{slide}
    \slideheading{MetaOCaml example}
    The classic: powering!
\begin{verbatim}
let square x = .< .~x * .~x >. ;;

let rec power n x =
  if n=0 then .< 1 >.
  else .< .~x * .~(power (n-1) x ) >. ;;

let p1 = .< fun y -> .~(power 5 .<y>.) >. ;;
# val p1 : ('a, int->int) code = 
  .< fun y1 -> (y1*(y1*(y1*(y1*(y1*1))))) >.
\end{verbatim}
\end{slide}

\begin{slide}
    \slideheading{MetaOCaml example (cont)}
\begin{verbatim}
let even n = (n mod 2) == 0;;
let square x = .< .~x * .~x >. ;;

let rec power n x =
      if n=0 then .< 1 >.
      else if n=1 then x
      else if even n then square(power (n/2) x )
      else .< .~x * .~(power (n-1) x ) >. ;;

let p1 = .< fun y -> .~(power 5 .<y>.) >. ;;
# val p1 : ('a, int->int) code = 
  .< fun y1 -> (y1*(y1*y1)*(y1*y1)) >.
\end{verbatim}
\end{slide}

\begin{slide}
    \slideheading{MetaOCaml example (cont)}
\begin{verbatim}
let even n = (n mod 2) == 0;;
let square x = .<let z=.~x in z*z>. ;;

let rec power n x =
      if n=0 then .< 1 >.
      else if n=1 then x
      else if even n then square(power (n/2) x )
      else .< .~x * .~(power (n-1) x ) >. ;;

let p1 = .< fun y -> .~(power 5 .<y>.) >. ;;
# val p1 : ('a, int->int) code = 
  .< fun y1 -> (y1* let y3= let y2=y1 in (y2*y2) in (y3*y3)) >.
\end{verbatim}
\end{slide}

\begin{slide}
    \slideheading{Design Points \emph{(aspects)}}
    \hspace*{-2cm}
        \begin{tabular}{|l|l|l|} \hline
        Design Dimension & Abstracts & Depends on \\ \hline
        Domain & Matrix values & -- \\ \hline
        Representation & Matrix representation & -- \\ \hline
        Fraction-free & use of division & Domain \\ \hline
        Pivoting & length measure & Domain \\ \hline
        Output & choice of output & -- \\ \hline
        Normalization & domain needs normalization & Domain \\ \hline
        Determinant & determinant tracking & Output \& Fraction-free \\ \hline
        ZeroEquivalence & decidability of $=0$ & -- \\ \hline
        UserInformation & user-feedback & ZeroEquivalence\\ \hline
        Augmented & matrix is augmented & -- \\ \hline
    \end{tabular}
\end{slide}

\begin{slide}
    \slideheading{Modules \& Functors}
\begin{verbatim}
module IntegerDomain = 
  struct
    type v = int
    type kind = domain_is_ring
    type 'a vc = ('a,v) code
    let zero = .< 0 >.  
    let one = .< 1 >. 
    let plus x y = ret .<.~x + .~y>. 
    let times x y = ret .<.~x * .~y>.
    let minus x y = ret .<.~x - .~y>.
    let uminus x = ret .< - .~x>.
    let div x y = ret .<.~x / .~y>. 
    let better_than = Some (fun x y -> ret .<abs .~x > abs .~y >. )
    let normalizerf = None 
    let normalizerg = fun x -> x
end
\end{verbatim}
\end{slide}

\begin{slide}
    \slideheading{Modules \& Functors}
\begin{verbatim}
module type DOMAIN = sig
  type v
  type kind (* Field or Ring ? *)
  type 'a vc = ('a,v) code
  val zero : 'a vc
  val one : 'a vc
  val plus : 'a vc -> 'a vc -> ('a vc, 's, 'w) monad
  val times : 'a vc -> 'a vc -> ('a vc, 's, 'w) monad
  val minus : 'a vc -> 'a vc -> ('a vc, 's, 'w) monad
  val uminus : 'a vc -> ('a vc, 's, 'w) monad
  val div : 'a vc -> 'a vc -> ('a vc, 's, 'w) monad
  val better_than : ('a vc -> 'a vc -> (('a,bool) code, 's, 'w) monad) option
  val normalizerf : (('a,v -> v) code ) option
  val normalizerg : 'a vc -> 'a vc
end 
\end{verbatim}
\end{slide}

\begin{slide}
    \slideheading{Modules \& Functors}
\begin{verbatim}
module GenericArrayContainer(Dom:DOMAIN) = struct
  type contr = Dom.v array array (* Array of rows *)
  type 'a vc = ('a,contr) code
  type 'a vo = ('a,Dom.v) code
  let get' x n m = .< (.~x).(.~n).(.~m) >.
  let get x n m = ret (get' x n m)
  let set' x n m y = .< (.~x).(.~n).(.~m) <- .~y >.
  let set x n m y = ret (set' x n m y)
  let dim2 x = .< Array.length .~x >.       (* number of rows *)
  let dim1 x = .< Array.length (.~x).(0) >. (* number of cols *)
  let mapper g a = match g with
      | Some f -> .< Array.map (fun x -> Array.map .~f x) .~a >.
      | None   -> a
  let copy = (fun a -> .<Array.map (fun x -> Array.copy x) 
                       (Array.copy .~a) >. )
  let swap_rows_stmt a r1 r2 =
      .< let t = (.~a).(.~r1) in
         begin 
             (.~a).(.~r1) <- (.~a).(.~r2);
             (.~a).(.~r2) <- t
         end >.
end
\end{verbatim}
\end{slide}

\begin{slide}
    \slideheading{Continuation-passing Style (CPS)}
\begin{verbatim}
let one = .<1>. and plus x y = .<.~x + .~y>.
let simplest_code = let gen x y = plus x (plus y one) in
  .<fun x y -> .~(gen .<x>. .<y>.)>.
# val .<fun x_1 -> fun y_2 -> (x_1 + (y_2 + 1))>.
let param_code1' plus one =
  let gen x y = let ce = (plus y one) in  plus ce (plus x ce) in
  .<fun x y -> .~(gen .<x>. .<y>.)>.
param_code1' plus one
# val .<fun x_1 -> fun y_2 -> ((y_2 + 1) + (x_1 + (y_2 + 1)))>.
\end{verbatim}
Problem: \verb=y_2+1= repeated.  But obvious solution doesn't work!
\end{slide}

\begin{slide}
    \slideheading{Continuation-passing Style (CPS)}
    But if you pass your future around\dots
\begin{verbatim}
let letgen exp k = .<let t = .~exp in .~(k .<t>.)>.
let param_code2 plus one =
  let gen x y k = letgen (plus y one) 
                  (fun ce -> k (plus ce (plus x ce)))
  and k0 x = x
  in .<fun x y -> .~(gen .<x>. .<y>. k0)>.
param_code2 plus one
#val .<fun x_1 -> fun y_2 -> 
           let t_3 = (y_2 + 1) in (t_3 + (x_1 + t_3))>.
\end{verbatim}
\end{slide}

\begin{slide}
    \slideheading{Monads}
    Also known (in category theory) as \emph{Kleisli triples}.\\
    Method to encapsulate ``computations'', including state.\\
    We need \emph{state} and \emph{continuations}
\begin{verbatim}
type ('v,'s,'w) monad = 's -> ('s -> 'v -> 'w) -> 'w
let ret a = fun s k -> k s a
let bind a f = fun s k -> a s (fun s' b -> f b s' k)

let k0 s v = v
let runM m = m [] k0

let retN a = fun s k -> .<let t = .~a in .~(k s .<t>.)>.

let ifL test th el = ret .< if .~test then .~th else .~el >.
let ifM test th el = fun s k ->
  k s .< if .~(test s k0) then .~(th s k0) else .~(el s k0) >.
\end{verbatim}
\end{slide}

\begin{slide}
    \slideheading{Extensible Grammar}
    In all programming languages, a lot of \emph{expressible abstractions}
    have attrocious syntax.  Definitely a problem of languages with
    just a few powerful primitives. \texttt{camlp4} for Ocaml allows one to add
    to (or redefine completely) the grammar of the language.  For monads:
\begin{verbatim}
let param_code3 plus one =
  let gen x y = bind (retN (plus y one)) (fun ce -> 
                ret (plus ce (plus x ce)))
  in .<fun x y -> .~(runM (gen .<x>. .<y>.))>.

let param_code4 plus one =
  let gen x y = mdo \{ ce <-- retN (plus y one);
                      ret (plus ce (plus x ce)) \}
  in .<fun x y -> .~(runM (gen .<x>. .<y>.))>.
\end{verbatim}
\end{slide}

\begin{slide}
    \slideheading{Solution}
    Have Modules and (higher-order) Functors, all using the CPS Monad, of
    \emph{code generators} that assemble the solution from base \emph{code
    combinators}.
\begin{verbatim}
module GenFA1 = 
    Gen(FloatDomain)
       (GenericArrayContainer)
       (RowPivot)
       (DivisionUpdate(FloatDomain)(GenericArrayContainer)
                      (NoDet(FloatDomain)))
       (OutJustMatrix(FloatDomain)(GenericArrayContainer)
                     (NoDet(FloatDomain)))
\end{verbatim}
\end{slide}

\begin{slide}
    \slideheading{Solution}
    \begin{small}
\begin{verbatim}
module Gen(Dom: DOMAIN)(C: CONTAINER2D)(PivotF: PIVOT)
          (Update: UPDATE with type baseobj = Dom.v 
                          and type ctr = C(Dom).contr)
          (Out: OUTPUT with type contr = C(Dom).contr 
                       and type D.indet = Dom.v 
                       and type 'a D.lstate = 'a Update.D.lstate) =
   struct
    module Ctr = C(Dom) module Pivot = PivotF(Dom)(C)(Out.D)
    type v = Dom.v
    let gen =
      let zerobelow b r c m n brc =
        let innerbody i = mdo {
            bic <-- Ctr.get b i c;
            whenM (l1 LogicCode.not (LogicCode.equal bic Dom.zero ))
                (seqM (retLoopM (Idx.succ c) (Idx.pred m)
                          (fun k -> Update.update b r c i k) )
                      (Ctr.set b i c Dom.zero)) } in 
        mdo { seqM (retLoopM (Idx.succ r) (Idx.pred n) innerbody) 
                   (Update.update_det brc) } in
      let dogen a = mdo {
          r <-- Out.R.decl ();
          c <-- retN (liftRef Idx.zero);
          b <-- retN (Ctr.mapper Dom.normalizerf (Ctr.copy a));
          m <-- retN (Ctr.dim1 a);
          n <-- retN (Ctr.dim2 a);
          () <-- Update.D.decl ();
          () <-- Out.P.decl ();
          seqM 
            (retWhileM (LogicCode.and_ (Idx.less (liftGet c) m)
                                       (Idx.less (liftGet r) n) )
               ( mdo {
               rr <-- retN (liftGet r);
               cc <-- retN (liftGet c);
               pivot <-- l1 retN (Pivot.findpivot b rr m cc n);
               seqM (retMatchM pivot (fun pv -> 
                        seqM (zerobelow b rr cc m n pv)
                             (Out.R.succ ()) )
                        (Update.D.zero_sign () ))
                    (Code.update c Idx.succ) } ))
            (Out.make_result b) } 
    in .<fun a -> .~(runM (dogen .<a>.)) >.
end
\end{verbatim}
\end{small}
\end{slide}
\begin{slide}
    \slideheading{Results}
$8$ domains, $5$ container types, $3$ pivoting strategies, 
$5$ output choices, $2$ elimination strategies\\
$\Longrightarrow$ over 800 (of 1200) type-safe possibilities.

Resulting code is \textbf{identical} to hand-written code.

Using \emph{offshoring}, can generate C (and sometimes Fortran) code
for results.
\end{slide}

\begin{slide}
    \slideheading{Conclusion}
    This is not Gaussian Elimination specific -- this should scale
    easily to all of Linear Algebra, and with a bit more work, to
    most algebraic code.

    Need more syntactic sugar\dots
\end{slide}

\end{document}
